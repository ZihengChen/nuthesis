\subsection{Event selection}
\label{sec:event}

The event selection begins by requiring an event pass the lowest
\pt theshold single electron or muon trigger that is not prescaled.
From these datasets it is possible to select on a number of $\PW\PW$-like
final states originating from \ttbar and tW production.  These final
states are constructed based on the number of reconstructed leptons,
jet multiplicity, and b tag multiplicity.  The categorization of these
events differs between the counting and shape analysis.  The common
definition of the categories are listed below.

\begin{itemize}
    \item $ee$:
    \begin{itemize}
        \item exactly two electrons
        \item $p_{T} > 30, 20$ GeV
        \item reject events with hadronic taus or muons
        \item Z boson veto ($N_{b} \geq 1$ only): $M_{ee} < 75$ or $M_{ee} > 105\,\GeV$
    \end{itemize}
    \item $\mu\mu$:
    \begin{itemize}
        \item exactly two muons
        \item $p_{T} > 25, 10$ GeV
        \item reject events with hadronic taus or electrons
        \item Z boson veto ($N_{b} \geq 1$ only): $M_{\mu\mu} < 75$ or $M_{\mu\mu} > 105\,\GeV$
    \end{itemize}
    \item $e\mu$:
    \begin{itemize}
        \item exactly one electron and one muon
        \item lead muon (electron) $p_{T} > 25 (30)\,\GeV$
        \item trailing muon (electron) $p_{T} > 10\,(20)\,\GeV$
        \item reject events with hadronic taus 
        \item events in electron datastream that fire muon trigger are vetoed
    \end{itemize}
    \item $e\tau$:
    \begin{itemize}
        \item exactly one electron and one hadronic tau
        \item $p_{T} > 30, 20$ GeV
        \item reject events with muons
    \end{itemize}
    \item $\mu\tau$:
    \begin{itemize}
        \item exactly one muon and one hadronic tau
        \item $p_{T} > 25, 20$ GeV
        \item reject events with electrons
    \end{itemize}
    \item e + jets:
    \begin{itemize}
        \item exactly one electron 
        \item $p_{T} > 30$ GeV
        \item reject events with muons or hadronic taus
        \item at least four jets
    \end{itemize}
    \item $\mu + jets$:
    \begin{itemize}
        \item exactly one muon 
        \item $p_{T} > 25$ GeV
        \item reject events with electrons or hadronic taus
        \item at least four jets
    \end{itemize}
\end{itemize}

These selections are designed to primarily target $\ttbar$ production
and specific \PW decay modes.  The final states will tend to only
contain events from a single datastream except for the $e\mu$ selection
which has non-negligible overlap between the electron and muon
datastreams.  Any overlap in events between the two datastreams are
removed by only taking the event from single muon datastream.  Because
the tau can decay to an electron, muon, or hadronically, each of these
channels has some mixing between terms arising from $W\rightarrow\ell$
decays and $W\rightarrow\tau\rightarrow\ell$ decays.  The mixing between
the selected final states and the underlying W boson decays are shown in
table~\ref{tab:signal_breakdown}.  These numbers are estimated from
simulated \ttbar events and are consequently dependent on the values of
branching fractions used in the simulation.  

For the counting analysis, there is always a requirement that there be
at least two jets and at least one b tagged jet. The categories are
partitioned based on whether there is exactly one b jet or there are two
or more b jets.  Additionally, the $e\mu$ selection is split into muon
triggered and electron triggered categories: if the muon has the highest
\pt and the muon trigger has fired it categorized as a $\mu e$ event; if
the electron has highest \pt and the electron trigger fired it is
categorized as an $e \mu$ event.

The shape analysis takes advantage of subdividing the data into several
additional b tag and jet multiplicity categories to constrain
systematics uncertainties, and to take advantage of higher tau
reconsruction purity in lower jet multiplicity bins.  The categories are
shown in table~\ref{tab:jet_categories}.  As described in the list
above, the $ee$ and $\mu\mu$ categories have a Z veto applied in the
case that there are one or more b tags; this requirement is not applied
in the zero b tag case.  There is also a set of requirements to enhance
the proportion of Drell-Yan in the $e\tau$ and $\mu\tau$ categories in
the case that the number of jets is 0 or 1 and there are no b tags.  The
requirements are constructed to mainly reduce the W boson contribution
and are :

\begin{itemize}
    \item $40 \GeV \leq M_{\ell\tau_{h}} \leq 100 \GeV$,
    \item $\Delta\phi(\ell, \tau_{h}) > 2.5$,
    \item $M_{T}^{\ell} < 60 \GeV$,
\end{itemize}

where $M_{T}^{\ell}$ is the transverse mass of the electron or muon,

\begin{equation}
\label{eq:trans_mass}
    M_{T,\ell} = \sqrt{2 p_{T}^{\ell}\MET (1-\cos\Delta\phi(p_{T}^{\ell}, \MET))}.
\end{equation}

\begin{table}[]
    \centering
    \setlength{\tabcolsep}{1.5em}
    \renewcommand{\arraystretch}{1.1}
    \caption{Categories used in the shape analysis based on jet and b
    tag multiplicities.}
    
    \begin{tabular}{l|c|c|c|c}
                                    & $N_{j} = 0$        & $N_{j} = 1$        & $N_{j} = 2$        & $N_{j} \geq 3$     \\
	\hline
    \multirow{2}{*}{$N_{b} = 0$}    & $e\tau$, $\mu\tau$ & $e\tau$, $\mu\tau$ & \multicolumn{2}{c}{$e\tau$, $\mu\tau$} \\
                                    & $e\mu$             & $e\mu$             & \multicolumn{2}{c}{$ee, \mu\mu, e\mu$} \\
	\hline
    \multirow{3}{*}{$N_{b} = 1$}    &                    & $e\tau$, $\mu\tau$ & $e\tau$, $\mu\tau$ & $e\tau$, $\mu\tau$ \\
	\cline{4-5}
                                    &                    & $e\mu$             & \multicolumn{2}{c}{$ee, \mu\mu, e\mu$}  \\
                                    &                    &                    & \multicolumn{2}{c}{$ej$, $\mu j$}  \\
	\hline
    \multirow{3}{*}{$N_{b} \geq 2$} & \multicolumn{2}{c|}{}                   & $e\tau$, $\mu\tau$ & $e\tau$, $\mu\tau$ \\
	\cline{4-5}
                                    & \multicolumn{2}{c|}{}                   & \multicolumn{2}{c}{$ee, \mu\mu, e\mu$}  \\
                                    & \multicolumn{2}{c|}{}                   & \multicolumn{2}{c}{$ej$, $\mu j$}  \\
	\hline
    \end{tabular}
    
    \label{tab:jet_categories}
\end{table}


Estimates of the data yields for each of the categories are shown in
table~\ref{tab:yields} and \ref{tab:yields_ltau}.  Comparisons between
the observed data and the expectation based on simulation and
data-driven background estimates are shown in
appendix~\ref{appendix:plots}. 


\begin{sidewaystable}[]
    \centering
    \setlength{\tabcolsep}{0.4em}
    \renewcommand{\arraystretch}{1.5}
    \begin{tabular}{l|cc|cc|cc|cc|cc|cc|cc}
    \hline
    channel                & \multicolumn{2}{c|}{$\mu\mu$} & \multicolumn{2}{|c|}{$ee$} & \multicolumn{2}{|c|}{$e\mu$} & \multicolumn{2}{|c|}{$\mu\tau$} & \multicolumn{2}{|c|}{$e\tau$} & \multicolumn{2}{|c|}{$\mu$+jets} & \multicolumn{2}{|c}{$e+jets$} \\
    \hline
    $\rm n_{b tag}$        & $=1$ & $\geq 2$ & $=1$ & $\geq 2$ & $=1$ & $\geq 2$ & $=1$ & $\geq 2$ & $=1$ & $\geq 2$ & $=1$ & $\geq 2$ & $=1$ & $\geq 2$ \\
    \hline                                                                                                                               
    $ee$                   & --   & --       & 85.1 & 85.8     & --   & --       & --   & --       & 0.8  & 0.6      & --   & --       & 3.4  & 3.6      \\
    $\mu\mu$               & 82.0 & 83.3     & --   & --       & --   & --       & 0.4  & 0.3      & --   & --       & 1.5  & 1.6      & --   & --       \\
    $e\mu$                 & 0.1  & --       & --   & --       & 85.8 & 86.3     & 0.9  & 0.5      & 0.3  & 0.2      & 3.4  & 3.6      & 1.5  & 1.6      \\
    $\tau_{e}\tau_{e}$     & --   & --       & 0.5  & 0.5      & --   & --       & --   & --       & --   & --       & --   & --       & --   & --       \\
    $\tau_{\mu}\tau_{\mu}$ & 0.6  & 0.7      & --   & --       & --   & --       & --   & --       & --   & --       & --   & --       & --   & --       \\
    $\tau_{e}\tau_{\mu}$   & --   & --       & --   & --       & 0.5  & 0.5      & --   & --       & --   & --       & --   & --       & --   & --       \\
    $\tau_{e}\tau_{h}$     & --   & --       & --   & --       & --   & --       & --   & --       & 2.9  & 3.0      & --   & --       & 0.2  & 0.2      \\
    $\tau_{\mu}\tau_{h}$   & --   & --       & --   & --       & --   & --       & 2.9  & 3.3      & --   & --       & 0.2  & 0.2      & --   & --       \\
    $\tau_{h}\tau_{h}$     & --   & --       & --   & --       & --   & --       & --   & --       & --   & --       & --   & --       & --   & --       \\
    $e\tau_{e}$            & --   & --       & 13.6 & 13.3     & --   & --       & --   & --       & 0.2  & 0.1      & --   & --       & 0.8  & 0.9      \\
    $e\tau_{\mu}$          & --   & --       & --   & --       & 5.6  & 5.5      & 0.1  & --       & 0.1  & 0.1      & 0.2  & 0.2      & 0.4  & 0.4      \\
    $e\tau_{h}$            & --   & --       & 0.2  & 0.1      & --   & --       & --   & --       & 52.6 & 59.0     & --   & --       & 3.3  & 3.5      \\
    $\mu\tau_{e}$          & --   & --       & --   & --       & 7.2  & 7.4      & 0.2  & 0.1      & --   & --       & 0.6  & 0.7      & 0.1  & 0.1      \\
    $\mu\tau_{\mu}$        & 15.6 & 15.6     & --   & --       & --   & --       & 0.1  & --       & --   & --       & 0.5  & 0.5      & --   & --       \\
    $\mu\tau_{h}$          & 0.1  & --       & --   & --       & 0.1  & 0.1      & 52.6 & 59.2     & --   & --       & 3.3  & 3.5      & --   & --       \\
    $eh$                   & --   & --       & 0.6  & 0.2      & 0.3  & 0.1      & --   & --       & 40.8 & 35.1     & --   & --       & 85.3 & 84.9     \\
    $\mu h$                & 1.4  & 0.4      & --   & --       & 0.3  & 0.2      & 40.5 & 34.7     & --   & --       & 84.5 & 84.1     & --   & --       \\
    $\tau_{e}h$            & --   & --       & --   & --       & --   & --       & --   & --       & 2.1  & 1.8      & --   & --       & 4.8  & 4.7      \\
    $\tau_{\mu}h$          & 0.1  & --       & --   & --       & --   & --       & 2.3  & 1.8      & --   & --       & 5.5  & 5.4      & --   & --       \\
    $\tau_{h}h$            & --   & --       & --   & --       & --   & --       & 0.1  & --       & 0.1  & --       & --   & --       & 0.1  & --       \\
    $hh$                   & --   & --       & --   & --       & --   & --       & --   & --       & --   & --       & 0.2  & --       & 0.1  & 0.1      \\
    \end{tabular}
    \caption{$\ttbar+\Pqt\PW$ sample composition for each of the
    seven categories and b tag multiplicities.  Values are in
    percent.\label{tab:signal_breakdown}}
\end{sidewaystable}

\begin{sidewaystable}[p]
    \centering
    \setlength{\tabcolsep}{0.4em}
    \renewcommand{\arraystretch}{2}
    \small
    \begin{tabular}{l|cccccccc|cc}
    \hline
        & QCD & VV  & $\gamma$ & Z & W & t & tW & tt & total & data      \\
    \hline
    
    $\mu e$, $n_b=1$                   &       --$\pm$     -- &     90.3$\pm$    4.2 &      0.9$\pm$    0.9 &    202.7$\pm$   37.6 &     13.4$\pm$    5.1 &      9.5$\pm$    2.6 &   2107.6$\pm$   53.1 &  38871.4$\pm$   87.5 &  41295.8$\pm$  109.2 &  41047.0$\pm$  202.6 \\ 
    $\mu e$, $n_b\geq2$                &       --$\pm$     -- &      5.9$\pm$    1.0 &       --$\pm$     -- &       --$\pm$     -- &      3.1$\pm$    2.2 &      2.3$\pm$    1.6 &    625.7$\pm$   28.9 &  22647.7$\pm$   66.8 &  23270.9$\pm$   74.1 &  23918.0$\pm$  154.7 \\ 
    \hline
    $\mu\mu$, $n_b=1$                  &       --$\pm$     -- &    370.4$\pm$    5.8 &      4.1$\pm$    1.8 &  18046.9$\pm$  455.4 &     52.4$\pm$   11.7 &     55.8$\pm$    6.7 &   3406.2$\pm$   68.8 &  62266.6$\pm$  112.4 &  84202.3$\pm$  474.3 &  84284.0$\pm$  290.3 \\ 
    $\mu\mu$, $n_b\geq2$               &       --$\pm$     -- &     45.8$\pm$    1.5 &      0.0$\pm$    0.0 &   1945.7$\pm$  142.0 &      3.6$\pm$    2.6 &      3.9$\pm$    1.8 &    959.3$\pm$   36.2 &  35685.2$\pm$   85.1 &  38643.4$\pm$  169.6 &  39253.0$\pm$  198.1 \\ 
    \hline
    $\mu\tau$, $n_b=1$                 &   1130.7$\pm$  108.8 &     52.3$\pm$    2.6 &     11.8$\pm$    3.2 &    866.7$\pm$   78.7 &    730.8$\pm$   42.9 &    182.6$\pm$   12.4 &   1291.0$\pm$   41.9 &  18430.0$\pm$   60.6 &  22695.9$\pm$  159.6 &  21621.0$\pm$  147.0 \\ 
    $\mu\tau$, $n_b\geq2$              &    346.6$\pm$   51.5 &      5.5$\pm$    0.7 &      0.9$\pm$    0.8 &    103.6$\pm$   29.6 &     56.9$\pm$   14.4 &     36.8$\pm$    5.6 &    322.6$\pm$   21.0 &   9647.6$\pm$   43.7 &  10520.4$\pm$   78.3 &   9934.0$\pm$   99.7 \\ 
    \hline
    $\mu$+jets, $n_b=1$                &  24300.4$\pm$ 3404.9 &    371.0$\pm$    5.2 &   1501.2$\pm$   67.5 &   7533.2$\pm$  265.9 &  49248.1$\pm$  327.3 &   8484.6$\pm$   85.3 &  24447.8$\pm$  187.0 & 514064.6$\pm$  327.2 & 629950.9$\pm$ 3453.3 & 630704.0$\pm$  794.2 \\ 
    $\mu$+jets, $n_b\geq2$             &   4650.7$\pm$ 1399.5 &     61.4$\pm$    2.0 &    248.3$\pm$   31.8 &   1331.9$\pm$  114.0 &   6524.2$\pm$  118.8 &   5172.2$\pm$   66.7 &  10335.6$\pm$  121.4 & 356185.1$\pm$  272.2 & 384509.5$\pm$ 1442.2 & 385397.0$\pm$  620.8 \\ 
    \hline
    $e e$, $n_b=1$                     &       --$\pm$     -- &    138.2$\pm$    3.6 &      2.8$\pm$    1.2 &   4726.5$\pm$  215.7 &      5.4$\pm$    2.8 &      1.1$\pm$    0.8 &   1382.0$\pm$   42.7 &  23447.3$\pm$   66.9 &  29703.3$\pm$  229.9 &  29491.0$\pm$  171.7 \\ 
    $e e$, $n_b\geq2$                  &       --$\pm$     -- &     16.2$\pm$    0.9 &      0.1$\pm$    0.1 &    500.5$\pm$   67.8 &      3.7$\pm$    2.6 &      2.1$\pm$    1.2 &    371.4$\pm$   22.1 &  13412.7$\pm$   50.7 &  14306.6$\pm$   87.5 &  14334.0$\pm$  119.7 \\ 
    \hline
    $e\mu$, $n_b=1$                    &       --$\pm$     -- &    127.2$\pm$    4.9 &     25.5$\pm$   13.2 &    411.9$\pm$   52.7 &     32.8$\pm$    7.2 &     37.6$\pm$    5.4 &   2917.6$\pm$   62.7 &  49878.6$\pm$   99.2 &  53431.1$\pm$  129.8 &  52362.0$\pm$  228.8 \\ 
    $e\mu$, $n_b\geq2$                 &       --$\pm$     -- &      9.0$\pm$    1.3 &      1.9$\pm$    1.1 &     59.0$\pm$   19.5 &      6.5$\pm$    3.2 &      6.1$\pm$    2.2 &    837.9$\pm$   33.8 &  28374.1$\pm$   74.9 &  29294.5$\pm$   84.6 &  29860.0$\pm$  172.8 \\ 
    \hline
    $e\tau$, $n_b=1$                   &    874.2$\pm$   90.3 &     38.0$\pm$    2.1 &    194.5$\pm$   38.8 &    677.8$\pm$   69.3 &    456.3$\pm$   32.9 &    125.3$\pm$   10.0 &    908.2$\pm$   34.6 &  12884.7$\pm$   49.7 &  16159.1$\pm$  139.0 &  15309.0$\pm$  123.7 \\ 
    $e\tau$, $n_b\geq2$                &     94.2$\pm$   46.3 &      3.0$\pm$    0.4 &     10.0$\pm$    2.9 &     53.4$\pm$   21.3 &     28.7$\pm$    8.5 &     43.4$\pm$    6.0 &    196.1$\pm$   15.9 &   6682.4$\pm$   35.8 &   7111.3$\pm$   65.1 &   7006.0$\pm$   83.7 \\ 
    \hline
    $e$+jets, $n_b=1$                  &  25625.1$\pm$ 2941.3 &    494.9$\pm$    5.1 &  12035.7$\pm$  173.0 &  13119.8$\pm$  323.2 &  34481.3$\pm$  266.1 &   5786.3$\pm$   68.8 &  17454.7$\pm$  154.8 & 360917.6$\pm$  268.5 & 469915.4$\pm$ 2992.9 & 464543.0$\pm$  681.6 \\ 
    $e$+jets, $n_b\geq2$               &   3327.4$\pm$ 1476.4 &     84.5$\pm$    2.0 &   2095.3$\pm$   78.4 &   2520.8$\pm$  138.5 &   4696.3$\pm$   98.0 &   3524.2$\pm$   53.7 &   7616.3$\pm$  102.3 & 249557.0$\pm$  223.4 & 273421.8$\pm$ 1509.3 & 274162.0$\pm$  523.6 \\ 
    \hline

    \end{tabular}
    \caption{Estimates of the yields. The estimate of the expected yield is compared to
    the yield observed from data.  Uncertainties are statistical only.
    \label{tab:yields}}
\end{sidewaystable}

\begin{sidewaystable}[]
    \centering
    \setlength{\tabcolsep}{0.4em}
    \renewcommand{\arraystretch}{1.5}
    
    \caption{Estimates of the yields for various processes in
    the $e\tau$ and $\mu\tau$ categories broken down by the number of b tags.
    The estimate of the expected yield is compared to the yield observed
    from data.  Uncertainties are statistical only.}
    
    \resizebox{\textwidth}{!}{
        \begin{tabular}{l|ccccccc|cc}
        \hline
                                        & QCD                  & Diboson (non-WW) & WW               & Z                      & W                    & tW                & $\sf t\bar{t}$     & Expected               & Observed \\
        \hline
        \multicolumn{10}{l}{$e\tau$}   \\
        \hline
        $N_{j} = 0, N_{b} = 0$          & $14609.7 \pm 843.7$  & $11.7 \pm 1.4$   & $102.2 \pm 7.2$  & $30670.4 \pm 3175.9$   & $9505.8 \pm 594.4$   & $11.1 \pm 3.7$    & $29.7 \pm 2.8$     & $54940.5 \pm 3339.4$   & $55591$  \\
        $N_{j} = 1, N_{b} = 0$          & $1512.7 \pm 125.2$   & $10.0 \pm 1.2$   & $20.9 \pm 2.3$   & $3237.1 \pm 355.2$     & $1159.9 \pm 98.0$    & $20.8 \pm 5.2$    & $76.3 \pm 5.7$     & $6037.5 \pm 389.2$     & $6074$   \\
        $N_{j} \geq 2, N_{b} = 0$       & $5519.7 \pm 363.2$   & $233.6 \pm 24.3$ & $269.8 \pm 16.8$ & $6721.8 \pm 724.1$     & $6906.0 \pm 410.6$   & $551.2 \pm 40.4$  & $5933.6 \pm 333.3$ & $26135.7 \pm 968.7$    & $25788$  \\
        $N_{j} = 1, N_{b} = 1$          & $789.5 \pm 77.4$     & $8.0 \pm 1.0$    & $16.4 \pm 2.0$   & $725.6 \pm 99.6$       & $650.5 \pm 60.3$     & $675.5 \pm 47.6$  & $3381.9 \pm 190.7$ & $6247.5 \pm 241.2$     & $6256$   \\
        $N_{j} = 2, N_{b} = 1$          & $421.6 \pm 59.9$     & $11.7 \pm 1.3$   & $10.8 \pm 1.6$   & $424.7 \pm 69.2$       & $305.0 \pm 33.4$     & $538.3 \pm 39.7$  & $5994.7 \pm 336.8$ & $7706.7 \pm 352.8$     & $7388$   \\
        $N_{j} \geq 3, N_{b} = 1$       & $315.4 \pm 56.0$     & $13.1 \pm 1.5$   & $5.0 \pm 1.0$    & $212.1 \pm 42.9$       & $169.3 \pm 23.1$     & $302.1 \pm 25.7$  & $6021.4 \pm 338.2$ & $7038.5 \pm 347.2$     & $6660$   \\
        $N_{j} = 2, N_{b} \geq 2$       & $48.4 \pm 16.4$      & $1.1 \pm 0.2$    & $0.3 \pm 0.2$    & $18.8 \pm 15.9$        & $10.6 \pm 5.8$       & $83.4 \pm 11.1$   & $2606.9 \pm 147.4$ & $2769.5 \pm 149.7$     & $2683$   \\
        $N_{j} \geq 3, N_{b} \geq 2$    & $81.3 \pm 28.8$      & $1.8 \pm 0.3$    & $0.3 \pm 0.2$    & $55.2 \pm 14.0$        & $18.0 \pm 6.9$       & $87.8 \pm 11.5$   & $3574.9 \pm 201.5$ & $3819.4 \pm 204.5$     & $3704$   \\
        \hline
        \multicolumn{10}{l}{$\mu\tau$} \\
        \hline
        $N_{j} = 0, N_{b} = 0$          & $19581.5 \pm 1133.6$ & $27.6 \pm 3.1$   & $244.6 \pm 15.3$ & $103926.9 \pm 10727.5$ & $20342.3 \pm 1205.2$ & $19.3 \pm 5.0$    & $66.2 \pm 5.1$     & $144208.5 \pm 10854.4$ & $146128$ \\
        $N_{j} = 1, N_{b} = 0$          & $2255.6 \pm 167.9$   & $24.0 \pm 2.6$   & $37.0 \pm 3.4$   & $8216.3 \pm 868.5$     & $2470.3 \pm 177.3$   & $33.8 \pm 6.8$    & $162.4 \pm 10.6$   & $13199.4 \pm 902.2$    & $13293$  \\
        $N_{j} \geq 2, N_{b} = 0$       & $5467.2 \pm 372.9$   & $313.5 \pm 32.5$ & $413.2 \pm 24.9$ & $10752.1 \pm 1139.7$   & $10989.1 \pm 640.3$  & $879.2 \pm 59.4$  & $9261.1 \pm 519.4$ & $38075.4 \pm 1457.1$   & $38184$  \\
        $N_{j} = 1, N_{b} = 1$          & $1452.3 \pm 113.6$   & $12.3 \pm 1.4$   & $27.8 \pm 2.8$   & $1632.3 \pm 193.8$     & $1199.1 \pm 96.4$    & $1112.9 \pm 72.6$ & $5266.7 \pm 296.1$ & $10703.3 \pm 390.8$    & $10628$  \\
        $N_{j} = 2, N_{b} = 1$          & $709.7 \pm 75.4$     & $17.6 \pm 1.9$   & $18.1 \pm 2.1$   & $708.4 \pm 101.7$      & $568.1 \pm 50.5$     & $769.3 \pm 53.1$  & $9493.5 \pm 532.4$ & $12284.6 \pm 552.1$    & $12048$  \\
        $N_{j} \geq 3, N_{b} = 1$       & $438.5 \pm 70.7$     & $19.5 \pm 2.1$   & $9.7 \pm 1.5$    & $384.5 \pm 62.6$       & $292.9 \pm 32.0$     & $480.7 \pm 36.5$  & $9413.5 \pm 527.9$ & $11039.3 \pm 538.5$    & $10314$  \\
        $N_{j} = 2, N_{b} \geq 2$       & $111.1 \pm 19.9$     & $1.7 \pm 0.2$    & $1.0 \pm 0.4$    & $58.6 \pm 23.6$        & $56.0 \pm 16.9$      & $153.8 \pm 16.5$  & $4157.7 \pm 234.1$ & $4539.9 \pm 237.3$     & $4321$   \\
        $N_{j} \geq 3, N_{b} \geq 2$    & $117.5 \pm 35.6$     & $3.0 \pm 0.4$    & $1.4 \pm 0.5$    & $79.4 \pm 22.2$        & $18.1 \pm 6.9$       & $157.9 \pm 16.7$  & $5599.2 \pm 314.7$ & $5976.5 \pm 318.0$     & $5705$   \\
        \hline
    \end{tabular}}

    \label{tab:yields_ltau}
\end{sidewaystable}



\FloatBarrier
