\section{Dataset and Simulated Events}
\label{sec:analysis:dataset}


\subsection{Data}
\label{sec:analysis:dataset:data}

In this analysis, data is selected based on the presence of at least one muon or one electron. The single muon dataset requires that events contain at least one muon with $\pt > 25\GeV$ and passing the loose track isolation criterion, $\rm Iso_{track} < 0.1$. The single electron dataset requires that there be at least one electron satisfying the requirement that $\pt > 30\GeV$ and that it passes the tight identification requirements as defined by the EGamma POG.  The specific dataset names and the associated integrated luminosities are listed in Table~\ref{tab:analysis:dataset:data2016}.

\begin{table}[ht]
    \centering
    \setlength{\tabcolsep}{1em}
    \renewcommand{\arraystretch}{1.1}
    \caption{Data samples produced by CMS in 2016.} \label{tab:analysis:dataset:data2016}
    
    % \small
    \begin{tabular}{l c c}
        \hline
        Sample                                              & Run ranges    & $\int L (\fbinv)$ \\
        \hline
        \texttt{SingleMuon/Run2016B-03Feb2017\_ver2-v2}     & 272007-275376 & 5.33                \\
        \texttt{SingleMuon/Run2016C-03Feb2017-v2}           & 275657-276283 & 2.4                 \\
        \texttt{SingleMuon/Run2016D-03Feb2017-v2}           & 276315-276811 & 4.26                \\
        \texttt{SingleMuon/Run2016E-03Feb2017-v2}           & 276831-277420 & 4.1                 \\
        \texttt{SingleMuon/Run2016F-03Feb2017-v2}           & 277772-278808 & 3.2                 \\
        \texttt{SingleMuon/Run2016G-03Feb2017-v2}           & 278820-280385 & 7.8                 \\
        \texttt{SingleMuon/Run2016H-03Feb2017\_ver*-v1}     & 281613-284044 & 9.2                 \\
        \hline
        \texttt{SingleElectron/Run2016B-03Feb2017\_ver2-v2} & 272007-275376 & 5.33                  \\
        \texttt{SingleElectron/Run2016C-03Feb2017-v2}       & 275657-276283 & 2.4                   \\
        \texttt{SingleElectron/Run2016D-03Feb2017-v2}       & 276315-276811 & 4.26                  \\
        \texttt{SingleElectron/Run2016E-03Feb2017-v2}       & 276831-277420 & 4.1                   \\
        \texttt{SingleElectron/Run2016F-03Feb2017-v2}       & 277772-278808 & 3.2                   \\
        \texttt{SingleElectron/Run2016G-03Feb2017-v2}       & 278820-280385 & 7.8                   \\
        \texttt{SingleElectron/Run2016H-03Feb2017\_ver*-v1} & 281613-284044 & 9.2                   \\
        \hline
    \end{tabular}


\end{table}



\noindent Run ranges where data quality is determined to be insufficient are filtered removed from the datset by applying a luminosity mask. The following file is provided in JSON format from the CMS PPD group:

\texttt{Cert\_271036-284044\_13TeV\_23Sep2016ReReco\_Collisions16\_JSON.txt}

\noindent The full dataset consists of 35.9\fbinv of integrated luminosity~\cite{cms:lumi2016:CMS-PAS-LUM-17-001}




\subsection{Simulated Dataset}
\label{sec:analysis:dataset:simulation}

Simulated datasets are used for modelling the major SM processes, including SM diboson, $\PW/\PZ/\PGg$ associated with jets, single-top and \ttbar. The background from multijet QCD is estimated by a data-driven approach discussed in section~\ref{sec:analysis:background} The simulated samples used in modelling the background and signal are shown in table~\ref{tab:analysis:dataset:mc2016}.  The production of the samples was carried during the Summer 2016 campaingn and the production of the mini Analysis Oriented Data format (miniAOD) was done using CMSSW release \texttt{8\_0\_26\_patch2}. The same release was used for processing both the data and the simulated samples. Lepton universality is assumed for the simulated datasets, namely $ \BWl = 10.8\%$. To account for the deviation from the data, some reweightings of the simulated dataset are applied, commonly including reweightings for pile-up, top \pt, \WW \pt and \PZ \pt.

\begin{table}[ht]
    \centering
    \setlength{\tabcolsep}{2em}
    \renewcommand{\arraystretch}{1.3}
    \caption{Simulated datasets.} \label{tab:analysis:dataset:mc2016}

    \begin{table}[ht]
    \centering
    \setlength{\tabcolsep}{2em}
    \renewcommand{\arraystretch}{1.1}
    \caption{Simulated MC samples.} \label{tab:dat:mc2016}
    % \small
    \begin{tabular}{l l c}
        \hline
        Process                                           & Generator         & $\sigma \times \text{BR} (pb)$ \\
        \hline
        $t\bar{t}$                                        & \POWHEG+\PYTHIA     & 831.76                        \\
        $t\bar{t}$  (leptonic)                            & \POWHEG+\PYTHIA     & 87.32                          \\
        $t\bar{t}$  (semi-leptonic)                       & \POWHEG+\PYTHIA     & 364.35                         \\
        $tW/\bar{t}W$                                     & \POWHEG+\PYTHIA     & 35.6                           \\
        \hline
        Z+jets                                            &                  &                                \\
        \hspace*{1em} $10 < m_{\ell\ell} < 50$ GeV        & \MCATNLO+\PYTHIA   & 18610                          \\
        \hspace*{1em} $m_{\ell\ell} > 50 $GeV             & \MCATNLO+\PYTHIA   & 5765                           \\
        \hspace*{1em} $m_{\ell\ell} > 50, N_{j} = 0 $GeV  & \MCATNLO+\PYTHIA   & 4757                           \\
        \hspace*{1em} $m_{\ell\ell} > 50, N_{j} = 1 $GeV  & \MCATNLO+\PYTHIA   & 884.4                          \\
        \hspace*{1em} $m_{\ell\ell} > 50, N_{j} = 2 $GeV  & \MCATNLO+\PYTHIA   & 338.9                          \\
        \hline
        W + 1 jet                                         & \MADGRAPH+\PYTHIA  & 11486.5                        \\
        W + 2 jet                                         & \MADGRAPH+\PYTHIA  & 3775.2                         \\
        W + 3 jet                                         & \MADGRAPH+\PYTHIA  & 1139.8                         \\
        W + 4 jet                                         & \MADGRAPH+\PYTHIA  & 655.82                         \\
        \hline
        $qq\rightarrow WW \rightarrow 2\ell 2\nu$         & \POWHEG           & 12.13                          \\
        $gg\rightarrow WW \rightarrow 2\ell 2\nu$         & \POWHEG           & 0.588                          \\
        WZ $\rightarrow 3\ell \nu$                        & \POWHEG+\PYTHIA    & 5.29                           \\
        WZ $\rightarrow 2\ell 2q$                         & \MCATNLO+\PYTHIA   & 5.595                          \\
        ZZ $\rightarrow 2\ell 2\nu$                       & \POWHEG+\PYTHIA    & 0.564                          \\
        ZZ $\rightarrow 2\ell 2q$                         & \MCATNLO+\PYTHIA   & 3.22                           \\
        ZZ $\rightarrow 4\ell$                            & \MCATNLO+\PYTHIA   & 1.21                           \\
        \hline
    \end{tabular}

    
\end{table}

\end{table}







% \subsection{Standard Reweightings}



% \FloatBarrier


