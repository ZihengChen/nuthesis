

\section{Branching Ratio}

Test of lepton flavour universality (LFU) between electron and muons in 
weak section has been performed to unprecedented precision
in the past two decades. The tests have been carried out on both
colliders and fix target experiments. Their results are shown
in Table \ref{tbl:testlfuemu}. In general, the measurements
branching ratios between electron and muon agree very well with 
SM prediction.

\input{section6/tables/emutest.tex}

In contract with agreement on LFU for $e$ and $\mu$ in weak section, LPU 
regarding $\tau$ versus $e$ and $\mu$, as is discussed in Chapter 1, 
is significantly challenged by 
measurements from ALEPH, DELPHI, OPAL and L3 with LEP e+e- collision, 
as well as Belle, Belle and LHCb with B meson decay.


Therefore, we are interesed in the ratio of $Br (W\to \tau \nu)$ with respect to electron
and muon channels,

\begin{equation}
    r = \frac{Br (W\to \tau \nu)}{Br (W\to l \nu)} , \text{ where } l=e,\mu
\end{equation}

based on the assumption that $Br (W\to \mu \nu) = Br( W\to e \nu )$, which
is well justified by the previous precision test of LFU between $e$ and $\mu$ in weak section.
This assumption is the same in Belle and BaBar measurements.

The key to the success of Belle and BaBar measurements is that $tau$ are reconstructed
by the same method as electron or muon, such that systematics regarding object
reconstruction and selection are cancelled.
Following this principle, we are measuring r in purely dilepton channels with muonic and electronic taus.
Comparing with hadronic taus, this avoids the systematic uncertainty related to hadronic tau efficiency
and misidentification.
By using leptonic taus, systematics regarding lepton reconstruction 
is canceled out to the first order, thus the precision of r is not limited systematically.

The evolved dilepton channels are $\mu\mu$, $ee$ and $e\mu$ with $n_j \geq 2$ and $n_b = 1,2$,
where $\mu\mu$, $ee$ also include $n_b = 0$ bin for Z background normalization purpose.
r is obtained by simultaneous fit to the pT spectrum of the trailing lepton in $\mu\mu$,
$ee$ and $e\mu$ channels. The methodology of this template fit is described in Section 5.3.
The result is in Eqn \ref{eqn:fitr}.

\begin{equation}
    \boxed{
    r = \frac{Br (W\to \tau \nu)}{Br (W\to l \nu)}
    = 1.000 \times \big[1 \pm 2.72\% \text{ (stat)} \pm 1.44\% \text{ (syst)} \big]
    }
    \label{eqn:fitr}
\end{equation}

The correlation matrix of the fit is shown in Fig \ref{fig:covr}.

The measurement of r using leptonic tau has small systematic uncertainty, thanks to the 
cancellation of reconstruction efficiency. The precision of r is statistically limited, 
which is expected to be improved when including 2017 data.


The improvement of r precision when including more channels is shown in
Fig \ref{fig:gain}. The gain of adding $e\mu$ and $\mu e$ channel is
significant, while adding $l \tau$ and $l4j$ channel is small.

\begin{figure}[H]
    \centering
    \includegraphics[width=12cm]{section6/figures/gain.png}
    \caption{ Improvement on the r precision when including more channels.}
    \label{fig:gain}
\end{figure}

\begin{figure}[p]
    \centering
    \includegraphics[width=14cm]{section6/figures/cov_r.png}
    \caption{Fitting the pT spectrum of trailing lepton in $ee$, $\mu\mu$ and $e\mu$ channel.
    The correlation matrix among r and systematic parameters.
    }
    \label{fig:covr}
\end{figure}