
This sections presents the results of the estimates of the branching
fractions using the two methods presented earlier, and a test of lepton
universality based on the measured values of the leptonic branching
fractions.

\subsection{W branching fraction measurement}

The measurement of the branching fractions is carried out using both the
MLE and the semi-analytic method.  The specifics of both measurements
are described in Section~\ref{sec:method}.  The datasets used are common
to both categories, but the approach for how the branching fractions are
determined are substantially different.  One important distinction
between the two measurements is that the MLE method measures the
branching fraction simultaneously in all channels while the
semi-analytic approach measures the branching fractions separately in
different trigger and b-tag categories.  The resulting values for the
\PW branching fractions are shown in
Table~\ref{tab:results}.

\begin{table}[h]
    \centering
    \topcaption{Values of branching fractions determined in both
        analysis approaches and the PDG values.  Errors combine
        statisitcal and systematics uncertainties.
    \label{tab:results}}
    \begin{tabular}{l|c|c|c}
                   & semi-analytic          & MLE                   & PDG                   \\
    \hline                                                                 
    $\beta_{e}$    & $ (10.80 \pm 0.22) \%$ & $(10.8 \pm 0.11) \%$ & $(10.71 \pm 0.16)$ \% \\
    $\beta_{\mu}$  & $ (10.80 \pm 0.23) \%$ & $(10.8 \pm 0.08) \%$ & $(10.63 \pm 0.15)$ \% \\
    $\beta_{\tau}$ & $ (10.80 \pm 0.68) \%$ & $(10.8 \pm 0.19) \%$ & $(11.38 \pm 0.21)$ \% \\
    $\beta_{h}$    & $-- \pm --$            & $(67.4 \pm 0.26) \%$ & $(67.41 \pm 0.27)$ \% \\
    \end{tabular}
\end{table}




\FloatBarrier
\subsection{Test of lepton universality (\textbf{WIP})}

Having established the method for estimating the branching fractions,
hypothesis tests can be carried out.  To do this, the profile likelihood
ratio is constructed assuming a null hypothesis of lepton universality
and two scenarios of unequal branching fractions for the alternative
hypothesis.  The two hypotheses that are tested are:

\begin{enumerate}
    \item $B_{e} = B_{\mu} \neq B_{\tau}$,
    \item $B_{e} \neq B_{\mu} \neq B_{\tau}$.
\end{enumerate}

The first hypothesis is of primary interest, and would be the scenario
more likely to be supported by the current measured values.  The
likelihood ratio for the two cases is therefore,

\begin{equation}
    q_{alt. 1} = -2(\mathcal{L}\left(B_{e,\mu}, B_{\tau},
        \boldsymbol{\theta}|\mathbf{x}\right) - \mathcal{L}\left(B_{e}, B_{\mu},
    B_{\tau}, \boldsymbol{\theta}|\mathbf{x}\right)}.
\end{equation}

and, 

\begin{equation}
    q_{alt. 2} = -2(\mathcal{L}\left(B_{\ell},
        \boldsymbol{\theta}|\mathbf{x}\right) - \mathcal{L}\left(B_{e}, B_{\mu},
    B_{\tau}, \boldsymbol{\theta}|\mathbf{x}\right)).
\end{equation}

The test allows us to quantify p values that correspond to the values of
$q$ determined from fitting the data.  In order to do this, the pdf of
$q$ for both scenarios needs to be estimated.  Assuming regularity
conditions~\cite{Wilks}, the distributions will be $chi^{2}$ with one
d.o.f. for $q_{alt. 1}$ and two d.o.f. for $q_{alt. 2}$.  This has been
tested on 100 toy datasets and it is found that the distribution for
smaller excursions conform with Wilk's theorem (see
figure~\ref{fig:lu_test}.  Based on this, the distribution of the
likelihood ratio is well approximated by a $\chi^{2}_{k}$ distribution.

\begin{figure}[h]
    \begin{center}
        \includegraphics[width=0.95\textwidth]{figures/lepton_universality_toys}
        \caption{Distribution for the profile likelihood ratio for the two
            hypothesis tests of interest based on toys data generated
            from the Asimov dataset: \emph{left} the minimum NLL for
            each evaluated toy dataset, and \emph{right} distribution for
            $q_{alt. 1}$ and $q_{alt. 2}$.
            \label{fig:lu_test}}
    \end{center}
\end{figure}
