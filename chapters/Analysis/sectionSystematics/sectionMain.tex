\section{Systematics}
\label{sec:analysis:systematics}


The various sources of uncertainty that have been considered are
described in the following sections. As described in Section~\ref{sec:analysis:method}, 
the treatment of systematics
differs between the two analysis approaches: the shape analysis makes
use of nuisance parameters (n.p.);
the counting analysis is carried out varying each systematic
individually and assessing the variation on the estimates of the
branching fractions.


\subsection{Source of Systematics}
\subsubsection{Luminosity} 
        
The uncertainty on the CMS luminosity measurement is estimated to be
2.5\% for the 2016 run~\cite{CMS-PAS-LUM-17-001}.  This uncertainty
effects the overall scale of the data model, and is included as a
normalization nuisance parameter that affects the overall normalization of all
channels in a fully correlated manner.


\subsubsection{Data-driven background estimates}

The uncertainty is split into four normalization parameters: one for
each of the final states $e\tau$, $\mu\tau$, $eh$, and $\mu h$.  The
prior uncertainty for each is assumed to be 25\%.

\subsubsection{Cross section for simulated processes}

\begin{itemize}
    \item \ttbar: 10\%, PDF, $\alpha_{s}$, $\mu_{R}/\mu_{F}$
    \item tW: 10\%
    \item WW: unconstrained
    \item Z+jets: 10\%, PDF, $\alpha_{s}$, $\mu_{R}/\mu_{F}$
    \item W+jets: 10\%
    \item diboson: 10\%
\end{itemize}

\subsubsection{WW \pt reweighting}

The reweighting of the WW \pt is accounted for by including two n.p. for
the resummation and factorization variations as described in
section~\ref{sec:analysis:dataset}

\subsubsection{top \pt reweighting}

As described in section~\ref{sec:analysis:dataset}, the uncertainty on the top \pt
scale is included as a one-sided morphing templates generated based on
previous top PAG studies.

\subsubsection{Pileup}

Each event is weighted with a scale factor to account for differences in
the pileup spectrum between data and simulation.  The uncertainty on
the event weights is mainly due to the uncertainty on the minimum bias
cross section.  The nominal minimum bias cross section is $69.2 \pm
3.18~\text{mb}$. The effect of the uncertainty is propagated through the
analysis by calculating the distribution of pileup in data while varying
the cross section up and down by one standard deviation.  

\subsubsection{Trigger efficiency}

The uncertainty due to the trigger efficiency scale factors is
accounted for by saving the uncertainty of each weight and including the
uncertainty in the bin uncertainty of the template histograms.  The scale
factors are then included as normalization nuisance parameters in the
fit model.

\begin{itemize}
    \item \textit{single muon}: 0.5\% normalization uncertainty on all
        categories where the triggering lepton is a muon.
    \item \textit{single electron}: accounted for as a shape uncertainty
        in events where the event is exclusively triggered by an
        electron.  This accounts for uncertainty in the determination of
        the scale factors including the statistical uncertainty, the
        variation due the triggering of the tag lepton, and variation
        due to the probe electron.  A detailed description of the
        electron trigger scale factors in included in
        \ref{sec:electron_trigger}.
\end{itemize}

\subsubsection{Muon reconstruction}
    \begin{itemize}
    \item \textit{identification/isolation}: the uncertainties are
        accounted for each muon and are based on values provided
        by the POG.  These are incorporated as a shape uncertainty that
        are $\pt$ dependent.  
    \item \textit{energy scale}: to account for the muon energy scale,
        the muon \pt is varied by $\pm 1 \sigma$ (0.2\%) and the effect of
        the variation on the acceptance and change in kinematics is included
        in the model through morphing templates.
    %\textit{resolution}
    \end{itemize}

\subsubsection{Electron reconstruction}
    \begin{itemize}
        \item \textit{identification/isolation}: the uncertainties
            provided per electron are taken from values provided by the
            POG.  These are incorporated as a shape uncertainty that are
            $\pt$ dependent.
        \item \textit{reconstructions}: treated the same as the
            identification uncertainty.  Scale factors and their
            uncertainties are only $\eta$ dependent. 
        \item \textit{energy scale}: The electron energy scale is
            assumed to be know at the 0.5\% level, and is assigned a
            nuisance parameter that modifies the change to the shape of
            the relevant kinematic quantity.
        %\item \textit{resolution}
    \end{itemize}

\subsubsection{Tau reconstruction}
    \begin{itemize}
        \item \textit{identification}: the $\tau$ POG recommends a 5\%
            uncertainty on the scale factor applied to simulation.
            Because a sideband is included to provide an \emph{in situ}
            evaluation of the $\tau$ efficiency scale factor, the scale
            factors are included as $\pt$-dependent n.p. in seven \pt
            bins.
        \item \textit{$jet\rightarrow\tau$}: scale factors and
            uncertainties for jets faking taus were derived based on a
            dilepton plus reconstructed tau control region.  A n.p. is
            assigned to each $\pt$ bin used to measure the scale factor
            and an overall normalization n.p. is assigned to account for
            any difference in rate between light and heavy jets.
        \item \textit{$e\rightarrow\tau$}: a single normalization n.p.
            is included to templates where an electron is
            misreconstructed as a hadronically decaying tau.
        \item \textit{energy scale}: the tau energy scale is corrected in
            correspondence with POG recommendations and an uncertainty
            of 1.2\% per decy mode is assigned.  These are included as
            three shape uncertainties depending on the reconstructed
            decay mode of the hadronically decaying tau.
    \end{itemize}

\subsubsection{Jet reconstruction}

    Jet systematics impact the analysis by modifying the acceptance of
    events in the various jet multiplicity categories.  With that in
    mind, the uncertainty is taken into account by varying the various
    sources of jet uncertainties and assessing the resulting effect on
    the jet and b-tag multiplicities.

    \begin{itemize}
        \item \textit{energy scale}: the jet energy scale is varied by
            the various uncertainty sources on the jet energy
            corrections provided by the JetMET POG.  These are included
            as 18 shape n.p.'s (N.B. the lepton kinematic quantities
            which are fit are generally not affected by variation of the
            jet energy scale, but migration between different b tag
            categories does happen)
        \item \textit{resolution}: the jet energy is corrected in
            simulation to account for the difference in resolution
            between data and simulation.  The correction is applied per
            jet and is dependent on the jet \pt.  Consequently, there is
            an associated uncertainty.  The overall effect of this is
            estimated by varying the scale factor up and down one
            standard deviation and propagating the effect to the
            morphing templates.
    \end{itemize}

\subsubsection{b-tagging}
    
The b tag modelling in simulation is corrected to better describe the
data based on scale factors.  The uncertainty on the correction is
assessed based on up and down variations of b tagging and mistagging
scale factors supplied by the b tag POG.  The b tag uncertainties are
factorized based on the various sources of uncertainty considered in the
calculation of the scale factors.  The variation is propagated through
the analysis through the inclusion of shape nuisance parameters for both
b tagging and mistagging variation.

\subsubsection{Theory/simulation modelling}

In addition to the normalization uncertainties coming from PDF, QCD
scale, and uncertainty on $\alpha_{s}$, several other theory
uncertainties are accounted for.  These are only included for $\ttbar$
processes and are applied as recommended by TOP PAG.

\begin{itemize}
    \item \textit{ISR/FSR}: variations to $\alpha_{S}$ affecting both
        ISR and FSR are evaluated based on dedicated \ttbar MC samples.
        This is done for ISR and FSR independently.  These variations are
        propagated through the analysis through morphing templates.
    \item \textit{ME-PS matching scale}: matrix element to parton shower
        matching is regulated at the generator level by the \textit{hdamp}
        parameter.  This parameter is varied from the nominal value of
        $1.58^{+0.66}_{-0.59}$ in dedicated MC samples and propagated 
        through morphing templates.
    \item \textit{Underlying event}: modelling of the underlying event
        is dependent on the Pythia tune that is used (in the case of this
        analysis, CUETP8M2T4)~\cite{CMS-PAS-TOP-16-021}.  Dedicated samples
        are generated varying the appropriate parameters and the variation
        in efficiency is propagated through the analysis with morphing
        templates.
        %(\emph{Not included in this iteration})
\end{itemize}

There are two issues with these uncertainty sources that are worth
considering.  The first is that the variations due to these sources of
uncertainty are estimated from dedicated MC samples.  This leads to a
fairly sizeable statistical uncertainty, and can lead to exagerated
uncertainties and strange behavior in the morphing templates (e.g., both
the up and down variation will predict yields below/above the nominal
sample).  Also, the size of the uncertainty resulting from the FSR
variation is very large (up to 20\%) in the $\ell\tau$ categories.  This
level of variation would be corrected for in the scale factors
accounting for the difference in ID/misID efficiency for identified
$\tau$ candidates, and that uncertainty in general would be much
smaller, $<5\%$.  In appendix~\ref{sec:tau_fsr}, a method for accounting
for this is described.




\subsection{Shape analysis systematics}

As described previously, each source of uncertainty is accounted for in
the shape analysis by including one or more associated nuisance
parameters.  After minimizing the likelihood, post-fit values for the
nuisance parameters and their associated uncertainties are obtained.
This is illustrated in figure~\ref{fig:pulls_all}.  In general, the
pulls on the nuisance parameters do not exceed $2\sigma$ of their
initial uncertainty, but many of the nuisance parameters do become
constrained.  Additionally, the correlations for each parameter can be
obtained and are displayed in figure~\ref{fig:corr_matrix}.  In order to
isolate the effect of each nuisance parameter on the uncertainty of the 
branching fractions, the minimization is repeated while individually fixing each
nuisance parameter either to its post-fit value plus or minus the
post-fit uncertainty.  The result of this process is shown in
figure~\ref{fig:impacts_all}.

\begin{figure}[ht]
    \centering
    \includegraphics[width=1.2\textwidth, angle=-90]{chapters/Analysis/sectionSystematics/figures/asimov_pulls_summary.pdf}
    \caption{Pulls and constrain of all non-MC statistic nuisance
        parameters after minimizing the likelihood.}
    \label{fig:pulls_all}
\end{figure}

\begin{figure}[ht]
    \centering
    \includegraphics[width=0.99\textwidth]{chapters/Analysis/sectionSystematics/figures/correlation_matrix_unblinded.pdf}
    \caption{Correlation matrix for branching fractions and nuisance
        parameters.  This does not include the nuisance parameters
        associated with bin-by-bin MC statistical uncertainty.}
    \label{fig:corr_matrix}
\end{figure}

\begin{figure}[ht]
    \centering
    \includegraphics[width=1.2\textwidth, angle=-90]{chapters/Analysis/sectionSystematics/figures/unblinded_impacts.pdf}
    \caption{impacts}
    \label{fig:impacts_all}
\end{figure}

\FloatBarrier



\subsection{Counting analysis systematics}

For the counting analysis, the systematics are assessed individually by
varying up and down the sources of systematic uncertainty listed in
Section~\ref{ec:analysis:systematics}. The same branching fraction 
extraction is repeated with the variated systematic parameter.
The change in the branching fractions 
with respect to the nominal value is treated as the systematical uncertainty
due to a given source of systematics.

Recall that channels are divided into 
four groups based on the trigger types and b-tag multiplicities,
($\mu1b$, $\mu2b$, $e1b$, $e2b$), each of which produces a $\beta$ measurement. 
Table~\ref{tab:syst_alt} shows the 
uncertainties of $\beta$ from these four groups due to each individual source of systematics.
The combine of the four groups assumes
\begin{itemize}
    \item one source of systematics is fully correlated among the four groups.
    \item different sources of systematics are mutually independent.
\end{itemize}

\noindent Therefore, the chi-squared in the combine can be written as
\begin{equation}
    \chi^2 (\beta) = (\beta_0 - \textbf{A} \beta )^T \textbf{V}^{-1} (\beta_0 - \textbf{A} \beta )
\end{equation}

\noindent where $\beta = [\beta_e, \beta_\mu, \beta_\tau]^T $ is the combined branching fraction, and
$\beta_0$ is the nominal value of the four measurements in the $\mu1b$, $\mu2b$, $e1b$, $e2b$ group, defined as
% 
\begin{equation}
    \beta_0 = \bigg [
    \beta_e^{\mu1b}, \beta_\mu^{\mu1b}, \beta_\tau^{\mu1b}, \quad 
    \beta_e^{\mu2b}, \beta_\mu^{\mu2b}, \beta_\tau^{\mu2b}, \quad 
    \beta_e^{e1b}, \beta_\mu^{e1b}, \beta_\tau^{e1b}, \quad
    \beta_e^{e2b}, \beta_\mu^{e2b}, \beta_\tau^{e2b}
    \bigg ]^T
\end{equation}

\noindent And $\textbf{A}=[I_{3\times3}, I_{3\times3}, I_{3\times3}, I_{3\times3}]^T$ is a $12 \times 3$ 
matrix consist of four $3\times 3$ identity matrices. $\textbf{V}$ is the variance matrix for the 12 
elements in $\beta_0$, which sums the statistical and the systematical uncertainties,

\begin{equation}
    \textbf{V} =
    \sum_{n \in \text{data,MC}} \big( \Delta_{n}\beta_0 \big) \otimes   \big( \Delta_{n}\beta_0 \big) +
    \sum_{\theta \in \text{syst}} \big( \Delta_{\theta}\beta_0 \big) \otimes  \big( \Delta_{\theta}\beta_0 \big).
\end{equation}

\noindent where $\Delta_{\theta}\beta_0$ are shown as rows in Table~\ref{tab:syst_alt} and the statistical and systematical part
of the $\textbf{V}$ matrix is shown in Figure~\ref{fig:corBetaBar}. The combined result can 
be analytically calculated 
\begin{equation}
    \beta =   (\textbf{A}^T \textbf{V}^{-1} \textbf{A})^{-1}(\textbf{A}^T \textbf{V}^{-1}) \beta_0 , \quad
    \text{with } \textbf{Var}\big[\beta\big]  =   (\textbf{A}^T \textbf{V}^{-1} \textbf{A})^{-1}
\end{equation}

\begin{figure}[ht]
    \centering
    \includegraphics[width=0.99\textwidth]{chapters/Analysis/sectionSystematics/figures/covarMatrix_total.png}
    \caption{ the statistical and systematical part of the $\textbf{V}$ matrix. }
    \label{fig:corBetaBar}
\end{figure}

% \subsubsection{Method and Result of Analytic Combining}

% The counting analysis extracts leptonic branching fractions
% $\{\beta_e, \beta_\mu, \beta_\tau\}$ simultaneously from yields of mutually exclusive channels, 
% grouped in four trigger-bjet categories, $\mu-1b$,  $\mu-2b$,  $e-1b$ and $e-2b$.
% Its parameter extraction outputs totally 4 sets of $\{\beta_e, \beta_\mu, \beta_\tau\}$,
% one set in each category. We denote the output 4 sets of $\{\beta_e, \beta_\mu, \beta_\tau\}$
% as a vector $\beta$.

% \begin{equation}
%     \beta = \bigg [
%     \beta_e^{\mu1b}, \beta_\mu^{\mu1b}, \beta_\tau^{\mu1b}, \quad 
%     \beta_e^{\mu2b}, \beta_\mu^{\mu2b}, \beta_\tau^{\mu2b}, \quad 
%     \beta_e^{e1b}, \beta_\mu^{e1b}, \beta_\tau^{e1b}, \quad
%     \beta_e^{e2b}, \beta_\mu^{e2b}, \beta_\tau^{e2b}
%     \bigg ]
% \end{equation}

% $\beta$ variates with respect to the statistical fluctuation 
% of event yields, as well as the variation of systematical parameters,
% leading to its statistical and systematical uncertainties.


% The statistical variance of $\beta$ is calculated by propagating 
% the statistical uncertainty of yield in each channel, 
% then summing them in quadrature. As is given in Eqn \ref{eqn:statVar}.
% The summation in quadrature is based on the fact that the statistical 
% fluctuation in each channel is fully independent.

% \begin{equation}
%     \textbf{V}_{stat} = \sum_{i \in ch} 
%     \bigg( \frac{\partial \beta}{\partial n_i} \delta_{n_i} \bigg) \otimes 
%     \bigg( \frac{\partial \beta}{\partial n_i} \delta_{n_i} \bigg)
%     \label{eqn:statVar}
% \end{equation}

% The systematic uncertainty due to a given systematical parameter is
% evaluated by variating this systematical parameter by its 
% own uncertainty and then taking the changes of outcoming $\beta$. 
% The total systematic variance is obtained by summing all systematic 
% uncertainties in quadrature. As is given in Eqn \ref{eqn:systVar}.
% This summation of quadrature is based on the assumption made in counting analysis 
% that all systematical parameters in consideration are independent from each other.


% \begin{equation}
%     \textbf{V}_{syst} = \sum_{\theta \in syst}
%     \big( \Delta_{\theta}\beta \big) \otimes 
%     \big( \Delta_{\theta}\beta \big)
%     \label{eqn:systVar}
% \end{equation}

% The outer product of variations is based on the fact that all elements of
% $\beta$ is fully correlated when tuning up and down a systematical parameters.
% Table \ref{tbl:errors} shows the squared root of diagonal elements of 
% $\textbf{V}_{stat}$ and $\textbf{V}_{syst}$. The full matrix of $\textbf{V}_{stat}$ 
% and $\textbf{V}_{syst}$ are shown in Fig \ref{fig:varStatSyst}.
% The total variance of $\beta$ is summation of statistics and systematics variance.
% The variance matrix not only represents the sensitivity of the measurement 
% to each systematics, but also characterizes the correction among 
% $\{\beta_e, \beta_\mu, \beta_\tau\}$ within each category and across all four categories. 

% \begin{equation}
%     \textbf{V} = \textbf{V}_{stat} + \textbf{V}_{syst}
% \end{equation}

% To combine the branching fraction in four categories, we construct $\chi^2$ parametrized
% by average branching fractions, $\bar{\beta}_e,\bar{\beta}_\mu, \bar{\beta}_\tau$ 
% in Eqn \ref{eqn:DefineChisquared}. 

% \begin{equation}
%     \chi^2 (\bar{\beta}_e, \bar{\beta}_\mu, \bar{\beta}_\tau) = 
%     (\beta - \bar{\beta} )^T \textbf{V}^{-1} (\beta - \bar{\beta} )
%     \label{eqn:DefineChisquared}
% \end{equation}

% where $\bar{\beta}$ is linearly parametrized by $\bar{\beta}_e, \bar{\beta}_\mu, \bar{\beta}_\tau$ 
% via a $12 \times 3$ matrix A which consists of four $3\times 3$ identity matrices
% distributing parameters linearly to 4 categories.

% \begin{equation}
%     \bar{\beta} = A \bigg [ \bar{\beta}_e, \bar{\beta}_\mu, \bar{\beta}_\tau \bigg ]
%     =
%     \bigg [
%     \bar{\beta}_e, \bar{\beta}_\mu, \bar{\beta}_\tau, \quad 
%     \bar{\beta}_e, \bar{\beta}_\mu, \bar{\beta}_\tau, \quad 
%     \bar{\beta}_e, \bar{\beta}_\mu, \bar{\beta}_\tau, \quad 
%     \bar{\beta}_e, \bar{\beta}_\mu, \bar{\beta}_\tau
%     \bigg ]
% \end{equation}

% Thanks to this linearly parametrizing of $\bar{\beta}$, the first and second 
% derivatives of $\chi^2$ can be calculated analytically.

% \begin{align}
%     &\nabla \chi^2   = -2 (A^T V^{-1} \beta - A^T V^{-1} \bar{\beta} )
%     \\
%     &\nabla^2 \chi^2 = 2 A^T V^{-1} A 
% \end{align}

% The central value of average branching fraction comes from minimizing this $\chi ^2$, or
% $\nabla \chi ^2 = 0$
% The variance of average branching fraction comes from Fisher Information, 
% $I = \frac{1}{2} \nabla^2 \chi^2 $ evaluated at 
% point with the least chi-squared $\bar{\beta}^{LS}$.
% In other words, the covariance of average branching fractions,  
% $U [\bar{\beta}^{LS}]$,
% equals to the inverse of
% Fisher Information at $\bar{\beta} = \bar{\beta}^{LS}$.

% \begin{equation}
%     \bar{\beta}^{LS} =   (A^T V^{-1} A)^{-1}(A^T V^{-1})  \cdot \beta
%     \label{eqn:combineMean}
% \end{equation}

% \begin{equation}
%     U \big[\bar{\beta}^{LS} \big]  =   (A^T V^{-1} A)^{-1}
%     \label{eqn:combineCovar}
% \end{equation}

% These analytic formula for LS estimator of linear parameters in 
% Eqn \ref{eqn:combineMean} and \ref{eqn:combineCovar} are derived as Eqn 7.10 and 7.11
% in Glen Cowan's Statistical Data Analysis.
% With this combining method, the values of average branching 
% fractions are shown in Eqn \ref{eqn:averagebf}. The correlation of 
% $\bar{\beta}_e,\bar{\beta}_\mu, \bar{\beta}_\tau$ is shown in Fig \ref{fig:corBetaBar}. 

% \begin{align}
%     \bar{\beta}_e^{LS}    &= 0.1080 \times \big[1 \pm 0.37\% \text{ (stat)} \pm 2.06\% \text{ (syst)} \big] \\
%     \bar{\beta}_\mu^{LS}  &= 0.1080 \times \big[1 \pm 0.33\% \text{ (stat)} \pm 2.15\% \text{ (syst)} \big] \\
%     \bar{\beta}_\tau^{LS} &= 0.1080 \times \big[1 \pm 0.81\% \text{ (stat)} \pm 6.35\% \text{ (syst)} \big]
%     \label{eqn:averagebf}
% \end{align}

% \begin{figure}[ht]
%     \centering
%     % \includegraphics[width=7cm]{section5/figures/covarMatrix_beta.png}
%     \caption{ The correlation of $\bar{\beta}_e,\bar{\beta}_\mu, \bar{\beta}_\tau$ }
%     \label{fig:corBetaBar}
% \end{figure}


\begin{sidewaystable}[p]
  \small
  \renewcommand{\arraystretch}{1.2}
  \centering

  \begin{tabular}{|l|ccc|ccc|ccc|ccc|ccc|}
  \hline
  Error Source & \multicolumn{3}{c|}{$\mu$-1b} & \multicolumn{3}{c|}{$\mu$-2b} & \multicolumn{3}{c|}{$e$-1b} & \multicolumn{3}{c|}{$e$-2b} \\
  \hline
                & $B_e$ & $B_\mu$ & $B_\tau$ & $B_e$ & $B_\mu$ & $B_\tau$ & $B_e$ & $B_\mu$ & $B_\tau$ & $B_e$ & $B_\mu$ & $B_\tau$ \\
  \hline
  StatErr of Data                            & 0.543 & 0.533 & 1.243 & 0.714 & 0.637 & 1.492 & 0.743 & 0.557 & 1.520 & 0.904 & 0.707 & 1.807 \\ 
  StatErr of bg MC                           & 0.178 & 0.745 & 0.767 & 0.110 & 0.411 & 0.501 & 0.897 & 0.257 & 1.065 & 0.494 & 0.137 & 0.521 \\ 
  StatErr of sg MC                           & 0.168 & 0.151 & 0.415 & 0.189 & 0.165 & 0.428 & 0.217 & 0.176 & 0.503 & 0.233 & 0.192 & 0.520 \\ 
  \hline
  PDG err of $Br^\tau_e$                     & 0.002 & 0.019 & 0.029 & 0.002 & 0.019 & 0.029 & 0.003 & 0.019 & 0.029 & 0.003 & 0.020 & 0.030 \\ 
  PDG err of $Br^\tau_\mu$                   & 0.047 & 0.017 & 0.098 & 0.047 & 0.017 & 0.099 & 0.041 & 0.013 & 0.101 & 0.043 & 0.013 & 0.106 \\ 
  2.5$\%$ err of luminosity                  & 0.330 & 0.461 & 0.120 & 0.093 & 0.064 & 0.049 & 0.135 & 0.390 & 0.204 & 0.002 & 0.101 & 0.092 \\ 
  5$\%$ err of tt XS                         & 0.002 & 0.000 & 0.151 & 0.009 & 0.015 & 0.032 & 0.021 & 0.011 & 0.148 & 0.011 & 0.002 & 0.003 \\ 
  5$\%$ err of tW XS                         & 0.002 & 0.001 & 0.157 & 0.010 & 0.015 & 0.033 & 0.022 & 0.012 & 0.155 & 0.011 & 0.002 & 0.004 \\ 
  5$\%$ err of t XS                          & 0.062 & 0.062 & 0.033 & 0.053 & 0.052 & 0.058 & 0.063 & 0.060 & 0.032 & 0.052 & 0.054 & 0.040 \\ 
  5$\%$ err of W+Jets XS                     & 0.343 & 0.354 & 0.325 & 0.068 & 0.068 & 0.066 & 0.349 & 0.347 & 0.366 & 0.065 & 0.066 & 0.084 \\ 
  10$\%$ err of Z+Jets XS                    & 0.495 & 2.655 & 0.237 & 0.122 & 0.491 & 0.055 & 1.576 & 0.501 & 0.173 & 0.275 & 0.104 & 0.041 \\ 
  10$\%$ err of $\gamma$+Jets XS             & 0.020 & 0.019 & 0.029 & 0.005 & 0.005 & 0.007 & 0.249 & 0.247 & 0.213 & 0.058 & 0.058 & 0.081 \\ 
  10$\%$ err of VV XS                        & 0.004 & 0.044 & 0.027 & 0.001 & 0.010 & 0.005 & 0.038 & 0.003 & 0.021 & 0.008 & 0.001 & 0.001 \\ 
  25$\%$ err of QCD in $e 4j$                & 0.000 & 0.000 & 0.000 & 0.000 & 0.000 & 0.000 & 1.164 & 1.118 & 2.410 & 0.219 & 0.218 & 0.406 \\ 
  25$\%$ err of QCD in $\mu 4j$              & 0.742 & 0.737 & 1.562 & 0.223 & 0.214 & 0.384 & 0.000 & 0.000 & 0.000 & 0.000 & 0.000 & 0.000 \\ 
  25$\%$ err of QCD in $e\tau$               & 0.000 & 0.000 & 0.000 & 0.000 & 0.000 & 0.000 & 0.372 & 0.498 & 2.651 & 0.069 & 0.092 & 0.503 \\ 
  25$\%$ err of QCD in $\mu\tau$             & 0.345 & 0.465 & 2.360 & 0.185 & 0.250 & 1.285 & 0.000 & 0.000 & 0.000 & 0.000 & 0.000 & 0.000 \\ 
  top pT reweighting                         & 0.000 & 0.000 & 0.032 & 0.002 & 0.003 & 0.007 & 0.004 & 0.002 & 0.031 & 0.002 & 0.000 & 0.001 \\ 
  0.6$\%$ err of $\epsilon_e$ reco           & 0.575 & 0.054 & 0.042 & 0.583 & 0.055 & 0.042 & 0.709 & 0.160 & 0.103 & 0.574 & 0.084 & 0.069 \\ 
  1.4$\%$ err of $\epsilon_e$ id             & 1.386 & 0.129 & 0.101 & 1.410 & 0.133 & 0.101 & 1.766 & 0.335 & 0.275 & 1.456 & 0.163 & 0.197 \\ 
  0.1$\%$ err of $\epsilon_\mu$ reco         & 0.015 & 0.125 & 0.016 & 0.008 & 0.095 & 0.011 & 0.009 & 0.078 & 0.008 & 0.008 & 0.077 & 0.008 \\ 
  0.2$\%$ err of $\epsilon_\mu$ id           & 0.052 & 0.496 & 0.066 & 0.021 & 0.370 & 0.045 & 0.033 & 0.299 & 0.029 & 0.032 & 0.299 & 0.031 \\ 
  5$\%$ err of $\epsilon_\tau$               & 0.745 & 1.004 & 5.091 & 0.694 & 0.937 & 4.823 & 0.723 & 0.967 & 5.146 & 0.700 & 0.937 & 5.111 \\ 
  4.7$\%$ err of $\epsilon_{j\to\tau}$       & 0.460 & 0.620 & 3.145 & 0.307 & 0.414 & 2.129 & 0.458 & 0.613 & 3.260 & 0.290 & 0.388 & 2.115 \\ 
  0.5$\%$ err of $ES_{e}$                    & 0.249 & 0.023 & 0.018 & 0.228 & 0.022 & 0.016 & 0.008 & 0.171 & 0.061 & 0.010 & 0.247 & 0.017 \\ 
  0.2$\%$ err of $ES_{\mu}$                  & 0.095 & 0.092 & 0.033 & 0.093 & 0.092 & 0.035 & 0.013 & 0.116 & 0.011 & 0.012 & 0.114 & 0.012 \\ 
  1.2$\%$ err of $ES_{\tau\to\pi^\pm}$       & 0.034 & 0.046 & 0.232 & 0.035 & 0.047 & 0.244 & 0.034 & 0.046 & 0.245 & 0.030 & 0.040 & 0.216 \\ 
  1.2$\%$ err of $ES_{\tau\to\pi^\pm\pi^0}$  & 0.086 & 0.116 & 0.587 & 0.069 & 0.093 & 0.477 & 0.066 & 0.088 & 0.469 & 0.075 & 0.100 & 0.548 \\ 
  1.2$\%$ err of $ES_{\tau\to3\pi^\pm}$      & 0.026 & 0.035 & 0.175 & 0.026 & 0.034 & 0.177 & 0.024 & 0.032 & 0.172 & 0.024 & 0.032 & 0.176 \\ 
  Single-e Trigger (probe syst)              & 0.218 & 0.020 & 0.016 & 0.222 & 0.021 & 0.016 & 0.029 & 0.032 & 0.004 & 0.036 & 0.004 & 0.009 \\ 
  Single-e Trigger (tag syst)                & 0.495 & 0.046 & 0.036 & 0.503 & 0.047 & 0.036 & 0.063 & 0.088 & 0.080 & 0.037 & 0.013 & 0.038 \\ 
  0.5$\%$ err of $Br_{\tau\to\pi^\pm}$       & 0.008 & 0.011 & 0.047 & 0.009 & 0.012 & 0.050 & 0.008 & 0.011 & 0.047 & 0.009 & 0.012 & 0.055 \\ 
  0.4$\%$ err of $Br_{\tau\to\pi^\pm\pi^0}$  & 0.018 & 0.024 & 0.102 & 0.019 & 0.025 & 0.108 & 0.019 & 0.025 & 0.110 & 0.020 & 0.025 & 0.117 \\ 
  1.1$\%$ err of $Br_{\tau\to\pi^\pm2\pi^0}$ & 0.022 & 0.029 & 0.124 & 0.022 & 0.029 & 0.120 & 0.022 & 0.028 & 0.123 & 0.024 & 0.031 & 0.143 \\ 
  0.5$\%$ err of $Br_{\tau\to3\pi^\pm}$      & 0.015 & 0.021 & 0.094 & 0.017 & 0.022 & 0.102 & 0.016 & 0.021 & 0.100 & 0.017 & 0.022 & 0.106 \\ 
  1.1$\%$ err of $Br_{\tau\to3\pi^\pm\pi^0}$ & 0.009 & 0.011 & 0.043 & 0.010 & 0.012 & 0.046 & 0.009 & 0.011 & 0.043 & 0.010 & 0.012 & 0.046 \\ 
  Pileup                                     & 0.041 & 0.183 & 0.777 & 0.231 & 0.026 & 0.891 & 0.428 & 0.474 & 0.592 & 0.248 & 0.137 & 0.835 \\ 
  JES                                        & 2.300 & 0.750 & 4.421 & 1.823 & 1.543 & 2.968 & 1.681 & 2.370 & 4.577 & 1.681 & 1.773 & 2.993 \\ 
  JER                                        & 0.238 & 0.180 & 0.265 & 0.143 & 0.146 & 0.356 & 0.259 & 0.249 & 0.406 & 0.148 & 0.138 & 0.538 \\ 
  Btag                                       & 0.098 & 0.772 & 0.643 & 0.111 & 0.023 & 0.114 & 0.181 & 0.091 & 0.762 & 0.024 & 0.109 & 0.088 \\ 
  Mistag                                     & 0.100 & 0.141 & 0.035 & 0.100 & 0.056 & 0.090 & 0.077 & 0.142 & 0.124 & 0.030 & 0.096 & 0.135 \\ 
  tt fsr                                     & 0.760 & 0.583 & 0.743 & 0.236 & 0.253 & 0.643 & 0.289 & 0.473 & 0.756 & 1.029 & 0.065 & 1.337 \\ 
  tt isr                                     & 0.724 & 0.747 & 1.105 & 0.720 & 0.723 & 0.876 & 0.317 & 1.060 & 1.414 & 0.043 & 0.830 & 0.062 \\ 
  tt UE                                      & 0.021 & 0.037 & 1.665 & 0.306 & 1.017 & 0.266 & 0.122 & 0.177 & 1.060 & 0.172 & 0.133 & 0.053 \\ 
  tt MEPS                                    & 0.198 & 0.653 & 1.699 & 1.117 & 0.645 & 0.129 & 0.033 & 0.743 & 1.812 & 0.163 & 1.279 & 1.196 \\ 
  \hline
  Total                                      & 3.378 & 3.646 & 8.655 & 3.047 & 2.579 & 6.609 & 3.643 & 3.459 & 9.135 & 2.884 & 2.728 & 6.967 \\ 
  \hline
  \end{tabular}
  \caption{ Statistical and systematic error of four categories. }
  \label{tab:syst_alt}
\end{sidewaystable}



