\section{Measurement of b-tag Efficiencies in Simulation}
\label{sec:app:btag}

%  to correct the b-tag efficiencies in the simulation with respective to data.

% \subsection{Corrections for b-tag Efficiencies}


To account for differences of the b-tag efficiency in data and simulation, a method that modifies the b-tag status of a jet is adopted in the simulation. In the method, the status is modified based on a set of data-to-simulation scale factors derived by the b-tag POG, and the efficiencies for simulation which have been measured independently in this section.  The method of the b-tag correction for simulation works as follows:

\begin{itemize}

    \item jets are identified as originating from the decay of a b quark, c quark, or ``light" parton (usdg) from generator truth information. Depending on the parton flavor and jet \pt, the appropriate scale factor $f_{\epsilon}$ and efficiency from simulation $\epsilon$ are looked up from a map.
    
    \item if $f_{\epsilon} < 1$, then a b-tagged jet is downgraded to a non-b tagged jet with probability,
        \begin{equation}
            p = 1 - f_{\epsilon}.
        \end{equation}
        \noindent if it is not b-tagged, nothing is changed.
    
    
    \item if $f_{\epsilon} > 1$, then a non-b-tagged jet is upgraded to a b-tagged jet with probability,
        \begin{equation}
            p = \frac{1 - f_{\epsilon}}{1 - \frac{1}{\epsilon}}.
        \end{equation}
        \noindent If it is already b-tagged, its status is unchanged.
\end{itemize}

% \subsection{b-tag Efficiencies in the Simulation}

Measurement of the b-tag efficiency in simulation relies on knowing the flavor of the parton that gives rise to the jet.  This is done with official CMS tools that assign a jet flavor based on the characteristics of the quark and gluon content of a jet~\cite{twiki:jet_mc_flavor}.  The efficiencies are measured for the case of b, c, and light (usdg) flavor jets, and as a function of the jet \pt.  That is,
\begin{equation}
    \epsilon(\pt, \mathrm{flavor}) = \frac{ N_{\mathrm{pass}}(\pt, \mathrm{flavor})} {N(\pt, \mathrm{flavor})},
\end{equation}

\noindent where the numerator is the number of jets passing the b-tag working point, and the denominator is the total number of jets considered. These quantities are measured in both \ttbar and Z plus jet samples. The CSVv2 discriminator value for the two samples are shown in figures~\ref{fig:btag_csvv2} for the three jet flavor categories. The efficiency measurement uses the middle working point of CSVv2 discriminator, the same as the selection in the measurement of \PW branching fraction. The result of b-tagging efficiencies is shown in figure~\ref{fig:btag_eff}.  There is some level of disagreement between the two samples for the b quark jets that likely could be attributed to the \ttbar sample being generated with an NLO generator (\POWHEG) and the Z plus jet sample being generated with a LO generator (\MADGRAPH). The efficiencies from \ttbar sample is used for the b-tag correction. The events used for the selection require at least muon passing our analysis requirements, and the four leading \pt jets are considered in the measurement.

\begin{figure}[h!]
    \centering
    \includegraphics[width=0.45\textwidth]{chapters/Appendix/sectionBtag/figures/bmva_mc.pdf}
    \caption{Distribution of ``csv" b-tag discriminator for the three flavor categories under consideration for Z + jet and \ttbar events.      
    \label{fig:btag_csvv2}}
\end{figure}

\begin{figure}[h!]
    \centering
    \includegraphics[width=0.3\textwidth]{chapters/Appendix/sectionBtag/figures/bmva_mceff_vs_pt_b}
    \includegraphics[width=0.3\textwidth]{chapters/Appendix/sectionBtag/figures/bmva_mceff_vs_pt_c}
    \includegraphics[width=0.3\textwidth]{chapters/Appendix/sectionBtag/figures/bmva_mceff_vs_pt_usdg}
    \caption{Efficiency to b-tag a jet originating from a b quark (left), c quark (middle), and light quark (right).
    \label{fig:btag_eff}
    }
\end{figure}

\FloatBarrier

