

\section{Measurement of Scale Factors for Jet Faking Hadronic Tau}
\label{sec:app:jToTauhSF}


\subsection{Measurement Approach}
In both the signal region and control region, $\mu \tau$ and $e \tau$
final states have sizable contributions from electron or muon plus a jet
misidentified as a hadronic tau.  For correction to tau identification
efficiency, tau POG has provided $SF (\tau_h \to \tau_h)$ measured with
tag-prob technique at different identification working points.  However,
regarding the $SF (j \to \tau_h)$, because the jet environment varies from
analysis to analysis, tau POG suggests to measure it ourselves in a
side-band region to correct $j \to \tau_h$ contributions and access
corresponding uncertainties.

Our measurement of $SF (j\to \tau_h)$ is based on two regions,
$t\bar{t}$ region which is enriched with b-jets identified as hadronic
taus, $Z+jets$ region which is enriched with light jets identified as
hadronic taus. Both Tight and VTight $\tau_h$ identification working points
are considered.

\begin{itemize}
    \item $t\bar{t}$ region: $e\mu+\tau_h$ final state is selected. The selection requires
    exactly one muon and one electron with tight identification and
    isolation, plus one hadronic tau passing Tight or VTight working point.
    Corrections to reconstruction and selection of electron and muon are
    applied. The events has to fire either single muon trigger or single electron
    trigger.  The \pt threshold for triggering muon (electron) is 25 (30)
    GeV, while for non-triggering muon (electron) is 10 (20) GeV.  This
    selects a sample enriched with $t\bar{t}$ with $b \to \tau_h$. The kinematics of $e\mu+\tau$ final
    state events are shown in Figure~\ref{fig:appendix:fakeTauId:emutau}.
    
    
    \item $Z+jets$ region: $\mu\mu+\tau_h$ and $ee+\tau_h$ final state are selected.  The
    selection requires exactly two muons or two electrons with tight
    identification and isolation, plus one hadronic tau passing Tight or VTight
    working point.  Corrections to reconstruction and selection of
    electron and muon are applied.
    The trigger and \pt thresholds of leptons are the same as
    $e\mu+\tau_h$ final state.  This selects a sample enriched with $Z +
    jet$ with a light jet misidentified as $\tau_h$. The kinematics of $\mu\mu+\tau$ and
    $ee+\tau$ final states are shown in Figure~\ref{fig:appendix:fakeTauId:mumutau} and \ref{fig:appendix:fakeTauId:eetau}.
\end{itemize}

\noindent Note that the selections of $e\mu+\tau$, $\mu\mu+\tau$, $ee+\tau$ are developed based on the $e\mu$, $\mu\mu$, $ee$ selection in the main Br analysis,
using the same dilepton selection but requiring one additional Tight or VTight $\tau_h$. The origins of selected $\tau_h$ are tagged based on MC truth.  For each
selected $\tau_h$, if there is a gen-level $\tau_h$ found within $\delta
R = 0.3$, the $\tau_h$ is tagged as true identification.  If not a true
identification, we try to match it with jet in the vetoed-jet collection
and tag it as $j \to \tau_h$, flavor of which equals to the MC flavor of
the jet correspondence.  In the rare cases where multiple jet
correspondences are found, the one with highest pT is considered.  If
neither gen-level $\tau_h$ match nor jet correspondence are found, the
$\tau_h$ is untagged, which mainly due to $e \to \tau_h$. The gen-level 
origin of $\tau_h$ are included in Figure~\label{fig:appendix:fakeTauId:emutau},
\label{fig:appendix:fakeTauId:mumutau},\label{fig:appendix:fakeTauId:eetau}.


\begin{figure}
    \centering
    \includegraphics[width=0.4\textwidth]{chapters/Appendix/sectionJetToTauh/figures/emutau_dilepton_mass_pickles_lltauTight.png}
    \includegraphics[width=0.4\textwidth]{chapters/Appendix/sectionJetToTauh/figures/emutau_dilepton_mass_pickles_lltauVTight.png}
    \includegraphics[width=0.4\textwidth]{chapters/Appendix/sectionJetToTauh/figures/emutau_tauPt_pickles_lltauTight.png}
    \includegraphics[width=0.4\textwidth]{chapters/Appendix/sectionJetToTauh/figures/emutau_tauPt_pickles_lltauVTight.png}
    \includegraphics[width=0.4\textwidth]{chapters/Appendix/sectionJetToTauh/figures/emutau_tauGenFlavor_pickles_lltauTight.png}
    \includegraphics[width=0.4\textwidth]{chapters/Appendix/sectionJetToTauh/figures/emutau_tauGenFlavor_pickles_lltauVTight.png}
    \caption{Distributions of $m_{e\mu}$, $\tau_h$ \pt and gen-level $\tau_h$ origin in the $e\mu+\tau$ channel. The left and right column shows the Tight and VTight $\tau_h$ WP respectively.}
    \label{fig:appendix:fakeTauId:emutau}
\end{figure}

\begin{figure}
    \centering
    \includegraphics[width=0.4\textwidth]{chapters/Appendix/sectionJetToTauh/figures/mumutau_dilepton_mass_pickles_lltauTight.png}
    \includegraphics[width=0.4\textwidth]{chapters/Appendix/sectionJetToTauh/figures/mumutau_dilepton_mass_pickles_lltauVTight.png}
    \includegraphics[width=0.4\textwidth]{chapters/Appendix/sectionJetToTauh/figures/mumutau_tauPt_pickles_lltauTight.png}
    \includegraphics[width=0.4\textwidth]{chapters/Appendix/sectionJetToTauh/figures/mumutau_tauPt_pickles_lltauVTight.png}
    \includegraphics[width=0.4\textwidth]{chapters/Appendix/sectionJetToTauh/figures/mumutau_tauGenFlavor_pickles_lltauTight.png}
    \includegraphics[width=0.4\textwidth]{chapters/Appendix/sectionJetToTauh/figures/mumutau_tauGenFlavor_pickles_lltauVTight.png}
    \caption{Distributions of $m_{\mu\mu}$, $\tau_h$ \pt and gen-level $\tau_h$ origin in the $\mu\mu+\tau$ channel. The left and right column shows the Tight and VTight $\tau_h$ WP respectively.}
    \label{fig:appendix:fakeTauId:mumutau}
\end{figure}


\begin{figure}
    \centering
    \includegraphics[width=0.4\textwidth]{chapters/Appendix/sectionJetToTauh/figures/eetau_dilepton_mass_pickles_lltauTight.png}
    \includegraphics[width=0.4\textwidth]{chapters/Appendix/sectionJetToTauh/figures/eetau_dilepton_mass_pickles_lltauVTight.png}
    \includegraphics[width=0.4\textwidth]{chapters/Appendix/sectionJetToTauh/figures/eetau_tauPt_pickles_lltauTight.png}
    \includegraphics[width=0.4\textwidth]{chapters/Appendix/sectionJetToTauh/figures/eetau_tauPt_pickles_lltauVTight.png}
    \includegraphics[width=0.4\textwidth]{chapters/Appendix/sectionJetToTauh/figures/eetau_tauGenFlavor_pickles_lltauTight.png}
    \includegraphics[width=0.4\textwidth]{chapters/Appendix/sectionJetToTauh/figures/eetau_tauGenFlavor_pickles_lltauVTight.png}
    \caption{Distributions of $m_{ee}$, $\tau_h$ \pt and gen-level $\tau_h$ origin in the $ee+\tau$ channel. The left and right column shows the Tight and VTight $\tau_h$ WP respectively.}
    \label{fig:appendix:fakeTauId:eetau}
\end{figure}


The \pt spectrum of $\tau_h$ in $\mu\mu+\tau_h$, $ee+\tau_h$ and
$e\mu+\tau_h$ final states are shown in figure~\ref{fig:misidprefit}.
Because jet modeling of Z+jet is off in $n_j=0$ but good $n_j \geq 1$,
the events are split into $n_j=0$ and $n_j \geq 1$ to deal with jet
modeling in the Z+jet simulation. Both Tight and VTight working
points for $\tau_h$ isolation are included.



\begin{figure}
    \centering
    \includegraphics[width=0.99\textwidth]{chapters/Appendix/sectionJetToTauh/figures/2020_tauID_prefit_lltauTight.png}
    \includegraphics[width=0.99\textwidth]{chapters/Appendix/sectionJetToTauh/figures/2020_tauID_prefit_lltauVTight.png}
    \caption{Prefit distributions of $\tau_h$ \pt. Upper and lower are the Tight and VTight WP.}
    \label{fig:appendix:fakeTauId:prefit}
\end{figure}

\begin{figure}
    \centering
    \includegraphics[width=0.99\textwidth]{chapters/Appendix/sectionJetToTauh/figures/2020_tauID_postfit_lltauTight.png}
    \includegraphics[width=0.99\textwidth]{chapters/Appendix/sectionJetToTauh/figures/2020_tauID_postfit_lltauVTight.png}
    \caption{Post distributions of $\tau_h$ \pt. Upper and lower are the Tight and VTight WP. }
    \label{fig:appendix:fakeTauId:postfit}
\end{figure}




To measure $SF (q\to \tau_h)$  and $SF (b\to \tau_h)$, a template fit to the $\tau_h$ \pt is performed.  
The free parameters are $SF (q\to \tau_h)$  and $SF (b\to \tau_h)$ in 5 \pt bins from 20-80 GeV. 
The systematical uncertainties, including cross sections, luminosity, electron/muon
efficiency, are taken into account as nuisance parameters in the fit.



\subsection{Result of the Scale Factors}



\begin{figure}
    \centering
    \includegraphics[width=0.49\textwidth]{chapters/Appendix/sectionJetToTauh/figures/fit2_ptflavor2_lltauTight.png}
    \includegraphics[width=0.49\textwidth]{chapters/Appendix/sectionJetToTauh/figures/fit2_ptflavor2_lltauVTight.png}
    \caption{$SF (j\to \tau_h)$ for Tight and VTight tau.}
    \label{fig:appendix:fakeTauId:fit}
\end{figure}



The result of $SF (j\to \tau_h)$ for Tight and VTight WP are shown in figure~\ref{fig:appendix:fakeTauId:fit} and table~\ref{tab:appendix:fakeTauId:fit}
The pulls and correlation matrix of the template fit are shown in Figure~\ref{fig:appendix:fakeTauId:fitparam}

In shape analysis, the uncertainty of $SF (j \to \tau_h)$ will be used as prefit uncertainty of
the corresponding nuisance parameters.  In counting analysis, 
the measured $SF (j \to
\tau_h)$ will be variate according to the measured uncertainties.




\begin{figure}
    \centering
    \includegraphics[width=0.49\textwidth]{chapters/Appendix/sectionJetToTauh/figures/corr2_lltauTight_splitJetFlavor.png}
    \includegraphics[width=0.49\textwidth]{chapters/Appendix/sectionJetToTauh/figures/corr2_lltauVTight_splitJetFlavor.png}
    \includegraphics[width=0.49\textwidth]{chapters/Appendix/sectionJetToTauh/figures/pull2_lltauTight_splitJetFlavor.png}
    \includegraphics[width=0.49\textwidth]{chapters/Appendix/sectionJetToTauh/figures/pull2_lltauVTight_splitJetFlavor.png}
    \caption{The correlation matrix and the pulls of the fitting parameters for Tight (left) and VTight (right) $\tau_h$. }
    \label{fig:appendix:fakeTauId:fitparam}
\end{figure}



\begin{table}[h]
    \setlength{\tabcolsep}{6pt} % Default value: 6pt
    \renewcommand{\arraystretch}{1.5} % Default value: 1
    \caption{ $SF (j\to \tau_h)$ for Tight and VTight tau.}
    
    \begin{tabular}{c|ccccc}
    \hline
    $p^T_{\tau_h}$ [GeV]  & 20-25         & 25-30         & 30-40         & 40-50         & 50-80         \\
    \hline
    $SF(b\to \rm{Tight} \; \tau_h)$  & $1.02\pm0.12$ & $1.16\pm0.12$ & $1.27\pm0.11$ & $1.21\pm0.13$ & $0.81\pm0.13$ \\
    $SF(q\to \rm{Tight} \;  \tau_h)$  & $1.04\pm0.08$ & $0.99\pm0.07$ & $0.99\pm0.06$ & $0.90\pm0.06$ & $0.91\pm0.07$ \\
    \hline
    $SF(b\to \rm{VTight} \; \tau_h)$ & $0.97\pm0.14$ & $1.19\pm0.16$ & $1.39\pm0.15$ & $0.96\pm0.14$ & $0.91\pm0.17$ \\
    $SF(q\to \rm{VTight} \; \tau_h)$ & $1.02\pm0.08$ & $0.95\pm0.07$ & $0.94\pm0.06$ & $0.89\pm0.07$ & $0.86\pm0.07$ \\
    \hline
    \end{tabular}
 
    \label{tab:appendix:fakeTauId:fit}
\end{table}


\FloatBarrier