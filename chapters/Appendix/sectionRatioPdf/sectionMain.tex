\section{Probability Distribution Function for the Ratios of Branching Fractions}
\label{sec:app:ratioPdf}


Converting the distriution of branching fractions to the distribution of
ratios of branching fractions requires some care given the significant
correlation between the branching fractions.  The procedure is
nonetheless straight forward and has been carried in statistics
literature for the one dimensional case~\cite{10.1093/biomet/24.3-4.428,
10.2307/2334671}.  The procedure of transforming variables is the same
for the two dimensional case, namely, calculate the expectation value of
the absolute value of the branching fraction in the denominator.  Given
the derivation is straight forward, the result is quoted here without
going through all of the steps.  Starting from the pdf of the leptonic
branching fractions,

\begin{equation}
    g(\beta_{e}, \beta_{\mu}, \beta_{\tau}) =
    \mathcal{N}\left(\boldsymbol{\beta}; \hat{\boldsymbol{\beta}}, \boldsymbol{\Sigma}\right)
\end{equation}

\noindent where $\hat{\boldsymbol{\beta}}$ is the maximum likelihood estimates of the
leptonic branching fractions and $\boldsymbol{\Sigma}$ is the covariance.
The transformation to the PDF of the ratios is then found by evaluating
the integral,




\begin{equation}
    f(r_{e\tau}, r_{\mu\tau}) = \int_{-\infty}^{\infty}
    \left| \beta_{\tau}\right|g(r_{e\tau}\beta_{\tau}, r_{\mu\tau}\beta_{\tau}, \beta_{\tau})
    d\beta_{\tau}
\end{equation}

\noindent Carrying out the integration is very similar to the one dimensional
case.  The resulting expression is,

\begin{equation}
    f(r_{e\tau}, r_{\mu\tau}) 
    = \frac{bd}{2\pi \sigma_{e}\sigma_{\mu}\sigma_{\tau} a^{3}}
        \left[ \Phi \left(\frac{b}{a\sqrt{\Psi}}\right)
         - \Phi\left(\frac{b}{a\sqrt{\Psi}}\right)\right] 
         +
         \frac{\sqrt{\Psi}}{\sqrt{2\pi^{3}}\sigma_{e}\sigma_{\mu}\sigma_{\tau}}e^{-\frac{c}{2\Psi}},
\end{equation}

\noindent where,

\begin{align} 
    a \equiv a\left(r_{e\tau}, r_{\mu\tau}\right) 
            &= \frac{r_{e\tau}^{2}\left(1 - \rho_{\mu\tau} \right)}{\sigma_{e}^{2}}
            + \frac{r_{\mu\tau}^{2}\left(1 - \rho_{e\tau} \right)}{\sigma_{\mu}^{2}}
            + \frac{\left(1 - \rho_{e\mu} \right)}{\sigma_{\tau}^{2}} \\
            \nonumber
            &+ \frac{2r_{e\tau}r_{\mu\tau}\left( \rho_{e\tau} \rho_{\mu\tau}   - \rho_{e\mu} \right)}{\sigma_{e}\sigma_{\mu}} 
            \nonumber
            + \frac{2r_{e\tau}\left( \rho_{e\mu} \rho_{\mu\tau}   - \rho_{e\tau} \right)}{\sigma_{e}\sigma_{\tau}} \\
            \nonumber
            &+ \frac{2r_{\mu\tau}\left( \rho_{e\mu} \rho_{e\tau}   - \rho_{\mu\tau} \right)}{\sigma_{\mu}\sigma_{\tau}}
\end{align}
\begin{align}
    b \equiv b(r_{e\tau}, r_{\mu\tau}) 
        &= \frac{r_{e\tau}\beta_{e}\left(1 - \rho_{\mu\tau} \right)}{\sigma_{e}^{2}}
        + \frac{r_{\mu\tau}\beta_{\mu}\left(1 - \rho_{e\tau} \right)}{\sigma_{\mu}^{2}}
        + \frac{\beta_{\tau}\left(1 - \rho_{e\mu} \right)}{\sigma_{\tau}^{2}} \\
        \nonumber
        &+ \frac{\left(r_{e\tau}\beta_{\mu} + r_{\mu\tau}\beta_{e}\right)\left( \rho_{e\tau} \rho_{\mu\tau} - \rho_{e\mu} \right)}{\sigma_{e}\sigma_{\mu}}
        \nonumber
        + \frac{\left(r_{e\tau}\beta_{\tau} + \beta_{e}\right)\left( \rho_{e\mu} \rho_{\mu\tau}   - \rho_{e\tau} \right)}{\sigma_{e}\sigma_{\tau}} \\ 
        \nonumber
        &+ \frac{\left(r_{\mu\tau}\beta_{\tau} + \beta_{\mu}\right)\left( \rho_{e\tau} \rho_{e\mu}   - \rho_{\mu\tau} \right)}{\sigma_{\mu}\sigma_{\tau}}
\end{align}
\begin{align}
    c &= \frac{\beta_{e}^{2}\left(1 - \rho_{\mu\tau} \right)}{\sigma_{e}^{2}}
    + \frac{\beta_{\mu}^{2}\left(1 - \rho_{e\tau} \right)}{\sigma_{\mu}^{2}}
    + \frac{\beta_{\tau}^{2}\left(1 - \rho_{e\mu} \right)}{\sigma_{\tau}^{2}} \\
        \nonumber
        &+ \frac{2\beta_{e}\beta_{\mu}\left( \rho_{e\tau} \rho_{\mu\tau} - \rho_{e\mu} \right)}{\sigma_{e}\sigma_{\mu}}
        \nonumber
        + \frac{2\beta_{e}\beta_{\tau}\left( \rho_{e\mu} \rho_{\mu\tau}   - \rho_{e\tau} \right)}{\sigma_{e}\sigma_{\tau}} \\ 
        \nonumber
        &+ \frac{2\beta_{\mu}\beta_{\tau}\left( \rho_{e\tau} \rho_{e\mu}   - \rho_{\mu\tau} \right)}{\sigma_{\mu}\sigma_{\tau}}
\end{align}

\begin{equation}
    \Psi = 1 - \rho_{e\mu}^{2} - \rho_{e\tau}^{2} - \rho_{\mu\tau}^{2} + 2\rho_{e\mu}\rho_{e\tau}\rho_{\mu\tau}
\end{equation}

\noindent This expression gives the analytic expression for the PDF of two ratios
derived from three normally distributed quantities accounting for their
correlations and is used in drawing figure~\ref{fig:ratio_contours_2D}.
\FloatBarrier