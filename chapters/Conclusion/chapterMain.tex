\chapter{Conclusion}
\label{sec:conclusion}


A precision measurement of three individual leptonic branching fractions and the inclusive hadronic branching fraction of \PW boson has been performed using the LHC $\sqrt{s}=13\TeV$ proton-proton collision data collected by the CMS detector during run 2016. Dataset are triggered with single electron and single muon trigger. The final states corresponding the topology of leptonic and semileptonic \ttbar are selected. The selected sample is split into mutually-exclusive channels based on the multiplicity of electron, muon and hadronic tau, requiring at least one muon or one electron that enables trigger. For each channel, further partition relying on the jet and \PQb tag multiplicities is designed to separate regions with different signal purity.

Two approaches are developed for this measurement. The shape analysis fits the \pt distribution of the sensitive leptons in different channels. It exploits the \WW and \wjets regions for more \PW statistics, and \zjets regions for controlling the systematics related to the identification of hadronic tau. The counting analysis constructs ratios of yields for channels with the same trigger and analytically solves three leptonic branching fractions from a set of quadratic equations. It eliminates the shape information and uses only the \ttbar concentrated regions. It is designed to cross-check the shape analysis. 


From shape analysis, the \PW branching fractions \BWe, \BWm, \BWt and \BWh are $10.83(10)\%$, $10.94(08)\%$, $10.77(21)\%$ and $67.46(28)\%$, respectively. From counting analysis, the \PW branching fractions \BWe, \BWm, \BWt and \BWh are $11.16(27)\%$, $11.13(22)\%$, $10.64(65)\%$ and $67.08(72)\%$, respectively. The values from the two approaches consistent with each other in one sigma. The shape analysis is about 3 times more precise than the counting analysis, owing to the sensitivity from the \pt spectrum, controlling the \PGth systematics and wider selection regions embracing \WW and \wjets events. 

Based on the more precise result from the shape analysis, the ratios between pair-wised leptonic channels are calculated. Assuming universality between electron and muon, the ratio between tauonic branching fraction and the average of electronic and muonic branching fraction is determined as
\begin{equation*}
    R_{\tau/(e,\mu)} = \frac{2 B^W_\tau} {B^W_e +  B^W_\mu} = 1.002\pm0.019,
\end{equation*}

\noindent consistent with the standard model lepton flavor universality. This resolves the LEP's tension with the standard model, which has been an open issue for more than a decade. Assuming lepton flavor universality among three generations, the leptonic and total hadronic branching fraction is estimated as $10.89(08)\%$ and $67.32(23)\%$ respectively, leading to the ratio of the total leptonic and total hadronic branching factions as $R^W_{h/l}=2.060 \pm 0.021$. With 
$
    R^\PW_{\mathrm{h}/\ell} = \BWh/(1- \BWh) = \big( 1 + \alpS(m_\PW)/\pi \big) \sumCKM
$,
three standard model quantities are subsequently derived: the sum square of elements in the first two rows of the Cabibbo--Kobayashi--Maskawa (CKM) matrix  $\sum{\left|V_{ij}\right|^{2}} = 1.991 \pm 0.019$, the CKM element $V_{cs} = 0.970 \pm 0.008$, and the strong coupling constant at the \PW mass pole $\alpha_{S}(m_\mathrm{W}) = 0.099 \pm 0.026$.


This CMS measurement of \PW branching fraction successfully improves upon the experimental precision from LEP. In addition, give the a good agreement with the standard model lepton flavor universality, it clearly resolves LEP's tension with the SM LU.