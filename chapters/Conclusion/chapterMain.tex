\chapter{Conclusion}
\label{sec:conclusion}


In this thesis, a precision measurement of the \PW leptonic and inclusive hadronic branching fraction has been performed using the LHC $\sqrt{s}=13\TeV$ p-p collision data collected by the CMS detector during run 2016. 
Dataset are triggered with single electron and single muon trigger. The final states corresponding the topology of leptonic and semileptonic \ttbar are selected. 
The selected sample is split into seven mutually-exclusive channels based on the multiplicity of electron, muon and hadronic tau, requiring at least one muon or one electron that enables trigger.
For each channel, further partition relying on the jet and b-tag multiplicities is designed to separate regions with different signal purity.

Two approaches are used for the measurement. The shape analysis template fit the \pt distribution of the sensitive leptons in different channels and $n_j n_b$ categories. It exploits the $WW$ and $w+jets$ region for more \PW statistics, and $Z+jets$ region for controlling the systematics related to the identification of hadronic tau. The counting analysis constructs ratios of yields for channels with the same trigger and analytically solve three leptonic branching fractions from a set of quadratic equations. It eliminates the shape information and uses only the \ttbar concentrated regions. It is designed to cross-check the shape analysis. 


From shape analysis, the $W\to e\nu$, $W\to \mu\nu$, $W\to \tau\nu$ branching fraction and \PW total hadronic branching fraction are $10.83(10)\%$, $10.94(08)\%$, $10.77(21)\%$ and $67.46(28)\%$, respectively.
From counting analysis, $11.15(27)\%$, $11.13(22)\%$, $10.63(65)\%$ and $67.08(72)\%$, respectively. The value from the two approaches consistent with each other in one sigma. The shape analysis is about 3 times more precise than the counting analysis, owing to the sensitivity from the \pt spectrum, controlling the $\tau_h$ systematics and wider selection regions embracing $WW$ and $W$+Jets events. 

Based on the more precise shape analysis, the ratios between pair-wised leptonic channels are calculated. Assuming universality between electron and muon, the ratio between tauonic branching fraction and the average of electronic and muonic branching fraction is determined as
\begin{equation*}
    R_{\tau/(e,\mu)} = 2 B^W_\tau /(B^W_e +  B^W_\mu) = 1.002\pm0.019,
\end{equation*}

\noindent consistent with the SM lepton universality. This solves the LEP's tension with the SM which is undetermined for more than a decade. Assuming lepton universality among three generations, the leptonic and total hadronic branching fraction is estimated as $10.89(08)\%$ and $67.32(23)\%$ respectively, leading to the ratio of \PW total leptonic and total hadronic $R^W_{h/l}=2.060 \pm 0.021$. With 
$
    R^W_{\rm h/l} = \frac{B_h}{1-B_h} = 
    \left[1 + \frac{\alpha_{S}(M^{2}_{\mathrm{W}})}{\pi}\right]
    \sum_{\substack{i = (\mathrm{u,c}), \\ j=(\mathrm{d, s, b})}}
    \left|\rm V_{ij}\right|^{2}
$, 
three standard model quantities are subsequently derived: the sum square of elements in the first two rows of the Cabibbo--Kobayashi--Maskawa (CKM) matrix  $\sum{\left|V_{ij}\right|^{2}} = 1.991 \pm 0.019$, the CKM element $V_{cs} = 0.970 \pm 0.008$, and the strong coupling constant at the W mass pole $\alpha_{S}(m_\mathrm{W}) = 0.099 \pm 0.026$.


This CMS measurement of \PW branching fraction successfully improves the experimental precision from LEP. In addition, by a good agreement with SM LU, it clearly responds to the LEP's tension with the SM LU undetermined for more than a decade.