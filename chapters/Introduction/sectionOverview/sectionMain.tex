
\section{Overview}
\label{sec:introduction:overview}

% + \bar{\psi}_R \gamma^\PGm (g' Y B_\PGm) \psi_R 
In the standard model (SM) of particle physics, the interaction between the charged weak bosons and the leptons is expressed by the Lagrangian term $\bar{\chi}_L \gamma^\mu \big( g T_a W^a_\mu +g'Y B_\mu \big) \chi_L $, where the coupling constant $g$ is the same for all three lepton generations. Namely,
\begin{equation*}
	g_\Pe = g_\PGm = g_\PGt \equiv g.
\end{equation*}
\noindent This property is part of a fundamental SM assumption, known as lepton flavor universality (LFU). Therefore, a test of LFU for the \PW boson's interaction is an important aspect to examine the SM and probe new physics. It has been performed by many particle physics experiments, approaches of which primarily include using the decay of \PW boson produced in colliders, using the weak decay of mesons, and using the weak decay of leptons. Section~\ref{sec:introduction:relatedWorks} in this introduction briefly reviews these activities. Among them, the most related experiments to this thesis are performed with the decay of \PW bosons produced in high-energy particle colliders.


% SPS+Tevatron
The earliest LFU test of this kind can be traced back to the experiments at the Super Proton Synchrotron (SPS), UA1~\cite{Albajar:1988ka} and UA2~\cite{appel1986measurement, Alitti:1991eh, Alitti:1992hv}, at CERN in the 1980s. The measurements were then improved by the Tevatron experiments, CDF~\cite{Abazov:2003sv,Abe:1990sd,Abe:1992ys, Abe:1991fb} and \DZERO~\cite{ Abbott:1999tt, Abachi:1995xc, Abbott:1999pk}, at Fermilab during Tevatron's run-1 from 1985 to 1995. Both SPS and Tevatron produced \PW bosons from \Pp-\PAp collisions. One of the common features of these experiments was that the primary measured quantities were the cross-sections of the inclusive \PW production in three \PW leptonic decay channels, $\sigma_{\Pp\PAp \to \PW} \times \BWl$ for $\ell \in \{\Pe,\PGm, \PGt \}$. Then LFU were tested by taking the ratios of the measured $\sigma_{\Pp\PAp \to \PW} \times \BWl$ between two different lepton generations. For \PW coupling to the electron and tau, the combined SPS and Tevatron result~\cite{Abbott:1999pk} showed
\begin{equation*}
    g^\PW_\PGt / g^\PW_\Pe = 0.988 \pm 0.025 \quad \text{(SPS+Tevatron)}.
\end{equation*}
\noindent Overall, the SPS and Tevatron results did not show any clear sign of LFU violation related to the \PW boson.



% LEP
The most precise measurements of the three \PW leptonic branching fractions came from the four LEP experiments, ALEPH~\cite{Heister:2004wr}, DELPHI~\cite{Abdallah:2003zm}, OPAL~\cite{Abbiendi:2007rs}, L3~\cite{Achard:2004zw} during the LEP's run-2 (1995-2000) which produced \PW\PW pairs from the electron-positron collisions. By the time of this thesis, the four LEP experiments are still the only \PW leptonic branching fraction measurements included in the world average by the Particle Data Group (PDG). The combined LEP result~\cite{Schael:2013ita} gave $10.71(16)\%$, $10.63(15)\%$, $11.38(21)\%$ for the electronic, muonic and tauonic branching fractions, respectively. Assuming partial universality between electron and muon, ratio between the tauonic and the combined electronic and muonic branching fractions was reported~\cite{Schael:2013ita} as
\begin{equation*}
    R^\PW_{\PGt/(\Pe,\PGm)} = \frac{2\times \BWt }{\BWe + \BWm } = 1.066 \pm 0.025 \quad \text{(LEP)}.
\end{equation*}
\noindent In comparison, the SM predictions~\cite{Denner:1991kt,Rtau,dEnterria:2016rbf} for $R^\PW_{\PGt/(\Pe,\PGm)}$ is 0.99912, taking into account the next-to-leading order electroweak corrections and the effect of lepton mass in the \PW decay phase space. LEP's $R^\PW_{\PGt/(\Pe,\PGm)}$ shows a 2.6 standard deviation ~\cite{Schael:2013ita} from the SM prediction. This moderate deviation motivates the measurement of the branching fractions more precisely.
%is $R_{\PGt/e} \equiv B^\PW_\PGt/B^\PW_\Pe = 0.99912$ and $R_{\PGm/e} \equiv B^\PW_\PGm/B^\PW_\Pe = 1.00000$ 



% LHC
Opportunities arrive in the LHC era. During the LHC run-2, proton and proton collision at $\sqrt{s}=13\TeV$ allows an unprecedentedly large cross-section for \ttbar production. Since a top quark decays almost exclusively into one \PW boson and one \PQb quark, with the help of \PQb tagging techniques~\cite{Chatrchyan:2012jua, Sirunyan:2017ezt, Bols:2020bkb}, it is possible to select a large and high-purity sample of \ttbar events with two \PW bosons, which can then be used to study \PW boson decays. A recent measurement by the ATLAS Collaboration~\cite{Aad:2020ayz} has exploited such a strategy to measure $R^\PW_{\PGt/\PGm} = \frac{ \BWt}{\BWm}$ by fitting the impact parameter distribution of the final state muons. The resulting value is $R^\PW_{\PGt/\PGm} = 0.992 \pm 0.013$, suggesting that lepton flavor universality is preferred. 


% CMS
In CMS, there has been a significant improvement on the identification of the hadronic tau leptons~\cite{Chatrchyan:2012zz, Khachatryan:2015dfa, Sirunyan:2018pgf}, which further opens the door to efficiently select \PW decays with \PGth final states in addition to electron and muon final states, and therefore to measure all the three leptonic branching fractions simultaneously. The CMS analysis in this thesis is performed under this context.

    \noindent \textbf{Motivations:}
        \begin{itemize}
            \item The measurements of three \PW leptonic branching fractions have not been improved for more than a decay since LEP;
            \item LEP's $R_{\PGt/(e,\PGm)}$ shows a $2.6\,\sigma$ deviation from the SM prediction.
        \end{itemize}
    
    \noindent \textbf{Opportunities:}
        \begin{itemize}
            \item LHC 13\TeV \Pp-\Pp collisions produce a large number of \ttbar events giving $\PW\PW$ pairs;
            \item The improved \PQb tagging allows to select \ttbar events with a high purity;
            \item The improved \PGth identification enables to efficiently select \PW tauonic decays.
        \end{itemize}
\noindent We have performed a simultaneous measurement of \BWe, \BWm, \BWt and the inclusive hadronic branching \BWh. $35.9\fbinv$ of data collected by the CMS at $\sqrt{s} =13\TeV$ during the 2016 run of the LHC are analyzed by selecting events consistent with the decay of $\ttbar\to \PW\PW+\PQb\PQb$ and $\tW\to \PW\PW+\PQb$. The final states resulting from either one or both of the \PW bosons decaying leptonically are considered.  To collect these events, single electron and single muon triggers are used, thus requiring that the final state must contain at least one prompt electron or muon. Based on the presence of final state objects, data sample are split into a few channels, including \cme, \cmm, \cmt, \cmh channel based on the single muon trigger and \cee, \cem, \cet, \ceh channel based on the single electron trigger. The estimation of the \PW branching fractions is carried out based on two separately-developed approaches: 

\begin{table}[!htbp]
    \centering
    \setlength{\tabcolsep}{0.5 em}
    \renewcommand{\arraystretch}{2}
    \begin{tabular}{ >{\centering}m{0.22\textwidth}|m{0.7\textwidth} }
        \hline
        \textbf{Shape analysis}      & template fit the \pt distribution of the sensitive leptons in all different channels simultaneously. \\ 
        \hline
        \textbf{Counting analysis}   & construct ratios of yields for channels with the same trigger and solve three leptonic branching fractions from a set of quadratic equations. \\ 
        \hline
    \end{tabular}
\end{table}


The shape analysis is designed to push forward the precision of W branching fractions beyond LEP. It makes the most advantage of the shape information of the lepton \pt spectrum to discriminate the electron and muon coming from \PW boson decay and from the decay of taus from \PW boson. Comparing with the lepton impact parameters, \pt can be calibrated more conveniently using the energy correction provided by the CMS physics object group (POG), and systematics uncertainty associated to \pt is also expected to be smaller. To achieve better precision, shape analysis includes extra orthogonal regions besides the \ttbar enriched regions, thus constraining some of the most offending systematics such as those related to \PGth reconstruction. 

The counting analysis is designed to cross-check the shape analysis, with more emphasis on the robustness over precision. It eliminates the shape information of the kinematics distribution and uses only \ttbar concentrated regions. By constructing ratios of yields for channels with the same trigger, it has the benefit of canceling some systematics uncertainties related to \ttbar cross section, trigger efficiency and luminosity, and being robust with the lepton energy calibration. However, its precision is significantly limited by the \PGth identification systematics, ultimately being less sensitive than the shape analysis. 

This thesis describes both the approaches and their results. For \BWt, the shape analysis achieves an absolute uncertainty of 2.1\%, while the uncertainty of counting analysis is about 6.7\%. The final result of shape and counting analysis agree with each other within one sigma. In the CMS publication, the more precise result from the shape analysis is reported as the official CMS result, and are compared and combined with the LEP result. 

With the simultaneously measured three individual leptonic branching fractions and their correlations, the pairwise ratios between two branching fractions are calculated to test the SM LFU predictions. Furthermore, the leptonic and inclusive hadronic branching fraction under the lepton flavor universality assumption are also estimated by repeating the shape analysis, which alternatively uses the same parameter for three leptonic branching fractions. From the measured \BWh, some SM quantities can be derived, as listed in Table~\ref{tab:introduction:overview:derivedQuantity}. Assuming the unitarity of the CKM matrix, the strong coupling constant $\alpS(m_\PW)$ can be calculated; alternatively, using the latest experimental measurement of $\alpS(m_\PW)$, the square sum of the six CKM elements in the first two rows can be calculated and compared with the unitarity. Among the six elements in the square sum \sumCKM, \absVcs has the least experimental precision, currently at percent level. So we can take a step further to determine \absVcs using the experimental values of five other CKM elements. The mathematics and physics related to these derivations are covered in Section~\ref{sec:physics:vcs}.




\begin{table}[!h]
    \setlength{\tabcolsep}{0.5em}
    \renewcommand{\arraystretch}{1.5}
    \centering
    \caption{Standard model quantities can be derived from the measured \BWh. }
    \resizebox{0.92\textwidth}{!}{
    \begin{tabular}{ccc}
        % \hline
        Assumption &  & Derived quantity \\
        \hline
        CKM Unitarity $\sumCKM = 2$                             & $\Longrightarrow$                     & $\alpS(m_\PW)$            \\%= 0.094\pm0.033$
        \hline
        PDG $\alpS(m_\PW) = 1.1200\pm0.010$~\cite{pdg2020}      & $\Longrightarrow$                     & \sumCKM                   \\ %= 1.985\pm0.021
        \hline
        PDG $\alpS(m_\PW) = 1.1200\pm0.010$~\cite{pdg2020}      & \multirow{2}{*}{$\Longrightarrow$}    & \multirow{2}{*}{\absVcs}  \\ %= 0.968\pm 0.011
        PDG $\sumCKMfive = 1.0490(18)$~\cite{pdg2020}           &                                       &                           \\
        \hline
    \end{tabular}}
    \label{tab:introduction:overview:derivedQuantity}
\end{table}




For the outlook of this measurement in the LHC run-3 and high luminosity LHC (HL-LHC)~\cite{Apollinari:2284929} runs, the further precision improvement can be contributed from the advancement of the \PGth identification, as well as the improvement of the impact parameter resolution if it is included in the future as additional discriminating observables or additional categorization dimensions. In the era of HL-LHC, a few exciting upgrades of the CMS detector will have been accomplished after the phase-2 upgrade~\cite{CMSCollaboration:2015zni}. A new tracker system~\cite{Klein:2017nke} will improve the resolution of the impact parameter. A new endcap calorimeter, the high granularity calorimeter (HGCAL)~\cite{Collaboration:2293646}, is expected to improve the \PGth identification by allowing novel deep learning algorithms based on its high-resolution jet images. This thesis also describes a new clustering algorithm developed for the HGCAL reconstruction and the corresponding high-performance computing using GPUs.



The rest of this introduction covers a brief review of the related LFU tests and \absVcs measurements. Chapter 2 and 3 describe the related SM/BSM physics and the key aspects of the CMS experiment, respectively. Chapter 4 presents the method and the result of the measurement of the \PW branching fractions, followed by the supplement studies in Chapter 5. Chapter 6 shows the clustering algorithm developed for the HGCAL reconstruction and its high performance computing using GPUs.
