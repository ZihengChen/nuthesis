
\section{Theories about LU}
\label{sec:relatedworks:theories}


\subsection{Standard Model}

The minimum action principle leads to the Euler-Lagrange equation of motion:
\begin{equation}
    \frac{\partial}{\partial x_\mu} \frac{\partial L}{\partial(\partial \phi / \partial x_\mu)} - \frac{L}{\partial \phi} = 0
\end{equation}

Given a lagrangian of a system, the dynamics of the system is fully dictated by the Euler-lagrangian Equation of Motion. Therefore the modeling of the physics of the fundamental particles is essentially the designing and interpreting the Lagrangian. For example, modeling a massive scaler field with $L=\frac{1}{2} \partial_\mu\phi \partial^\mu \phi - \frac{1}{2} m^2 \phi^2$ leads to Klein-Gordon equation

\begin{equation}
    \partial_\mu \partial^\mu \phi + m^2 \phi = 0
\end{equation}



it dictates the dynamics of a system which is modeled by the Lagrangian L. Standard Model is essentially such a Lagrangian for fundamental particles.



What is the world made of? So far the ultimate answer to the question human device is the standard model (SM). We pursue the most concise fumula to discribe the underline Physics rules, trying to unviewing the most fundmental rules as much as possible. In the framework of lagrangian formulism, the design of the world is written as the lagrangian and then implemented by the equation of motion resulted from the minimal action princile. 


Group is a set of operations that are closed, has unity and inverse, and are associative. Lie group is a continuous and smooth group.  The topology of lie group can be completely expressed as a set of generator and corresponding real parameters. Any element in the lie group is equavilent to a consective infinitesmal operations.

\begin{equation}
    U(\theta) = e^{\theta J} = e^{\sum_i \theta_i J^i }  =  \Pi_i \lim_{n \to \infty} (1+\frac{\theta_i}{n})^n
\end{equation}


where $J_i$ is the generator of the lie group and its mathmatical meaning is the tangent at unity. The lie algibra of the lie group is nothing but the commutation relation of the generators. Any groups with the same lie algibra is isotrophic, such as SO(3) and SU(2) both lie algribra is cross product. Standard model is U(1)xSU(2)xSU(3).

The fundation of standard model is Yang-mills gauage field and higgs mechanism. Yang-mills theory provide a frame work for nonablilian gauge field, based on which electroweak and QCD are established. The higgs mechanism solves the mass of gauge boson in the gauge field via spontaneous symmetry breaking and also generate fermion mass via their interaction with higgs field. 

The lagrangian of QED is 

\begin{equation}
    \mathit{L} = \bar{\psi} i\gamma^\mu \partial_\mu \psi - m \bar{\psi} \psi + e \bar{\psi} \gamma^\mu A_\mu \psi - \frac{1}{4}F_{\mu\nu}F^{\mu\nu} = \bar{\psi} (i\gamma^\mu D_\mu - m) \psi - \frac{1}{4}F_{\mu\nu}F^{\mu\nu}
\end{equation}

It is sysmetric under local gauge transformation $\psi(x) \to e^{i \alpha(x) } \psi(x) $





elements can be parameterized by a set of parameters and


gauge invariance of 
GWS



Local gauge invariance


spontaneous symmetry breaking

higgs mechnism
