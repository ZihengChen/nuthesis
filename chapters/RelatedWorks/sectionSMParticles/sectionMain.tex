
\section{Standard Model Particles}
\label{sec:relatedWorks:smParticles}

To answer the question of "what is the world made of", numerous theories have been developed throughout history. The ancient Greeks modeled the matter with four fundamental elements: air, water, fire, and earth, while the ancient Chinese believed in the five most essential building blocks: metal, wood, water, fire, and earth. In the modern age, the emergence of science provides a systematic approach to develop and test the models for the constituents of matter. As the experimental technology allows us to probe smaller and smaller structures, our understanding of the fundamental elements of matter evolves from molecules to atoms, then to nucleons and atomic electrons, finally to quarks and leptons, which is our wildly-acknowledged answer to this question in the modern physics. This level of understanding was achieved by a set of exciting progress in both physics theory and experiments in the most recent century, which together led us to the Standard Model (SM), a systematic and elegant answer to the question of "what is matter", as well as "what is force".

Since its establishment, SM, on one hand, has been very successful in making predictions for the experimental phenomena, such as the existence, properties, and behaviors of particles. On the other hand, it has been tested with remarkably high precision in many aspects in experiments such as fixed-target experiments, collider experiments, and neutrino experiments. Despite its tremendous success, SM is still not the perfect ultimate theory to settle down because it still has many limitations: the gravitational force which is one of the four fundamental interactions in nature is not included in the SM; the dark matter hinted by many astronomy observations is not modeled by any SM fermions; electromagnetic and weak forces are unified, but it does not unify the strong force; moreover, the running couplings of the three forces do not join at one point in the high energy scale. So seeking new physics beyond the standard model is one of the important topics in particle physics nowadays. 


\begin{figure}[ht]
    \centering
    \includegraphics[width=0.6\textwidth]{chapters/RelatedWorks/sectionSMParticles/figures/sm.png}
    \caption{Particles in the Standard Model. Fermions include quarks and leptons both having three generations, while bosons include four gauge bosons and one Higgs boson.}
    \label{fig:relatedWorks:smParticles:sm}
\end{figure}

Standard Model treats matter and the force among the matter as a set of different quantum fields, excited states of which correspond to different indivisible fundamental particles. Figure~\ref{fig:relatedWorks:smParticles:sm} shows the table of SM particles. The "matter particles" are fermions with spin-1/2 including quarks and leptons, while the "force particles" are bosons with spin-1 account for the electromagnetic, strong, weak force. Additionally, there is a spin-0 Higgs boson that generates mass for fermions and gauge boson via Higgs Mechanism. The mathematical foundation of these SM particle in Figure~\ref{fig:relatedWorks:smParticles:sm} is the quantum field theory (QFT), a theory combining quantum mechanics and special relativity, which is briefly described in Section~\ref{sec:relatedWorks:qft}. This section gives an overall description of the particles in SM. 


\subsection{Fermions}
\label{sec:relatedWorks:smParticles:fermion}

\begin{table}[ht]
    \centering
    \setlength{\tabcolsep}{1.2em}
    \renewcommand{\arraystretch}{1.2}
    \begin{tabular}{ccccc|ccccc}
    \hline
    lepton      & $T$           & $T^3$          & $Q$ & $Y$ & quark  & $T$           & $T^3$          & $Q$            & $Y$            \\
    \hline
    $\nu_{e,L}$ & $\frac{1}{2}$ & $\frac{1}{2}$  & 0   & -1  & $u_L$  & $\frac{1}{2}$ & $\frac{1}{2}$  & $\frac{2}{3}$  & $\frac{1}{3}$  \\
    $e_L$       & $\frac{1}{2}$ & $-\frac{1}{2}$ & -1  & -1  & $d_L$  & $\frac{1}{2}$ & $-\frac{1}{2}$ & $-\frac{1}{3}$ & $\frac{1}{3}$  \\
    \hline
    -           & -             & -              & -   & -   & $u_R$  & 0             & 0              & $\frac{2}{3}$  & $\frac{4}{3}$  \\
    $e_R$       & 0             & 0              & -1  & -2  & $d_R$  & 0             & 0              & $-\frac{1}{3}$ & $-\frac{2}{3}$ \\
    \hline
    \end{tabular}
    \caption{The electroweak quantum number of lepton and quarks. The second and third family of lepton and quark have the same EW quantum number as the first family. The origins and meanings of these electroweak quantum numbers are discussed in Section~\ref{sec:relatedWorks:qft:gws}}
    \label{tab:relatedWorks:smParticles:ewQuantumNumber}
\end{table}


\noindent \textbf{Quarks} Three generations of quarks have been discovered: up ($u$) and down ($d$) being the first generation, charm ($c$) and strange ($s$) being the second generation, top ($t$) and bottom ($b$) being the third generation. The terminology ``generation" is often referred to as ``family" or ``flavor" as well. The up, charm and top quarks have charge $-\frac{2}{3}$, while the electric charge of down, strange, and bottom quark is $\frac{1}{3}$. Other than electric charge, a quark also carries color quantum number and thus participates in the strong interaction. The color quantum number in the strong force including red, green, and blue, is analogous to the electric charge in the electromagnetic force. Each quark has its corresponding antiquark carrying an opposite electric charge and anti-color. However, neither the fractional charge nor individual color charge is observed in nature, because quarks never exist alone. Quarks and their properties only reveal during the high-energy short-distance local interactions. At low energy levels, they are always combined in two-quark or three-quark final bounded states, called mesons and baryons respectively, which are color-neutral and integer-charged. This phenomenon is the so-called quark confinement, mathematical form of which is presented in Section~\ref{sec:relatedWorks:qft:qcd}. Quarks not only couple to the electromagnetic and strong force, but are also involved in the weak interaction. In fact, quarks are the only SM particles that couple to all four fundamental forces including gravity since they are massive. The weak hypercharge and isospin of quarks are listed in the Table~\ref{tab:relatedWorks:smParticles:ewQuantumNumber}. The mass of quarks arranges from a few MeV to 173 GeV, increasing with the quark generations. The heavy quarks have a short lifetime and decay into light quarks via the weak force with quark mixing. Therefore, the matter in our everyday life includes only the first generation light quarks. The most massive quark $t$ decays almost instantaneously to $b$ and weak gauge boson upon its production before hadronizing into bounded states. Quark model has been successful in the classification of mesons and baryons, and explaining the observations in the lepton-nucleon deep-inelastic scattering, electron-positron annihilation and proton-proton hard collision.



\noindent \textbf{Leptons} Three generations leptons have been discovered: electrons ($e$) and electron neutrino ($\nu_e$) being the first family, muon ($\mu$) and muon neutrino ($\nu_\mu$) being the second family, tau ($\tau$) and tau neutrino ($\nu_\tau$) being the third family. Electron, muon, and tau have -1 electric charge and couple to the electromagnetic force, while all neutrinos are not charged and thus do not interact electromagnetically. Charged leptons can be both left-handed and right-handed, while the neutrinos can only be left-handed because right-handed neutrinos have not been experimentally observed so far. Due to the chiral nature of weak interaction, the left-handed leptons couples to both W and Z weak gauge bosons, while right-handed charged leptons have zero weak isospin and do not couple to the W boson. The quantum number of leptons are also shown in the Table~\ref{tab:relatedWorks:smParticles:ewQuantumNumber}. The mass of the charged leptons increases with the lepton generation. Charged leptons in the second and third generations have finite lifetimes. Therefore, electrons are the only charged lepton in everyday matter. The mass of neutrinos had been thought of as zero in the Standard Model until the discovery of neutrino oscillation. The neutrino masses were then added to the SM. But the exact values of neutrino masses are still not determined yet.




\subsection{Boson}
\label{sec:relatedWorks:smParticles:boson}

The bosons in SM consist of four spin-1 gauge bosons and one spin-0 Higgs boson. Four gauge bosons are responsible for the forces among fermions: the electromagnetic force is mediated by the photon $\gamma$; the strong nuclear force is propagated by exchanging gluons $g$, the weak force is carried by the $W^\pm$ and $Z$ bosons. In the QFT, the existence of gauge boson originates from engaging the local symmetries or gauge symmetry to the fermions Lagrangian. In other words, force is the consequence of gauging the matter. For example, the existence of gluons or the strong force can be derived from the gauge symmetry of three colors of the quarks. The gauge symmetry is further discussed in Section~\ref{sec:relatedWorks:qft:gaugeSymmetry}. Among the four gauge bosons, photon and gluons are massless while the $W\pm$ and $Z$ boson are massive with $m_W = 80.385\pm0.015$ GeV and $m_Z = 90.183\pm 0.002$ GeV \cite{pdg2020} respectively. The mass of the gauge boson breaks the gauge symmetry and become not renormalizable unless the masses of gauge bosons are generated by the Higgs boson. The Higgs boson is a spin-0 boson that was predicted back in the 1960s to solve the issues related to the mass of gauge bosons and finally confirmed exists at $m_H=125.09\pm 0.24$ GeV by the CMS \cite{Chatrchyan:2012ufa} and the ATLAS \cite{Aad:2012tfa} experiment at LHC in 2012. It is the last missing piece founded and added in the SM and its discovery completes the table of SM particles shown in Figure~\ref{fig:relatedWorks:smParticles:sm}. Besides the mass of gauge boson, the Higgs boson also generates the mass for all the fermions via Yukawa couplings. Therefore Higgs is often called the ``God" particles that generate mass in the universe.

