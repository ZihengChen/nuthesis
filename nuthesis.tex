%%%%%%%%%%%%%%%%%%%%%%%%%%%%%%%%%%%%%%%%%%%%%%%%%%%%%%%%%%%%%%%%%%%%%%
% nuthesis-template.tex - Miguel A, Lerma - 4/23/2018
%                         mlerma@math.northwestern.edu
%%%%%%%%%%%%%%%%%%%%%%%%%%%%%%%%%%%%%%%%%%%%%%%%%%%%%%%%%%%%%%%%%%%%%%

%%%%%%%%%%%%%%%%%%%%%%%% DISCLAIMER %%%%%%%%%%%%%%%%%%%%%%%%%%%%%%%%%%
% 
% In spite of the effort to accommodate this template and the nuthesis
% class to the official requirements of the university to write a 
% Ph.D. dissertation, it is not possible to guarantee that it will 
% always work, and the author of the dissertation remains responsible
% for checking that such requirements are actually fulfilled by 
% his/her final work.
%
%%%%%%%%%%%%%%%%%%%%%%%%%%%%%%%%%%%%%%%%%%%%%%%%%%%%%%%%%%%%%%%%%%%%%%
\RequirePackage{rotating}
% The nuthesis class is based on % amsbook.cls.
\documentclass[12pt,reqno]{nuthesis}

% \onehalfspacing
\usepackage{rotating}
\usepackage{graphicx}
\usepackage{placeins}
\usepackage{siunitx}
\usepackage{cite}

\usepackage{textcomp}
\usepackage{amsmath}
\usepackage{xspace}


%%% ====================================================================
% from cms-tdr.cls 
% PENNAMES
\usepackage[low-sup]{subdepth}  % corrects normal super/sub after subdepth: https://tex.stackexchange.com/questions/352936/wrong-superscript-placement-with-hepparticles
% \usepackage[italic,italicGreek]{heppennames2}
\usepackage[pazoGreek]{heppennames2}
\usepackage{ptdr-definitions}
%%% ====================================================================


\usepackage{fullpage,rotating,booktabs,dcolumn}
\usepackage{natbib}
\usepackage[com pat = 1.1.0]{tikz-feynman}
\usepackage{multirow}
\usepackage{slashed}
\usepackage{lscape}
\usepackage[ruled,vlined]{algorithm2e}
\usepackage{tabularx}

\usepackage{geometry}
\geometry{letterpaper, margin=1in, tmargin=1.3in, headsep=24 pt}

% Page numbers should conform to margin requirements and be placed at least 1” from the top and 
% right sides of the page, as appears in this document. A tutorial for page number placement appears later in this document
\hypersetup{hidelinks}






%%%%%%%%%%%%%%%%%%%%%%%%%%%%%%%%%%%
% DATA OF AUTHOR AND DISSERTATION %
%%%%%%%%%%%%%%%%%%%%%%%%%%%%%%%%%%%
\author{Ziheng Chen}
\title{Measurements of the W Boson Branching Fractions in Proton-Proton collisions at 13\TeV Center-of-mass Energy with the CMS Experiment}
\degree{DOCTOR OF PHILOSOPHY}  
\field{Physics}
\graduationmonth{April}
\graduationyear{2021}

% \includeonly{chapters/Introduction/chapterMain}	


\begin{document}
    
    
    %%%%%%%%%%%%%%%%%%%%%%
    % Some initial stuff %
    %%%%%%%%%%%%%%%%%%%%%%


    % Preliminary pages start here.
    \frontmatter

    % Produces the title page.
    \maketitle

    % Creates the copyright page.
    \copyrightpage

    % Abstract page.
    \abstract
    The leptonic and inclusive hadronic decay branching fractions of the W boson are studied using 35.9\fbinv of proton-proton collision data collected at $\sqrt{s}=13\TeV$ during the 2016 run of the CMS experiment. Events characterized by the production of two W boson pairs, or of a W boson accompanied by jets, are selected. Multiple event categories sensitive to the signal processes are defined based on the presence of energetic isolated charged leptons, the number of hadronic jets, and the number of b tagged jets.  A maximum likelihood estimate of the W branching fractions is carried out by fitting to the data in each event category simultaneously. The branching fractions of the W boson decaying into electron, muon, and tau lepton final states amount to $(10.83  \pm 0.10)\%$, $(10.94  \pm 0.08)\%$, and $(10.77 \pm 0.21)\%$, respectively, supporting the hypothesis of lepton universality for the weak interaction. Under the assumption of lepton universality, the inclusive leptonic and hadronic decay branching fractions are found to be $(10.89 \pm 0.08)\%$ and $(67.32 \pm 0.23)\%$, respectively. From these results, three standard model quantities are subsequently derived: the sum square of elements in the first two rows of the Cabibbo--Kobayashi--Maskawa (CKM) matrix  $\sum{\abs{\mathrm{V_{ij}}}^{2}} = 1.991 \pm 0.019$, the CKM element $\abs{\mathrm{V_{cs}}} = 0.969 \pm 0.011$, and the strong coupling constant at the W mass scale, $\alpS(m_\mathrm{W}) = 0.094 \pm 0.033$.


    % Acknowledgements page.
    \acknowledgements

    First and foremost, I would like to express my deepest and most sincere gratitude to my supervisor \textbf{Mayda Velasco}, Professor in the Physics Department, Northwestern University, who provides me with excellent guidance and training to explore and research the wonderful world of particle physics. 

    Next, I would like to thank \textbf{Nathaniel Odell}, a postdoc fellow in the Physics Department, Northwestern University, and the colleague I collaborate the most closely with on this thesis, for providing me enormous support and advice during the studies in this thesis. Without him, this thesis would not be completed. 

    Also, many people within the CMS have given me beneficial guidance. I could not appreciate more the ARC committee members and the SMP colleagues for sharing their constructive suggestions and insights about our analysis work. Besides, I also want to acknowledge my HGCAL DPG colleagues and BRIL BCM1F colleagues for providing me with marvelous opportunities and experiences on the detector research and developing.

    Finally, I would like to thank my family members, including my wife \textbf{Shuyuan Hu}, who is currently studying for her Ph.D. in Heidelberg, Germany, as well as my parents in China. Despite the long distance, the love, joy, happiness, and support they brought to me during my doctoral journey is one of the most invaluable and beautiful things that ever happen to me.



    % needed for the hyperlinks to work correctly
    \clearpage\phantomsection 
    % Table of Contents
    \setcounter{tocdepth}{2}
    \begin{spacing}{1.5}
        \tableofcontents	
    \end{spacing}

    \clearpage\phantomsection 
    \listoftables

    \clearpage\phantomsection 
    \listoffigures



    %%%%%%%%%%%%%%%%%%%%%%%%%%%
    % Actual text starts here %
    %%%%%%%%%%%%%%%%%%%%%%%%%%%
    \mainmatter
    \chapter{Introduction}
\label{sec:introduction}

\section{Overview}
\label{sec:introduction:overview}

In the standard model (SM), the interaction between the weak boson and the leptons is expressed by the Lagrangian term $\bar{\chi}_L \gamma^\mu \big( g T_a W^a_\mu +g'Y B_\mu \big) \chi_L + \bar{\psi}_R \gamma^\mu (g' Y B_\mu) \psi_R $, where the coupling constant $g$ is the same for all three lepton generations. Namely,
\begin{equation*}
	g_e = g_\mu = g_\tau \equiv g.
\end{equation*}

\noindent The lepton universality (LU) test in the \PW boson's interaction is an important aspect to test the SM and probe new physics. This has been performed by many particle physics experiments, approaches of which primarily include using the decay of \PW boson produced in the colliders, using the weak decay of mesons, and using the weak decay of leptons. Section~\ref{sec:relatedWorks:lu} in this introduction gives a review of these activities. Among them, the most related experiments to this thesis are performed with the decay of the \PW bosons produced in the high-energy colliders.


% SPS+Tevatron
The earliest LU test of this kind can be traced back to the SPS experiments, UA1~\cite{Albajar:1988ka} and UA2~\cite{appel1986measurement, Alitti:1991eh, Alitti:1992hv}, at CERN in the 1980s. The measurement was then improved by the Tevatron experiments, CDF~\cite{Abazov:2003sv, Abe:1990sd, Abe:1992ys, Abe:1991fb} and D0~\cite{ Abbott:1999tt, Abachi:1995xc, Abbott:1999pk}, at Fermilab during Tevatron's run-1 from 1985 to 1995. Both SPS and Tevatron produced \PW bosons from the $p\bar{p}$ collisions. One of the common features of these experiments was that the measured quantity was the product of the inclusive \PW cross-section and the \PW leptonic branching fraction $\sigma_W \times B^W_\ell$ for the three lepton generations. The LU test was performed by taking the ratio of the measured $\sigma_W \times B^W_\ell$ between two different lepton generations. For \PW coupling to the electron and tau, the combined SPS and Tevatron result~\cite{Abbott:1999pk} showed
\begin{equation*}
    R_{\tau/e} = g^W_\tau / g^W_e = 0.988 \pm 0.025 \qquad \text{(SPS+Tevatron)}.
\end{equation*}
\noindent Overall, the SPS and Tevatron results hinted no clear sign of LU violation related to the \PW boson. 



% LEP
The the most precise measurement of the three \PW leptonic branching fractions $B^W_e, B^W_\mu, B^W_\tau$ comes from the four LEP experiments, ALEPH~\cite{Heister:2004wr}, DELPHI~\cite{Abdallah:2003zm}, OPAL~\cite{Abbiendi:2007rs}, L3~\cite{Achard:2004zw} during the LEP's run-2 (1995-2000) which produced $WW$ pairs from the electron-positron collisions. By now, the four LEP experiments are still the only \PW leptonic branching fraction measurements included in the PDG. The experimental precisions of $B^W_e, B^W_\mu, B^W_\tau$ have not been improved since LEP. The combined LEP result~\cite{Schael:2013ita} gave $B^W_e = 10.71(16)\%$, $B^W_\mu = 10.63(15)\%$, $B^W_\tau = 11.38(21)\%$. Assuming partial universality for electron and muon, ratio between the tauonic and the average of electronic and muonic branching fraction was reported~\cite{Schael:2013ita} as
\begin{equation*}
    R_{\tau/(e,\mu)} = \frac{2\times B^W_\tau }{B^W_e +  B^W_\mu} = 1.066 \pm 0.025 \qquad \text{(LEP)}.
\end{equation*}

\noindent In comparison, the SM predictions~\cite{Denner:1991kt,Rtau,dEnterria:2016rbf} for $R_{\tau/(e,\mu)}$ is 0.99912,  
%is $R_{\tau/e} \equiv B^W_\tau/B^W_e = 0.99912$ and $R_{\mu/e} \equiv B^W_\mu/B^W_e = 1.00000$ 
taking into account the electromagnetic radiation correction and effect of the lepton mass in the \PW decay phase space. LEP's $R_{\tau/(e,\mu)}$ shows a 2.6 standard deviation from the SM prediction. This moderate deviation motivates the measurement of the branching fractions more precisely.





% LHC
The opportunities arrive in the LHC era. During the LHC run-2, proton and proton collision at $\sqrt{s}=13~\TeV$ allows an unprecedentedly large cross-section for the top quark pair production and produces a large number of \ttbar events. Since the top quark decays almost exclusively into the \PW boson and the $b$ quark, with the help of the improved b-tagging techniques \cite{Chatrchyan:2012jua, Sirunyan:2017ezt, Bols:2020bkb}, it is possible to select a large and pure sample of \ttbar events with two \PW bosons, which can then be used to study \PW boson decays. A recent measurement by the ATLAS collaboration~\cite{Aad:2020ayz} has exploited such a strategy to measure the $R_{\tau/\mu}=B^W_\tau/B^W_\mu$ ratio by fitting the impact parameter distribution of the final state muons. The resulting value is $R_{\tau/\mu} = 0.992 \pm 0.013$, suggesting that lepton universality is preferred. 


% CMS
In the CMS, there has been a significant improvement on the identification of the hadronic tau leptons \cite{Chatrchyan:2012zz, Khachatryan:2015dfa, Sirunyan:2018pgf}, which further opens the door to efficiently select \ttbar events with the $\tau_h$ final state in addition to electron and muon final state, and therefore to measure all the three leptonic branching fractions simultaneously. The analysis in this thesis is performed with the CMS experiment under this context ---

% \begin{itemize}
%     \item Motivations:
%         \begin{itemize}
%             \item Measurement of $B^W_e, B^W_\mu ,B^W_\tau$ has not been improved for more than a decay since LEP.
%             \item $2.6~\sigma$ deviation between LEP's $R_{\tau/(e,\mu)}$ and the SM requires additional measurement.
%         \end{itemize}
    
%     \item Opportunity:
%         \begin{itemize}
%             \item $\sqrt{s}=13~\TeV$ p-p collision at the LHC produces an increased number of \ttbar events which give $WW$ pairs.
%             \item Improved b-tagging allows to select \ttbar events with a high purity.
%             \item Improved $\tau_h$ identification allows more efficient selection involving $\tau_h$ final state.
%         \end{itemize}
% \end{itemize}


    \noindent \textbf{Motivations:}
        \begin{itemize}
            \item $B^W_e, B^W_\mu ,B^W_\tau$ measurement has not been improved for more than a decay since LEP;
            \item $2.6~\sigma$ deviation between LEP's $R_{\tau/(e,\mu)}$ and the SM requires additional measurement.
        \end{itemize}
    
    \noindent \textbf{Opportunities:}
        \begin{itemize}
            \item $\sqrt{s}=13~\TeV$ p-p collision at the LHC produces an increased number of \ttbar events which give $WW$ pairs;
            \item the improved b-tagging allows to select \ttbar events with a high purity;
            \item the improved $\tau_h$ identification enables to efficiently select \ttbar events with $\tau_h$ final state in addition to electron and muon final state.
        \end{itemize}

    
    


\noindent We have performed a simultaneous measurement of $B^W_e, B^W_\mu ,B^W_\tau$ and the inclusive hadronic branching $B^W_h$. $35.9~\fbinv$ of data collected by the CMS at $\sqrt{s} =13~\TeV$ during the 2016 run of the LHC are analyzed by selecting events consistent with the decay of \ttbar and $tW$. The final states resulting from either one or both of the \PW bosons decaying leptonically are considered.  To collect these events, single electron and single muon triggers are used, thus requiring that the final state must contain at least one prompt electron or muon. Based on the final state objects, data sample are split into a few channels including $e e, \mu\mu,  e\mu, e \tau_h, \mu\tau_h, e\text{+jets}, \mu\text{+jets}$. The estimation of the W branching fractions is carried out based on two separately-developed approaches: 

\begin{table}[!htbp]
    \centering
    \setlength{\tabcolsep}{0.5 em}
    \renewcommand{\arraystretch}{2}
    \begin{tabular}{ >{\centering}m{0.22\textwidth}|m{0.7\textwidth} }
        \hline
        \textbf{Shape analysis}      & template fit the \pt distribution of the sensitive leptons in different channels. \\ 
        \hline
        \textbf{Counting analysis}   & construct ratios of yields for channels with the same trigger and analytically solve three leptonic branching fractions from a set of quadratic equations. \\ 
        \hline
    \end{tabular}
\end{table}


The shape analysis is designed to push forward the precision beyond LEP. It makes the most advantage of the shape information of the lepton \pt spectrum to discriminate the electron and muon coming from $W\to e/\mu$ and from $W\to\tau \to e/\mu$. Comparing with the lepton impact parameter, \pt can be calibrated more conveniently using the energy correction provided the CMS physics object group (POG), and systematics uncertainty associated with the \pt is also expected to be smaller. To achieve better precision, shape analysis includes extra orthogonal regions besides the \ttbar enriched regions, thus constraining some of the most offending systematics such as those related to $\tau_h$ reconstruction. 

The counting analysis is designed to cross-check the shape analysis, with more emphasis on the robustness over precision. It eliminates the shape information and uses only the \ttbar concentrated regions. By constructing the ratios of yields for channels with the same trigger, it has the benefit of canceling some systematics uncertainties related to \ttbar cross section, trigger efficiency and luminosity, and being robust with the lepton energy calibration. However, its precision is highly bottlenecked by the tau identification systematics, ultimately being less sensitive than the shape analysis. 

This thesis describes both the approaches and their results. For $B^W_\tau$, the shape analysis achieves an absolute uncertainty of $2.1\%$, while the uncertainty of counting analysis about $6.7\%$. The final result of shape and counting analysis agree with each other within 1 sigma. In the CMS publication, the more precise result from the shape analysis is reported as the official CMS result to the physics community, and to be compared to and combined with the LEP, while briefly mentioning the cross-check from the counting analysis. 

With the simultaneously measured $B^W_e, B^W_\mu ,B^W_\tau$ and their correlation, the pair-wised ratios between two branching fractions are calculated to testing the SM LU prediction. Furthermore, the leptonic and inclusive hadronic branching fraction under the lepton universality assumption are also estimated by repeating the shape analysis with the same parameter for three leptons. From the measured $B^W_\ell$ or $B^W_h$, some SM quantities can be derived, as listed in Table~\ref{tab:introduction:derivedQuantity}. With the SM CKM unitarity, the strong coupling constant $\alpha_s(m_W)$ can be calculated; alternatively, using the latest experimental measurement of $\alpha_s(m_W)$, the square sum of the six CKM elements in the first two rows can be calculated and compared with the unitarity. Among the six elements in the square sum $\sum_{uc,dsb} |V_{ij}|^2 $, $V_{cs}$ has the least experimental precision, currently at percent level. So we can take a step further to determine $V_{cs}$ using the experimental values of five other CKM elements. The mathematics and physics related to these derivations are covered in Section~\ref{sec:relatedWorks:vcs}.


% The result of CMS simultaneous measurement of the three W leptonic branching fractions are


% \begin{equation*}
%     B^W_e=10.83(10)\%, ~ B^W_\mu=10.94(08)\%, ~ B^W_\tau=10.77(21)\% \qquad \text{(CMS)}.
% \end{equation*}



% \begin{figure}[!htbp]
%     \centering
%     % \includegraphics[width=0.5\textwidth]{chapters/Introduction/figures/cms1d.png}
%     \includegraphics[width=0.99\textwidth]{chapters/Introduction/figures/image.png}
%     \caption{Our simultaneous measurement of the three W leptonic branching fraction with the CMS experiment. }
%     \label{fig:introduction:cmsMoneyPlot}
% \end{figure}


% \noindent Figure~\ref{fig:introduction:cmsMoneyPlot} illustrates the CMS result on the 2D plane of pairs of leptonic branching fraction, together with the comparison and combine with the LEP result. Our result indicates no clear LU violation. If assuming LU, we get $B^W_l=10.89(08)\%, ~ B^W_h=67.32(23)\%$. Comparing with the LEP result, the CMS precision is about 1.6x (2.0x) better on the electronic (muonic) \PW branching fraction, while achieving the same precision on the tauonic one. Combining the CMS with LEP, the world average of $B^W_e,B^W_\mu,B^W_\tau$ measurement is significantly improved, which is the first time for more than a decade since LEP. 


% Based on the measured inclusive hadronic branching fraction $B^W_h$ under the LU hypothesis, some other related SM quantities can be derived. The thory related to the derivation is shown in Section. The full result of the derived quantity is presented in Section. Assuming 

\begin{table}[!h]
    \setlength{\tabcolsep}{0.5em}
    \renewcommand{\arraystretch}{1.5}
    \centering
    \caption{SM quantities can be derived from the measured $B^W_\ell$ or $B^W_h$. }
    \resizebox{0.92\textwidth}{!}{
    \begin{tabular}{ccc}
        % \hline
        Assumption &  & Derived quantity \\
        \hline
        CKM Unitarity $\sum_{uc,dsb} |V_{ij}|^2 = 2$ & $\Longrightarrow$ & $\alpha_s(m_W)$  \\%= 0.094\pm0.033$
        \hline
        PDG $\alpha_s(m_W) = 1.1120\pm0.010$~\cite{pdg2020}     & $\Longrightarrow$ & $\sum_{uc,dsb} |V_{ij}|^2 $ \\ %= 1.985\pm0.021
        \hline
        PDG $\alpha_s(m_W) = 1.1120\pm0.010$~\cite{pdg2020}    & \multirow{2}{*}{$\Longrightarrow$} & \multirow{2}{*}{$V_{cs}$}  \\ %= 0.968\pm 0.011
        PDG $|V_{ud}|^2 + |V_{us}|^2 +|V_{ub}|^2 +|V_{cd}|^2 +|V_{cb}|^2=1.0490(18)$~\cite{pdg2020} & &\\
        \hline
    \end{tabular}}
    \label{tab:introduction:derivedQuantity}
\end{table}




For the outlook of this measurement in the LHC run~3 and High Luminosity LHC (HL-LHC)~\cite{Apollinari:2284929} runs, the further precision improvement can be contributed from the advancement of the $\tau_h$ identification, as well as the improvement of the impact parameter resolution if it is included as additional discriminating observables in the future. In the era of HL-LHC, a few exciting upgrades of the CMS detector will have complished after the phase-2 upgrade~\cite{CMSCollaboration:2015zni}. A new tracker~\cite{Klein:2017nke} will improve the resolution of the impact parameter. A new endcap calorimeter, the high granularity calorimeter (HGCAL)~\cite{Collaboration:2293646}, is expected to improve the $\tau_h$ identification by allowing novel deep learning algorithms based on its high-resolution jet image. This thesis also describes a new clustering algorithm developed for HGCAL reconstruction and the corresponding high-performance computing using GPUs.



The rest of this introduction section covers brief reviews of the LU tests and $V_{cs}$ measurements. 
Section 2 and 3 describe the related SM/BSM physics and the key aspects of the CMS experiment, respectively.
Section 4 presents the method and the result of the measurement of the \PW branching fractions with the CMS experiment, followed by the related supplement studies in Section 5.
Section 6 shows the clustering algorithm developed for the HGCAL reconstruction and its high performance computing using GPUs.
















% A fundamental test of lepton universality in the electroweak sector is to measure the branching fractions of the W boson.  The most precise measurements of these quantities were measured at LEP~\cite{Schael:2013ita} and the results of all four experiments are combined to form the world averages~\cite{Patrignani:2016xqp}.  The values measured at each experiment and their combined values are shown in figure~\ref{fig:introduction:wbr}.

% \begin{figure}[ht]
%     \centering
%     \includegraphics[width=0.6\textwidth]{chapters/Introduction/figures/wdecay.png}
%     \caption{Measured leptonic and inclusive hadronic W branching fractions from LEP.}
%     \label{fig:introduction:wbr}
% \end{figure}


% These measurements show a $\sim 2.6~\sigma$ difference between the tau branching fraction in the case that the fit is carried out assuming lepton universality and the case that each of the branching fractions can vary independently.  This motivates measuring the branching fractions more precisely.  Because of the large number of \ttbar events produced at the LHC, it is possible to select a high statistics, high purity dataset containing two W boson decays which can be used to precisely measure the W branching fractions.  


% In this analysis, $35.9\fbinv$ of data collected by CMS at $\sqrt{s} =13$ TeV during the 2016 run of the LHC are analyzed by selecting events consistent with the decay of pair-produced top quarks or W bosons are selected.  Final states resulting from either one or both of the W bosons decaying leptonically are considered.  To collect these events, single lepton triggers are used thus requiring that the final state must contain at least one prompt electron or muon. 

% Estimation of the values of the W branching fractions is carried out based on two approaches: 

% \begin{itemize}
%     \item maximum likelihood estimation (MLE), binning the data based on the number of b-tags and channel-dependent kinematic information. This will be referred to as the \emph{shape analysis}.
%     \item a semi-analytic approach that constructs ratios of yields in the various channels in a manner that mimics a direct construction of the branching fraction.  This approach does not use kinematic shape information, but does divide channels based on the number of b tags. This will be referred to as the \emph{counting analysis}.
% \end{itemize}




% 
\section{Test of Lepton Universality in the Weak Sector}
\label{sec:relatedWorks:lu}

In the SM, the lepton universality (LU) of the interaction with the weak gauge boson is expressed by the Lagrangian term $\bar{\chi}_L \gamma^\mu \big( i \partial_\mu -g \frac{\tau_a}{2} W^a_\mu -g'\frac{Y}{2} B_\mu \big) \chi_L $ in Equation~\ref{eqn:relatedWorks:qft:gws:lagragian}, where the coupling constant $g$ is the same for all three lepton generations. Namely,
\begin{equation}
	g_e^W = g_\mu^W = g_\tau^W \equiv g.
\end{equation}

\noindent The precision measurement of the weak couplings to the three lepton generations provides an excellent test of the SM. Any deviation from the lepton universality could hint new physics beyond the SM. Nowadays, the precision test of LU in the weak sector has been performed in a wide variety of particle physics experiments, including but not limited to experiments at the $p\bar{p}, e^+ e^-, pp$ colliders, meson factories, and tau factories. The most related experiments to this thesis are performed in the colliders with the decay of the on-shell \PW bosons. This section summarizes the LU tests with on-shell \PW bosons in the colliders, including SPS, Tevatron, LEP, and LHC, followed by a brief discussion about the tests in the favor changing charge current (FCCC) decays of the mesons and taus. Among all these tests, there are two most well-known anomalies, the LEP results and the semileptonic decay of B mesons, which are also presented in this section. 


\subsection{Test with on-shell W Decay}
\label{sec:relatedWorks:lu:W}


The on-shell \PW boson can be produced in the colliders. Comparing the leptonic decay of on-shell \PW boson is the most direct LU tests of $g^W_l$. By now, the LU test with the on-shell \PW boson can be divided into three stages:

\begin{itemize}
    \item CERN SPS (1981-1991) and Fermilab Tevatron (Run-I 1985-1995) using \mbox{$p \bar{p} \to W \, X$}.
    \item CERN LEP (Run-II 1995-2000) using $e^+e^- \to W^+  W^-$.
    \item CERN LHC since 2011. W is produced via many processes. The major processes for the LU test include $pp \to W +X$ (Run-I) and $pp \to t \bar{t} \to Wb+Wb$ (Run-II)
\end{itemize}

\noindent The earliest test can be traced back to the SPS experiments at CERN in the 1980s. Then the measurement was improved by the Tevatron experiments at Fermilab during its Run-I 1985-1995. Both the colliders produced \PW bosons from the $p\bar{p}$ collisions. One of the common features of the SPS and Tevatron experiments was that the measured quantity was $\sigma_W \times B(W\to l\nu)$ instead of the three individual $B(W\to l\nu)$. This is because $W$ boson was just discovered, and its production cross-section $\sigma_W$ was not known well enough to be disentangled. No clear violation was observed in the SPS and Tevatron experiments. The second era was marked by the LEP experiments, which gave the latest and most precise test before this thesis. One of the key features of the LEP result was that the three leptonic branching fractions were measured simultaneously, together with the corresponding correlations. The combined result of LEP experiments showed a 2.6 $\sigma$ deviation from SM lepton universality. This observation is one of the major motivations of pushing forward the tests in the LHC era, such as this analysis.


\subsubsection{SPS and Tevatron Experiments}

Both SPS and Tevatron collide protons and anti-protons. SPS operated at CERN from 1981 to 1991 at a center-of-mass energy of 0.546~\TeV and 0. 630~\TeV. The SM electroweak bosons, W and Z, were first discovered in the SPS in 1983. In 1985, Tevatron at Fermilab began operations at a higher center-of-mass energy at 1.8~\TeV, which was later upgraded to 1.96~\TeV in its second run since 2001. Tevatron was in service for more than 20 years until 2010 to give ways to the LHC. Based on the discovery and studies of weak bosons on the SPS, Tevatron experiments continued on more precise measurements of the properties of the W and Z bosons. Here lists the key results from the SPS and Tevatron experiments related to the lepton universality test.


% SPS Tevatron result plot
\begin{figure}[ht]
    \centering
    \includegraphics[width=0.5\textwidth]{chapters/RelatedWorks/sectionLU/figures/spsTevatron.png}
    \caption{ $g^W_\tau / g^W_e$ measured in the SPS and Tevatron experiments \cite{Abbott:1999pk}. In all the four experiments, the lepton universality between electron and tau, concerning the weak coupling to \PW boson, is observed and is consistent with the SM prediction $g^W_\tau / g^W_e=1$ within one experimental uncertainty. The combine was done by D0 collaboration \cite{Abbott:1999pk}.}
    \label{fig:relatedWorks:lu:W:spsTevatronCombinedRatio}
\end{figure}

The UA1, UA2 experiment at the CERN SPS and CDF, D0 experiment at the Fermilab Tevatron measured the $p\bar{p} \to W + X$ production cross-section in the three leptonic channels, the ratios of which provided a test of lepton universality concerning the couplings to the on-shell \PW boson. Figure~\ref{sec:relatedWorks:lu:W:spsTevatron} shows the measurement of $\sigma_W \times B(W\to l \nu)$ in the SPS and Tevatron experiments. Figure~\ref{fig:relatedWorks:lu:W:spsTevatronCombinedRatio} from \cite{Abbott:1999pk} summarizes the results of $g^W_\tau / g^W_e$ measurements in the SPS and Tevatron experiments. All four measurements confirmed consistency with the SM lepton universality within one experimental uncertainty. The combined average is calculated by D0 in \cite{Abbott:1999pk}, the last published result among the four. In the combine, the systematics in different experiments are assumed to be uncorrelated. The combined result is also consistent with the lepton universality with a precision level at 2.5\%, which can be translated into a precision level of about  5\% on the W branching ratio $B(W\to \tau) / B(W\to e)$. 



% and had two major experiments UA1 and UA2, which discovered the electroweak bosons W and Z predicted by the GWS EW theory in 1983. The collision energy of SPS was later surpassed by Tevatron at Fermilab in 1986. Since then Tevatron operated more than 20 years until 2010 to give ways to LHC. The two major experiments at Tevatron are D0 and CDF which measured the properties of W and Z boson with improved precision. With respect to testing the lepton universality in the W sector, the SPS and Tevatron experiments share many similarities. They did not directly measured the three leptonic decay branching fractions of W, mainly because the W cross section was not measured precisely at their experimental period. Instead, they measured the cross section of W production in the three leptonic channel. Namely, the product of the W+X produciton cross section and three leptonic W decay branching fractions. 
% UA1 and UA2 experiments in the SPS performed the early measurement of W production cross section in the three leptonic decay channels of \PW boson. 

UA1 was a general-purpose particle detector at the CERN SPS, consisting of the inner tracker, ECAL HCAL, and a muon system, sequentially from the inside to the outside.  It took 0.546~\TeV and 0.63~\TeV data during 1982-1983 and 1984-1985, respectively. Its result of \PW boson studies is listed in \cite{Albajar:1988ka}. $W \to e \nu$ events were selected based on single-electron plus met selection. The QCD and $W\to \tau_e \nu$ background were estimated with data-driven and MC approach, respectively. In total, 59 and 240 $W \to e \nu$ events were selected from the 0.546~\TeV and 0.63~\TeV collision, respectively.  $W \to \mu \nu$ events were selected based on single muon plus met selection. The background involving muons from tau and meson decays was estimated by proper simulations. In total, 10 and 57 $W\to \mu\nu$ events were selected from the 0.546~\TeV and 0.63~\TeV data.  $W\to \tau \nu$ were selected with a single hadronic tau plus met selection. The hadronic taus were identified by highly collimated narrow jets with low charged-track multiplicity.  A $\tau$-likelihood was calculated for each jet candidate based on the its shape and charged tracks. In total, 32 events were selected from the combined 0.546~\TeV and 0.63~\TeV dataset. Based on the yields, UA1 reported the $\sigma_W \times Br(W\to l\nu) $ for the three leptons $l=e,\mu,\tau$ at 0.546~\TeV and 0.63~\TeV center-of-mass energy. Pair-wise ratios of  $\sigma_W \times Br(W\to l\nu) $ were calculated to test the lepton universality. Table~\ref{tab:relatedWorks:lu:W:sps} lists the $\sigma_W \times Br(W\to l\nu) $ and ratios from UA1.



UA2 was a particle detector at the CERN SPS, consisting of a tracking system surrounded by a calorimetry system with EM and hadronic compartments. Unlike UA1, UA2 was not a multipurpose detector; its focus was on the calorimeters and did not have a muon detector. Therefore, lepton universality test on UA2 mainly involved the $W \to e\nu$ and $W \to \tau \nu$. \cite{appel1986measurement} summarized the $\sigma_W \times Br(W\to e \nu) $ measurements from the UA2 using 0.546 TeV and 0.63 TeV data collected during 1982-1983 and 1984-1985. The measurement was based on single-electron plus met trigger. This  $\sigma_W \times Br(W\to e \nu) $ result is shown in Table~\ref{tab:relatedWorks:lu:W:sps}. After the UA2 upgrade during 1985-1987,  the tau channel was added and a test of the lepton universality between $\tau$ and $e$ was performed \cite{Alitti:1991eh, Alitti:1992hv}, using the 0.63 TeV data collected during 1988-1990. The hadronic taus were reconstructed from jet candidates with selections on relative hadronic energy and the lateral energy profile. The data was triggered with the met trigger in 1988-1989 and hadronic tau trigger in 1990. \cite{Alitti:1991eh} analyzed the 1988-1989 data, while \cite{Alitti:1992hv} combined the 1988-1989 data with 1990 data. The result \cite{Alitti:1992hv} for the ratio between tauonic and electronic W decays is shown in the Table~\ref{tab:relatedWorks:lu:W:sps}. 

\begin{figure}[ht]
    \centering
    \includegraphics[width=0.35\textwidth]{chapters/RelatedWorks/sectionLU/figures/SPS.png}
    \includegraphics[width=0.6\textwidth]{chapters/RelatedWorks/sectionLU/figures/tevatron.png}
    \caption{Measurement of $\sigma_W \times B(W\to l \nu)$ in the SPS \cite{Albajar:1988ka} and Tevatron experiments. [RIGHT plot needs reproduce] }
    \label{sec:relatedWorks:lu:W:spsTevatron}
\end{figure}

% SPS result table
\begin{table}[ht]
    \setlength{\tabcolsep}{ 0.5 em}
    \renewcommand{\arraystretch}{1.5}
    \centering
    \caption{The measurement of $\sigma_W \times B(W\to l \nu)$ and the ratios between leptonic channels in the UA1 and UA2 experiment at the CERN SPS. }
    \resizebox{\textwidth}{!}{
    \begin{tabular}{ |c|l l| } 
         
         % UA1 result
         \hline
         \multicolumn{3}{|c|}{UA1 \cite{Albajar:1988ka} }  \\
         \hline
         & $p\bar{p}$ at $\sqrt{s}=0.546$ TeV &  $p\bar{p}$ at $\sqrt{s}=0.630$ TeV \\
         \hline
         $\sigma_W \times Br(W\to e    \nu)$  [nb]  & 0.55 $\pm$ 0.08 (stat) $\pm$ 0.09 (syst) & 0.63 $\pm$ 0.06 (stat) $\pm$ 0.10 (syst) \\ 
         $\sigma_W \times Br(W\to \mu  \nu)$  [nb]  & 0.56 $\pm$ 0.18 (stat) $\pm$ 0.12 (syst) & 0.63 $\pm$ 0.08 (stat) $\pm$ 0.11 (syst) \\ 
         $\sigma_W \times Br(W\to \tau \nu)$  [nb]  & \multicolumn{2}{c|}{ 0.63 $\pm$ 0.13 (stat) $\pm$ 0.12 (syst) }  \\ 
         \hline
         $Br(W\to \mu  \nu)/ Br(W\to e \nu)$  & \multicolumn{2}{c|}{1.00  $\pm$ 0.14 (stat) $\pm$ 0.08 (syst) } \\
         $Br(W\to \tau \nu)/ Br(W\to e \nu)$  & \multicolumn{2}{c|}{1.02  $\pm$ 0.20 (stat) $\pm$ 0.10 (syst) } \\
         
         \hline
         \multicolumn{2}{c}{} \\
         
         % UA2 result
         \hline
         \multicolumn{3}{|c|}{UA2}  \\
         \hline
         & $p\bar{p}$ at $\sqrt{s}=0.546$ TeV &  $p\bar{p}$ at $\sqrt{s}=0.630$ TeV \\
         \hline
         $\sigma_W \times Br(W\to e    \nu)$  [nb] \cite{appel1986measurement} & 0.50 $\pm$ 0.09 (stat) $\pm$ 0.05 (syst) & 0.53 $\pm$ 0.06 (stat) $\pm$ 0.05 (syst) \\ 
         % This is UA2 result reported in the UA1 summary
        %  $\sigma_W \times Br(W\to e    \nu)$  [nb] \cite{Albajar:1988ka} & 0.61 $\pm$ 0.10 (stat) $\pm$ 0.07 (syst) & 0.57 $\pm$ 0.04 (stat) $\pm$ 0.07 (syst) \\ 
         \hline
         $Br(W\to \tau \nu)/ Br(W\to e \nu)$ \cite{Alitti:1992hv} & - & 1.04  $\pm$ 0.08 (stat) $\pm$ 0.08 (syst) \\
         
         \hline
    \end{tabular}}
    \label{tab:relatedWorks:lu:W:sps}
\end{table}




% Tevatron result table
\begin{table}[ht]
    \setlength{\tabcolsep}{0.5 em}
    \renewcommand{\arraystretch}{1.5}
    \centering
    \caption{The measurement of $\sigma_W \times B(W\to l \nu)$ and the ratios between leptonic channels in the CDF and D0 experiment at the Fermilab Tevatron.}
    \resizebox{\textwidth}{!}{
    \begin{tabular}{ |c|l| } 
         
         % D0 result
         \hline
         \multicolumn{2}{|c|}{D0 with $p\bar{p}$ at $\sqrt{s}=1.8$ TeV} \\
         \hline
         $\sigma_W \times Br(W\to e    \nu)$  [nb] \cite{Abbott:1999tt} & 2.31 $\pm$ 0.01 (stat) $\pm$ 0.05 (syst) $\pm$ 0.10 (lum) \\ 
         $\sigma_W \times Br(W\to \mu  \nu)$  [nb] \cite{Abachi:1995xc} & 2.09 $\pm$ 0.23 (stat) $\pm$ 0.11 (syst) \\ 
         $\sigma_W \times Br(W\to \tau \nu)$  [nb] \cite{Abbott:1999pk} & 2.22 $\pm$ 0.09 (stat) $\pm$ 0.10 (syst) $\pm$ 0.10 (lum)  \\ 
         \hline
         $Br(W\to \mu  \nu)/ Br(W\to e \nu)$ \cite{Abachi:1995xc} & 0.89  $\pm$ 0.10 \\
         $Br(W\to \tau \nu)/ Br(W\to e \nu)$ \cite{Abbott:1999pk} & 0.961 $\pm$ 0.061 \\
         
         \hline
         \multicolumn{2}{c}{}  \\
         
         
         %  CDF result
         \hline
         \multicolumn{2}{|c|}{CDF with $p\bar{p}$ at $\sqrt{s}=1.8$ TeV} \\
         \hline
         $\sigma_W \times Br(W\to e    \nu)$  [nb] \cite{Abe:1990sd}    & 2.19 $\pm$ 0.04 (stat) $\pm$ 0.21 (syst) \\ 
         $\sigma_W \times Br(W\to \mu  \nu)$  [nb] \cite{Abe:1992ys}    & 2.21 $\pm$ 0.07 (stat) $\pm$ 0.21 (syst) \\ 
         $\sigma_W \times Br(W\to \tau \nu)$  [nb] \cite{Abe:1991fb}    & 2.05 $\pm$ 0.27 \\ 
         \hline
         $Br(W\to \mu  \nu)/ Br(W\to e \nu)$ \cite{Abe:1992ys} & 1.02  $\pm$ 0.08 \\
         $Br(W\to \tau \nu)/ Br(W\to e \nu)$ \cite{Abe:1991fb} & 0.94  $\pm$ 0.14 \\

         \hline
    \end{tabular}}
    \label{tab:relatedWorks:lu:W:tevatron}
\end{table}



CDF was an azimuthally and forward-backward symmetric general-purpose detector at the Fermilab Tevatron. It was consist of several subdetector layers, including a silicon tracker, gas chamber as the central outer tracker, solenoid magnet, ECAL/HCAL, and muon detector. CDF began taking its first data in 1985 and started Run I after its first upgrade in 1989. For $W \to e  \nu$, \cite{Abe:1990sd} presented a measurement of $\sigma_W \times B(W\to e \nu)$ using the single-electron trigger with a selection of single isolated electron plus met. For $W \to \mu  \nu$, \cite{Abe:1992ys} presented a measurement of $\sigma_W \times B(W\to \mu \nu)$ and the ratio of muon and electron channel. This measurement used the single-muon trigger with a selection of single isolated muon plus met. Citing the previous CDF result on $\sigma_W \times B(W\to e \nu)$ in \cite{Abe:1990sd}, it obtained the ratio of the muonic and electronic weak coupling as $\frac{g^W_\mu}{g^W_e}=1.01\pm0.04$, consistent with the lepton universality. For $W \to \tau \nu$, \cite{Abe:1991fb} measured the $\sigma_W \times B(W\to \tau \nu)$ and its ratio to the electronic channel previous obtained in the \cite{Abe:1990sd}. The tau channel was based on two triggers, met trigger and single-tau trigger, which yielded 132 and 47 final events after selections. Comparing with the met trigger, the tau trigger required an additional tau jet cluster with a lower met threshold. The tau identification required 0-3 tracks with no tracks in the \ang{10} - \ang{30} region separate from the seeding track. Combining the met triggered and tau triggered data, the ratio between tau channel and electron channel was reported as $g^W_\tau/g^W_e=0.97\pm0.07$  agreeing with the SM lepton universality, as shown in Figure~\ref{fig:relatedWorks:lu:W:spsTevatronCombinedRatio}. Table~\ref{tab:relatedWorks:lu:W:tevatron} lists the CDF's results about the three $\sigma_W \times B(W\to l \nu)$ and the pair-wise ratios.





D0 was a general-purpose particle detector at the Fermilab Tevatron. Its structure was similar to CDF, consisting of a hybrid tracking system with silicon inner tracker and scintillator fiber outer tracker, superconducting solenoid, ECAL/HCAL, and the muon system. The detector was completed in 1991 and was placed in the Tevatron in February 1992. D0 collected its 1.8 TeV collision data during 1992-1995. With data collected in 1992-1993, D0 presented a measurement of $\sigma_W \times B(W\to e\nu)$, $\sigma_W \times B(W\to \mu \nu)$ and their ratio \cite{Abachi:1995xc}. Later, in the year 1994-1995, about 6 times more data was collected, and accordingly $\sigma_W \times B(W\to e\nu)$ was updated with better precision \cite{Abbott:1999tt}. It is worth noticing that this update \cite{Abbott:1999tt} also reported the branching fraction of W decay into electrons separately from the $\sigma_w$, as $B(W\to e\nu)=(10.66\pm0.15\pm0.21\pm0.11\pm0.11)\%$, where the uncertainties were for statistics, systematics, theory, and NLO. Also, with the 1994-1995 data, D0 measured $\sigma_W \times B(W\to \tau \nu)$ and test the lepton universality between tau and electron \cite{Abbott:1999pk}, shown in Figure ~\ref{fig:relatedWorks:lu:W:spsTevatronCombinedRatio}. For $W \to e \nu$ and $W \to \mu \nu$, the measurement selected events based on single-electron plus met and single-muon plus met. For $W \to \tau \nu$, D0 used a dedicated hadronic tau trigger, which included requirements on the met, the leading narrow jet pt, and no jet opposite to the leading narrow jet. The hadronic taus were reconstructed as boosted narrow jets with cuts on the $E_T$ and the jet width (an energy-weighted tower size in the jet). For each jet candidate, the energy in the leading two towers over the total energy was used to discriminate the tau jets over the background QCD jets. Table~\ref{tab:relatedWorks:lu:W:tevatron} lists the D0 results about the three $\sigma_W \times B(W\to l \nu)$ and the pair-wise ratios. 









% this result is indirect calculation using the LEP inputs
% \cite{Abazov:2003sv} summaries the W mass and witdh measurement on Tevatron by D0 and CDF and reports a tevatron combined W to electron branching fraction as 
% \begin{equation}
%     B(W\to e\nu)=(10.61 \pm 0.28) \% \; \text{Tevatron}
% \end{equation}


\subsubsection{LEP Experiments}
The Large Electron-Positron collider (LEP) at CERN increased its collision center-of-mass energy from the \PZ pole (LEP-I 1989-1995) to a maximum of 209~\GeV during its second running phase (LEP-II 1995-2000). In some parts of 1995 and 1997, the LEP was operated at center-of-mass energies below the WW resonance at 130.3, 136.3, and 140.2~\GeV. The rest runs of LEP-II scanned at 10 different energies above the WW resonance ranging in 161.3 - 209~\GeV. During the full second run scanning the center-of-mass energy from 130~\GeV to 209~\GeV, the four LEP experiments ALEPH, DELPHI, L3, and OPAL, collected a total data of 3~\fbinv integrated luminosity. 

The four detectors at LEP were designed to explore the physics at the \PZ pole during the LEP-I and from WW mass up to 203 GeV during the LEP-II. ALEPH was a cylindrical symmetric detector. It had a tracking system  (drift chamber and TPC) and ECAL inside a supper conducting solenoid. Outside the solenoid were streamer tubes inserted in the iron return yokes for the hadron and muon detection. DELPHI was also a cylindrical general-purpose detector consisting of the vertex detector, TPC tracker, Ring-Imaging Cherenkov detector, ECAL, solenoid, HCAL, muon chamber. OPAL's subdetector structures were formed by vertex detector, tracker, magnetic solenoid, crystal ECAL/HCAL, and muon detector. Unlike the other 3 detectors, L3 had its magnetic solenoid as the outmost layer; inside were trackers (silicon strip micro vertex detector and time expansion chamber), ECAL, HCAL, and muon chamber. 

The WW production in the electron positron collision was mainly induced by the EW process in the t-channel exchanging $\nu_e$, and the triple gauge boson coupling process in the s-channel mediated by Z or photon. The measurement of WW production cross-section from the four LEP experiments combined is shown in Figure~\ref{fig:relatedWorks:lu:W:lepWWxs}. There is a clear turn on the WW production at the 161.3 GeV. The combined result of WW cross-section is consistent with the theoretic prediction by YFSWW and RACOONWW.

\begin{figure}[ht]
    \centering
    \includegraphics[width=0.49\textwidth]{chapters/RelatedWorks/sectionLU/figures/lep_ww.png}
    \caption{The LEP measurement of WW production cross-section. The measurement was a combine of the four LEP experiments, with a total 3 $fb^{-1}$  data. The WW production at LEP was mainly induced by exchanging neutrinos in the t-channel and quark annihilation to $Z/\gamma$  in the s-channel. The measured cross-section agreed with the theoretical calculation.}
    \label{fig:relatedWorks:lu:W:lepWWxs}
\end{figure}

Each experiment determined the leptonic \PW decay branching fractions from the WW cross-sections measurement, with and without the lepton universality assumption \cite{Schael:2013ita}. The hadronic branching fraction was determined from the leptonic ones based on the unitarity. When combining the four experiments, the theoretical uncertainties of signal and background, as well as the theoretical uncertainties of the luminosity, were treated as correlated; in contrast, the experimental uncertainties on the luminosity, detector effects, and MC statistics are treated as uncorrelated. The details of the $B(W\to l \nu)$ results and the correlations, in individual experiment and after being combined, are summarized in Table~\ref{tab:relatedWorks:lu:W:lep} and in Figure~\ref{fig:relatedWorks:lu:W:lep}. A clear excess of the lepton universality was observed in the result. While the branching fractions to electron and muon agree well with each other, the branching fraction to tau is more than $2 \sigma$ larger than the average of the branching fraction to electron and muon. Assuming only partial lepton universality, the ratio between the $B(W\to \tau \nu)$ and the average of $B(W\to e \nu)$ and $B(W\to \mu \nu)$ were reported as \cite{Schael:2013ita}
\begin{equation*}
    \frac{2\times Br(W\to \tau \nu)}{Br(W\to e \nu)+ Br(W\to \mu  \nu)} = 1.066 \pm 0.025,
\end{equation*}

\noindent showing a 2.6 standard deviation from the lepton universality.


% LEP result plot
\begin{figure}[ht]
    \centering
    \includegraphics[width=0.99\textwidth]{chapters/RelatedWorks/sectionLU/figures/lepResult.png}
    \caption{\PW leptonic and hadronic branching fractions from the four LEP experiments. In the combined result, $B(W\to \tau \nu)$  is 2.6 $\sigma$ larger than the average of $B(W\to e \nu)$ and $B(W\to \mu \nu)$ \cite{Schael:2013ita}. }
    \label{fig:relatedWorks:lu:W:lep}
\end{figure}


% LEP result table
\begin{table}[ht]
    \setlength{\tabcolsep}{.5 em}
    \renewcommand{\arraystretch}{1.5}
    \centering
    \caption{Three $B(W\to l \nu)$ and the 3x3 correlation matrix of the four LEP experiments and the combined result \cite{Schael:2013ita}. *) The pair-wised ratios are derived from the $B(W\to l \nu)$ measurements. The central value and uncertainty is from PDG and the correlation is estimated by us.}
    \resizebox{\textwidth}{!}{
    \begin{tabular}{ |c| c  c | } 
         %  LEP ALEPH
         \hline
         \multicolumn{3}{|c|}{ALEPH \cite{Heister:2004wr}} \\
         \hline
         $Br(W\to e    \nu)$    & 10.78 $\pm$ 0.27 (stat) $\pm$ 0.10 (syst) & 
         \multirow{3}{*}{
            \begin{footnotesize}
            $\begin{bmatrix}
                +1.000 &-0.009 &-0.332 \\ 
                -0.009 &+1.000 &-0.268 \\
                -0.332 &-0.268 &+1.000 
            \end{bmatrix}$ 
            \end{footnotesize} 
         } \\
         $Br(W\to \mu  \nu)$    & 10.87 $\pm$ 0.25 (stat) $\pm$ 0.08 (syst) & \\ 
         $Br(W\to \tau \nu)$    & 11.25 $\pm$ 0.32 (stat) $\pm$ 0.20 (syst) & \\
         \hline
         \multicolumn{3}{c}{} \\
         
         
         %  LEP DELPHI
         \hline
         \multicolumn{3}{|c|}{DELPHI \cite{Abdallah:2003zm}} \\
         \hline
         $Br(W\to e    \nu)$    & 10.55 $\pm$ 0.31 (stat) $\pm$ 0.14 (syst) & 
         \multirow{3}{*}{
            \begin{footnotesize}
            $\begin{bmatrix}
                +1.000 &+0.030 &-0.340 \\ 
                +0.030 &+1.000 &-0.170 \\
                -0.340 &-0.170 &+1.000 
            \end{bmatrix}$ 
            \end{footnotesize} 
         } \\
         $Br(W\to \mu  \nu)$    & 10.65 $\pm$ 0.26 (stat) $\pm$ 0.08 (syst) & \\ 
         $Br(W\to \tau \nu)$    & 11.46 $\pm$ 0.39 (stat) $\pm$ 0.19 (syst) & \\
         \hline
         \multicolumn{3}{c}{} \\
         
         
         %  LEP L3
         \hline
         \multicolumn{3}{|c|}{L3 \cite{Achard:2004zw}} \\
         \hline
         $Br(W\to e    \nu)$    & 10.78 $\pm$ 0.29 (stat) $\pm$ 0.13 (syst) & 
         \multirow{3}{*}{
            \begin{footnotesize}
            $\begin{bmatrix}
                +1.000 &+0.136 &-0.201 \\ 
                +0.136 &+1.000 &-0.122 \\
                -0.201 &-0.122 &+1.000 
            \end{bmatrix}$ 
            \end{footnotesize} 
         } \\
         $Br(W\to \mu  \nu)$    & 10.03 $\pm$ 0.29 (stat) $\pm$ 0.12 (syst) & \\ 
         $Br(W\to \tau \nu)$    & 11.89 $\pm$ 0.40 (stat) $\pm$ 0.20 (syst) & \\
         \hline
         
         \multicolumn{3}{c}{} \\
         
         %  LEP OPAL
         \hline
         \multicolumn{3}{|c|}{OPAL \cite{Abbiendi:2007rs}} \\
         \hline
         $Br(W\to e    \nu)$    & 10.71 $\pm$ 0.25 (stat) $\pm$ 0.11 (syst) & 
         \multirow{3}{*}{
            \begin{footnotesize}
            $\begin{bmatrix}
                +1.000 &+0.135 &-0.303 \\ 
                +0.135 &+1.000 &-0.230 \\
                -0.303 &-0.230 &+1.000 
            \end{bmatrix}$ 
            \end{footnotesize} 
         } \\
         $Br(W\to \mu  \nu)$    & 10.78 $\pm$ 0.24 (stat) $\pm$ 0.10 (syst) & \\ 
         $Br(W\to \tau \nu)$    & 11.14 $\pm$ 0.31 (stat) $\pm$ 0.17 (syst) & \\
         \hline
         
         \multicolumn{3}{c}{} \\
         %  LEP Average
         \hline
         \multicolumn{3}{|c|}{LEP Average \cite{Schael:2013ita}} \\
         \hline
         $Br(W\to e    \nu)$    & 10.71 $\pm$ 0.14 (stat) $\pm$ 0.07 (syst) & 
         \multirow{3}{*}{
            \begin{footnotesize}
            $\begin{bmatrix}
                +1.000 &+0.136 &-0.201 \\ 
                +0.136 &+1.000 &-0.122 \\
                -0.201 &-0.122 &+1.000 
            \end{bmatrix}$ 
            \end{footnotesize} 
         } \\
         $Br(W\to \mu  \nu)$    & 10.63 $\pm$ 0.13 (stat) $\pm$ 0.07 (syst) & \\ 
         $Br(W\to \tau \nu)$    & 11.38 $\pm$ 0.17 (stat) $\pm$ 0.11 (syst) & \\
         \hline
         *$Br(W\to \mu  \nu)/ Br(W\to e \nu)$ & 0.993  $\pm$ 0.019 & 
         \multirow{3}{*}{
            \begin{footnotesize}
            $\begin{bmatrix}
                +1.000 &+0.440 &-0.314 \\ 
                +0.440 &+1.000 &+0.714 \\
                -0.314 &+0.714 &+1.000 
            \end{bmatrix}$ 
            \end{footnotesize} 
         } \\
         *$Br(W\to \tau \nu)/ Br(W\to e \nu)$ & 1.063  $\pm$ 0.027 & \\
         *$Br(W\to \tau \nu)/ Br(W\to\mu\nu)$ & 1.070  $\pm$ 0.026 &  \\
         
         \hline
    \end{tabular}}
    \label{tab:relatedWorks:lu:W:lep}
\end{table}


\subsubsection{LHC Experiments}

During the LHC Run-I at a center-of-mass energy of 7 TeV and 8 TeV, the lepton universality test in the EW sector was studied in the electron and muon channel, taking W+jets events as the signal. Two such measurements were published by the ATLAS and LHCb. ATLAS measured the $\sigma_W \times B(W \to e \nu)$ and $\sigma_W \times B(W \to \mu \nu)$ \cite{Aaboud:2016btc} with 7 TeV proton-proton collision data with 4.6~\fbinv collected in 2011. The events in the electron and muon channel were triggered with the single-lepton trigger and selected with several lepton isolation and identification cut, as well as the met cut. The ratio between muon and electron was determined as $\frac{ B(W  \to \mu \nu) }{ B(W \to e \nu)} = 1.003\pm 0.010$. LHCb also measured the $\sigma_W \times B(W \to e \nu)$ \cite{Aaij:2016qqz} and $\sigma_W \times B(W \to \mu \nu)$ \cite{Aaij:2015zlq} in two analysis with 8 TeV LHC data corresponding to 2~\fbinv integrated luminosity. The events were also triggered with the single-lepton, and the selections required on lepton quality and met. To test universality between the electron and muon channel, the second analysis \cite{Aaij:2016qqz} compared the electron channel with the muon channel published in the first analysis \cite{Aaij:2015zlq}, taking into account the experimental correlations. The comparison included both the total cross-section and the differential cross-section with respect to pseudorapidity. The differential cross-section agreed well in the electron and muon channel. The ratio of the two total cross-section led to $\frac{ B(W  \to \mu \nu) }{ B(W \to e \nu)}  = 0.980 \pm 0.018 $.




During the LHC Run-II at a unprecedentedly high center-of-mass energy of 13~\TeV, \PW bosons from the \ttbar events are treated as the major signal in the LU test for the first time, thanks to the high \ttbar cross-section at 13~\TeV. In addition, compared to the LHC Run-I, the improvement of the hadronic tau reconstruction also allows better precision tests involving the tau channel. This is the context of our analysis. Related to our analysis, ATLAS recently measured the ratio between W to tau and W to muon branching fraction $B(W  \to \tau \nu) / B(W \to \mu \nu) $ using the LHC Run-II data from 2016-2018 at 13 TeV corresponding to 139~\fbinv. The analysis selects \ttbar events with a single-muon trigger and applies additional requirements on muon quality, jet multiplicity, and b-tag multiplicity for a \ttbar-enriched region. Tau is probed with tau's muonic decay, which is 17\% of the total tau decay width. The key technique of this ATLAS measurement is fitting to the vertex displacement of the selected muon to discriminate $W \to \mu$ and $W \to \tau \to \mu$. Probing tau with muonic decay helps cancel the systematical uncertainties related to the muon reconstruction. Also, the systematics concerning hadronic tau reconstruction is avoided. The limitation of this approach is that only the $B(W  \to \tau \nu) / B(W \to \mu \nu) $ ratio is measured but the three individual leptonic branching fractions are not. The reported result of the $\tau / \mu $ branching ratio is

$$ \frac{ B(W  \to \tau \nu) }{ B(W \to \mu \nu)}  = 0.992 \pm 0.013 \text{ (ATLAS Run-II) }$$





\subsection{Test with Meson Decay}
\label{sec:relatedWorks:lu:meson}


% meson decay table
\begin{table}[ht]
    \setlength{\tabcolsep}{.5 em}
    \renewcommand{\arraystretch}{1.5}
    \centering
    \caption{SM prediction and the experimental measurements of the leptonic or semi-leptonic branching ratios of the pseudoscalar mesons. \cite{Bifani:2018zmi} }
    \resizebox{\textwidth}{!}{
    \begin{tabular}{|c|c|c|c|}
        \hline
         & SM Prediction & World Average & Included measurements \\
        \hline
        % pi
        $R^\pi_{e/\mu} \; [10^{-4}]$ &  1.2352 $\pm$ 0.0001 \cite{Cirigliano:2007xi} & 1.2327 $\pm$ 0.0023  & 
            \tiny{ TRIUMF \cite{Numao:1992ve, Britton:1992pg}, PiENu \cite{Aguilar-Arevalo:2015cdf}, BGO-OD \cite{Czapek:1993kc}} \\
        % K
        $R^K_{e/\mu} \; [10^{-5}]$ &  2.477  $\pm$ 0.001 \cite{Cirigliano:2007xi} & 2.488  $\pm$ 0.009 & 
            \tiny{NA62 \cite{Lazzeroni:2012cx}, KLOE \cite{Ambrosino:2009aa} }\\
        % D_s
        $R^{D_s}_{\tau/\mu} $ &  9.76 $\pm$ 0.10 \cite{Dobrescu:2008er} & 9.95 $\pm$ 0.61  & 
            \tiny{ HFLAV \cite{Amhis:2016xyh} ave of CLEO, BASIII, BELLE, BABAR} \\
        
        \hline
        % B D
        $R^{B}_{D, \tau/l} $ &  0.299  $\pm$ 0.003  \cite{Bifani:2018zmi} & 0.340  $\pm$ 0.030 & 
            \tiny{BABAR \cite{Lees:2012xj, Lees:2013uzd}, Belle \cite{Huschle:2015rga} }\\
            
        % B D*
        $R^{B}_{D*, \tau/l} $ &  0.258  $\pm$ 0.005 \cite{Bifani:2018zmi} & 0.295  $\pm$ 0.014 & 
            \tiny{BABAR \cite{Lees:2012xj, Lees:2013uzd}, Belle \cite{Huschle:2015rga, Sato:2016svk, Hirose:2016wfn}, LHCb\cite{Aaij:2015yra,Aaij:2017uff, Aaij:2017deq} }\\
            
        \hline
    \end{tabular}}
    \label{tab:relatedWorks:lu:meson:ratio}
\end{table}

The FCCC decay of pseudoscalar mesons provides tests of LU.  The most stringent constraints come from the study of leptonic decay of the charged pions or kaons, which are helicity suppressed in the SM.  The level of helicity suppression depends on the mass of the outcoming lepton. Pions and kaons can decay into electrons and muons, but not taus which are heavier than $\pi, K$. The ratio between the electronic and muonic branching fractions $R^\pi_{e/\mu},R^K_{e/\mu}$ are measured and compared with the SM prediction in the $\pi$ factories \cite{Numao:1992ve, Britton:1992pg, Aguilar-Arevalo:2015cdf, Czapek:1993kc} and $K$ factories \cite{Lazzeroni:2012cx, Ambrosino:2009aa} listed in Table~\ref{tab:relatedWorks:lu:meson:ratio}. For D meson, tauonic decay is possible, and the ratio between tauonic and muonic branching fraction $R^D_{\tau/\mu}$ is measured in the charm factories including CLEO, BASIII, Belle, and BaBar. Table~\ref{tab:relatedWorks:lu:meson:ratio} shows these experimental measurements and the comparison to the SM theoretical calculations. The experimental result of purely leptonic decay of light and charm pseudoscalar meson agree well with the theoretical prediction. 

Additionally, the tests of LU can be performed by comparing semileptonic FCCC transitions, such as $D\to K l\nu$, involving different lepton flavors. Heavy Flavor Averaging Group (HGLAV) provides a summary of the LU test using the semileptonic FCCC transitions of D meson and B meson \cite{Amhis:2019ckw}. An anomaly is observed in the semileptonic decay of B meson. $R^{B}_{D^{(*)}, \tau/l}$ is measured in the Belle \cite{Huschle:2015rga, Sato:2016svk, Hirose:2016wfn}, BaBar  \cite{Lees:2012xj, Lees:2013uzd} and LHCb \cite{Aaij:2015yra,Aaij:2017uff, Aaij:2017deq}, where ratio is defined as the semileptonic branching fraction with tau over with electron and muon. $R^{B}_{D, \tau/l} = \frac{Br(B\to D\tau \nu)}{Br(B\to Dl \nu)}$ and $R^{B}_{D^*, \tau/l} = \frac{Br(B\to D^*\tau \nu)}{Br(B\to D^*l \nu)}$ where $l=e,\mu$ . The experimental results are listed in Table~\ref{tab:relatedWorks:lu:meson:ratio}. Figure~\ref{fig:relatedWorks:lu:meson:bMesonDecay} illustrates this anomaly in the B meson semileptonic decay. The world average of  Belle, BaBar and LHCb is about 4 sigma deviated from the theoretical prediction with SM.

% However, these tests require knowledge of the ratio of the form factors of the scalar and vector meson, f0/f+, with very high accuracy to be competitive with the leptonic decays, where the main hadronic input (meson decay constants) drops out of the LU ratios. 



% B meson decay plot
\begin{figure}[ht]
    \centering
    \includegraphics[ width = 0.7 \textwidth ]{chapters/RelatedWorks/sectionLU/figures/bmeson.png}
    \caption{ Anomaly of lepton universality in the semi-leptonic decay of B meson \cite{Amhis:2019ckw}. SM prediction and the world average of $R^{B}_{D, \tau/l}$  and $R^{B}_{D^*, \tau/l}$  shows a 4 sigma deviation.}
    \label{fig:relatedWorks:lu:meson:bMesonDecay}
\end{figure}




\subsection{Test with Tau Decay}
\label{sec:relatedWorks:lu:lepton}

The LU of FCCC transitions can also be tested by the tau precision measurement \cite{Pich:2013lsa}. In the SM, the only expected difference between the $\tau^- \to e^- \bar{\nu}_e \nu_\tau$  and $\tau^- \to \mu^- \bar{\nu}_\mu \nu_\tau$  decay is due to decay kinematic phase space due to the mass of the outcoming leptons. $g_\mu  / g_e $ can be obtained by precision measurement of the tau decay in the electron and muon channels. Similarly,  $g_\tau  / g_\mu $ can be obtained by precision measurement of electronic tau decay  $\tau^- \to e^- \bar{\nu}_e \nu_\tau$ and electronic muon decay  $\mu^- \to e^- \bar{\nu}_e \nu_\mu$.  The ratio of the FCCC couplings to the third and first family can be obtained from the measurements of the $\tau^- \to \mu^- \bar{\nu}_\mu \nu_\tau$  and $\mu^- \to e^- \bar{\nu}_e \nu_\mu$  branching fraction and the $\tau,\mu$ lifetimes.  These represent the most stringent experimental tests available today for LU tests in the EW sector. From the tau precision measurement, the ratios of EW coupling constant among the three leptons are \cite{Pich:2013lsa}

\begin{align}
    g_\tau / g_\mu &= 1.0010 \pm 0.0014 \\
    g_\tau / g_e   &= 1.0029 \pm 0.0014 \\
    g_\mu  / g_e   &= 1.0018 \pm 0.0014 
\end{align}






\section{Related Experimental Results}

This section gives a brief reviews of two sets of related experiments: 1) the LU test in the charged weak sector; 2) the measurements of $V_{cs}$.

\subsection{Test of Lepton Universality of the Charged Weak Current}
\label{sec:relatedWorks:lu}


\subsubsection{Test with \PW Boson Decay} 
\label{sec:relatedWorks:lu:W}

.
% Test with \PW boson decay can be summarized into three eras: 1) SPS and Tevatron, 2) LEP, 3) LHC.




% SPS and Tevatron Experiments
\underline{SPS and Tevatron}

Both SPS and Tevatron collide protons and anti-protons. SPS operated at CERN from 1981 to 1991 at a center-of-mass energy of 0.546~\TeV and 0.630~\TeV. The SM electroweak bosons, \PW and \PZ, were first discovered in the SPS in 1983 \cite{ARNISON1983103, BANNER1983476}. In 1985, Tevatron at Fermilab began operations at a higher center-of-mass energy at 1.8~\TeV, which was later upgraded to 1.96~\TeV in its second run since 2001. Tevatron was in service for more than 20 years until 2010 to give ways to the LHC. 
% The properties of the weak bosons were measured with higher precision by Tevatron experiments. 


\begin{figure}[ht]
    \centering
    \includegraphics[height=0.35\textheight]{chapters/RelatedWorks/sectionLU/figures/SPS.png} \qquad
    \includegraphics[height=0.35\textheight]{chapters/RelatedWorks/sectionLU/figures/tevatron.png}
    \caption{Measurement of $\sigma_{p\bar{p}\to W} \times B^W_{l \nu}$ by the SPS \cite{Albajar:1988ka} and Tevatron~\cite{Abazov:2003sv, Abbott:1999tt, Abachi:1995xc, Abbott:1999pk, Abe:1990sd, Abe:1992ys, Abe:1991fb} experiments.}
    \label{sec:relatedWorks:lu:W:spsTevatron}
\end{figure}


% SPS Tevatron result plot
\begin{figure}[ht]
    \centering
    \includegraphics[width=0.5\textwidth]{chapters/RelatedWorks/sectionLU/figures/spsTevatron.png}
    \caption{ $g^W_\tau / g^W_e$ measured in the SPS and Tevatron experiments \cite{Abbott:1999pk}. In all the four experiments, the ratio of the weak coupling constant between electron and tau was extracted by the ratio of $\sigma_{p\bar{p}\to W} \times B^W_{l \nu}$ measurement in the electron and tau channel. The average was combined by D0 collaboration~\cite{Abbott:1999pk}, the last published result among the four.}
    \label{fig:relatedWorks:lu:W:spsTevatronCombinedRatio}
\end{figure}


The UA1, UA2 experiment at the CERN SPS and the CDF, D0 experiment at the Fermilab Tevatron measured the $p\bar{p} \to W \to \ell\nu$ cross-section in the different leptonic channels. Figure~\ref{sec:relatedWorks:lu:W:spsTevatron} shows the measurement of $\sigma_W \times B^W_{l \nu}$ in the SPS and Tevatron experiments. The LU test is performed by taking the ratios of two different leptonic channels. Figure~\ref{fig:relatedWorks:lu:W:spsTevatronCombinedRatio} from \cite{Abbott:1999pk} summarizes the results of $g^W_\tau / g^W_e$ measurements in the SPS and Tevatron experiments. The combined average was calculated by the D0 collaboration~\cite{Abbott:1999pk}, which was the last published result among the four. The combine assumed uncorrelated systematical and statistical uncertainties. And all four measurements confirmed consistency with the SM lepton universality within one experimental uncertainty.







% SPS result table
\begin{table}[ht]
    \setlength{\tabcolsep}{ 0.5 em}
    \renewcommand{\arraystretch}{1.5}
    \centering
    \caption{The measurement of $\sigma_W \times B(W\to l \nu)$ and the ratios between leptonic channels in the UA1 and UA2 experiment at the CERN SPS. }
    \resizebox{\textwidth}{!}{
    \begin{tabular}{ |c|l l| } 
         
         % UA1 result
         \hline
         \multicolumn{3}{|c|}{UA1 \cite{Albajar:1988ka} }  \\
         \hline
         & $p\bar{p}$ at $\sqrt{s}=0.546$ TeV &  $p\bar{p}$ at $\sqrt{s}=0.630$ TeV \\
         \hline
         $\sigma_W \times Br(W\to e    \nu)$  [nb]  & 0.55 $\pm$ 0.08 (stat) $\pm$ 0.09 (syst) & 0.63 $\pm$ 0.06 (stat) $\pm$ 0.10 (syst) \\ 
         $\sigma_W \times Br(W\to \mu  \nu)$  [nb]  & 0.56 $\pm$ 0.18 (stat) $\pm$ 0.12 (syst) & 0.63 $\pm$ 0.08 (stat) $\pm$ 0.11 (syst) \\ 
         $\sigma_W \times Br(W\to \tau \nu)$  [nb]  & \multicolumn{2}{c|}{ 0.63 $\pm$ 0.13 (stat) $\pm$ 0.12 (syst) }  \\ 
         \hline
         $Br(W\to \mu  \nu)/ Br(W\to e \nu)$  & \multicolumn{2}{c|}{1.00  $\pm$ 0.14 (stat) $\pm$ 0.08 (syst) } \\
         $Br(W\to \tau \nu)/ Br(W\to e \nu)$  & \multicolumn{2}{c|}{1.02  $\pm$ 0.20 (stat) $\pm$ 0.10 (syst) } \\
         
         \hline
         \multicolumn{2}{c}{} \\
         
         % UA2 result
         \hline
         \multicolumn{3}{|c|}{UA2}  \\
         \hline
         & $p\bar{p}$ at $\sqrt{s}=0.546$ TeV &  $p\bar{p}$ at $\sqrt{s}=0.630$ TeV \\
         \hline
         $\sigma_W \times Br(W\to e    \nu)$  [nb] \cite{appel1986measurement} & 0.50 $\pm$ 0.09 (stat) $\pm$ 0.05 (syst) & 0.53 $\pm$ 0.06 (stat) $\pm$ 0.05 (syst) \\ 
         % This is UA2 result reported in the UA1 summary
        %  $\sigma_W \times Br(W\to e    \nu)$  [nb] \cite{Albajar:1988ka} & 0.61 $\pm$ 0.10 (stat) $\pm$ 0.07 (syst) & 0.57 $\pm$ 0.04 (stat) $\pm$ 0.07 (syst) \\ 
         \hline
         $Br(W\to \tau \nu)/ Br(W\to e \nu)$ \cite{Alitti:1992hv} & - & 1.04  $\pm$ 0.08 (stat) $\pm$ 0.08 (syst) \\
         
         \hline
    \end{tabular}}
    \label{tab:relatedWorks:lu:W:sps}
\end{table}




% Tevatron result table
\begin{table}[ht]
    \setlength{\tabcolsep}{0.5 em}
    \renewcommand{\arraystretch}{1.5}
    \centering
    \caption{The measurement of $\sigma_W \times B(W\to l \nu)$ and the ratios between leptonic channels in the CDF and D0 experiment at the Fermilab Tevatron.}
    % \resizebox{0.95\textwidth}{!}{
    \begin{tabular}{ |c|l| } 
         

         
         %  CDF result
         \hline
         \multicolumn{2}{|c|}{CDF with $p\bar{p}$ at $\sqrt{s}=1.8$ TeV} \\
         \hline
         $\sigma_W \times Br(W\to e    \nu)$  [nb] \cite{Abe:1990sd}    & 2.19 $\pm$ 0.04 (stat) $\pm$ 0.21 (syst) \\ 
         $\sigma_W \times Br(W\to \mu  \nu)$  [nb] \cite{Abe:1992ys}    & 2.21 $\pm$ 0.07 (stat) $\pm$ 0.21 (syst) \\ 
         $\sigma_W \times Br(W\to \tau \nu)$  [nb] \cite{Abe:1991fb}    & 2.05 $\pm$ 0.27 \\ 
         \hline
         $Br(W\to \mu  \nu)/ Br(W\to e \nu)$ \cite{Abe:1992ys} & 1.02  $\pm$ 0.08 \\
         $Br(W\to \tau \nu)/ Br(W\to e \nu)$ \cite{Abe:1991fb} & 0.94  $\pm$ 0.14 \\

         \hline
         
         \multicolumn{2}{c}{}  \\
         
         % D0 result
         \hline
         \multicolumn{2}{|c|}{D0 with $p\bar{p}$ at $\sqrt{s}=1.8$ TeV} \\
         \hline
         $\sigma_W \times Br(W\to e    \nu)$  [nb] \cite{Abbott:1999tt} & 2.31 $\pm$ 0.01 (stat) $\pm$ 0.05 (syst) $\pm$ 0.10 (lum) \\ 
         $\sigma_W \times Br(W\to \mu  \nu)$  [nb] \cite{Abachi:1995xc} & 2.09 $\pm$ 0.23 (stat) $\pm$ 0.11 (syst) \\ 
         $\sigma_W \times Br(W\to \tau \nu)$  [nb] \cite{Abbott:1999pk} & 2.22 $\pm$ 0.09 (stat) $\pm$ 0.10 (syst) $\pm$ 0.10 (lum)  \\ 
         \hline
         $Br(W\to \mu  \nu)/ Br(W\to e \nu)$ \cite{Abachi:1995xc} & 0.89  $\pm$ 0.10 \\
         $Br(W\to \tau \nu)/ Br(W\to e \nu)$ \cite{Abbott:1999pk} & 0.961 $\pm$ 0.061 \\
         
         \hline
         
         
    \end{tabular}
    % }
    \label{tab:relatedWorks:lu:W:tevatron}
\end{table}


UA1 was a general-purpose particle detector at the CERN SPS, consisting of the inner tracker, ECAL HCAL, and a muon system, sequentially from the inside to the outside.  It took 0.546~\TeV and 0.63~\TeV data during 1982-1983 and 1984-1985, respectively. Its result of \PW boson studies is listed in \cite{Albajar:1988ka}. $W \to e \nu$ events were selected based on single-electron plus met selection. The QCD and $W\to \tau_e \nu$ background were estimated with data-driven and MC approach, respectively. In total, 59 and 240 $W \to e \nu$ events were selected from the 0.546~\TeV and 0.63~\TeV collision, respectively.  $W \to \mu \nu$ events were selected based on single muon plus met selection. The background involving muons from tau and meson decays was estimated by proper simulations. In total, 10 and 57 $W\to \mu\nu$ events were selected from the 0.546~\TeV and 0.63~\TeV data.  $W\to \tau \nu$ were selected with a single hadronic tau plus met selection. The hadronic taus were identified by highly collimated narrow jets with low charged-track multiplicity.  A $\tau$-likelihood was calculated for each jet candidate based on the its shape and charged tracks. In total, 32 events were selected from the combined 0.546~\TeV and 0.63~\TeV dataset. Based on the yields, UA1 reported the $\sigma_W \times Br(W\to l\nu) $ for the three leptons $l=e,\mu,\tau$ at 0.546~\TeV and 0.63~\TeV center-of-mass energy. Pair-wise ratios of  $\sigma_W \times Br(W\to l\nu) $ were calculated to test the lepton universality. Table~\ref{tab:relatedWorks:lu:W:sps} lists the $\sigma_W \times Br(W\to l\nu) $ and ratios from UA1.



UA2 was a particle detector at the CERN SPS, consisting of a tracking system surrounded by a calorimetry system with EM and hadronic compartments. Unlike UA1, UA2 was not a multipurpose detector; its focus was on the calorimeters and did not have a muon detector. Therefore, lepton universality test on UA2 mainly involved the $W \to e\nu$ and $W \to \tau \nu$. \cite{appel1986measurement} summarized the $\sigma_W \times Br(W\to e \nu) $ measurements from the UA2 using 0.546 TeV and 0.63 TeV data collected during 1982-1983 and 1984-1985. The measurement was based on single-electron plus met trigger. This  $\sigma_W \times Br(W\to e \nu) $ result is shown in Table~\ref{tab:relatedWorks:lu:W:sps}. After the UA2 upgrade during 1985-1987,  the tau channel was added and a test of the lepton universality between $\tau$ and $e$ was performed \cite{Alitti:1991eh, Alitti:1992hv}, using the 0.63 TeV data collected during 1988-1990. The hadronic taus were reconstructed from jet candidates with selections on relative hadronic energy and the lateral energy profile. The data was triggered with the met trigger in 1988-1989 and hadronic tau trigger in 1990. \cite{Alitti:1991eh} analyzed the 1988-1989 data, while \cite{Alitti:1992hv} combined the 1988-1989 data with 1990 data. The result \cite{Alitti:1992hv} for the ratio between tauonic and electronic W decays is shown in the Table~\ref{tab:relatedWorks:lu:W:sps}. 






CDF was an azimuthally and forward-backward symmetric general-purpose detector at the Fermilab Tevatron. It was consist of several subdetector layers, including a silicon tracker, gas chamber as the central outer tracker, solenoid magnet, ECAL/HCAL, and muon detector. CDF began taking its first data in 1985 and started Run I after its first upgrade in 1989. For $W \to e  \nu$, \cite{Abe:1990sd} presented a measurement of $\sigma_W \times B(W\to e \nu)$ using the single-electron trigger with a selection of single isolated electron plus met. For $W \to \mu  \nu$, \cite{Abe:1992ys} presented a measurement of $\sigma_W \times B(W\to \mu \nu)$ and the ratio of muon and electron channel. This measurement used the single-muon trigger with a selection of single isolated muon plus met. Citing the previous CDF result on $\sigma_W \times B(W\to e \nu)$ in \cite{Abe:1990sd}, it obtained the ratio of the muonic and electronic weak coupling as $\frac{g^W_\mu}{g^W_e}=1.01\pm0.04$, consistent with the lepton universality. For $W \to \tau \nu$, \cite{Abe:1991fb} measured the $\sigma_W \times B(W\to \tau \nu)$ and its ratio to the electronic channel previous obtained in the \cite{Abe:1990sd}. The tau channel was based on two triggers, met trigger and single-tau trigger, which yielded 132 and 47 final events after selections. Comparing with the met trigger, the tau trigger required an additional tau jet cluster with a lower met threshold. The tau identification required 0-3 tracks with no tracks in the \ang{10} - \ang{30} region separate from the seeding track. Combining the met triggered and tau triggered data, the ratio between tau channel and electron channel was reported as $g^W_\tau/g^W_e=0.97\pm0.07$  agreeing with the SM lepton universality, as shown in Figure~\ref{fig:relatedWorks:lu:W:spsTevatronCombinedRatio}. Table~\ref{tab:relatedWorks:lu:W:tevatron} lists the CDF's results about the three $\sigma_W \times B(W\to l \nu)$ and the pair-wise ratios.





D0 was a general-purpose particle detector at the Fermilab Tevatron. Its structure was similar to CDF, consisting of a hybrid tracking system with silicon inner tracker and scintillator fiber outer tracker, superconducting solenoid, ECAL/HCAL, and the muon system. The detector was completed in 1991 and was placed in the Tevatron in February 1992. D0 collected its 1.8 TeV collision data during 1992-1995. With data collected in 1992-1993, D0 presented a measurement of $\sigma_W \times B(W\to e\nu)$, $\sigma_W \times B(W\to \mu \nu)$ and their ratio \cite{Abachi:1995xc}. Later, in the year 1994-1995, about 6 times more data was collected, and accordingly $\sigma_W \times B(W\to e\nu)$ was updated with better precision \cite{Abbott:1999tt}. It is worth noticing that this update \cite{Abbott:1999tt} also reported the branching fraction of W decay into electrons separately from the $\sigma_w$, as $B(W\to e\nu)=(10.66\pm0.15\pm0.21\pm0.11\pm0.11)\%$, where the uncertainties were for statistics, systematics, theory, and NLO. Also, with the 1994-1995 data, D0 measured $\sigma_W \times B(W\to \tau \nu)$ and test the lepton universality between tau and electron \cite{Abbott:1999pk}, shown in Figure ~\ref{fig:relatedWorks:lu:W:spsTevatronCombinedRatio}. For $W \to e \nu$ and $W \to \mu \nu$, the measurement selected events based on single-electron plus met and single-muon plus met. For $W \to \tau \nu$, D0 used a dedicated hadronic tau trigger, which included requirements on the met, the leading narrow jet pt, and no jet opposite to the leading narrow jet. The hadronic taus were reconstructed as boosted narrow jets with cuts on the $E_T$ and the jet width (an energy-weighted tower size in the jet). For each jet candidate, the energy in the leading two towers over the total energy was used to discriminate the tau jets over the background QCD jets. Table~\ref{tab:relatedWorks:lu:W:tevatron} lists the D0 results about the three $\sigma_W \times B(W\to l \nu)$ and the pair-wise ratios. 






% \subsubsubsection{LEP Experiments}
\underline{LEP}

The LEP at CERN increased its collision center-of-mass energy from the \PZ pole (LEP-I 1989-1995) to a maximum of 209~\GeV during its second running phase (LEP-II 1995-2000). In some parts of 1995 and 1997, the LEP was operated at center-of-mass energies below the WW resonance at 130.3, 136.3, and 140.2~\GeV. The rest runs of LEP-II scanned at 10 different energies above the WW resonance ranging in 161.3 - 209~\GeV. During the full second run scanning the center-of-mass energy from 130~\GeV to 209~\GeV, the four LEP experiments ALEPH, DELPHI, L3, and OPAL, collected a total data of 3~\fbinv integrated luminosity. 

The four detectors at LEP were designed to explore the physics at the \PZ pole during the LEP-I and from WW mass up to 203 GeV during the LEP-II. ALEPH was a cylindrical symmetric detector. It had a tracking system  (drift chamber and TPC) and ECAL inside a supper conducting solenoid. Outside the solenoid were streamer tubes inserted in the iron return yokes for the hadron and muon detection. DELPHI was also a cylindrical general-purpose detector consisting of the vertex detector, TPC tracker, Ring-Imaging Cherenkov detector, ECAL, solenoid, HCAL, muon chamber. OPAL's subdetector structures were formed by vertex detector, tracker, magnetic solenoid, crystal ECAL/HCAL, and muon detector. Unlike the other 3 detectors, L3 had its magnetic solenoid as the outmost layer; inside were trackers (silicon strip micro vertex detector and time expansion chamber), ECAL, HCAL, and muon chamber. 

The WW production in the electron positron collision was mainly induced by the EW process in the t-channel exchanging $\nu_e$, and the triple gauge boson coupling process in the s-channel mediated by Z or photon. The measurement of WW production cross-section from the four LEP experiments combined is shown in Figure~\ref{fig:relatedWorks:lu:W:lepWWxs}. There is a clear turn on the WW production at the 161.3 GeV. The combined result of WW cross-section is consistent with the theoretic prediction by YFSWW and RACOONWW.

\begin{figure}[ht]
    \centering
    \includegraphics[width=0.49\textwidth]{chapters/RelatedWorks/sectionLU/figures/lep_ww.png}
    \caption{The LEP measurement of WW production cross-section. The measurement was a combine of the four LEP experiments, with a total 3 $fb^{-1}$  data. The WW production at LEP was mainly induced by exchanging neutrinos in the t-channel and quark annihilation to $Z/\gamma$  in the s-channel. The measured cross-section agreed with the theoretical calculation.}
    \label{fig:relatedWorks:lu:W:lepWWxs}
\end{figure}

Each experiment determined the leptonic \PW decay branching fractions from the WW cross-sections measurement, with and without the lepton universality assumption \cite{Schael:2013ita}. The hadronic branching fraction was determined from the leptonic ones based on the unitarity. When combining the four experiments, the theoretical uncertainties of signal and background, as well as the theoretical uncertainties of the luminosity, were treated as correlated; in contrast, the experimental uncertainties on the luminosity, detector effects, and MC statistics are treated as uncorrelated. The details of the $B(W\to l \nu)$ results and the correlations, in individual experiment and after being combined, are summarized in Table~\ref{tab:relatedWorks:lu:W:lep} and in Figure~\ref{fig:relatedWorks:lu:W:lep}. A clear excess of the lepton universality was observed in the result. While the branching fractions to electron and muon agree well with each other, the branching fraction to tau is more than $2 \sigma$ larger than the average of the branching fraction to electron and muon. Assuming only partial lepton universality, the ratio between the $B(W\to \tau \nu)$ and the average of $B(W\to e \nu)$ and $B(W\to \mu \nu)$ were reported as \cite{Schael:2013ita} $1.066 \pm 0.025$, showing a 2.6 standard deviation from the lepton universality.


% LEP result plot
\begin{figure}[ht]
    \centering
    \includegraphics[width=0.99\textwidth]{chapters/RelatedWorks/sectionLU/figures/lepResult.png}
    \caption{\PW leptonic and hadronic branching fractions from the four LEP experiments. In the combined result, $B(W\to \tau \nu)$  is 2.6 $\sigma$ larger than the average of $B(W\to e \nu)$ and $B(W\to \mu \nu)$ \cite{Schael:2013ita}. }
    \label{fig:relatedWorks:lu:W:lep}
\end{figure}


% LEP result table
\begin{table}[!ht]
    \setlength{\tabcolsep}{.5 em}
    \renewcommand{\arraystretch}{1.5}
    \centering
    \caption{Three individual leptonic branching fractions from the four LEP experiments and the combined result \cite{Schael:2013ita}.}
    % \resizebox{\textwidth}{!}{
    \begin{tabular}{ |c| c  c | } 
         %  LEP ALEPH
         \hline
         \multicolumn{3}{|c|}{ALEPH \cite{Heister:2004wr}} \\
         \hline
         \BWe    & 10.78 $\pm$ 0.27 (stat) $\pm$ 0.10 (syst) & 
         \multirow{3}{*}{
            \begin{footnotesize}
            $\begin{bmatrix}
                +1.000 &-0.009 &-0.332 \\ 
                -0.009 &+1.000 &-0.268 \\
                -0.332 &-0.268 &+1.000 
            \end{bmatrix}$ 
            \end{footnotesize} 
         } \\
         \BWm    & 10.87 $\pm$ 0.25 (stat) $\pm$ 0.08 (syst) & \\ 
         \BWt    & 11.25 $\pm$ 0.32 (stat) $\pm$ 0.20 (syst) & \\
         \hline
         \multicolumn{3}{c}{} \\
         
         
         %  LEP DELPHI
         \hline
         \multicolumn{3}{|c|}{DELPHI \cite{Abdallah:2003zm}} \\
         \hline
         \BWe    & 10.55 $\pm$ 0.31 (stat) $\pm$ 0.14 (syst) & 
         \multirow{3}{*}{
            \begin{footnotesize}
            $\begin{bmatrix}
                +1.000 &+0.030 &-0.340 \\ 
                +0.030 &+1.000 &-0.170 \\
                -0.340 &-0.170 &+1.000 
            \end{bmatrix}$ 
            \end{footnotesize} 
         } \\
         \BWm    & 10.65 $\pm$ 0.26 (stat) $\pm$ 0.08 (syst) & \\ 
         \BWt    & 11.46 $\pm$ 0.39 (stat) $\pm$ 0.19 (syst) & \\
         \hline
         \multicolumn{3}{c}{} \\
         
         
         %  LEP L3
         \hline
         \multicolumn{3}{|c|}{L3 \cite{Achard:2004zw}} \\
         \hline
         \BWe    & 10.78 $\pm$ 0.29 (stat) $\pm$ 0.13 (syst) & 
         \multirow{3}{*}{
            \begin{footnotesize}
            $\begin{bmatrix}
                +1.000 &+0.136 &-0.201 \\ 
                +0.136 &+1.000 &-0.122 \\
                -0.201 &-0.122 &+1.000 
            \end{bmatrix}$ 
            \end{footnotesize} 
         } \\
         \BWm    & 10.03 $\pm$ 0.29 (stat) $\pm$ 0.12 (syst) & \\ 
         \BWt    & 11.89 $\pm$ 0.40 (stat) $\pm$ 0.20 (syst) & \\
         \hline
         
         \multicolumn{3}{c}{} \\
         
         %  LEP OPAL
         \hline
         \multicolumn{3}{|c|}{OPAL \cite{Abbiendi:2007rs}} \\
         \hline
         \BWe    & 10.71 $\pm$ 0.25 (stat) $\pm$ 0.11 (syst) & 
         \multirow{3}{*}{
            \begin{footnotesize}
            $\begin{bmatrix}
                +1.000 &+0.135 &-0.303 \\ 
                +0.135 &+1.000 &-0.230 \\
                -0.303 &-0.230 &+1.000 
            \end{bmatrix}$ 
            \end{footnotesize} 
         } \\
         \BWm    & 10.78 $\pm$ 0.24 (stat) $\pm$ 0.10 (syst) & \\ 
         \BWt    & 11.14 $\pm$ 0.31 (stat) $\pm$ 0.17 (syst) & \\
         \hline
         
         \multicolumn{3}{c}{} \\
         %  LEP Average
         \hline
         \multicolumn{3}{|c|}{LEP Average \cite{Schael:2013ita}} \\
         \hline
         \BWe    & 10.71 $\pm$ 0.14 (stat) $\pm$ 0.07 (syst) & 
         \multirow{3}{*}{
            \begin{footnotesize}
            $\begin{bmatrix}
                +1.000 &+0.136 &-0.201 \\ 
                +0.136 &+1.000 &-0.122 \\
                -0.201 &-0.122 &+1.000 
            \end{bmatrix}$ 
            \end{footnotesize} 
         } \\
         \BWm    & 10.63 $\pm$ 0.13 (stat) $\pm$ 0.07 (syst) & \\ 
         \BWt    & 11.38 $\pm$ 0.17 (stat) $\pm$ 0.11 (syst) & \\
         \hline
        %  *$Br(W\to \mu  \nu)/ Br(W\to e \nu)$ & 0.993  $\pm$ 0.019 & 
        %  \multirow{3}{*}{
        %     \begin{footnotesize}
        %     $\begin{bmatrix}
        %         +1.000 &+0.440 &-0.314 \\ 
        %         +0.440 &+1.000 &+0.714 \\
        %         -0.314 &+0.714 &+1.000 
        %     \end{bmatrix}$ 
        %     \end{footnotesize} 
        %  } \\
        %  *$Br(W\to \tau \nu)/ Br(W\to e \nu)$ & 1.063  $\pm$ 0.027 & \\
        %  *$Br(W\to \tau \nu)/ Br(W\to\mu\nu)$ & 1.070  $\pm$ 0.026 &  \\
         
        %  \hline
    \end{tabular}
    % }
    \label{tab:relatedWorks:lu:W:lep}
\end{table}


% \subsubsubsection{LHC Experiments}

\underline{LHC}

During the LHC run~1 at a center-of-mass energy of 7 TeV and 8 TeV, the lepton universality test in the EW sector was studied in the electron and muon channel, taking W+jets events as the signal. Two such measurements were published by the ATLAS and LHCb. ATLAS measured the $\sigma_W \times B(W \to e \nu)$ and $\sigma_W \times B(W \to \mu \nu)$ \cite{Aaboud:2016btc} with 7 TeV proton-proton collision data with 4.6~\fbinv collected in 2011. The events in the electron and muon channel were triggered with the single-lepton trigger and selected with several lepton isolation and identification cut, as well as the met cut. The ratio between muon and electron was determined as $\frac{ B(W  \to \mu \nu) }{ B(W \to e \nu)} = 1.003\pm 0.010$. LHCb also measured the $\sigma_W \times B(W \to e \nu)$ \cite{Aaij:2016qqz} and $\sigma_W \times B(W \to \mu \nu)$ \cite{Aaij:2015zlq} in two analysis with 8 TeV data corresponding to 2~\fbinv integrated luminosity. The events were also triggered with the single-lepton, and the selections required on lepton quality and met. To test universality between the electron and muon channel, the second analysis \cite{Aaij:2016qqz} compared the electron channel with the muon channel published in the first analysis \cite{Aaij:2015zlq}, taking into account the experimental correlations. The comparison included both the total cross-section and the differential cross-section with respect to pseudorapidity. The differential cross-section agreed well in the electron and muon channel. The ratio of the two total cross-section led to $\frac{ B(W  \to \mu \nu) }{ B(W \to e \nu)}  = 0.980 \pm 0.018 $.


During the LHC Run-II at a unprecedentedly high center-of-mass energy of 13~\TeV, \PW bosons from the \ttbar events are treated as the major signal in the LU test for the first time, thanks to the high \ttbar cross-section at 13~\TeV. ATLAS~\cite{Aad:2020ayz} measured the ratio between W to tau and W to muon branching fraction $B(W  \to \tau \nu) / B(W \to \mu \nu) $ using the LHC run~2 data from 2016-2018 at 13 TeV corresponding to 139~\fbinv. The analysis selects \ttbar events with a single-muon trigger and applies additional requirements on muon quality, jet multiplicity, and b-tag multiplicity for a \ttbar-enriched region. Tau is probed with tau's muonic decay, which is 17\% of the total tau decay width. The key technique of this ATLAS measurement is fitting to the vertex displacement of the selected muon to discriminate $W \to \mu$ and $W \to \tau \to \mu$. Probing tau with muonic decay helps cancel the systematical uncertainties related to the muon reconstruction. Also, the systematics concerning hadronic tau reconstruction is avoided. The limitation of this approach is that only the $B(W  \to \tau \nu) / B(W \to \mu \nu) $ ratio is measured but the three individual leptonic branching fractions are not. The reported result of the $\tau / \mu $ branching ratio is

$$ \frac{ B(W  \to \tau \nu) }{ B(W \to \mu \nu)}  = 0.992 \pm 0.013 \text{ (ATLAS) }$$





\subsubsection{Test with Meson Decay}
\label{sec:relatedWorks:lu:meson}

\input{chapters/Introduction/tables/mesonLU}

The charged weak current decay of mesons also provides tests of LU.  The most stringent constraints come from the study of leptonic decay of the charged pions or kaons, which are helicity suppressed in the SM.  The level of helicity suppression depends on the mass of the outcoming lepton. Pions and kaons can decay into electrons and muons, but not taus which are heavier than $\pi, K$. The ratio between the electronic and muonic branching fractions $R^\pi_{e/\mu},R^K_{e/\mu}$ are measured in the $\pi$ factories \cite{Numao:1992ve, Britton:1992pg, Aguilar-Arevalo:2015cdf, Czapek:1993kc} and $K$ factories \cite{Lazzeroni:2012cx, Ambrosino:2009aa} listed in Table~\ref{tab:relatedWorks:lu:meson:ratio}. The ratios are compared to the SM prediction with LU. For D meson, tauonic decay is possible, and the ratio between tauonic and muonic branching fraction $R^D_{\tau/\mu}$ is measured in the charm factories including CLEO, BASIII, Belle, and BaBar. Table~\ref{tab:relatedWorks:lu:meson:ratio} shows these experimental measurements and the comparison to the SM theoretical calculations. The experimental results of purely leptonic decay of the light and charm pseudoscalar mesons agree well with the SM theoretical prediction. 

Additionally, the tests of LU can be performed by comparing semileptonic charged weak decays, such as $D\to K l\nu$. Heavy Flavor Averaging Group (HGLAV) provides a summary of the LU test using the semileptonic charged weak decay of the D meson and the B meson \cite{Amhis:2019ckw}. An anomaly is observed in the $B\to D^{(*)} l\nu$ semileptonic decay in the tau channel versus the electron and muon channel. $R^{B}_{D^{(*)}, \tau/l}$ is measured in the Belle \cite{Huschle:2015rga, Sato:2016svk, Hirose:2016wfn}, BaBar  \cite{Lees:2012xj, Lees:2013uzd} and LHCb \cite{Aaij:2015yra,Aaij:2017uff, Aaij:2017deq}, where ratio is defined as $R^{B}_{D, \tau/l} = \frac{Br(B\to D\tau \nu)}{Br(B\to Dl \nu)}$ and $R^{B}_{D^*, \tau/l} = \frac{Br(B\to D^*\tau \nu)}{Br(B\to D^*l \nu)}$ where $l=e,\mu$ . The experimental results are listed in Table~\ref{tab:relatedWorks:lu:meson:ratio}. Figure~\ref{fig:relatedWorks:lu:meson:bMesonDecay} illustrates this anomaly of the B meson semileptonic decay. The world average of Belle, BaBar and LHCb is about 4 sigma deviated from the SM theoretical prediction.

% However, these tests require knowledge of the ratio of the form factors of the scalar and vector meson, f0/f+, with very high accuracy to be competitive with the leptonic decays, where the main hadronic input (meson decay constants) drops out of the LU ratios. 



% B meson decay plot
\begin{figure}[ht]
    \centering
    \includegraphics[ width = 0.7 \textwidth ]{chapters/RelatedWorks/sectionLU/figures/bmeson.png}
    \caption{ Anomaly of lepton universality in the semi-leptonic decay of B meson \cite{Amhis:2019ckw}. SM prediction and the world average of $R^{B}_{D, \tau/l}$  and $R^{B}_{D^*, \tau/l}$  shows a 4 sigma deviation.}
    \label{fig:relatedWorks:lu:meson:bMesonDecay}
\end{figure}




\subsubsection{Test with Tau Decay}
\label{sec:relatedWorks:lu:lepton}

The LU of charged weak current can also be tested by the tau precision measurements \cite{Pich:2013lsa,Amhis:2019ckw}. In the SM, the only expected difference between the $\tau^- \to e^- \bar{\nu}_e \nu_\tau$  and $\tau^- \to \mu^- \bar{\nu}_\mu \nu_\tau$  decay is due to decay kinematic phase space due to the mass of the outcoming leptons. $g_\mu  / g_e $ can be obtained by precision measurement of the tau decay in the electron and muon channels. Similarly,  $g_\tau  / g_\mu $ can be obtained by precision measurement of electronic tau decay  $\tau^- \to e^- \bar{\nu}_e \nu_\tau$ and electronic muon decay  $\mu^- \to e^- \bar{\nu}_e \nu_\mu$.  The ratio of the charged weak boson couplings to the third and first family can be obtained from the measurements of the $\tau^- \to \mu^- \bar{\nu}_\mu \nu_\tau$  and $\mu^- \to e^- \bar{\nu}_e \nu_\mu$  branching fraction and the $\tau,\mu$ lifetimes.  These represent one of the most stringent experimental tests for LU in the EW sector. By global fit to the tau precision measurements, HFLAV~\cite{Amhis:2019ckw} determines the ratios of EW coupling constant among the three leptons as 

\begin{align*}
    g_\tau / g_\mu &= 1.0010 \pm 0.0014 \\
    g_\tau / g_e   &= 1.0029 \pm 0.0014 \\
    g_\mu  / g_e   &= 1.0018 \pm 0.0014 
\end{align*}





\subsection{Measurements of $V_{cs}$ }
\label{sec:relatedWorks:vcsMeasurements}

The CKM matrix originates from the Yukawa couplings in the SM Higgs sector. It represents the mixing between quarks' mass eigenstates and the flavor eigenstates. When physical quarks in their mass eigenstates participate in the weak interaction, they are projected to the flavor eigenstates by the corresponding element in the CKM matrix. More details about the CKM in the standard model are discussed in Section~\ref{sec:relatedWorks:qft:gws}. 

\begin{table}[ht]
    \centering
    \setlength{\tabcolsep}{1.5em}
    \renewcommand{\arraystretch}{1.5}
    \caption{The current experimental world average of the 9 elements in the CKM matrix in the PDG \cite{pdg2020}.  }
    \resizebox{\textwidth}{!}{
    \begin{tabular}{c|c|c }
        \hline
        $|V_{ud}|=0.97370 \pm 0.00014 $     & $|V_{us}|=0.2245 \pm 0.0008$      &  $|V_{ub}|=0.00382 \pm 0.00024$   \\ \hline
        $|V_{cd}|=0.221 \pm 0.004 $         & $|V_{cs}|=0.987 \pm 0.011$        &  $|V_{cb}|=0.0410 \pm 0.0014$     \\ \hline
        $|V_{td}|=0.0080 \pm 0.0003 $       & $|V_{ts}|=0.0388 \pm 0.0011$      &  $|V_{tb}|=1.013 \pm 0.030$       \\
        \hline
    \end{tabular}}
    \label{tab:relatedWorks:vcs:ckm}
\end{table}


The current experimental measurement of the 9 elements in the CKM matrix \cite{pdg2020} is shown in Table~\ref{tab:relatedWorks:vcs:ckm}. Among the 6 elements in the first two rows, $|V_{cs}|$ is measured with the least precision. The average of $|V_{cs}|$ measurements is shown in Figure~\ref{fig:relatedWorks:vcs:measurements}. Currently, there are two direct approaches to measure $|V_{cs}|$, using the D meson decay in the charm factories and using the on-shell $W\to c s$  with jet tagging in the collider experiments.

The best direct determination of $|V_{cs}|$ is from the semileptonic decay of $D$ and the leptonic decay of $D_s$ produced in the charm factory. For the results from the leptonic decay of $D_s$ meson, the branching fraction of $D_s^+ \to \mu^+ \nu$ and $D_s^+ \to \tau^+ \nu$ are both measured in the Belle \cite{Zupanc:2013byn}, CLEO \cite{Alexander:2009ux,Onyisi:2009th,Naik:2009tk}, BaBar \cite{delAmoSanchez:2010jg} and BESIII \cite{Ablikim:2016duz, Ablikim:2018jun}. Using the experimental value of mass and lifetime of $D_s$, as well as the lattice QCD calculation of the form factor $f_{D_s}$, $|V_{cs}|$ can be determined from the $D_s$ leptonic decay and yields a world average of $|V_{cs}|=0.992\pm 0.012$ \cite{Amhis:2019ckw}, where the dominating uncertainty is from the experimental error. For the results from the semileptonic decay of $D$ meson, the branching fraction of $D\to K l\nu$ is measured by CLEO-c \cite{Besson:2009uv}, Belle \cite{Widhalm:2006wz}, BaBar \cite{Aubert:2007wg} and BESIII \cite{Ablikim:2015ixa, Ablikim:2018evp}, which gives an average of $|V_{cs}|$ of $|V_{cs}|=0.939\pm 0.038$ \cite{Amhis:2019ckw} in the D meson decay. The dominant uncertainty is form the theoretical calculations of the D meson form factor with latice QCD. Combining the result from the $D$ and $D_s$ decay, the charm factories measures $|V_{cs}|=0.987\pm 0.011$ \cite{Amhis:2019ckw}. This is also the value considered as the world average by the PDG~\cite{pdg2020}.

The second direct measurement of $|V_{cs}|$ is from the on-shell $W\to c s$ decays in the collider experiments. This approach relies on jet tagging to identifies the jets originating from the c and s quarks, which is relatively difficult, especially in the hadron collider with a more complex hadron environment. Therefore, this approach is less explored compared with the $D/D_s$ approach. So far, the only published result based on the $W\to c s$  approach is from the DELPHI~\cite{Abreu:1998ap} in the LEP. DELPHI identified the charged mesons based on their ionization energy loss while traversing through the TPC tracking system. Since s and c jets tend to include energetic kaons, c and s jets were first tagged by the \pt and the particle id ($\pi$ or K) of the leading meson in the jet. Then the impact parameters of the tracks in the jet were considered to discriminate c jets against other quarks. The result from DELPHI were reported as $|V_{cs}|=0.94 ^{+0.32}_{-0.26}\pm 0.13$. 

% In Figure~\ref{fig:relatedWorks:vcs:measurements}, the indirect measurement of $|V_{cs}|$ is from the global fit by CKMFitter to all the measured CKM elements assuming the four SM parameters. 

In addition, LEP published one indirect result. LEP measures the $Br(W\to l \nu) = (10.83 \pm 0.07 \pm 0.07) \%$ \cite{Schael:2013ita}, based on which calculates the sum of all six CMK element in the first two rows as $\sum |V_{ij}|^2 = 2.002 \pm 0.027$. Since $|V_{cs}|$ is the least precisely measured element, LEP subtract other five elements from $|V_{ij}|^2 $ and produces an indirect measurement of $|V_{cs}|=0.969\pm 0.013$. With the latest CKM values for the other five element in the first two rows, repeating LEP's calculation gives $|V_{cs}|=0.972\pm 0.013$. This thesis measures the $Br(W\to l \nu)$. Therefore, the same calculation as LEP can be done for our result to get $|V_{cs}|$ from $Br(W\to l \nu)$.


 \begin{figure}
    \centering
%     \includegraphics[width=0.45\textwidth]{vcs_meson_ds.png} \qquad
    \includegraphics[width=0.6\textwidth]{chapters/RelatedWorks/sectionVcs/figures/vcs_world_average.png}
    \caption{The $|V_{cs}|$ measurements~\cite{pdg2020}. The 2020 PDG average~\cite{pdg2020} combines the results from $D$ and $D_s$ decay.}
    \label{fig:relatedWorks:vcs:measurements}
\end{figure}





% \section{Overview of $V_{cs}$ Measurements}

The CKM matrix originates from the Yukawa couplings in the SM Higgs sector. It represents the mixing between quarks' mass eigenstates and the flavor eigenstates. When physical quarks in their mass eigenstates participate in the weak interaction, they are projected to the flavor eigenstates by the corresponding element in the CKM matrix. More details about the CKM in the standard model are discussed in Section~\ref{sec:relatedWorks:qft:gws}. 

\begin{table}[ht]
    \centering
    \setlength{\tabcolsep}{1.5em}
    \renewcommand{\arraystretch}{1.5}
    \caption{The current experimental world average of the 9 elements in the CKM matrix in the PDG \cite{pdg2020}.  }
    \resizebox{\textwidth}{!}{
    \begin{tabular}{c|c|c }
        \hline
        $|V_{ud}|=0.97370 \pm 0.00014 $         & $|V_{us}|=0.2245 \pm 0.0008$       &  $|V_{ub}|=0.00382 \pm 0.00024$         \\ \hline
        $|V_{cd}|=0.221 \pm 0.004 $                    & $|V_{cs}|=0.987 \pm 0.011$       &  $|V_{cb}|=0.00410 \pm 0.0014$                    \\ \hline
        $|V_{td}|=0.0080 \pm 0.0003 $               & $|V_{ts}|=0.0388 \pm 0.0011$       &  $|V_{tb}|=1.013 \pm 0.030$                    \\
        \hline
    \end{tabular}}
    \label{tab:relatedWorks:vcs:ckm}
\end{table}


The current experimental measurement of the 9 elements in the CKM matrix \cite{pdg2020} is shown in Table~\ref{tab:relatedWorks:vcs:ckm}. Among the 6 elements in the first two rows, $|V_{cs}|$ is measured with the least precision. The average of $|V_{cs}|$ measurements is shown in Figure~\ref{fig:relatedWorks:vcs:measurements}. Currently, there are two direct approaches to measure $|V_{cs}|$, using the D meson decay in the charm factories and using the on-shell $W\to c s$  with jet tagging in the collider experiments.

The best direct determination of $|V_{cs}|$ is from the semileptonic decay of $D$ and the leptonic decay of $D_s$ produced in the charm factory. For the results from the leptonic decay of $D_s$ meson, the branching fraction of $D_s^+ \to \mu^+ \nu$ and $D_s^+ \to \tau^+ \nu$ are both measured in the Belle \cite{Zupanc:2013byn}, CLEO \cite{Alexander:2009ux,Onyisi:2009th,Naik:2009tk}, BaBar \cite{delAmoSanchez:2010jg} and BESIII \cite{Ablikim:2016duz, Ablikim:2018jun}. Using the experimental value of mass and lifetime of $D_s$, as well as the lattice QCD calculation of the form factor $f_{D_s}$, $|V_{cs}|$ can be determined from the $D_s$ leptonic decay and yields a world average of $|V_{cs}|=0.992\pm 0.012$ \cite{Amhis:2019ckw}, where the dominating uncertainty is from the experimental error. For the results from the semileptonic decay of $D$ meson, the branching fraction of $D\to K l\nu$ is measured by CLEO-c \cite{Besson:2009uv}, Belle \cite{Widhalm:2006wz}, BaBar \cite{Aubert:2007wg} and BESIII \cite{Ablikim:2015ixa, Ablikim:2018evp}, which gives an average of $|V_{cs}|$ of $|V_{cs}|=0.939\pm 0.038$ \cite{Amhis:2019ckw} in the D meson decay. The dominant uncertainty is form the theoretical calculations of the D meson form factor with latice QCD. Combining the result from the $D$ and $D_s$ decay, the charm factories measures $|V_{cs}|=0.987\pm 0.011$ \cite{Amhis:2019ckw}.

The second direct measurement of $|V_{cs}|$ is from the on-shell $W\to c s$ decays in the collider experiments. This approach relies on jet tagging to identifies the jets originating from the c and s quarks, which is relatively difficult, especially in the hadron collider with a more complex hadron environment. Therefore, this approach is less explored compared with the $D/D_s$ approach. So far, the only published result based on the $W\to c s$  approach is from the DELPHI in the LEP, which reports $|V_{cs}|=0.94 ^{+0.32}_{-0.26}\pm 0.13$. \cite{Abreu:1998ap}

In Figure~\ref{fig:relatedWorks:vcs:measurements}, the indirect measurement of $|V_{cs}|$ is from the global fit by CKMFitter to all the measured CKM elements assuming the four SM parameters. In addition, LEP published another indirect result. LEP measures the $Br(W\to l \nu) = (10.83 \pm 0.07 \pm 0.07) \%$ \cite{Schael:2013ita}, based on which calculates the sum of all six CMK element in the first two rows as $\sum |V_{ij}|^2 = 2.002 \pm 0.027$. Since $|V_{cs}|$ is the least precisely measured element, LEP subtract other five elements from $|V_{ij}|^2 $ and produces an indirect measurement of $|V_{ij}|=0.969\pm 0.013$. This thesis measures the $Br(W\to l \nu) $ as well. Therefore, the same calculation as LEP can be done for our result to get $|V_{cs}|$ from $Br(W\to l \nu)$. The next part of this section covers about the steps of the such calculations.


 \begin{figure}
    \centering
%     \includegraphics[width=0.45\textwidth]{vcs_meson_ds.png} \qquad
    \includegraphics[width=0.6\textwidth]{chapters/RelatedWorks/sectionVcs/figures/vcs.png}
    \caption{The world average of $|V_{cs}|$ measurements. }
    \label{fig:relatedWorks:vcs:measurements}
\end{figure}

    \chapter{Physics Foundations}
\label{sec:relatedWorks}



\section{Standard Model Particles}
\label{sec:physics:smParticles}

% What is the world made of? Throughout the history, numerous theories were developed to answer the question. The ancient Greeks modeled the matter with four fundamental elements: air, water, fire, and earth, while the ancient Chinese believed in the five most essential building blocks: metal, wood, water, fire, and earth. In the modern age, the emergence of science provides a systematic approach to develop and test these models. As the experimental technology allows us to probe smaller and smaller structures, our understanding of the fundamental elements of matter evolves from molecules to atoms, then to nucleons and electrons, finally to quarks and leptons. This level of understanding was achieved by a set of exciting progress in both physics theory and experiments in the most recent century, which together led us to the Standard Model (SM), a systematic and elegant answer to the question of "what is matter", as well as "what is force".

% Standard Model (SM), since its establishment, has been very successful in making predictions for the experiments, such as the existence, properties, and behaviors of particles. On the other hand, it has been tested with remarkably high precision in many aspects in experiments such as fixed-target experiments, collider experiments, and neutrino experiments. Despite its tremendous success, SM is still not the perfect ultimate theory to settle down because it still has many limitations: the gravitational force which is one of the four fundamental interactions in nature is not included in the SM; the dark matter hinted by many astronomy observations is not modeled by any SM fermions; electromagnetic and weak forces are unified, but it does not unify the strong force; moreover, the running couplings of the three forces do not join at one point in the high energy scale. So seeking new physics beyond the standard model is one of the important topics in particle physics nowadays. 


\begin{figure}[ht]
    \centering
    \includegraphics[width=0.6\textwidth]{chapters/Physics/sectionSMParticles/figures/sm.png}
    \caption{Particles in the Standard Model. Fermions include three generations of quarks and leptons, while bosons include four gauge bosons and one Higgs boson.}
    \label{fig:physics:smParticles:sm}
\end{figure}

The standard model treats matter and the force as a set of quantum fields, excited states of which correspond to different fundamental particles. Figure~\ref{fig:physics:smParticles:sm} shows the table of the SM particles. The ``matter particles" are spin-$\frac{1}{2}$ fermions including three generations of quarks and leptons. The ``force particles" are spin-1 gauge bosons accounting for the electromagnetic, strong, weak forces. Additionally, there is a spin-0 Higgs boson that generates mass for fermions and gauge bosons. The theoretical foundation of the SM particles in Figure~\ref{fig:physics:smParticles:sm} is the quantum field theory (QFT), a theory joining quantum mechanics and special relativity.  Appendix~\ref{sec:physics:qft} summarizes the backbone of the quantum field theory for SM, covering Yang-Mills Gauge Theory, Higgs Mechanism, Glashow-Weinberg-Salam (GWS) electroweak model and Quantum Chromodynamics (QCD). This section provides an overall description of the SM particles. 


\subsection{Fermions}
\label{sec:physics:smParticles:fermions}

\begin{table}[ht]
    \centering
    \setlength{\tabcolsep}{1 em}
    \renewcommand{\arraystretch}{1.5}
    \caption{The electroweak quantum number of the first generation leptons and quarks. The second and third generation have the same electroweak quantum number as the first generation. The shown quantum numbers are isospin $T$, the third component of isospin $T^3$, charge $Q$ and hypercharge $Y$. }
    % \resizebox{\textwidth}{!}{
    \begin{tabular}{ccccc|ccccc}
    \hline
    lepton      & $T$           & $T^3$          & $Q$ & $Y$ & quark  & $T$           & $T^3$          & $Q$            & $Y$            \\
    \hline
    $\nu_{e,L}$ & $\frac{1}{2}$ & $\frac{1}{2}$  & 0   & -1  & $u_L$  & $\frac{1}{2}$ & $\frac{1}{2}$  & $\frac{2}{3}$  & $\frac{1}{3}$  \\
    $e_L$       & $\frac{1}{2}$ & $-\frac{1}{2}$ & -1  & -1  & $d_L$  & $\frac{1}{2}$ & $-\frac{1}{2}$ & $-\frac{1}{3}$ & $\frac{1}{3}$  \\
    \hline
    -           & -             & -              & -   & -   & $u_R$  & 0             & 0              & $\frac{2}{3}$  & $\frac{4}{3}$  \\
    $e_R$       & 0             & 0              & -1  & -2  & $d_R$  & 0             & 0              & $-\frac{1}{3}$ & $-\frac{2}{3}$ \\
    \hline
    \end{tabular}
    % }
    \label{tab:physics:smParticles:ewQuantumNumber}
\end{table}


\textbf{Quarks}. Three generations of quarks have been discovered: up (\PQu) and down (\PQd) being the first generation, charm (\PQc) and strange (\PQs) being the second generation, top (\PQt) and bottom (\PQb) being the third generation. The terminology ``generation" is often referred to as ``family" as well. The up, charm and top quarks have electric charge of $\frac{2}{3}$, while the electric charge of down, strange and bottom quark is $-\frac{1}{3}$. Other than electric charge, quark also carries color quantum number and thus participates in the strong interaction. The color quantum number (red, green, and blue) is analogous to the electric charge in the electromagnetic force. Each quark has its corresponding antiquark carrying an opposite electric charge and anti-color. However, neither the fractional charge nor individual color charge is observed in nature, because quarks never exist alone. Quarks and their properties only reveal during the high-energy short-distance local interactions. In the low energy scale, they are always combined in two-quark or three-quark bounded states, called mesons and baryons respectively, which are color-neutral and integer-charged. This phenomenon is known as ``quark confinement". Quarks not only couple to the electromagnetic and strong force, but are also involved in the weak interaction. The weak hypercharge and isospin of quarks are listed in the Table~\ref{tab:physics:smParticles:ewQuantumNumber}. The masses of quarks arrange from a few \MeV to 173\GeV. The heavy quarks decay into light quarks via the weak force with quark mixing. Therefore, the matter in our everyday life includes only the first generation light quarks. The most massive quark \PQt quark decays almost instantaneously to one \PQb quark and one \PW gauge boson upon its production before hadronizing into bounded states. Quark model has been successful in the classification of mesons and baryons, and explaining the observations in many experiments such as the lepton-nucleon deep-inelastic scattering, electron-positron annihilation and proton-proton hard collision.



\textbf{Leptons}. Three generations of leptons have been discovered: electrons (\Pe) and electron neutrino (\PGne) being the first generation, muon (\PGm) and muon neutrino (\PGnGm) being the second generation, tau (\PGt) and tau neutrino (\PGnGt) being the third generation. Electron, muon and tau have negative one electric charge and couple to the electromagnetic force, while all neutrinos are not charged and do not interact electromagnetically. Charged leptons can be both left-handed and right-handed, while the neutrinos can only be left-handed because right-handed neutrinos have not been experimentally observed so far. Due to the chiral nature of weak interaction, the left-handed leptons couples to both W and Z weak gauge bosons, while right-handed charged leptons do not couple to the W bosons. The quantum number of leptons are also shown in the Table~\ref{tab:physics:smParticles:ewQuantumNumber}. The masses of the charged leptons increases with the lepton generations. Charged leptons in the second and third generations have finite lifetimes. Therefore, electrons are the only charged lepton in our everyday matter. Neutrinos in the SM had been thought as massless until the discovery of neutrino oscillation. The neutrino masses were then added to the SM. But the exact values of neutrino masses are still not fully determined yet. 




\subsection{Bosons}
\label{sec:physics:smParticles:bosons}

The bosons in the SM consist of four spin-1 gauge bosons and one spin-0 Higgs boson. Four gauge bosons are responsible for the forces between fermions: the electromagnetic force is mediated by the photon $\gamma$; the strong nuclear force is propagated by gluons $g$, the weak force is carried by the \PW and \PZ bosons. In the QFT, the existence of gauge boson originates from engaging the local symmetries for the fermions Lagrangian. In other words, force is modelled as a consequence of ``gauging" the matter. Among the four gauge bosons, photon and gluon are massless while the \PW and \PZ boson are massive with $m_\PW = 80.385\pm0.015$\GeV and $m_\PZ = 90.183\pm 0.002$\GeV \cite{pdg2020}, respectively. The non-zero mass of the gauge boson breaks the gauge symmetry and causes renormalization issues, unless the mass of gauge boson is generated by the Higgs mechanism. The Higgs boson is a spin-0 boson predicted in the 1960s to solve the problems related to the gauge boson mass. It was finally confirmed exist at $m_H=125.09\pm 0.24$\GeV by the CMS~\cite{Chatrchyan:2012ufa} and the ATLAS~\cite{Aad:2012tfa} experiment at LHC in 2012. It was the last missing piece added in the SM particles. Besides the mass of gauge bosons, Higgs boson also generates fermions' masses via Yukawa couplings.




\section{Physics in the Hadron Colliders}
\label{sec:physics:ppCollision} 

When protons collide in hadron collider, it is actually the quarks and gluons inside the protons, called partons, that collide at the high center-of-mass energies. This high energy collision between partons is called the hard process and can be calculated perturbatively with the quantum field theory. However, the experimental observables from the proton-proton collision not only involves the physics in the hard process, but also includes many low-energy QCD processes happening before and after the hard process that cannot be treated with perturbative QCD (pQCD). Therefore, to properly make predictions to experiments, efforts are made to understand how partons distribute in the collided protons before the hard process, and how outcoming particles evolve in the long-distance range after the hard process. These studies yield the topics of the parton distribution function (PDF) and jet physics. In this section, brief descriptions of the PDF, hard process, and jet physics are presented. This provides a basic picture to understand event generators described in Section~\ref{sec:cmsExperiment:simulation}. Also, it helps understand the sources of some theoretical uncertainties in the \PW branching fraction analysis.

\subsection{Parton Distribution Functions}
\label{sec:physics:ppCollision:pdf} 


A proton can be pictured as three valence quarks surrounded by a cloud of soft gluon and sea quarks. For a proton with a given momentum, the probability distribution of finding a certain type of parton is described by the parton distribution functions $f_i(x)$, where $x$ denotes the fraction of the total proton momentum $p$ carried by the parton. When colliding protons, partons are actually participating in the high energy collision. As a result, the cross-section of the collision is a convolution of the cross-section of the hard process and the PDFs of the two collided partons:
\begin{equation}
    \sigma_{pp\to X } = \sum_{ij}\int dx_1 dx_2 ~ f_i(x_1, \mu^2) ~ f_j(x_2, \mu^2) ~ \hat{\sigma}_{ij\to X } (x_1 p_1, x_2 p_2,\mu^2) .
    \label{eqn:physics:qft:ppCollision:factorization}
\end{equation}

\noindent This factorizes the total cross-section into the hard collision and PDFs, where $\mu$ is the factorization scale. To make theoretical predictions in the LHC, the PDFs are one of the necessary inputs. The measurements of PDFs are primarily accomplished by the lepton-hadron deep inelastic scattering (DIS) experiments. The DIS cross-section yields the structure functions of the hadrons $F_2(x)$, which theoretically equals to the sum of PDFs weighted by the quark momentum and charge squared: $F_2(x) = \sum_i x Q^2_i f_i(x)$. The PDFs of different quarks $f_i(x)$ are extracted from the electron DIS off protons and neutron targets, using the quark symmetries between the proton and neutron. The PDFs of anti-quarks are extracted from the DIS experiments with the neutrino and anti-neutrino beams, since the intermediating $W^\pm$ bosons are capable of probing specific charge conjugated states. However, the gluon distribution is not directly measured in the DIS experiments because both the mediating photon and \PW bosons in the DIS process do not carry color charge and thus do not probe the gluons. Instead, the information about the gluon distribution is indirectly extracted from the evolution of quark PDFs in different energy scales based on the DGLAP equation:
\begin{equation}
    \mu \frac{d}{d\mu} \begin{bmatrix} f_i(x,\mu^2) \\ f_g(x,\mu^2) \end{bmatrix} = 
    \sum_j \frac{\alpha_s}{\pi} \int_x^1 
    \frac{dy}{y}
    \begin{bmatrix} P_{q_i q_j}(\frac{x}{y}) & P_{q_i g}(\frac{x}{y}) \\ P_{g q_j}(\frac{x}{y}) & P_{gg}(\frac{x}{y})) \end{bmatrix} \begin{bmatrix} f_j(y,\mu^2) \\ f_g(y, \mu^2) \end{bmatrix} ,
    \label{eqn:physics:qft:ppCollision:dglap}
\end{equation}

\noindent where $P_{ab}$ is the DGLAP splitting function, representing the probability of parton $a$ radiating another parton $b$ with a fraction momentum $z=\frac{p_b}{p_a}$. The splitting  function $P_{ab}$ is calculated by considering the tree-level Feynman diagram and reads as
\begin{equation}
\begin{split}
	P_{qq}(z) &= \frac{4}{3}\bigg[\frac{1+z^2}{1-z} \bigg]_+, P_{qg}(z)=\frac{1}{2} \bigg[z^2+(1-z)^2 \bigg], P_{gq}(z)=\frac{4}{3}\bigg[\frac{1+(1-z)^2}{z} \bigg]\\
    P_{gg}(z) &= 6 \bigg[ \frac{z}{1-z}_+ + \frac{1-z}{z} + z(1-z) \bigg] +(11-\frac{n_f}{3})\delta(1-z) .
\end{split}
\label{eqn:physics:qft:ppCollision:splitting}
\end{equation}

\noindent The DGLAP equation is a renormalization group equation (RGE) for the scale-dependent evolution of PDFs, similar to the RGE for the running of couplings in Equation~\ref{eqn:physics:qft:qcd:rge}. The driving for the PDF evolution is the splitting functions in Equation~\ref{eqn:physics:qft:ppCollision:splitting}, analogous to the role of beta function in the running of coupling constants.  An intuitive understanding of the PDF evolution is that as the probing energy increases, more and more ``soft cloud" of sea quarks and gluons are revealed. As a result, the PDFs of sea quarks and gluons increase in the low $x$ region. The DGLAP equation in Equation~\ref{eqn:physics:qft:ppCollision:dglap} is a set of two first-order linear differential equations, and the solution is
\begin{equation}
    \begin{bmatrix} f_i(x,\mu^2) \\ f_g(x,\mu^2) \end{bmatrix} = \begin{bmatrix} f_i(x,\mu_0^2) \\ f_g(x,\mu_0^2) \end{bmatrix} + 
    \frac{\alpha_s}{2\pi} \log\bigg(\frac{\mu^2}{\mu_0^2}\bigg) 
    \sum_j \int_x^1 
    \frac{dy}{y}
    \begin{bmatrix} P_{q_i q_j}(\frac{x}{y}) & P_{q_i g}(\frac{x}{y}) \\ P_{g q_j}(\frac{x}{y}) & P_{gg}(\frac{x}{y})) \end{bmatrix} \begin{bmatrix} f_j(y,\mu^2_0) \\ f_g(y,\mu^2_0) \end{bmatrix}, 
\end{equation}

\noindent where the $\mu^2$ is the variable factorization scale in Equation~\ref{eqn:physics:qft:ppCollision:factorization}, and $\mu^2_0$ is the reference scale of the renormalization group. The measured PDFs at two different energy scales  $\mu^2=10 \text{ GeV}^2$ and $\mu^2=10^4 \text{ GeV}^2$ are shown in Figure~\ref{fig:physics:ppCollision:pdf}
\begin{figure}[ht]
    \centering
    \includegraphics[width = 0.7 \textwidth]{chapters/Physics/sectionPPCollision/figures/pdf.png}
    \caption{PDF of valence quark, sea quark and gluon at $\mu^2=10 \GeV^2$ and $\mu^2=10^4 \GeV^2$ \cite{pdg2020}. The valence quarks dominate the high-x region and in total only account for about 38\% of the proton momentum. The low-x region is dominated by the sea quarks and gluons, forming a ``soft cloud'' around the valence quarks. Gluons are the major components of the ``soft cloud'', and in total carry over 40\% of the proton momentum. Comparing PDF in \emph{(left)} and \emph{(right)}, the increasing of the energy scale dramatically populates the soft gluons and soft sea quarks.  Intuitively speaking, more and more ``soft cloud" of sea quarks and gluons are revealed as the probing energy increases.}
    \label{fig:physics:ppCollision:pdf}
\end{figure}



\subsection{Hard Processes}
\label{sec:physics:ppCollision:hardProcess} 


The hard processes between partons happen in the short-distance range and can be calculated perturbatively with quantum field theories. In the LHC, the hard processes allowed by the SM include the electroweak, QCD, and Higgs interactions. Figure~\ref{fig:physics:ppCollision:hardxs} shows a summary of the total cross-section of the SM processes in the LHC measured by the experiments and predicted by the SM. For this thesis, the signal processes producing a pair of \PW bosons include \ttbar, \tW and \WW.


\begin{figure}[ht]
    \centering
    \includegraphics[width=0.99\textwidth]{chapters/Physics/sectionPPCollision/figures/SigmaNew_v0.pdf}
    \caption{Summary of the cross-sections of the SM processes in the LHC. The grey bar shows the theoretical predictions. The red, blue and green points indicate the CMS measurements or the exclusion limits at 7, 8, 13\TeV.}
    \label{fig:physics:ppCollision:hardxs}
\end{figure}



\begin{figure}[ht]
    \centering
        \feynmandiagram[small,horizontal=a to b]{
        i1 [particle=\PQq] -- [fermion] a -- [fermion] i2 [particle=\PQq],
        a -- [gluon, edge label=\Pg] b,
        f1 [particle=\PQt] -- [fermion] b -- [fermion] f2 [particle=\PQt],
        % top decay
        f1b[particle=\PQb] -- [fermion] f1 -- [photon] f1W [particle=\PW, red],
        f2b[particle=\PQb] -- [anti fermion] f2 -- [photon] f2W [particle=\PW, red],
        f1 -- [opacity=0.0] f2,
        f1W -- [opacity=0.0] f2W,
        f1b -- [opacity=0.0] f1W,
        f2b -- [opacity=0.0] f2W,
    }; \qquad
    \feynmandiagram[small,horizontal=a to b]{
        i1 [particle=\Pg] -- [gluon] a -- [gluon] i2 [particle=\Pg],
        a -- [gluon, edge label=\Pg] b,
        f1 [particle=\PQt] -- [fermion] b -- [fermion] f2 [particle=\PQt],
        % top decay
        f1b[particle=\PQb] -- [fermion] f1 -- [photon] f1W [particle=\PW, red],
        f2b[particle=\PQb] -- [anti fermion] f2 -- [photon] f2W [particle=\PW, red],
        f1 -- [opacity=0.0] f2,
        f1W -- [opacity=0.0] f2W,
        f1b -- [opacity=0.0] f1W,
        f2b -- [opacity=0.0] f2W,
    }; \qquad
    \feynmandiagram[small, vertical=a to b, horizontal=a to f1]{
        i1 [particle=\Pg] -- [gluon] a -- [anti fermion] f1 [particle=\PQt],
        a -- [fermion, edge label=\PQt] b,
        i2 [particle=\Pg] -- [gluon] b -- [fermion] f2 [particle=\PQt],
        % top decay
        f1b[particle=\PQb] -- [fermion] f1 -- [photon] f1W [particle=\PW, red],
        f2b[particle=\PQb] -- [anti fermion] f2 -- [photon] f2W [particle=\PW, red],
        f1 -- [opacity=0.0] f2,
        f1W -- [opacity=0.0] f2W,
        f1b -- [opacity=0.0] f1W,
        f2b -- [opacity=0.0] f2W,
        % i1 -- [opacity=0.0] i2,
        % f1 -- [opacity=0.0] f2,
    };
    \caption{The tree-level processes of \ttbar production in the LHC. In all three diagrams, \ttbar is produced with the QCD interaction. In the LHC, the dominant production processes are the two diagrams on the right, with two incoming gluons colliding in the s-channel and t-channel, respectively. The top quark decays into one \PW boson and one \PQb quark immediately after the production.}
    \label{fig:physics:ppCollision:tt}
\end{figure}
\noindent For \ttbar, the top quark pairs are produced with the QCD interaction. The tree-level diagrams for the \ttbar production is shown in Figure~\ref{fig:physics:ppCollision:tt}. The quark-antiquark annihilation, shown as Figure~\ref{fig:physics:ppCollision:tt} left, was the dominant process in the Tevatron, where the quark and antiquark are the valence quark in the proton and anti-proton. But in the LHC, which collides proton-proton at a higher center-of-mass energy, gluon-gluon fusion in the s channel and t channel, shown as  Figure~\ref{fig:physics:ppCollision:tt} middle and right, are the dominant diagrams. The top quark decays into one \PQb quark and one \PW boson instantly after being produced. The resulting pair of \PW bosons are used to measure \PW branching fractions in this thesis. Meanwhile, the outcoming \PQb quarks are used to tag the \ttbar events. 

% The theoretical prediction of \ttbar cross-section at LHC 13\TeV is
% \begin{equation}
%     \sigma_{tt} = 00.00 \pm 0.00 \text{~(scale)} \pm 0.00 \text{~(PDF, \alpS)~pb [FIXME]} .
% \end{equation}



\begin{figure}[ht]
    \centering
        \feynmandiagram[scale=0.7][horizontal=a to b]{
        i1 [particle=\PQb] -- [fermion] a -- [gluon] i2 [particle=\Pg],
        a -- [fermion, edge label=\PQb] b,
        f1 [particle=\PW , red] -- [photon] b -- [fermion] f2 [particle=\PQt],
        f2b[particle=\PQb] -- [anti fermion] f2 -- [photon] f2W [particle=\PW, red],
        f1 -- [opacity=0.0] f2W,
    }; \quad
            
    \feynmandiagram[scale=0.7][vertical=a to b]{
        i1 [particle=\PQb] -- [fermion] a -- [photon] f1 [particle=\PW, red],
        a -- [fermion, edge label=\PQb] b,
        i2 [particle=\Pg] -- [gluon] b -- [fermion] f2 [particle=\PQt],
        f2b[particle=\PQb] -- [anti fermion] f2 -- [photon] f2W [particle=\PW, red],
        i1 -- [opacity=0.0] i2,
        f1 -- [opacity=0.0] f2W -- [opacity=0.0] f2b,
    };
    \caption{The tree-level processes of \tW production. The incoming \PQb quark scatters off a gluon with QCD interaction and gets excited into a top quark via the electroweak interaction.}
    \label{fig:physics:ppCollision:tw}
\end{figure}
\noindent The \tW production is induced via weak interactions and has smaller cross-section compared to \ttbar production. The tree-level \tW processes are shown in Figure~\ref{fig:physics:ppCollision:tw}. One \PW boson is produced associated with top quark. The outcoming top quark decays into one \PQb quark and another \PW boson. The two \PW bosons are used for \PW measurements, while the \PQb quarks is used to tag the \tW event. 

% The SM theoretical cross section for \tW is 
% \begin{equation}
%     \sigma_{tW} = 00.00 \pm 0.00 \text{~(scale)} \pm 0.00 \text{~(PDF, \alpS)~pb [FIXME]}
% \end{equation}


\begin{figure}[ht]
    \centering
        \feynmandiagram[small,horizontal=a to b]{
        i1 [particle=\PQq] -- [fermion] a -- [fermion] i2 [particle=\PQq],
        a -- [photon, edge label=\PZ] b,
        f1 [particle=\PW, red] -- [photon] b -- [photon] f2 [particle=\PW, red],
    }; \qquad
    \feynmandiagram[small, horizontal=i1 to a, vertical=a to b]{
        i1 [particle=\PQq] -- [fermion] a -- [photon] f1 [particle=\PW, red],
        a -- [fermion, edge label=\(\PQq^{\prime\prime}\)] b,
        i2 [particle=\(\PQq^{\prime}\)] -- [anti fermion] b -- [photon] f2 [particle=\PW, red],
        i1 -- [opacity=0.0] i2,
        f1 -- [opacity=0.0] f2,
    };
    \caption{The tree-level process of \WW production. These two diagrams are both induced by electroweak interactions. }
    \label{fig:physics:ppCollision:ww}
\end{figure}
\noindent The \WW events are produced by electroweak interactions. Figure~\ref{fig:physics:ppCollision:ww} shows two major tree-level processes of the \WW production. In the first diagram, quark and antiquark annihilate into a virtual $Z/\gamma$, which then decays into \WW via the electroweak triple-gauge-coupling (TGC). In the second diagram, \WW is produced in the t-channel of quark-quark scattering. The \WW processes contribute at small cross-sections to the event selection with $n_b=0$. The treatment of the \WW process is different between the shape analysis and the counting analysis: the shape analysis treat it as a signal process, while the counting analysis which does not have any $n_b=0$ category treats it as a background. 

% The SM theoretical cross-section for \WW is 
% \begin{equation}
%     \sigma_{\WW} = 00 \pm 00  \text{~(scale)}  \pm 00  \text{~(PDF, \alpS)~pb [FIXME]}
% \end{equation}




\subsection{Jet Forming}
\label{sec:physics:ppCollision:jetForming} 

% Quarks and gluons produced by the hard process have colors, but only the colorless particles can finally reach the detector.
A few processes take place between the hard process and the particle reaching the detectors. These processes mainly include the parton shower, hadronization, and meson/baryon decay. In addition, final state radiations (FSR) of isolated photons and gluons are also possible.

In an energy scale larger than $\Lambda_{QCD}$, quarks emit gluons, and subsequently, gluons convert into quark-antiquark pairs. Therefore, an initial parton ends up to be a bunch of secondary partons. And the initial momentum is split among all the secondary partons. This process is called the parton shower. The differential phase space of the gluon emission is
\begin{equation}
    dS = \frac{2\alpha_s C_F}{\pi} \frac{dE}{E}\frac{d\theta}{\sin \theta} \frac{d\phi}{2\pi}.
\end{equation}

\noindent The gluon emission cross-section diverges at a small $\theta$ and a small energy $E$, often referred to as ``inferred colinear" divergence. Therefore the parton shower tends to soft and colinear within a narrow cone of the initial parton. 
% But these divergences are canceled by the contributions from virtual diagrams, discussed in Section~\ref{sec:physics:vcs}, resulting in a finite total cross-section of the gluon emission. 
When the energy scale drops below $\Lambda_{QCD}$, the partons start to group together and form mesons and baryons. This process is called hadronization or fragmentation. There are two popular models for hadronization, the cluster approach and the string approach, both of which contain a few algorithm parameters derived from the experimental data. For example, \PYTHIA uses the string model for hadronization \cite{ANDERSSON198331}. After hadronization, unstable mesons and baryons decay into stable particles like \PGp, \PK, \PGg, \Pe, \PGm based on the corresponding life-times and certain decaying matrix elements, such as electroweak decaying. The decay processes can be also handle by \PYTHIA. More details about the simulation of these processes in the CMS simulated events are discussed in Section~\ref{sec:cmsExperiment:simulation}.

After parton shower, hadronization, meson/baryon decay, as well as possible FSR, a parton from the hard process finally ends up to be a narrow cone of stable particles, including charge hadrons, neutral hadrons, photons, and leptons. This cone of particles can be clustered together to represent the initial seeding parton. This cluster of colinear particles is called a jet. A jet is defined by a clustering algorithm (e.g. anti-\kt) and scale parameter (e.g. $\delta R=0.4$). To reliably represent the seeding parton, a jet algorithm has to be insensitive to the soft-colinear parton showering, so-called ``inferred colinear safe''. The jet algorithm in the CMS reconstruction is discussed in Section~\ref{sec:cmsExperiment:reconstruction}.




\section{Beyond the Standard Model for Lepton Universality Violation}
\label{sec:physics:bsm}

The primary signal process for the measurement of \PW branching fractions is \ttbar. If any beyond the standard model (BSM) physics leads to a small amount of lepton universality violation (LUV) in the top decay, it will be observed as a non-universality of the \BWl result. Besides mediating top quark decay, it is possible for the BSM to affect the experimental observables in other ways. For example, BSM particle \textrm{X} could be produced on-shell in association with \PQt and \PQb quark in the t-channel $\Pp\Pp \to \PQt \PQb \mathrm{X}$, where \textrm{X} is usually at \TeV scale and decays with a non-universality manner. In this circumstance, the heavy on-shell \textrm{X} decay gives rise to highly boosted leptons. However, since this \BWl measurement is primarily based on the trailing leptons in the dilepton channels and low energy taus in the lepton-plus-tau channels, the observables in our analysis are insensitive to such scenario. Thus when estimating the sensitivity of our measurement to BSMs, we focus on BSMs' LUV effects in the top decay. 

\begin{figure}[ht]
    \centering
    % Using the layered layout
    \feynmandiagram [layered layout, small,horizontal=a to b] {
      a [particle=\PQt] -- [fermion] b -- [fermion] f1 [particle=\PQb],
      b -- [red, boson, edge label'=\PWpr] c,
      c -- [anti fermion] f2 [particle=\PGnGt],
      c -- [fermion] f3 [particle=\PGt],
    };\qquad
    \feynmandiagram [layered layout, small,horizontal=a to b] {
      a [particle=\PQt] -- [fermion] b -- [fermion] f1 [particle=\PQb],
      b -- [red, scalar, edge label'=\PSHp] c,
      c -- [anti fermion] f2 [particle=\PGnGt],
      c -- [fermion] f3 [particle=\PGt],
    };\qquad
    \feynmandiagram [layered layout, small,horizontal=a to b] {
      a [particle=\PQt] -- [fermion] b -- [fermion] f1 [particle=\PGnGt],
      b -- [red, scalar, edge label'=\PLQ] c,
      c -- [anti fermion] f2 [particle=\PQb],
      c -- [fermion] f3 [particle=\PGt],
    };
    \caption{Beyond-the-standard-Models that could enhance the tau channel in the top decay. }
   \label{fig:physics:bsm:topdecayBSM}
\end{figure}

The BSMs, which can cause LUV in the top decay, are shown in the Figure~\ref{fig:physics:bsm:topdecayBSM}. These models include but not are limited to the \PWpr with a non-universal gauge coupling, \PSHp in the two higgs doublets model (2HDM), and leptoquark or \PLQ in many grand unification models. The underline mechanism to induce LUV in these BSMs are different: for \PWpr, the coupling to the third generation lepton is elevated by the special structure of the gauge symmetry extension; for \PSHp, the coupling to tau is stronger due to tau's heavier mass comparing with electron and muon; for leptoquark, \PLQ can be generation-conserved and thus the third generation \PLQ from the top decay tends to produce the third-generation fermions. This section first defines the model-independent top decay kinematics, followed by a QFT calculation of the SM top decay width. Then it discusses the three LUV BSMs in Figure~\ref{fig:physics:bsm:topdecayBSM} and estimates the sensitivity of the CMS \BWl measurement to these BSMs.



\subsection{Kinematics of Top Quark Decay}
\label{sec:physics:bsm:kinematics}

\begin{figure}[ht]
    \centering
    % Using the layered layout
    \feynmandiagram [layered layout,horizontal=a to b] {
      a [particle=\(\PQt:\, u_a(p_a)\)] -- [fermion] b -- [fermion] f3 [particle=\(\PQb: \,\bar{u}_3(k_3)\)],
      b -- [boson, edge label=\PW, momentum'=\(q\)] c,
      c -- [fermion] f2 [particle=\(\PGnGt :\, \bar{u}_2(k_2)\)],
      c -- [anti fermion] f1 [particle=\(\PGt^+ :\, v_1(k_1)\)],
    };
    \caption{Top quark decay in the Standard Model.}
    \label{fig:physics:bsm:topdecaySM}
\end{figure}

Figure~\ref{fig:physics:bsm:topdecaySM} is the tree-level diagram of the SM top quark decay. The incoming top quark is denoted as a u-type spinor $u_a(p_a)$ with four-momentum $p_a$; the outcoming anti-tau is denoted as a v-type spinor $v_1(k_1)$ with four-momentum $k_1$; the outcoming  $\PGnGt$ and $\PQb$ are represented by the $\bar{u}$-type spinnors $\bar{u}_2(k_2)$ and $\bar{u}_3(k_3)$ with four-momentum $k_2$ and $k_3$, respectively. Note that this correspondence between the spinnors $\{ \bar{u}_2(k_2), \bar{u}_3(k_3)\}$ and particles $\{\PGnGt, \PQb\}$ are the same when \PWpr or \PSHp mediates the top decay, but are swapped for the leptoquark. The four-momentum exchanged between the fermion currents is denoted as $q=k_1+k_2 = p_a - k_3$. Using the spinnor representation in Figure~\ref{fig:physics:bsm:topdecaySM}, the pair-wised inner products of the four-momentums in the center-of-mass frame can be calculated based on the energy-momentum conservation:
\begin{equation}
\begin{split}
	q^2 &=  m_\PQt^2 + m_3^2  -2 m_\PQt E_3  \\
    p_a \cdot k_1 &= m_\PQt E_1 \\
    p_a \cdot k_2 &= m_\PQt^2  -m_\PQt E_3-m_\PQt E_1  \\
    p_a \cdot k_3 &= m_\PQt E_3 \\
    k_1 \cdot k_2 &= m_\PQt^2/2 - m_1^2/2 + m_3^2/2 - m_\PQt E_3 \\
    k_2\cdot k_3 &=  m_\PQt^2/2 + m_1^2/2 - m_3^2/2 + m_\PQt E_1 \\
    k_1\cdot k_3 &=   -m_\PQt^2/2 - m_1^2/2 - m_3^2/2 + m_\PQt E_1 + m_\PQt E_3 ,
\end{split}
\label{eqn:physics:bsm:innerProduct}
\end{equation}

\noindent where all terms related to $m_2$ are neglected because neutrino is massless, and all the inner products are represented in terms of $ ( E_1,E_3 )$ with mass constants. It is equivalent to use $ ( E_2,E_3 )$ or $ ( E_1,E_2 )$ as variables. The top total width is proportional to the integral of the matrix element $\mSqBar $ over the three-body decay phase space $\mathrm{PS_3}$
\begin{equation}
	\Gamma = \frac{1}{2 m_\PQt} \int \mSqBar \; d\mathrm{PS_3} = \frac{1}{64 \pi^3 m_\PQt} \int_{0}^{m_\PQt/2} dE_3 \int_{m_\PQt/2-E_3}^{m_\PQt/2} dE_1 \mSqBar ,
    \label{eqn:physics:bsm:decayWidth}
\end{equation}




\noindent where the integral over $\mathrm{PS_3}$ is parametrized by $ ( E_1,E_3 )$.



\subsection{Standard Model Top Width}
\label{sec:physics:bsm:smTopDecay}

In the SM, the total decay width of top quark in Figure~\ref{fig:physics:bsm:topdecaySM} can be calculated in two ways: the narrow width approximation and the $\mSqBar $ integral in Equation~\ref{eqn:physics:bsm:decayWidth}. With narrow width approximation (NWA), the top width approximately equals to the product of top width $\Gamma_{\PQt \to \PQb \PW}$ and \PW tauonic branching fraction \BWt.
\begin{equation}
\begin{split}
    \Gamma^{\rm NWA}_{\PQt \to \PQb \PGt \PGn} &= \Gamma_{\PQt \to \PQb \PW} \times \BWt = \frac{g^2 m_\PQt }{64\pi} \cdot \frac{m_\PQt^2}{m_\PW^2} \bigg[ 1+2 \frac{m_\PW^2}{m_\PQt^2}\bigg] \bigg[1-\frac{m_\PW^2}{m_\PQt^2} \bigg]^2 \times 10.8\%  = 0.157 \GeV
\end{split}
\label{eqn:physics:bsm:nwa}
\end{equation}

\noindent Because the width of \PW is indeed relatively small, the NWA provides a good approximation to top tauonic width. But for the BSMs, the propagator's narrow width condition does not necessarily hold. Thus, it is useful to calculate tree-level diagram using the $\mSqBar $ integral in the SM case and then repeat it for BSMs. In the $\mSqBar $ integral, the finite non-zero width of \PW is taken into account. At tree-level, the \PW width is
\begin{equation}
	\Gamma_{\PW} = 9 \times \frac{g^2 m_\PW}{48 \pi} = 2.07 \GeV,
    \label{eqn:physics:bsm:wWidth}
\end{equation}

\noindent where the factor $9=3+2\times 3$ is the multiplicity of the three lepton generations plus two possible quark generations with three colors. The next leading order width $\Gamma_{\PW} ^{NLO}$ due to QCD corrections is discussed in the next section. In Feynman rule, the propagator of massive gauge vector boson has a gauge parameter $\xi$. Under the unitary gauge, $\xi$ is set to infinity $\xi\to \infty$ and the Feynman rule for the massive vector propagator is $\frac{g^{\mu \nu} - q^\mu q^\nu/M^2}{q^2-M^2 }$. The propagator has a pole on the real axis at the boson mass $M$. When the massive vector is not stable and thus has a total decay width of $\Gamma$, its self-energy induces a finite but non-zero imaginary part of the pole which is proportional to the propagator's mass $M$ and the total width $\Gamma$. This leads to the Breit–Wigner propagator:
\begin{equation}
    \feynmandiagram [inline=(d.base), horizontal=d to b] {
        d -- [boson, edge label=\PW, momentum'=\(q\)] b ,
    }; = -i \frac{g^{\mu \nu} - (1-\xi) \frac{q^\mu q^\nu}{q^2 - \xi m^2_{\PW} } }{q^2 - m^2_{\PW} + i m_\PW \Gamma_{\PW} }
    \overset{\mathrm{ \xi \to \infty}}{=} 
    -i \frac{g^{\mu \nu} - q^\mu q^\nu/m^2_{\PW}  }{q^2 - m^2_{\PW} + i m_\PW \Gamma_{\PW} }
   	\label{eqn:physics:bsm:wPropagator}
\end{equation}



\noindent Now we can build the full matrix element with the Feynman rule. The amplitude takes the form of a Breit–Wigner propagator sandwiched by two fermion currents and scaled by the coupling constant squared. The amplitude and its  complex conjugate read as
\begin{equation}
	\mathcal{M}  =  -i (\frac{g }{\sqrt{2}})^2 \cdot 
	\big[ \bar{u}_3 \gamma_\mu P_L u_a \big] 
	\frac{g^{\mu \nu} - q^\mu q^\nu/M^2_{\PW}}{q^2-m^2_{\PW} + i m_\PW \Gamma_{\PW}} 
	\big[ \bar{u}_2 \gamma_\nu P_L v_1 \big] 
    \label{eqn:physics:bsm:smTopDecay:smTopDecay:m}
\end{equation}
\begin{equation}
	\mathcal{M}^*  =  i (\frac{g }{\sqrt{2}})^2 \cdot 
	\big[ \bar{v}_1 P_R \gamma_\rho  u_2 \big] 
	\frac{g^{\rho \sigma} - q^\rho q^\sigma /M^2_{\PW}}{q^2-m^2_{\PW} - i m_\PW \Gamma_{\PW}} 
	\big[ \bar{u}_a P_R \gamma_\sigma  u_3 \big] 
\end{equation}

\noindent The amplitude squared is the product of $\mathcal{M}$  and $\mathcal{M^*} $ :
\begin{equation}
\begin{split}
	|\mathcal{M}|^2  = &  \frac{g^4}{4} \frac{1}{ (q^2-m^2_{\PW})^2 +  m^2_{\PW} \Gamma^2_{\PW} } \cdot \big (g^{\mu \nu} - q^\mu q^\nu/M^2_{\PW} \big) \big(g^{\rho \sigma} - q^\rho q^\sigma / M^2_{\PW}\big)  \cdot  \\
    &\big[ \bar{u}_3 \gamma_\mu P_L u_a \big]  \big[ \bar{u}_a P_R \gamma_\sigma  u_3 \big] \cdot \big[ \bar{u}_2 \gamma_\nu P_L v_1 \big]  \big[ \bar{v}_1 P_R \gamma_\rho  u_2 \big] \\
    = & \frac{g^4}{4} \frac{1}{ (q^2-m^2_{\PW})^2 +  m^2_{\PW} \Gamma^2_{\PW} } \cdot \\
    & \big (g^{\mu \nu} - (p_a^\mu-k_3^\mu) (k_1^\nu+k_2^\nu) /M^2_{\PW} \big) \big(g^{\rho \sigma} - (k_1^\rho+k_2^\rho) (p_a^\sigma -k_3^\sigma )/ M^2_{\PW}\big)  \cdot \\
    & \big[ \bar{u}_3 \gamma_\mu P_L u_a \big]  \big[ \bar{u}_a P_R \gamma_\sigma  u_3 \big] \cdot \big[ \bar{u}_2 \gamma_\nu P_L v_1 \big]  \big[ \bar{v}_1 P_R \gamma_\rho  u_2 \big] .
\end{split}
\end{equation}

\noindent The top quarks produced in the LHC shown in Figure~\ref{fig:physics:ppCollision:tt} can be be treated as unpolarized. So for the average of $|\mathcal{M}|^2$, we average over initial state spins and sum over final state spins. This step uses some properties of the spinnors and the trace calculation of gamma matrices, including trace for swapping spinnors $\bar{u} u = Tr[ u \bar{u}] $ , the spin completeness relation for spin sum $\sum_s u(p,s) \bar{u}(p,s) = \slashed{p} + m $  and $\sum_s v(p,s) \bar{v}(p,s) = \slashed{p} - m$. The average amplitude squared reads as
\begin{equation}
\begin{split}
	\mSqBar = & \frac{1}{2} \sum_s \mathcal{M}^* \mathcal{M} =
% 	\frac{g^4}{8} \frac{1}{ (q^2-m^2_{\PW})^2 +  m^2_{\PW} \Gamma^2_{\PW} } \bigg \{  
% 	\text{Tr}[\gamma^\nu \slashed{p}_a P_R \gamma^\rho \slashed{k}_3] \, \text{Tr}[\gamma_\nu \slashed{k}_2 P_R \gamma_\rho \slashed{k}_1] \\
% 	& \qquad - \frac{1}{m^2_{\PW}} \text{Tr}[\slashed{q} \slashed{p}_a P_R \gamma^\rho \slashed{k}_3] \, \text{Tr}[\slashed{q} \slashed{k}_2 P_R \gamma_\rho \slashed{k}_1] 
% 	- \frac{1}{m^2_{\PW}} \text{Tr}[\gamma^\nu \slashed{p}_a P_R \slashed{q} \slashed{k}_3] \,\text{Tr}[\gamma_\nu \slashed{k}_2 P_R \slashed{q} \slashed{k}_1] \\
% 	& \qquad + \frac{1}{m^4_{\PW}} \text{Tr}[\slashed{q} \slashed{p}_a P_R \slashed{q} \slashed{k}_3]\, \text{Tr}[\slashed{q} \slashed{k}_2 P_R \slashed{q} \slashed{k}_1] 
% 	\bigg\} \\
% 	& =  \frac{g^4}{8} \frac{1}{ (q^2-m^2_{\PW})^2 +  m^2_{\PW} \Gamma^2_{\PW} } \bigg \{  
% 	\text{Tr}[\gamma^\nu \slashed{p}_a P_R \gamma^\rho \slashed{k}_3] \, 
% 	\text{Tr}[\gamma_\nu \slashed{k}_2 P_R \gamma_\rho \slashed{k}_1] \\
% 	& \qquad - \frac{1}{m^2_{\PW}} \text{Tr}[(\slashed{p}_a -\slashed{k}_3) \slashed{p}_a P_R \gamma^\rho \slashed{k}_3] \, 
% 	\text{Tr}[(\slashed{k}_1 + \slashed{k}_2) \slashed{k}_2 P_R \gamma_\rho \slashed{k}_1] \\
% 	& \qquad - \frac{1}{m^2_{\PW}} \text{Tr}[\gamma^\nu \slashed{p}_a P_R (\slashed{p}_a -\slashed{k}_3)\slashed{k}_3] \,
% 	\text{Tr}[\gamma_\nu \slashed{k}_2 P_R (\slashed{k}_1 + \slashed{k}_2) \slashed{k}_1] \\
% 	& \qquad + \frac{1}{m^4_{\PW}} \text{Tr}[(\slashed{p}_a -\slashed{k}_3)\slashed{p}_a P_R (\slashed{p}_a -\slashed{k}_3) \slashed{k}_3]\, 
% 	\text{Tr}[(\slashed{k}_1 + \slashed{k}_2) \slashed{k}_2 P_R (\slashed{k}_1 + \slashed{k}_2) \slashed{k}_1]
% 	\bigg\}  \\
	\frac{g^4}{8} \frac{1}{ (q^2-m^2_{\PW})^2 +  m^2_{\PW} \Gamma^2_{\PW} } \bigg \{  \\
	& 16  (  k_2 \cdot k_3) (  k_1 \cdot p_a) + \frac{8}{m^2_{\PW}} [ m_1^2 m_3^2 (k_2 \cdot p_a) - m_1^2 m_\PQt^2 (k_2 \cdot k_3)] \\
	&  + \frac{4}{m^4_{\PW}} [ (m_1^2 m_\PQt^2 + m_1^2 m_3^2) (k_1 \cdot k_2) (k_3 \cdot p_a) - 2  m_1^2 m_3^2 m_\PQt^2  (k_1 \cdot k_2) ]\bigg\} .
\end{split}
\end{equation}

\noindent Since tau and b quark mass are much less than the top mass, $m_1 \ll m_\PQt $ and $m_3 \ll m_\PQt$, the terms with $m_1$ and $m_3$ can be neglected. Eventually, the average amplitude squared becomes
\begin{equation}
	\mSqBar =  g^4 \frac{2  (  k_2 \cdot k_3) (  k_1 \cdot p_a) }{ (  q^2-m^2_{\PW})^2 +  m^2_{\PW} \Gamma^2_{\PW} }  
    \label{eqn:physics:bsm:smTopDecay:smTopDecay:m2}
\end{equation}

\noindent where the inner products $(  k_2 \cdot k_3)$ and $ (  k_1 \cdot p_a) $ can be represent in terms of $(E_1,E_3)$ using Equation~\ref{eqn:physics:bsm:innerProduct}, which essentially parametrizes $\mSqBar $ as a function on the $(E_1,E_3)$ plane shown in Figure~\ref{fig:physics:bsm:smTopDecay:smM2}. The decay kinematic constraint corresponds to a triangle area on the  $(E_1,E_3)$ plane. The top total width is obtained from the $\mSqBar $ integral within this triangle area
\begin{equation}
    \begin{split}
         \WidthTopTauSM = \frac{g^4}{64 \pi^3 m_\PQt} \int_{0}^{m_\PQt/2} d E_3 \int_{m_\PQt/2-E_3}^{m_\PQt/2} d E_1 \frac{m_\PQt^3 E_1 - 2 m_\PQt^2 E_1^2}{(-2 m_\PQt E_3  + m_\PQt^2  -m_\PW^2)^2 + m^2_{\PW} \Gamma^2_{\PW}},
    \end{split}
\end{equation}



\begin{figure}
\centering
    \includegraphics[width=0.4 \textwidth]{chapters/Physics/sectionBSM/figures/SM.png}
    \caption{The Standard Model $\mSqBar $ as a function on the $(E_1,E_3)$ plane. It shows a sharp peak at $E_3 = \frac{m^2_\PQt - m^2_{\PW}}{2 m_\PQt} = 67.8 $\GeV. \WidthTopTauSM is proportional to the integral of $\mSqBar $ within a triangle region on the $(E_1,E_3)$ plane. }
    \label{fig:physics:bsm:smTopDecay:smM2}
\end{figure}







\noindent where one can plug in $m_\PQt= 173.1 \GeV, m_\PW= 80.6 \GeV, g=0.64 $  and $\Gamma_{\PW} = 2.07\GeV$ and evaluate the numerical value of the integral:
\begin{equation}
         \WidthTopTauSM = 0.154 \GeV .
\end{equation} 


\noindent This result from the tree-level QFT calculation with Feynman rule  $\WidthTopTauSM = 0.154 \GeV$ agrees well with the one calculated with the narrow width approximation. 





 
\subsection{Models with $\rm W^\prime$}
\label{sec:physics:bsm:WPrime}

\subsubsection{Model Overview}
\PWpr is a hypothetical massive gauge boson that couples to the electroweak charge current in many BSMs. Many direct searches for \PWpr boson have been conducted in the CMS and ATLAS at the LHC, including searches in the $\PWpr\to \PGt \PGn $  channel (tau plus \MET) \cite{Sirunyan:2018lbg, Khachatryan:2015pua,Aaboud:2018vgh}, $\PWpr\to \ell \PGn$ channel (electron or muon plus \MET) \cite{Sirunyan:2018mpc, Aaboud:2017efa}, $\PWpr\to \PW \PZ$ channel\cite{Sirunyan:2018ivv, Aaboud:2017eta}, $\PWpr\to \PQq_1 \PQq_2$ channel \cite{Sirunyan:2016iap, Aaboud:2017yvp}, and $\PWpr\to \PQt \PQb$ channel \cite{Sirunyan:2017vkm, Aaboud:2018juj}. The direct searches for \PWpr are usually model-independent, where an excess of data with respect to the SM prediction is searched in the corresponding mass spectrum. Then the limits of the \PWpr mass are set in the context of Sequential Standard Model (SSM), in which no specific assumption is made on the BSM gauge structures, and \PWpr coupling to fermions is the same as \PW coupling to fermion in the SM. So SSM is the most generic \PWpr BSM with the least extra assumption on top of SM. The ATLAS experiment has excluded an SSM \PWpr for masses below 3.7\TeV in the $\PGt+\MET$ channel. The CMS experiment has excluded an SSM \PWpr for masses below 5.2\TeV in the combination of electron and muon channels. The current PDG combined limit of the $m_{\PWpr}$ is also 5.2\TeV in the context of SSM. In contrast to the direct search, interpreting the potential LUV in the \PWpr-mediated top decay is more model-dependent. This subsection presents an overview of the \PWpr BSMs and then focuses on the most relevant \PWpr models, the Nonuniversal G221 model.




One of the most common ways to model the new physics is to extend the structure of the SM gauge symmetry. The extension of the gauge symmetry consequently introduces new gauge bosons, such as \PWpr. Though there are many possible ways to extend the SM $SU(2)_L \times U(1)_Y $ to a larger symmetry group, such as the grand unification models with a SU(5) symmetry, one of the simplest and the most widely studied extension is $SU(2)_1 \times SU(2)_2 \times U(1)_X $. Models based on such gauge extension are commonly called G221 models \cite{Hsieh:2010zr}. Though having the same underline gauge structure, different G221 models can embed different fermion doublets in the $SU(2)_1$ and $SU(2)_2$ groups, thus predicting different physics. Depending on the physical contents assigned to the $SU(2)_1 \times SU(2)_2 \times U(1)_X $ group, the G221 models are classified into three types: Left-Right \cite{PhysRevD.11.2558}, Ununified \cite{Chivukula:1994qw, GEORGI1990541} and Nonuniversal \cite{PhysRevLett.47.1788, MULLER1997192, PhysRevD.81.015006}.




\begin{itemize}
    \item Left-Right. The $SU(2)_1$ and $SU(2)_2$ in the left-right G221 describe the left-handed and right-handed fermion doublets, respectively. The fermion doublets involve both lepton and quarks. There are variations in which the right-handed fermion doublets are comprised of only leptons or quarks. The left-handed fermion doublets in the $SU(2)_1$ group are the same as those in the SM; the right-handed fermion doublet in the $SU(2)_2$ assumes the existence of right-handed neutrinos with masses beyond the\TeV scale. In the low energy domain, the combination of the $SU(2)_2$ and the BSM $U(1)_X$ symmetry spontaneously breaks into the SM hypercharge symmetry $U(1)_Y$. Namely, the BSM breaking step is $SU(2)_2 \times U(1)_X  \to U(1)_Y$, which gives masses to the new gauge bosons like \PWpr. While the \PW couples to left-handed doublets like SM, \PWpr couples to the right-handed doublets. Besides, \PWpr could have a suppressed coupling to the left-handed doublets via the W-\PWpr mixing. 
    
    \item Ununified. The $SU(2)_1$ and $SU(2)_2$ in the ununified G221 describe the lepton doublets and quark doublets, respectively. The $U(1)_X$ is the same as the SM hypercharge symmetry,  $U(1)_X=U(1)_Y$. The name ``ununified" highlights that leptons and quarks have separate $SU(2)$ symmetries. The two separate quark and lepton symmetries break into a single symmetry, the SM $SU(2)_L$, in the low energy domain. Namely, the BSM breaking step is $SU(2)_1 \times SU(2)_2 \to SU(2)_L$, which gives masses to the new gauge bosons like \PWpr. The coupling of \PWpr is mainly to the quark currents, which leads to a potential enhancement in the quark-involved processes.

    \item Nonuniversal. The $SU(2)_1$ and $SU(2)_2$ in the nonuniversal G221, sometimes referred to as nonuniversal gauge interaction models (NUGIM), describe the fermion doublets in the first two generations and fermion doublets in the third generation, respectively. The $U(1)_X$ is the same as the SM hypercharge symmetry,  $U(1)_X=U(1)_Y$. The name ``nonuniversal" implies that the first two generations and the third generation are embedded into separate $SU(2)$ groups. The BSM breaking is the same as the ununified G221, $SU(2)_1 \times SU(2)_2 \to SU(2)_L$. There is a mixing angle $\theta_E$ between the $SU(2)_1$ and $SU(2)_2$ groups. Thus, \PWpr couplings to the third generation and the first two generations are scaled by $\cot \theta_E$ and $\tan \theta_E$ respectively, leading to potential nonuniversal effects in the weak processes.
\end{itemize}




Among the three types, the most relevant model is NUGIM which is intended to violate the universality in the weak sector. Now, we focus on our sensitivity to this model. The interaction Lagrangian in the NUGIM originates from the covariant derivatives for light (gen=1,2) and heavy (gen=3) fermions. Denoting the coupling constants in the $SU(2)_1 , SU(2)_2, U(1)_X$ as $g_1, g_2, g'$, the covariant derivatives are
\begin{equation}
\begin{split}
	D_\mu \psi_{gen=1,2} &= \big[ \partial_\mu -ig_1W^a_{1\mu} T^a P_L - ig'YB_\mu\big] \psi_{gen=1,2}  \\
    D_\mu \psi_{gen=3} &= \big[ \partial_\mu -ig_2W^a_{2\mu} T^a P_L - ig'YB_\mu\big] \psi_{gen=3} 
\end{split}
\label{eqn:physics:bsm:WPrime:derivative}
\end{equation}

\noindent Given the mixing angle $\theta_E$ between the $SU(2)_1$ and $SU(2)_2$ group, the underlying coupling constants $g_1, g_2$ are related to the SM weak coupling constant $g$ by
\begin{equation}
	g_1=g/ \cos \theta_E, \quad g_2=g/ \sin \theta_E.
\end{equation}

\noindent And the SM $W^a$ triplet fields and new BSM triplet fields $\PWpr^a$  are the mixing states of $W_1^a$ and $W_2^a$:
\begin{equation}
	\PW^a = \frac{g_1 W^a_1 + g_2 W_2^a}{\sqrt{g_1^2+g_2^2}} \quad \PWpr^a = \frac{-g_1 W^a_1 + g_2 W_2^a}{\sqrt{g_1^2+g_2^2}}
\end{equation}

\noindent NUGIM employs two higgs doublets to generate mass for gauge bosons, including \PW and \PWpr. Different from the standard two higgs doublets model (2HDM) which is discussed in Section~\ref{sec:physics:bsm:chargedHiggs}, here the two higgs doublets are responsible for the first two generations and the third generation fermions. A large Higgs vacuum expectation value (VEV)  ratio between the two doublets, $\tan \beta$, can explain the relative smallness of masses in the first two generations compared to the top quark, whereas it does not explain the hierarchy $m_\PQt > m_\PQb$. This is in contrast to the 2HDM type-II models where large $\tan \beta$ can explain the hierarchy $m_\PQb\ll m_\PQt$, but not $m_\PQu, m_\PQc  \ll m_\PQt$. Given the covariant derivatives in Equation~\ref{eqn:physics:bsm:WPrime:derivative} and the mixing angle $\theta_E$, the Feynman rule for \PWpr couplings \cite{Edelhauser:2014yra} to the light and heavy fermion currents are 
\begin{equation}
\begin{split}
	\PAQt  \PQb \PWpr^+ , \bar{\PGt} \PGnGt \PWpr^+ \Longrightarrow -\frac{i g}{\sqrt{2}} \cot_E \gamma^\mu P_L \\
    \PAQu \PQd \PWpr^+ , \bar{\Pe} \PGne \PWpr^+ \Longrightarrow -\frac{i g}{\sqrt{2}} \tan_E \gamma^\mu P_L
\end{split}
\label{eqn:physics:bsm:WPrime:feynmanRule}
\end{equation}

\noindent and the Feynman rule for the propagator \PWpr takes a similar form to the SM \PW in Equation~\ref{eqn:physics:bsm:wPropagator}:
\begin{equation}
    \feynmandiagram [inline=(d.base), horizontal=d to b] {
        d -- [boson, edge label=\PWpr, momentum'=\(q\)] b ,
    }; =
    -i \frac{g^{\mu \nu} - q^\mu q^\nu/m^2_{\PWpr}  }{q^2 - m^2_{\PWpr} + i m_{\PWpr} \Gamma_{\PWpr} } .
\end{equation}


In Equation~\ref{eqn:physics:bsm:WPrime:feynmanRule}, the larger is the parameter $\cot_E$, the more enhancement to the third generation and more suppression to the first two generations there would be. So $\cot_E$ is the key parameter in the NUGIM to control the degree of ``nonuniversality". Also based on  Equation~\ref{eqn:physics:bsm:WPrime:feynmanRule}, the total width and the branching fraction of \PWpr can be calculated \cite{Edelhauser:2014yra}. If we denote the decay rate of SM \PW to one generation of lepton and neutrino in Equation~\ref{eqn:physics:bsm:wWidth} as $\Gamma_\ell=\frac{g^2 m_\PW}{48 \pi}$, the partial widths of \PWpr to the first two generations and the third generation fermion are 
\begin{equation}
\begin{split}
	\Gamma_{\PWpr\to \PQt \PQb}   &= 3  \cot_E^2  \frac{\PWpr}{\PW}\Gamma_\ell  = 3  \cot_E^2  \frac{g^2 m_{\PWpr}}{48 \pi}, \\
    \Gamma_{\PWpr\to \PQu \PQd}   &= 3  \tan_E^2  \frac{\PWpr}{\PW}\Gamma_\ell  = 3  \tan_E^2  \frac{g^2 m_{\PWpr}}{48 \pi}, \\
	\Gamma_{\PWpr\to \PGt \PGnGt} &= \cot_E^2  \frac{\PWpr}{\PW}\Gamma_\ell  =  \cot_E^2  \frac{g^2 m_{\PWpr}}{48 \pi}, \\
    \Gamma_{\PWpr\to \Pe \PGne}   &= \tan_E^2  \frac{\PWpr}{\PW}\Gamma_\ell  =  \tan_E^2  \frac{g^2 m_{\PWpr}}{48 \pi}.
\end{split}
\end{equation}

\noindent It is also allowed that \PWpr decays into SM Higgs and \PW boson, $\PWpr \to \PH +\PW$, width of which is $ \Gamma_{\PWpr\to \PH \PW}  = \frac{1}{4} \Gamma_{\PWpr\to \PGt \PGnGt} $  in the limit of large $\tan \beta$ corresponding to the SM fermion hierarchy \cite{KIM2012367}. Thus the total width of \PWpr sums up the hadronic, leptonic and Higgs emitting width;
\begin{equation}
	\Gamma_{\PWpr} = \frac{g^2 m_{\PWpr}}{48 \pi} \big [ (3+1+\frac{1}{4}) \cot^2_E + (6+2)\tan_E^2 \big] .
\end{equation}

\noindent where the width is proportional to the $ \cot^2_E $ or $ 1/\cot^2_E$ when $\cot^2_E \gg 1$ or $1 \gg \cot^2_E > 0$, respectively.  So NUGIM has a built-in upper and lower boundaries for $\cot_E$, constrained by the width of \PWpr. The parameter space beyond these boundaries predicts a \PWpr with a width too wide to be physical.  Take a conservatively large width as an example. If the width is smaller than half of the \PWpr mass, then 
\begin{equation}
    0.21 < \cot_E < 6.48.
\end{equation}


\begin{figure}[ht]
    \centering
    \includegraphics[width=0.4\textwidth]{chapters/Physics/sectionBSM/figures/WPDecayBr.png}
    \caption{The branching fraction of \PWpr in the nonuniversal gauge interaction model (NUGIM) as a function of $\cot_E$ \cite{Sirunyan:2018lbg}, where $\cot_E$ is the parameter controlling the degree of nonuniversality.}
    \label{fig:physics:bsm:WPrime:braching}
\end{figure}



\noindent The branching fraction of \PWpr is shown in Figure~\ref{fig:physics:bsm:WPrime:braching}. When $\cot_E > 2$, \PWpr decays dominantly to the third family fermions and the branching ratio to the first two generation is neglectable. The situation is reversed when $\cot_E<0.5$ .


\subsubsection{Enhancement of Tauonic Decay}

The total $\mSqBar$ in the NUGIM consists of the SM contribution in Section~\ref{sec:physics:bsm:smTopDecay}, plus contributions from \PWpr propagator (denoted as $\mSqBar _{\PWpr} $) and the interference between \PW and \PWpr (denoted as $\mSqBar_{\mathrm{int}} $). Namely,
\begin{equation}
	\mSqBar  = \mSqBar _{\PW} +  \mSqBar _{\PWpr} +  \mSqBar _{\mathrm{int}} .
\end{equation}

\noindent where the SM term $\mSqBar _{\PW} $  has been calculated in Equation~\ref{eqn:physics:bsm:smTopDecay:smTopDecay:m2}, while $\mSqBar _{\PWpr} +  \mSqBar _{\mathrm{int}}$ is the new BSM contribution to evaluate. This BSM contribution enhances the tau channel in the top decay. Eventually, the ratio of this BSM contribution over the SM contribution will be estimated and compared with our experimental precision. Now consider $\cot_E > 2$ where the BSM branching fraction to first and second generation of leptons are neglectable. With Feynman rule, the matrix element of top quark's tauonic decay mediated by \PWpr, similar to Equation~\ref{eqn:physics:bsm:smTopDecay:smTopDecay:m}, is spelt as the \PWpr propagator sandwiched by two fermion currents and scaled by the coupling constant squared:
\begin{equation}
\begin{split}
	i \mathcal{M}_{\PWpr}  & =  (\frac{g \cot_E}{\sqrt{2}})^2 \cdot 
	\big[ \bar{u}_3 \gamma_\mu P_L u_a \big] 
	\frac{g^{\mu \nu} - q^\mu q^\nu/M^2_{\PWpr}}{q^2-m^2_{\PWpr} + i m_{\PWpr} \Gamma_{\PWpr}} 
	\big[ \bar{u}_2 \gamma_\nu P_L v_1 \big] .
\end{split}
\end{equation}

\noindent The calculation of the average amplitude squared  $\mSqBar _{\PWpr} $ and $\mSqBar _{\mathrm{int}}$ has the same mathematical process as that in Section~\ref{sec:physics:bsm:smTopDecay}. First sum the spins; then evaluate the traces of gamma matrices. The result of such calculation reads as
\begin{equation}
	\mSqBar_{\PWpr} =  g^4 \cot_E^4 \frac{2  (  k_2 \cdot k_3) (  k_1 \cdot p_a) }{ (  q^2-m^2_{\PWpr})^2 +  m^2_{\PWpr} \Gamma^2_{\PWpr} }  
\end{equation}

\noindent and
\begin{equation}
\begin{split}
    \mSqBar _{\mathrm{int}} = &   2 \overline{ Re[\mathcal{M}^*_{\PW} \mathcal{M}_{\PWpr}] }  \\
    =& 2 g ^4\cot^2_E  \cdot  [2  (  k_2 \cdot k_3) (  k_1 \cdot p_a) ] \cdot \\
    &\frac 
    {( q^2-m^2_{\PW}) ( q^2-m^2_{\PWpr}) + m^2_{\PW}  m^2_{\PWpr}  \Gamma^2_{\PW} \Gamma^2_{\PWpr} }
    { \big[ ( q^2-m^2_{\PW})^2 +  m^2_{\PW} \Gamma^2_{\PW} \big] \big[ (  q^2-m^2_{\PWpr})^2 +  m^2_{\PWpr} \Gamma^2_{\PWpr} \big] }   
    ,
\end{split}
\end{equation}

\noindent where the inner products $(  k_2 \cdot k_3)$ and $ (  k_1 \cdot p_a) $ can be represented in terms of $(E_1,E_3)$ using Equation~\ref{eqn:physics:bsm:innerProduct}. Then $\mSqBar _{\PWpr} $ and $\mSqBar _{\mathrm{int}}$ become 2D functions on the $(E_1,E_3)$ plane.  Figure~\ref{fig:physics:bsm:WPrime:m2} shows a visualization of the $\mSqBar _{\PWpr} $ and $\mSqBar _{\mathrm{int}}$ on the $(E_1,E_3)$ plane, overlapped with the SM $\mSqBar _{\PW}$ from Section~\ref{sec:physics:bsm:smTopDecay}. The $\mSqBar _{\PW}$, $\mSqBar _{\PWpr} $ and $\mSqBar _{\mathrm{int}}$ are shown as orange, blue and green surface, respectively. The upper left plot in Figure~\ref{fig:physics:bsm:WPrime:m2} shows the scenario of $\cot_E=1, m_{\PWpr}=140\GeV$, where \PWpr is light enough to be on-shell from the top decay and  $\mSqBar _{\PWpr} $ has a clear peak at $E_3 = (m^2_\PQt - m^2_{\PWpr})/2 m_\PQt $. Due to this on-shell peak,  $\mSqBar _{\PWpr} $ is much larger than the interference term  $\mSqBar _{\mathrm{int}}$  and is the dominating BSM contribution. As $m_{\PWpr}$ increases, the on-shell \PWpr peak moves to left towards smaller $E_3$ and eventually disappear in the triangle area when $m_{\PWpr} > m_{t}$. Upper right plot in Figure~\ref{fig:physics:bsm:WPrime:m2} shows the scenario of $\cot_E=1, m_{\PWpr}=300 \GeV$, where the blue peak is out side the triangle area and the leading BSM contribution is from the interference. The lower plot in Figure~\ref{fig:physics:bsm:WPrime:m2} increases the $\cot_E$  with $\cot_E=4, m_{\PWpr}=300 \GeV$, where \PWpr width is wider and the interference becomes stronger. 





\begin{figure}[ht]
    \centering
    \includegraphics[width=0.49\textwidth]{chapters/Physics/sectionBSM/figures/WPrime_140_1.png}
    \includegraphics[width=0.49\textwidth]{chapters/Physics/sectionBSM/figures/WPrime_300_1.png}
    \includegraphics[width=0.49\textwidth]{chapters/Physics/sectionBSM/figures/WPrime_300_4.png}
    \caption{ $\mSqBar _{\PW}$, $\mSqBar _{\PWpr} $ and $\mSqBar _{\mathrm{int}}$ on the $(E_1,E_3)$ plane, shown as orange, blue and green surface, respectively. Changing the two model parameters $(\cot_E, m_{\PWpr})$ leads to different scenarios. Upper left, upper right and lower plots illustrate  $(\cot_E=1, m_{\PWpr}=140  \GeV )$, $(\cot_E=1, m_{\PWpr}=300  \GeV )$ , and $(\cot_E=4, m_{\PWpr}=300  \GeV )$ cases. }
    \label{fig:physics:bsm:WPrime:m2}
\end{figure}



Upon integrating  $\mSqBar _{\PWpr} +  \mSqBar _{\mathrm{int}}$ on the $(E_1,E_3)$ plane using Equation~\ref{eqn:physics:bsm:decayWidth}, one gets the BSM width of the taunic decay $\WidthTopTauBSM$. 
\begin{equation}
	\WidthTopTauBSM = \frac{1}{64 \pi^3 m_\PQt} \int_{0}^{m_\PQt/2} dE_3 \int_{m_\PQt/2-E_3}^{m_\PQt/2} dE_1  \bigg\{ \mSqBar _{\PWpr} +  \mSqBar_{\mathrm{int}}  \bigg \},
\end{equation}


\noindent Then the relative tau enhancement from the BSM with respect to SM can be obtained by $\WidthTopTauBSM/ \WidthTopTauSM$. For example, when $\cot_E=6$  and $m_{\PWpr}=1\TeV$, $\mSqBar _{\PWpr} +  \mSqBar _{\mathrm{int}}$ integral yields $\WidthTopTauBSM = 0.86 $ MeV and tau enhancement is
\begin{equation}
	\frac{ \WidthTopTauBSM}{ \WidthTopTauSM} \sim 0.55 \%, \quad (\cot_E=6, m_{\PWpr}=1 \text{\TeV}).
\end{equation}

\noindent In addition to this example, the tau enhancement $\WidthTopTauBSM/  \WidthTopTauSM $ can be calculated in the model parameters space $(\cot_E, m_{\PWpr})$ , shown as Figure~\ref{fig:physics:bsm:WPrime:tauEnhancement}. The relative enhancement decreases as $m_{\PWpr}$ increases and as $\cot_E$ approaches to 1. 
% For $\cot_E=4,6,10$, the $\WidthTopTauBSM/  \WidthTopTauSM $ as a 1D function of \PWpr mass, decomposed into $\mSqBar _{\PWpr} $ and $\mSqBar _{\mathrm{int}}$  terms, is shown in Figure~\ref{fig:physics:bsm:WPrime:tauEnhancement1d}. 
Our analysis confirms lepton universality with a $2\%$ uncertainty, which translate to exclusion of $ \WidthTopTauBSM/  \WidthTopTauSM  >  2\%$ shown as the left side of the green contour. For $m_{\PWpr}>600$\GeV, this exclusion region is beyond the the model's upper boundary of $\cot_E$ and not as competitive as the direct searches, shown in Figure~\ref{fig:physics:bsm:WPrime:directSearch}.




\begin{figure}
    \centering
    \includegraphics[width=0.6\textwidth]{chapters/Physics/sectionBSM/figures/RelEnhance.png} 
    \caption{The relative tau enhancement $ \WidthTopTauBSM/  \WidthTopTauSM $ in the NUGIM parameter space $(\cot_E, m_{\PWpr})$. \WidthTopTauBSM is calculated from integrating $\mSqBar _{\PWpr} +  \mSqBar _{\mathrm{int}}$.  Our analysis confirms LU with 2\% uncertainty, excluding the $ \WidthTopTauBSM/  \WidthTopTauSM > 2\% $ with one sigma, shown as the left side of the green contour.}
    \label{fig:physics:bsm:WPrime:tauEnhancement}
\end{figure}


% \begin{figure}
%     \centering
%     \includegraphics[width=0.6\textwidth]{chapters/Physics/sectionBSM/figures/RelEnhance1D.pdf} 
%     \caption{For $\cot_E=\{4,6,10\}$, $\WidthTopTauBSM/  \WidthTopTauSM $ as a function of \PWpr mass. Dash and solid lines are $\mSqBar _{\PWpr} $ and $\mSqBar _{\mathrm{int}}$  terms, respectively. For \PWpr heavier than top quark, the leading BSM contribution is from the \PW-\PWpr interference term. }
%     \label{fig:physics:bsm:WPrime:tauEnhancement1d}
% \end{figure}



\begin{figure}
    \centering
    \includegraphics[width=0.49\textwidth]{chapters/Physics/sectionBSM/figures/WPrime_search0.png} 
    \includegraphics[width=0.49\textwidth]{chapters/Physics/sectionBSM/figures/WPrime_search.png} 
    \caption{Direct Search of \PWpr on the Atlas and CMS.}
    \label{fig:physics:bsm:WPrime:directSearch}
\end{figure}






\FloatBarrier






\subsection{Models with $\rm H^+$}
\label{sec:physics:bsm:chargedHiggs}

\subsubsection{Model Overview}
The charged Higgs boson \PSHp is a hypothetical particle in the BSMs with extended scalar field structures. In the SM, the scalar sector has the simplest possible structure, one SU(2) doublet. Meanwhile the fermion structure with three mixing generations is not simple. It is possible that the scalar structure can have some more complex form. Two Higgs Doublet Model (2HDM) provides the next simplest structure for the SM scalar sector. It assumes one more scalar doublet in addition to that in the SM. The two scalar doublets are responsible for the masses of upper and lower fermions separately. There are three major motivations to 2HDM \cite{BRANCO20121}: generating mass in the Minimal Supersymmetric Standard Model (MSSM) \cite{HABER198575}, explaining the strong CP in the QCD \cite{KIM19871, PhysRevLett.38.1440}, and adding extra CP-violation source for the baryon asymmetry \cite{Trodden:1998qg, TUROK1991471, Joyce:1994zt}. 

% (From Review \cite{BRANCO20121}) The first and the best-known motivation is supersymmetry \cite{HABER198575}. In the supersymmetric theories, the scalars belong to chiral multiplets and their complex conjugates belong to multiplets of the opposite chirality; since multiplets of different chiralities cannot couple together in the Lagrangian, a single Higgs doublet cannot give mass simultaneously to the upper-type and down-type quarks. Moreover, since scalars sit in chiral multiplets together with the chiral spin-$\frac{1}{2}$ fields, the cancellation of anomalies also requires an additional doublet. The second motivation for 2HDMs comes from axion models \cite{KIM19871}. Peccei and Quinn \cite{PhysRevLett.38.1440} noted that a possible CP-violating term in the QCD Lagrangian, which is phenomenologically known to be very small, can be rotated away if the Lagrangian contains a global symmetry. However, imposing this symmetry is only possible if there are two Higgs doublets. While the simplest versions of the Peccei–Quinn model (in which all the New Physics was at the\TeV scale) are experimentally ruled out, there are variations with singlets at a higher energy scale that are acceptable, and the effective low-energy theory for those models still requires two Higgs doublets \cite{KIM19871}. The third motivation for 2HDMs is that the CV violation in the SM electroweak sector is not enough \cite{Trodden:1998qg} to generate a baryon asymmetry of the Universe of sufficient size. Two-Higgs-doublet models can do so due to the flexibility of their scalar mass spectrum \cite{Trodden:1998qg} and the existence of additional sources of CP violation. There have been many works on baryogenesis in the 2HDM \cite{TUROK1991471, Joyce:1994zt}. 

The direct searches for \PSHp have been conducted in two parts of the phase space, $m_{\PSHp} < m_\PQt$  and $m_{\PSHp} > m_\PQt$ \cite{pdg2020}. For $m_{\PSHp} < m_\PQt$, LEP \cite{Abbiendi:2013hk}, CMS \cite{Khachatryan:2015qxa} and ATLAS \cite{Aad:2014kga} have exclude \PSHp with mass below 80\GeV, 155\GeV, and 140\GeV with 95\% confidence level respectively. For $m_{\PSHp} > m_\PQt$, ATLAS has provide a $\tan\beta$-dependant exclusion of $m_{\PSHp}$ \cite{Aaboud:2018gjj}, more explicitly $m_{\PSHp}>181\GeV, 129\GeV, 390\GeV, 894\GeV, 1017\GeV, 1103\GeV$ at $\tan\beta=10,20,30,40,50,60$, respectively. Here in this section, we explore the effect of 2HDM \PSHp in the top decay and evaluate the corresponding ``tau enhancement" with respect to SM.






In 2HDMs, there are two complex scalar doublets with eight fields:
\begin{equation}
	\Phi_1 = \begin{bmatrix} \phi_1^+ \\ \frac{\nu_1}{\sqrt{2}} + \frac{\rho_1+i\eta_1}{\sqrt{2}}  \end{bmatrix} , 
    \Phi_2 = \begin{bmatrix} \phi_2^+ \\ \frac{\nu_2}{\sqrt{2}} + \frac{\rho_2+i\eta_2}{\sqrt{2}}  \end{bmatrix}
    \label{eqn:physics:bsm:chargedHiggs:scalarFields}
\end{equation}

\noindent where the ratio of the VEV of the two doublets are
\begin{equation}
\tan \beta = \frac{\nu_2}{\nu_1},
\end{equation}

\noindent and $\tan \beta$  is an important parameter in the model. The potential for the two scalar doublets reads as
\begin{equation}
\begin{split}
V=& m_{11}^2 \Phi_1^\dagger \Phi_1 + m_{22}^2 \Phi_2^\dagger \Phi_2 - m_{12}^2 ( \Phi_1^\dagger \Phi_2+\Phi_2^\dagger \Phi_1) \\
&  +\frac{\lambda_1}{2}(\Phi_1^\dagger \Phi_1)^2 +\frac{\lambda_2}{2}(\Phi_2^\dagger \Phi_2)^2
+\lambda_3 \Phi_1^\dagger \Phi_1 \Phi_2^\dagger \Phi_2 +\lambda_4 \Phi_1^\dagger \Phi_2 \Phi_2^\dagger \Phi_1
+\frac{\lambda_5}{2}[ (\Phi_1^\dagger \Phi_2)^2 + (\Phi_2^\dagger \Phi_1)^2 ]
\end{split}
\end{equation}

\noindent which is minimized when choosing the vacuum expectation value $\Phi_1= [0,v_1/\sqrt{2}]^T$ and $\Phi_2= [0,v_2/\sqrt{2}]^T$.



\noindent Out of the eight fields in Equation~\ref{eqn:physics:bsm:chargedHiggs:scalarFields}, three are ‘eaten’ to give mass to the \PW and \PZ gauge bosons; the remaining five are physical scalar fields. These are a charged scalar, two neutral scalars, and one pseudoscalar \cite{BRANCO20121}. The Lagrangian for the mass of the charged scalars is given by
\begin{equation}
	\mathcal{L}_{H^{\pm} mass} = \big( m_{12}^2 -(\lambda_4+\lambda_5) v_1 v_2 \big) 
    \begin{bmatrix} \phi_1^- & \phi_2^-  \end{bmatrix}
    \begin{bmatrix} \frac{v_2}{v_1} & -1 \\ -1 & \frac{v_1}{v_2} \end{bmatrix}
    \begin{bmatrix} \phi_1^+ \\ \phi_2^+ \end{bmatrix} ,
\end{equation}

\noindent which implies $m_{\PSHpm} = \sqrt{ [m_{12}^2 /(v_1 v_2) - \lambda_4 - \lambda_5] [v_1^2+v_2^2]} $ and the mass eigenstate of the charged Higgs is a linear mixing of $\phi^\pm_1$ and $\phi^\pm_2$:
\begin{equation}
\PSHpm = \phi_2^\pm \cos \beta - \phi_1^\pm \sin \beta .
\end{equation}


There are several types of 2HDM. If imposing flavor conservation, there are four possibilities (type I–IV) for the two Higgs doublets to couple to the SM fermions. Each of the four types gives rise to rather different phenomenology. In these four types of 2HDM, the generic form of the coupling between \PSHp and SM fermions can be expressed as a superposition of right- and left-chiral coupling components \cite{PhysRevD.41.3421}. The related Lagrangian term mediating the top decay is given by
\begin{equation}
	\mathcal{L}_{I} =  \frac{g  }{\sqrt{2} m_\PW} \PSHp \bigg[  \bar{t} (A \, P_R + B \, P_L) b + \bar{\nu}  (C\, P_L)  l \bigg]
    \label{eqn:physics:bsm:chargedHiggs:intLagrangian}
\end{equation}

\noindent In the first possibility (type-I), the $\Phi_2$ doublet gives masses to all quarks and leptons, so the other one, doublet $\Phi_1$, essentially decouples from fermions. In the second scenario (type-II), the $\Phi_2$ doublet gives mass to the right-handed up-type quarks, and the $\Phi_1$-doublet gives mass to the right-handed down-type quarks and charged leptons. In the type-III, both up- and down-type quarks couple to the second doublet $\Phi_2$, and all leptons couple to the first one $\Phi_1$. In the fourth scenario (type-IV), the roles of two doublets are reversed with respect to type-II. The explicit arrangements to generate fermion mass with $\Phi_1,\Phi_2$ in the four types are listed in Table~\ref{tab:physics:bsm:chargedHiggs:types}. Also the coupling constants A, B, C in Equation~\ref{eqn:physics:bsm:chargedHiggs:intLagrangian} are shown in  Table~\ref{tab:physics:bsm:chargedHiggs:types} for the four types. Among these four types, the most interesting one is type-II because it is the 2HDM for the MSSM. So here as an example, we provide a interpretation of our result in the context of type-II 2HDM. Other types could be easily explored by using the corresponding A, B, C coefficients and going through the same process.

\begin{table}[ht]
    \centering
    \setlength{\tabcolsep}{1em}
    \renewcommand{\arraystretch}{1.5}
    \caption{ There are four possibilities of 2HDM if imposing flavor conservation. The four types differ from each other by the specific ways fermion masses are generated with $\Phi_1,\Phi_2$ .The second and third column show the fermion masses which $\Phi_1,\Phi_2$ are responsible for in the four types. The last three columns show the coupling constants A, B, C in the interaction Lagrangian in Equation~\ref{eqn:physics:bsm:chargedHiggs:intLagrangian}.}
%     \resizebox{\textwidth}{!}{
    \begin{tabular}{c|cc | ccc }
        \hline
        Type & $\Phi_1$ Doublet & $\Phi_2$ Doublet & A               & B                 & C                    \\
        \hline
        I    & --               & \PQu, \PQd, \Pe    & $m_\PQt \cot \beta$ & $-m_\PQb \cot \beta$ & $-m_\PGt \cot \beta$ \\
        II   & \PQd, \Pe        & \PQu               & $m_\PQt \cot \beta$ & $m_\PQb \tan \beta$  & $m_\PGt \tan \beta$  \\
        III  & \Pe              & \PQu, \PQd         & $m_\PQt \cot \beta$ & $m_\PQb \tan \beta$  & $-m_\PGt \cot \beta$ \\
        IV   & \PQu             & \PQd, \Pe          & $m_\PQt \cot \beta$ & $-m_\PQb \cot \beta$ & $m_\PGt \tan \beta$  \\
        \hline
    \end{tabular}
%     }
    \label{tab:physics:bsm:chargedHiggs:types}
\end{table}



In type-II 2HDM, given the interaction Lagrangian in Equation~\ref{eqn:physics:bsm:chargedHiggs:intLagrangian} and coupling constants in Table~\ref{tab:physics:bsm:chargedHiggs:types}, the total width of \PSHp can be calculated as \cite{PhysRevD.99.095012}
\begin{equation}
\begin{split}
    \Gamma_{\PSHp} =& \frac{g^2 m_{\PSHp}}{32 \pi} \frac{1}{m^2_{\PW}} \times \\
    &\begin{cases}
        m_\PGt^2 \tan^2 \beta+ 3 m_\PQs^2 \tan^2 \beta  + 3 m_\PQc^2 \cot^2 \beta , & m_{\PSHp} < m_\PQt \\
        m_\PGt^2 \tan^2 \beta+ 3 (m_\PQs^2+m_\PQb^2) \tan^2 \beta  + 3 (m_\PQc^2+m_\PQt^2) \cot^2 \beta  , & m_{\PSHp} > m_\PQt \\
    \end{cases}
    ,
 \end{split}
\end{equation}

\noindent where $\PSHp \to \PQs \PQc, \PGt \PGnGt$ are considered when $m_{\PSHp} < m_\PQt$ and $\PSHp \to \PQt \PQb, \PQs \PQc, \PGt \PGnGt$ are considered when $m_{\PSHp} > m_\PQt$. The Feynman rule for the \PSHp propagator takes into account its mass and width:
\begin{equation}
    \feynmandiagram [inline=(d.base), horizontal=d to b] {
        d -- [scalar, edge label=\(\PSHp\), momentum'=\(q\)] b ,
    }; =
    \frac{1}{q^2 - m^2_{\PSHp} + i m_{\PSHp} \Gamma_{\PSHp} }
\end{equation}



\subsubsection{Enhancement of Tauonic Decay}
The relative tau enhancement with respect to SM, $\WidthTopTauBSM/  \WidthTopTauSM $, can be calculated in the context of type-II 2HDM. This is done be by evaluating the tree-level Feynman diagram for the \PSHp mediated top decay in the tau channel. For $\PQt \to \PQb \PGt \PGn$, the total $ \mSqBar  $ not only has the contributions from the SM \PW propagator discussed in Section~\ref{sec:physics:bsm:smTopDecay}, but also includes the \PSHp part $\mSqBar _{\PSHp} $  and the \PW-\PSHp interference part $\mSqBar _{\mathrm{int}} $ . Namely,
\begin{equation}
	\mSqBar  = \mSqBar _{\PW} +  \mSqBar _{\PSHp} +  \mSqBar _{\mathrm{int}} ,
\end{equation}

\noindent where $\mSqBar _{\PSHp} +  \mSqBar _{\mathrm{int}}$ is the new BSM contribution to evaluate. This BSM contribution enhances tau channel in the top decay because of much heavier tau mass. In contrast, the muon and electron receives neglectable enhancement due to their much lighter masses. The calculation of $\mSqBar _{\PSHp} +  \mSqBar _{\mathrm{int}}$ is similar to that in Section~\ref{sec:physics:bsm:WPrime}. The differences are: the propagator is now a scalar; the masses of $\PQb,\PGt$  cannot be neglected because they are origins of the \PSHp couplings in the 2HDM. With the Feynman rule, we spell the tree-level amplitude and its conjugate for the \PSHp mediated tauonic top decay:
\begin{equation}
	\mathcal{M}  =  (\frac{g  }{\sqrt{2}})^2 \frac{1}{m^2_{\PW}}  \cdot
	\frac{\big[ \bar{u}_2 ( C  \, P_R) v_1 \big] \big[ \bar{u}_3  (A \, P_R + B  \, P_L) u_a \big]  }{q^2-m^2_{\PSHp} + i m_{\PSHp} \Gamma_{\PSHp}} 
\end{equation}

\noindent and
\begin{equation}
	\mathcal{M}^*  =  (\frac{g   }{\sqrt{2}})^2 \frac{1}{m^2_{\PW}}  \cdot 
    \frac{ \big[ \bar{u}_a  (A \, P_L + B  \, P_R) u_3 \big] \big[ \bar{v}_1 ( C  \, P_L) u_2 \big]  }{q^2-m^2_{\PSHp} - i m_{\PSHp} \Gamma_{\PSHp}} 
\end{equation}

\noindent Then the average amplitude squared can be obtained by summing spins and evaluating the trace of gamma matrices. This process is the same as Section~\ref{sec:physics:bsm:smTopDecay} and \ref{sec:physics:bsm:WPrime}. So the middle steps are not shown here. The final result reads as
\begin{equation}
	\mSqBar_{\PSHp} = \frac{g^4}{2 m^4_{\PW}} \frac{1}{ (q^2-m^2_{\PSHp})^2 +  m^2_{\PSHp} \Gamma^2_{\PSHp}} 
    C^2 (k_1 \cdot k_2) \bigg[ (A^2 + B^2) (k_3 \cdot p_a ) + 2 AB \, m_3  m_\PQt\bigg]
\end{equation}


\noindent and for the interference between the vector \PW and scalar \PSHp 
\begin{equation}
\begin{split}
    \mSqBar _{\mathrm{int}} = &   2 \overline{ Re[\mathcal{M}^*_{\PW} \mathcal{M}_{\PSHp}] }  \\
    =& \frac{g ^4}{m_\PW^4} \frac
    {( q^2-m^2_{\PW}) ( q^2-m^2_{\PSHp}) + m^2_{\PW}  m^2_{\PSHp}  \Gamma^2_{\PW} \Gamma^2_{\PSHp} }
    { \big[ ( q^2-m^2_{\PW})^2 +  m^2_{\PW} \Gamma^2_{\PW} \big] \big[ (  q^2-m^2_{\PSHp})^2 +  m^2_{\PSHp} \Gamma^2_{\PSHp} \big] }  \cdot \\
    & \bigg\{
    m_\PW^2  C m_1 \big[A  m_\PQt (k_2 \cdot k_3) - B m_3 (k_2 \cdot p_a) ) \big] + \\
    & \big[A m_\PQt - B m_3 \big] C m_1  (k_1 \cdot k_2) (k_3 \cdot p_a)   +  \big[B m_\PQt - A m_3\big]  C m_1 m_3  m_\PQt (k_1 \cdot k_2)  
    \bigg\}
\end{split}
\end{equation}

\noindent The inner products, such as $(k_3 \cdot p_a) $, can be rewritten in terms of $E_1, E_3$ using Equation~\ref{eqn:physics:bsm:innerProduct}, such that $\mSqBar _{\PSHp}$ and $ \mSqBar _{\mathrm{int}}$ become 2D functions of $(E_1, E_3)$  with two model parameters $(m_{\PSHp}, \tan\beta)$. Figure~\ref{fig:physics:bsm:chargedHiggs:m2} shows the $\mSqBar _{\PSHp}$ and $ \mSqBar _{\mathrm{int}}$ as well as the SM $\mSqBar _{\PW}$ on the $(E_1, E_3)$  plane. The valid decay phase space is a triangle area on the  $(E_1, E_3)$ plane. When \PSHp is lighter than top quark, $\mSqBar _{\PSHp}$ has a peak at $E_3=(m_\PQt^2-m_{\PSHp}^2)/2m_\PQt^2$ for on-shell \PSHp propagator. The peak moves left towards smaller $E_3$ as the $m_{\PSHp}$ approaches $m_\PQt$. When $m_{\PSHp}$ exceeds $m_\PQt$, the on-shell \PSHp peak moves outside the valid kinematic triangle region. In this case, \PSHp impacts the matrix element via its width; the wider, the larger $\mSqBar _{\PSHp}$ and $ \mSqBar _{\mathrm{int}}$ becomes.

\begin{figure}[ht]
    \centering
    \includegraphics[width=0.35\textwidth]{chapters/Physics/sectionBSM/figures/2HDM_120_10.png}
    \includegraphics[width=0.35\textwidth]{chapters/Physics/sectionBSM/figures/zoom_2HDM_140_10.png}
    \includegraphics[width=0.35\textwidth]{chapters/Physics/sectionBSM/figures/2HDM_140_40.png}
    \includegraphics[width=0.35\textwidth]{chapters/Physics/sectionBSM/figures/zoom_2HDM_140_40.png}
    \includegraphics[width=0.35\textwidth]{chapters/Physics/sectionBSM/figures/2HDM_200_40.png}
    \includegraphics[width=0.35\textwidth]{chapters/Physics/sectionBSM/figures/zoom_2HDM_200_40.png}
    \caption{  $\mSqBar _{\PSHp}$ and $ \mSqBar _{\mathrm{int}}$ are 2D functions of $(E_1, E_3)$ with two parameters $(m_{\PSHp}, \tan\beta)$. The $\mSqBar _{\PW}$, $\mSqBar _{\PSHp} $ and $\mSqBar _{\mathrm{int}}$ are shown as orange, blue and green surface, respectively. The valid kinematics is a triangle area on the $(E_1, E_3)$  plane. The first, second, and third row uses model parameters $(m_{\PSHp} = 140 \GeV, \tan\beta=10)$, $(m_{\PSHp} = 140 \GeV, \tan\beta=40)$, and $(m_{\PSHp} = 200 \GeV, \tan\beta=40)$. The right column is zoom-in views of the left column to show the small interference term $ \mSqBar _{\mathrm{int}}$.}
    \label{fig:physics:bsm:chargedHiggs:m2}
\end{figure}




Finally, using Equation~\ref{eqn:physics:bsm:decayWidth}, the extra top width due the BSM \PSHp equals the integral of  $\mSqBar _{\PSHp} +  \mSqBar _{\mathrm{int}}$
\begin{equation}
	\WidthTopTauBSM = \frac{1}{64 \pi^3 m_\PQt} \int_{0}^{m_\PQt/2} dE_3 \int_{m_\PQt/2-E_3}^{m_\PQt/2} dE_1  \bigg\{ \mSqBar _{\PSHp} +  \mSqBar_{\mathrm{int}}  \bigg \},
\end{equation}

\noindent Upon integrating over the triangle area on  $(E_1, E_3)$ plane, we get the BSM effect as a function of model parameters, $\WidthTopTauBSM (m_{\PSHp}, \tan\beta)$. Take $m_{\PSHp} = 140 $\GeV and $\tan\beta=8$ as an example, $\WidthTopTauBSM= 14.6 \MeV - 2.1 \keV$, where 14.6\MeV and  -2.1\keV correspond the  $\mSqBar _{\PSHp}$ and $\mSqBar _{\mathrm{int}}$ integral, respectively. When the charged Higgs is  lighter than top quark, the absolutely dominant term is the $\mSqBar _{\PSHp}$ integral and the $\PW-\PSHp$ interference is neglectable. Take $m_{\PSHp} = 200 $\GeV and $\tan\beta=8$ as an another example: $\WidthTopTauBSM = 0.6 \keV - 0.8 \keV$, which is extremely small comparing with the $\WidthTopTauSM= 154 \text{ MeV}$. The relative tau enhancement $\WidthTopTauBSM/\WidthTopTauSM$ can be calculated at different model parameters $(m_{\PSHp}, \tan\beta)$. 
% Figure~\ref{fig:physics:bsm:chargedHiggs:relEnhance1d} shows  $\WidthTopTauBSM/\WidthTopTauSM$ as a function of $m_{\PSHp}$ for $\tan\beta= \{8,20,40,60\}$, decomposed into $\mSqBar _{\PSHp}$ and $\mSqBar _{\mathrm{int}}$  components. 
Figure~\ref{fig:physics:bsm:chargedHiggs:relEnahnce2d} shows $\WidthTopTauBSM/\WidthTopTauSM$ in the 2D parameter space $(m_{\PSHp}, \tan\beta)$. Our analysis confirms the LU with $2\%$ uncertainty. In  Figure~\ref{fig:physics:bsm:chargedHiggs:relEnahnce2d}, the contours correspond to one and two experimental sigma, $\WidthTopTauBSM/\WidthTopTauSM = 2\%$ and $\WidthTopTauBSM/\WidthTopTauSM = 4\%$, are shown as green and blue dash line. Generally, $m_{\PSHp} < 150$\GeV is excluded for all $\tan\beta$. But our analysis does not probe the $m_{\PSHp} >m_\PQt$ parameter space, which is more suitable for a direct search with boosted tau. As a comparison, the run-I CMS direct search \cite{Khachatryan:2015qxa} is shown in Figure~\ref{fig:physics:bsm:chargedHiggs:directsearch}.


% \begin{figure}[ht]
%     \centering
%     \includegraphics[width=0.89\textwidth]{chapters/Physics/sectionBSM/figures/RelEnhance2HDM_1d.pdf}
%     \caption{ Absolute values of $\WidthTopTauBSM/\WidthTopTauSM$ as function of $m_{\PSHp}$ for $\tan\beta= 8,20,40,60$, decomposed into $\mSqBar _{\PSHp}$ and $\mSqBar _{\mathrm{int}}$  terms shown as solid and dash lines, respectively. }
%     \label{fig:physics:bsm:chargedHiggs:relEnhance1d}
% \end{figure}







\begin{figure}[ht]
    \centering
    \includegraphics[width=0.49\textwidth]{chapters/Physics/sectionBSM/figures/RelEnhance2.png}
    \includegraphics[width=0.49\textwidth]{chapters/Physics/sectionBSM/figures/RelEnhance2_heavy.png}
    \caption{$\WidthTopTauBSM/\WidthTopTauSM$  in the 2D parameter space $(m_{\PSHp}, \tan\beta)$. Our analysis confirms the LU with a relative uncertainty of $2\%$. The contours correspond to one and two experimental sigma, $\WidthTopTauBSM/\WidthTopTauSM = 2\%$ and $\WidthTopTauBSM/\WidthTopTauSM = 4\%$, are shown as the green and blue dash line, the left side of which is excluded.  }
    \label{fig:physics:bsm:chargedHiggs:relEnahnce2d}
\end{figure}

\begin{figure}[ht]
    \centering
    \includegraphics[width=0.9\textwidth]{chapters/Physics/sectionBSM/figures/2HDM_search.png}
    \caption{Result of the direct search for \PSHp in the CMS Run-I \cite{Khachatryan:2015qxa}. }
    \label{fig:physics:bsm:chargedHiggs:directsearch}
\end{figure}


\FloatBarrier


\subsection{Models with Leptoquark}
\label{sec:physics:bsm:leptoquark}


Leptoquarks (\PLQ) is a hypothetical particle with both the lepton number and baryon number, motivated by the GUT and predicted by many theories unifying quarks and leptons. In some GUT, such as Georgi–Glashow SU(5) unification, Pati-Salam model with SU(4) color, leptoquark is a gauge vector boson to mediate forces between the lepton-quark current. In some models, such as extended technicolor models, leptoquark states appear as the bound scalar of techni-fermions. So \PLQ can be either a scalar or vector boson, which interacts with fermions via $\lambda \cdot (\bar{\PQq} \gamma^\mu \ell) \PLQ_\mu$ if $s_{\PLQ}=1$ or via Yukawa interaction $\lambda \cdot (\bar{\PQq}\ell) \PLQ $ if spin $s_{\PLQ}=0$. If the leptoquark couples both to left and right fermions, it is non-chiral. Otherwise, it is possibly to couple only to the left- or right-handed fermions and be chiral. There are also possibilities that it couples to only one the fermion generation or couples to different fermion generations simultaneously. 

There are many direct searches for the leptoquark at the LHC. A pair of leptoquarks could be produced via quark-quark annihilation and gluon-gluon fusion. Meanwhile, single leptoquark production may be possible via gluon-quark scattering. CMS and ATLAS have searched leptoquark decaying into the first, second, or third generation of fermions. The search with pair production of leptoquarks excludes $m_{\PLQ}<1.05$\TeV, while the search with single produced leptoquark gives a slightly higher mass limit at 1.755\TeV. 

Besides direct search, leptoquark would also cause BSM effective four-point interactions, allowing indirect searches. Searches for flavor-changing neutral current (FCNC) put a strong constraint on the leptoquark that simultaneously involves different lepton generations. Besides, pion's electronic decay and electron anomalous magnetic moment are also sensitive to non-chiral scalar leptoquarks. Electron-positron collider producing quark pairs in the t-channel also highly constrains \PLQ.  Currently, with these indirect limits, it is believed that the leptoquark is more likely to be a chiral scalar or vector coupling to a single family of fermions.

However, the interpretation of our results in the context of \PLQ is very model dependent. So here, we do not provide a interpretation specific to any \PLQ models. But in principle, the interpretation could follow the same process as that in Section~\ref{sec:physics:bsm:WPrime} and \ref{sec:physics:bsm:chargedHiggs}, where BSM vector and scalar propagator are considered, respectively.





\section{Derivation of $\mathrm{V}_{cs}$ from $\mathrm{W}$ leptonic branching fraction}
\label{sec:physics:vcs}


The coupling strength between \PW boson and the fermion current is $g$. However, due to the quark mixing, the vertex between  \PW boson and quark current is further scaled by a CMK element \absVij. Namely,
\begin{equation}
    \feynmandiagram [inline=(d.base), small, horizontal=d to b] {
        a[particle=\PGn] -- [fermion] b [dot] -- [fermion] c[particle=\Pe],
        b -- [boson] d [particle=\PW],
    };
    = i g \gamma^{\mu} , \qquad
    \feynmandiagram [inline=(d.base), small, horizontal=d to b] {
        a[particle=\(\PQq_j\)] -- [fermion] b [dot] -- [fermion] c[particle=\(\PQq_i\)],
        b -- [boson] d [particle=\PW],
    };
    = i g \absVij.
\end{equation}

\noindent Denoting the partial width of \PW decaying into one generation of lepton current as $\Gamma_\ell$, the tree-level calculation gives
\begin{equation}
    \Gamma_\ell \equiv \Gamma_{\PW \to \ell \PGn} =  \frac{g^2 m_W}{48 \pi} .
\end{equation}




\noindent The NLO electroweak correction of $\Gamma_\ell$ is at $10^{-5}$ relative level~\cite{dEnterria:2020cpv}.  The hadronic \PW width decaying into $q_i,q_j$ at the leading order of QCD correction can be expressed in terms of  $\Gamma_\ell$
\begin{equation}
    \Gamma_{\PW \to \PQq_i \PQq_j}^{\rm LO} = 3 \absVij^2 \frac{g^2 m_W}{48 \pi}  = 3 \absVij^2 \Gamma_\ell ,
\end{equation}


\noindent  where the factor 3 accounts for the three colors. The ratio between the total hadronic and the total leptonic \PW width, at the tree-level, then equals to the square sum of the CKM elements in the first two rows:
\begin{equation}
    \frac{\Gamma_{\rm had}^{\rm LO}}{\Gamma_{\rm lep}} = \frac{\sum_{ij=(uc)(dsb)} \Gamma_{\PW \to \PQq_i \PQq_j}^{\rm LO} }{ \sum_{\Pe,\mu,\tau} \Gamma_\ell } = \frac{\sum_{ij=(uc)(dsb)} 3 \absVij^2 \Gamma_\ell  }{ 3 \Gamma_\ell }= \sumCKM.
\end{equation}




% \subsection{Next-to-leading Order of \alpS}

\begin{figure}
    \centering
    \includegraphics[width=0.8\textwidth]{chapters/Physics/sectionVcs/figures/realVirtual.png}
    \caption{ The real and virtual diagram of\PW decay at the next-to-leading order of \alpS. }
    \label{fig:physics:vcs:realVirtual}
\end{figure}


\noindent At next-to-leading order (NLO) in \alpS, the QCD corrections related to the quark current are taken into account. More specifically, the real and virtual diagram shown in Figure~\ref{fig:physics:vcs:realVirtual} add extra contributions to the leading order width $\Gamma_{\PW \to \PQq_i \PQq_j}^{\rm LO} $. The real diagram corresponds to the gluon final state radiation from the outcoming quarks. The virtual diagram corresponds to the interference between the tree level diagram and the virtual gluon bubbles in the quark current and at the vertex. The calculations of the real and virtual contribution can be expressed as a factor multiplied on the tree-level width  $\Gamma_{\PW \to \PQq_i \PQq_j}^{\rm LO} $.
 \begin{align}
 	\Gamma^{\rm V}_{\PW \to \PQq_i \PQq_j}  &= \Gamma_{\PW \to \PQq_i \PQq_j}^{\rm LO} \times \frac{\alpS}{2\pi}\frac{4}{3} \bigg \{  -\ln^2\frac{m_g}{Q} -3 \ln\frac{m_g}{Q} + \frac{\pi^2}{3}-\frac{7}{2} \bigg\} \\
    \Gamma^{\rm R}_{\PW \to \PQq_i \PQq_j}  &= \Gamma_{\PW \to \PQq_i \PQq_j}^{\rm LO} \times \frac{\alpS}{2\pi}\frac{4}{3} \bigg \{  +\ln^2\frac{m_g}{Q} + 3 \ln\frac{m_g}{Q} - \frac{\pi^2}{3}+ 5 \bigg\}
\end{align}
 
\noindent  where $Q$ is the energy of the \PW boson and $m_g=0$ is the mass of the gluon, which makes both the real and virtual width diverge. But the divergences in the real and virtual width exactly cancel each other, leading to a finite total contribution. This QCD correction turns out to be a factor of $k=(1+\frac{\alpS}{\pi})$ :
\begin{equation}
\begin{split}
    \Gamma_{\PW \to \PQq_i \PQq_j}^{\rm NLO} =& \Gamma_{\PW \to \PQq_i \PQq_j}^{\rm LO} + \Gamma^{\rm V}_{\PW \to \PQq_i \PQq_j}  + \Gamma^{\rm R}_{\PW \to \PQq_i \PQq_j}
            =   \Gamma_{\PW \to \PQq_i \PQq_j}^{\rm LO} \big( 1+ \frac{\alpS(M_W)}{\pi}\big)
\end{split} .
\end{equation}

\noindent Therefore at NLO in \alpS, the ratio between the hadronic and leptonic \PW widths also includes the $k=(1+\frac{\alpS}{\pi})$ factor:
\begin{equation}
    \frac{\Gamma_{\rm had}^{\rm NLO}}{\Gamma_{\rm lep}} =  \underbrace{(1+\frac{\alpS}{\pi})}_{k} \sumCKM. %\sum_{ij=(uc)(dsb)} \absVij^2.
\end{equation}


\noindent For higher order \alpS corrections of the hadronic\PW width, the state-of-art factor has been calculated by considering additional QCD loops. At $\rm N^3LO$, the ratio between the hadronic and leptonic\PW width reads as 
\begin{equation}
    \frac{\Gamma_{\rm had}^{\rm N^3LO}}{\Gamma_{\rm lep}} =   \underbrace{ \bigg [ 1+1.045 ( \frac{\alpS}{\pi} ) + 0.94  ( \frac{\alpS}{\pi} ) ^2 -15  ( \frac{\alpS}{\pi} ) ^3 \bigg ]}_{k} \sumCKM.
\end{equation}

\noindent Finally, the sum square of the CKM elements in the first two rows can be calculated by the experimental measurement of \BWh
\begin{equation}
    \sumCKM= \frac{1}{k}\times \frac{\BWh }{1- \BWh}
\end{equation}



\noindent where \alpS at the \PW pole can be calculated with $\alpS(\mu_R = m_Z)=0.1178\pm0.0010$~\cite{pdg2020} and the QCD renormalization: $\alpS(m_W) = \alpS(\mu_R) - \alpha^2_s(\mu_R) \frac{ \beta_0}{2\pi} \ln \frac{m_W}{\mu_R} = 0.1199 \pm 0.0010$. The square sum of the five more precisely measured CKM elements can be calculated from the latest experimental results \cite{pdg2020} shown in Table~\ref{tab:introduction:relatedWorks:ckm}. $\mathrm{SS_5} = \sumCKMfive = 1.0490 \pm 0.0018$, which leads to the expression for \absVcs:
\begin{equation}
\absVcs = \sqrt{ \frac{1}{k}\times \frac{\BWh }{1- \BWh} - \mathrm{SS_5} } .
\end{equation}






    \chapter{The CMS Experiment}
\label{sec:cmsExperiment}


\section{The Large Hadron Collider}
\label{sec:cmsexperiment:lhc}

% overview
The \acrfull{lhc} \cite{exhep:lhc:Evans:2008zzb} is a 27 km circular particle collider located at the \acrfull{cern} across the border between France and Switzerland. The LHC was constructed during 1998-2008 in a 100-meter-deep underground tunnel previously used by the \acrfull{lep} \cite{exhep:lep:Myers:1991ym}. Inside the LHC, two proton beams collide at a maximum center-of-mass energy of $\sqrt{s}=14$ TeV with a designed instant luminosity of \SI{e34}{\per\cm\squared \per\s}. Around the ring path of the LHC, four collision positions are designed corresponding to four LHC experiments: CMS \cite{exhep:cms:Chatrchyan:2008aa} (Point 5), ATLAS \cite{exhep:atlas:Aad:2008zzm} (Point 1), LHCb \cite{exhep:lhcb:Alves:2008zz} (Point 8) and Alice \cite{exhep:alice:Aamodt:2008zz} (Point 2).


% constituents
The main components of LHC include two tubes with ultrahigh vacuum and about ten thousand superconducting magnets with various sizes installed alone the ring, including 1232 dipole magnets with length of 15 m to bend the beams and 392 quadrupole magnets with length of 5-7 m to focus the beams \cite{exhep:lhcFactsFigures}. Magnets of higher multipole orders are also used for corrections of the magnetic field. A liquid helium cooling system is used to cool the superconducting electromagnets at a cryogenic temperature of -271.3 \si{\degreeCelsius}. 


\begin{figure}[ht]
    \centering
    \includegraphics[width=0.8\textwidth]{chapters/CMSExperiment/sectionLHC/figures/lhc.png}
    \caption{Schematic overview of the LHC and related accelerator complex. The accelerator chain includes proton source, \acrfull{rfq}, LINAC, \acrfull{psb}, \acrfull{ps}, \acrfull{sps}, and finally the LHC \cite{exhep:lhcInject:Benedikt:2004wm}. Around the ring path of the LHC locate four LHC experiments: CMS \cite{exhep:cms:Chatrchyan:2008aa} (Point 5), ATLAS \cite{exhep:atlas:Aad:2008zzm} (Point 1), LHCb \cite{exhep:lhcb:Alves:2008zz} (Point 8) and Alice \cite{exhep:alice:Aamodt:2008zz} (Point 2)}
    \label{fig:cmsexperiment:lhc:map}
\end{figure}


% beam pipe
Before injected into the LHC, protons are accelerated to 450 GeV by a few existing accelerator facilities at CERN. Figure~\ref{fig:cmsexperiment:lhc:map} shows a schematic overview of the LHC with its related accelerator complex at CERN \cite{exhep:lhcInject:Benedikt:2004wm}. First, protons are produced by the ionization hydrogen gas and are extracted by a 90 keV voltage to inject into the \acrfull{rfq} where protons are divided into bunch crossings and are accelerated to 750 keV. A linear accelerator (Linac2) then energizes them to 50 MeV. The \acrfull{psb}, which has four superimposed synchrotron rings, brings the protons to 1.4 GeV for the injection to the \acrfull{ps}, a 628 m synchrotron outputting beams with energy of 25 GeV. The \acrfull{sps} further boosted to 450 GeV in its 7-km-long ring and deliver the beam to LHC. When accelerating proton beam from 450 GeV at the LHC injection to 6.5 TeV for the physics collision in the Run-2, the dipole magnetic field is increased from 0.54 T to 7.7 T to increase the banding power to circulating energized beams. During a physics run, luminosity of LHC decays with a lifetime about 14.9 hours \cite{exhep:lhc:Evans:2008zzb} due to effects of physics cross-section, photon emittances alone the circular path and scattering off the air remains. Therefore, every one or two days



% operation schedule
The operation of LHC from 2010 to 2035 consists of 6 runs with shutdown periods for upgrading and maintenance during the run intervals. In the Run-1 from 2010 to 2013, LHC delivered about 6 $fb^{-1}$ pp collision at $\sqrt{s}=7$ TeV in 2010, 2011 and 23.3 $fb^{-1}$ pp collision at $\sqrt{s}=8$ TeV in 2012 \cite{cms:publicLumiInfo}, with which the discovery of Higgs boson was made by the ATLAS \cite{exhep:atlasHiggsDisc:Aad:2012tfa} and the CMS \cite{exhep:cmsHiggsDisc:Chatrchyan:2012ufa}. In the Run-2 from 2016 to 2018, LHC produced 144 $fb^{-1}$ pp collisions at $\sqrt{s}=13$ TeV \cite{cms:publicLumiInfo}. Currently in 2020, the LHC is in its second long shutdown period, expecting Run3 starting in 2021 to operate at the maximum collision energy of $\sqrt{s}=14$ TeV. After Run3, LHC will be upgraded to higher luminosity or the \acrfull{hllhc} to reach an instant luminosity of \SI{5e34}{\per\cm\squared \per\s}, five times as much as current value. In the era of HL-LHC, three extra runs are scheduled during 2026-2035. In the long term future beyond the HL-LHC era, the \acrfull{fcc} \cite{exhep:fcc:Benedikt:2715354} plan is proposed to build a 100 km hadron collider next to the LHC and further increase the collision energy to the level of 100 TeV.


\section{Detector Apparatus}
\label{sec:cmsexperiment:detector}



% overview
The CMS \cite{exhep:cms:Chatrchyan:2008aa} detector is a general-purpose apparatus operating at the LHC and is located about 100 meters underground at Point 5 of the LHC, close to the French village of Cessy, between Lake Geneva and the Jura mountains. As a general purpose detector, the CMS detector is designed to observe any new physics phenomena that the LHC might reveal \cite{cms:tdr2:Ball:2007zza}. At the designed LHC luminosity of \SI{e34}{\per\cm\squared \per\s}, on average about 20 inelastic collisions are superimposed on the event of interest every collision of bunch crossings, leading to a large flux of particles originating from the detector center to enter the detector every 25 ns. In order to discern them and trigger the interested events within 25 ns latency over the LHC run period until 2035, CMS detector is designed to be highly-segmented, radiation-hard and with good timing resolution \cite{exhep:cms:Chatrchyan:2008aa}.

\begin{figure}[ht]
    \centering
    \includegraphics[width=0.98\textwidth]{chapters/CMSExperiment/sectionDetector/figures/cmsDetector.png}
    \caption{The layout of the CMS detector \cite{cms:detectorOverview}.}
    \label{fig:cmsexperiment:detector:detectorOverview}
\end{figure}

% overview of structure
The apparatus layout of CMS detector is shown in Figure~\ref{fig:cmsexperiment:detector:detectorOverview} \cite{cms:detectorOverview}. The central feature of the CMS apparatus is a superconducting solenoid of 6 m internal diameter, providing a magnetic field of 3.8 T. Within the superconducting solenoid volume are a silicon pixel and strip tracker, a lead tungstate crystal electromagnetic calorimeter, and a brass and scintillator hadron calorimeter, each composed of a barrel and two endcap sections. Muons are measured in gas-ionization detectors embedded in the steel flux-return yoke outside the solenoid. Additional forward calorimetry complements the coverage provided by the barrel and endcap detectors. The more details of these sub-systems, including magnet,  are discussed in this section. 

% To achieve the physics goal in the LHC environment, the design of each CMS sub-detectors are driven by the following performance requirement.

% \begin{itemize}
%     \item Magnets and Muon chamber: Good muon identification, energy resolution, charge determination.
%     \item Pixel and Tracker: Good charge hadron track reconstruction efficiency. Good displaced vertex tagging for $b$, $\tau$ identification.
%     \item ECAL: Good electromagnetic energy resolution. $\pi^0$ rejection. Efficient photon and lepton isolation at high luminosities.
%     \item HCAL: Good missing-transverse-energy and dijet-mass resolution
% \end{itemize}



\subsection{Magnet}
% overview
The CMS superconducting magnet \cite{cms:magnetTdr:Acquistapace:1997fm} is used to provide bending to the charged particles as they traverse, which is crucial to both particle identification and momentum measurement. The internal magnetic filed is 4 Tesla with 2.6 GJ stored energy and is generated by a superconducting solenoid surrounded by a set of return york. The solenoid is 12.5 m in length, 6.3 m in diameter and 200 ton in weight, consist of 41.7 MA-turn of wire. The radiation thickness of the solenoid is 4.9 $\chi_0$, which further prevents hadrons from entering the muon system. The solenoid is surrounded and mechanically supported by the iron return yokes, which direct the outer magnetic in the muon system. The yoke, consist of 5 barrel wheels and two endcaps, has an outer diameter of 14 m and a total weight of 10000 tons. Both barrel and endcap return yoke have three iron layers with thickness of 300/630/630 mm and 250/600/600 mm, respectively.



\subsection{Inner Tracking System}
% overview
The inner tracking system \cite{cms:trackerTdr:CMS:1997tlf} is used to measure the trajectories of charged particles. Covering region with $|\eta|<2.5$, it is consist of two major parts: pixel detector and Silicon strip tracker, layout of which is shown in Figure~\ref{fig:cmsexperiment:detector:tracker}. 


\begin{figure}[ht]
    \centering
    \includegraphics[width=0.98\textwidth]{chapters/CMSExperiment/sectionDetector/figures/tracker.png}
    \caption{The layout of the CMS inner tracking system \cite{exhep:cms:Chatrchyan:2008aa}. It is consist of pixel detector and silicon strip tracker (TIB/TID, TOB, TEC), covering regions with $|\eta|<2.5$. }
    \label{fig:cmsexperiment:detector:tracker}
\end{figure}


\begin{figure}[ht]
    \centering
    \includegraphics[width=0.98\textwidth]{chapters/CMSExperiment/sectionDetector/figures/trackerMaterial.png}
    \caption{The material thickness of inner tracking system before ECAL.}
    \label{fig:cmsexperiment:detector:trackerMaterial}
\end{figure}


% pixel
The pixel detector, shown in the center of Figure~\ref{fig:cmsexperiment:detector:tracker}, is consist of three cylindrical layers of pixel detector modules at radii of 4.4, 7.3, and 10.2 cm, totaling 66 million pixels with area of 1 \si{\m \squared}. It is capable of producing three high precision 3D spacial hits for each charged particle. 

% tracker
 The silicon strip tracker system is right outside the pixel detector in the region of $20<r<116$ cm and $|z|<282$ cm. The tracker system has three parts: \acrfull{tibtid}, \acrfull{tob} and \acrfull{tec}, with a total of 9.3 million channels and 198 \si{\m \squared} active silicon area. The silicon strip modules in the barrel are lied out in cylindrical shapes with their strips parallel to the Z direction to measure $r-\phi$ coordinates. In comparison, those in the endcap region are in the shape of disks and place their strips in radial direction to measure the $Z-\phi$ coordinates. In addition to measuring the 2D coordinates, the first two cylinders of TIB and TOB, the first two disks of TID and TEC, as well as the fifth ring of TEC, are double-sided by placing a second micro-strip detector module back-to-back to the first module with a stereo angle of 100 mrad. This small stereo angle allows the measurement of the third spacial coordinates: $Z$ in the barrel (TIB and TOB) and $r$ on the endcap (TID and TEC). Such tracker design ensures to acquire at least 9 hits in the silicon strip tracker with at least 4 of them being stereo measurements. 




\begin{figure}[ht]
    \centering
    \includegraphics[width=0.98\textwidth]{chapters/CMSExperiment/sectionDetector/figures/detectorLayout.png}
    \caption{The layout of the CMS tracker, ECAL, HCAL, Magnets and muon system on the Z-r plane \cite{cms:muonChamberWebsite}. The full coverage of pseudorapidity is up to $\eta=5$. The details of tracker is shown in Figure~\ref{fig:cmsexperiment:detector:tracker}. }
    \label{fig:cmsexperiment:detector:detectorLayout}
\end{figure}





\subsection{Electromagnetic Calorimeter (ECAL)}
%  overview
The CMS electromagnetic calorimeter (ECAL) \cite{cms:ecalTdr:CMS:1997ysd} is used to measure the energy of electromagnetic showers. As shown in Figure~\ref{fig:cmsexperiment:detector:detectorLayout}, ECAL is located right outside the tracking system. ECAL consists of the barrel part (EB), the endcap part (EE) and a preshower system (EP) in front of EE. EB and EE are hermetic homogeneous calorimeter made of lead tungstate crystals with \acrfull{apd} and \acrfull{vpt} as readout sensors respective, while the PS is a sampling calorimeter with lead-silicon alternating layers to enhance the spatial resolution in the EE region. The total ECAL material thickness is larger than 25 $\chi_0$ and about 1.1 $\lambda_I$.

% EB
The barrel part of the ECAL (EB) covers the pseudorapidity range of $|\eta|< 1.479$ and consists of 61200 crystals arranged in a 170x360 $\eta - \phi$ grid, with 8.14 $m^3$ of total crystal volume and 67.4 tons of weight. The crystals have a tapered shape mounted in a quasi-projective distribution, in which the crystal axis has 3\degree angle with respect to the vector from the origin to minimize chances of cracks aligned with the particle trajectories. The crystal cross-section corresponds to approximately $0.0174 \times 0.0174$ in $\eta - \phi$ or $22 \times 22$ $mm^2$ at the front face of crystal, and $26\times26$ $mm^2$ at the rear face. The crystal length is 230 mm corresponding to 25.8 $\chi_0$.

% EE
The endcaps (EE) cover the rapidity range $1.479 < |\eta| < 3.0$ and is consist of 7324 identically shaped crystals grouped in mechanical units of 5x5 crystals (supercrystals, or SCs), with 2.90 $m^3$ of total crystal volume and 24.0 tons of weight. The crystals are arranged in a rectangular x-y grid, with the crystals pointing at a focus 1300 mm beyond the interaction point, giving off-pointing angles ranging from 2 to 8 degrees. The crystals have a front face cross section $28.62\times28.62$ $mm^2$, a rear face cross section $30\times30$ $mm^2$ and a length of 220 mm corresponding to 24.7 $\chi_0$.

% preshower
A preshower detector (EP) is placed in front of EE in $1.479 < |\eta| < 2.6$ to increase the space resolution of electromagnetic showers and better identify neutral pions $\pi^0 \to \gamma \gamma$ in the endcap. EP is a sampling calorimeter with two lead-silicon layers. On each layer, the lead radiators initiate electromagnetic showers from incoming photons and electrons, while silicon strip sensors placed after each radiator measure the deposited energy and the transverse shower profiles. The directions of silicon strips on the two layers are orthogonal to each other. The material thickness of the first and second layer are 2 $\chi_0$ and 1 $\chi_0$ respective, with a total mechanical thickness of 20 cm.



\subsection{Hadron Calorimeter (HCAL)}
% overview
The CMS Hadron Calorimeter \cite{cms:hcalTdrCMS:1997xji} is used to measure the energy of hadrons and determine the missing transverse energy. HCAL consists of four parts: the HCAL in barrel region (HB) between ECAL and magnets, HCAL in the endcap region (HE), the forward hadronic calorimeter (HF) and a small section outside the magnetic (HO) in the barrel region to catch rare hadronic punch through before muon system. As shown in Figure~\ref{fig:cmsexperiment:detector:detectorLayout}, HB and HE are designed right outside the ECAL; FH is in the high pseudorapidity region outside the whole CMS endcap.

% HB, HE, HO
HB and HE are a sampling calorimeter covering $|\eta|< 1.3$, $1.3<|\eta|< 3.0$ respectively. They use brass absorber ($70\%$ Cu and $30\%$ Zn) and plastic scintillators for readout. HO covers the same $|\eta|< 1.3$ range as HB but uses iron as absorber enhance the material thickness in HB especially in the low $\eta$ region. With HO, the total material thickness of the HCAL is about 11.8 $\lambda_I$ make sure the hadronic leakage to muon is very rare. Totally, HCAL has about 7000 scintillators channels, granularity of which is $\Delta \eta \times \Delta \phi = 0.087 \times 0.087$ in the HB, OB and $1.3<|\eta|< 1.6$ part of HE and $\Delta \eta \times \Delta \phi = 0.017 \times 0.017$ in the rest part of HE.

% HF
HF is a sampling calorimeter covering $3.0 < |\eta| < 5$. It is essentially a cylindrical steel structure with an outer radius of 130.0 cm and with fibers piecing from the back in Z direction at two different depths. The front face of the calorimeter is located at 11.2 m from the interaction point. The absorber are made of steel installed perpendicular to beam pipe with a total material depth of 10 $\lambda_I$. The active material is quartz fibres (fused-silica core and polymer hard-cladding) installed in parallel with the beam pipe. When particle showers in the HF, a small part of Cherenkov light generated at the surface of quartz fibres is captured. To distinguish the electromagnetic shower vs hadronic shower, two different the penetration depth of fibers are used, long fiber span the entire HF, while short fibers start from 22 cm behind the HF front surface and extend to the back. These fibers are bundled to form $\Delta \phi \times \Delta \eta = 0.175 \times 0.175$ towers. 


\subsection{Muon System}
% overview
The CMS muon system \cite{cms:muonChamberTdr:CMS:1997iti} is designed outside the solenoid hosted in the return yoke to measure the tracks of muons. The system consists of barrel detector (MB) covering $|\eta|<1.2$ and endcap detectors (ME) covering $0.9 < |\eta| < 2.4$. 

% MB RPC
The barrel detector has 250 chambers in total which hosts 250 \acrfull{dt} and 480 \acrfull{rpc}. The chambers are  arranged in 4 concentric stations in the yoke, each of which is in turn divided into 5 wheels with 12 sectors on each wheel. The two inner most stations, labeled as MB1 and MB2 in Figure~\ref{fig:cmsexperiment:detector:detectorLayout}, has two RPCs sandwiching a DT, while The 2 outermost stations, MB3 and MB4 in Figure~\ref{fig:cmsexperiment:detector:detectorLayout}, consist of packages of a DT coupled to a layer on the inner side made of 1, 2, or 4 RPCs, depending on the sector and station.
Each DT in MB1, MB2 and MB3 has 12 layers of drift tubes divided into 3 groups of 4 consecutive layers or 3 superlayers. Two with wire parallel to Z direction measure $r-\phi$ coordinates, the middle one with wire perpendicular to Z direction measures $r-z$ coordinates. DTs in MB4 only have two superlayers for measurement of $r-\phi$ coordinates. RPCs are attached to DTs to improve the responding time necessary for triggers. Each RPC detector has a bakelite chamber with two 2 mm wide gaps and operates in avalanche mode biased by a high voltage. 

% ME
The endcap detector has 469 \acrfull{csc}s and 432 RPCs on two sides in the yokes that close the solenoid. The ME consists of 4 stations ME1-ME4. The disk of ME1 is divided into 3 concentric rings, while two rings for ME2-ME4. The details of the layout of CSCs and RPCs in ME are shown in Figure~\ref{fig:cmsexperiment:detector:detectorLayout}. Each CSC is trapezoidal in shape and consists of 6 gas gaps, each gap having a plane of radial cathode strips and a plane of anode wires running almost perpendicularly to the strips and measuring hits with 3D coordinates.


% ---------------------------------------
% section : Event Triggering
% ---------------------------------------
\section{Trigger System}
\label{sec:cmsexperiment:trigger}

% overview
CMS applies a two-tiered trigger system \cite{cms:trigger:Khachatryan:2016bia} to select the events of interest. The \acrfull{l1t}, composed of custom hardware processors, uses information from the calorimeters and muon detectors to reduce the event rate from 40 MHz to 100 kHz, within a latency less than 4 $\mu s$. The second level, known as the \acrfull{hlt}, consists of a farm of processors running a version of the full event reconstruction software optimized for fast processing. The HLT further reduces the event rate from 100 kHz to 1 kHz and output for data storage.



\subsection{Level-1 Trigger}

\begin{figure}[ht]
    \centering
    \includegraphics[width=0.8\textwidth]{chapters/CMSExperiment/sectionTrigger/figures/trigger.png}
    \caption{The logic structure of L1T.}
    %  Calorimeters and muon detectors raise local trigger primitives to compute a regional trigger signal. Then \acrfull{gmt} summarizes regional information from DT, SCS and PRC, while \acrfull{gct} concentrates the regional information from ECAL, HCAL and HF. Finally \acrfull{gt} makes the final decision based on the object orientated information from GMT and GCT and transport data in the sub-detectors to the HLT if L1T accept is made. 
    \label{fig:cmsexperiment:trigger:structure}
\end{figure}

% overview
L1 Trigger is designed to cope with the high collision frequency in the LHC, reducing the event rate from 40 MHz to 100 kHz, keeping only potential events of physics interest. To achieve this, the L1 trigger is designed with three components: local, regional, and global trigger. The logic structure is shown in Figure~\ref{fig:cmsexperiment:trigger:structure}. The local triggers, also called \acrfull{tpg}, are based on energy deposits in the calorimeter trigger towers as well as the track segments or hit patterns in muon chambers. Regional triggers combine the information from the local triggers in limited regions. They use pattern logic to determine the sorted trigger objects, such as electron or muon candidates. The \acrfull{gct} and \acrfull{gmt} determine the highest-rank calorimeter and muon objects across the entire experiment and transfer them to the \acrfull{gt}, the top entity of the Level-1 hierarchy. GT decides to reject an event or to accept it for further evaluation by the HLT. The Level-1 Accept decision is communicated to the sub-detectors through the \acrfull{ttc} system. Before decisions reach the front-end, the raw data are stored in FIFO pipelined memories in the front end electronics. Limited by the memory size, only a latency of \SI{3.2}{\us} is allowed between a given bunch crossing and the distribution of the L1T decision to the detector front-end electronics. The L1T electronics are housed partly on the detectors, partly in the underground control room located at approximately 90 m from the experimental cavern.


% calo
The \acrfull{tpg} make up the local step of the Calorimeter Trigger pipeline. For triggering purposes, the calorimeters are subdivided into trigger towers. Each TPG sums up the transverse energies measured in ECAL crystal tower or HCAL read-out tower to obtain the trigger tower's $E_T$, and then it attaches the correct bunch crossing number. The TPG electronics are integrated with the calorimeter read-out. The TPGs are transmitted through high-speed serial links to the regional calorimeter trigger, which determines regional candidate electrons/photons, transverse energy sums, $\tau$-veto bits, and information relevant for muons in the form of \acrfull{mip} and isolation bits. The Global Calorimeter Trigger determines the highest-rank calorimeter trigger objects across the entire calorimeter system, including 8 $e/\gamma$, 8 jet, 4 $\tau$, $\sum E_T$, $H_T$, 12 $n_j$, met.

% muon
All three muon systems – the DT, the CSC, and the RPC – take part in the L1T. The barrel DT chambers provide local trigger information in the form of track segments in the $\phi$-projection and hit patterns in the $\eta$-projection. The endcap CSCs deliver 3-dimensional track segments. All chamber types also identify the bunch crossing from which an event originated. The regional muon trigger consists of the DT and CSC track finders, which join segments to complete tracks and assign physical parameters. Besides, the RPC trigger chambers, which have excellent timing resolution, deliver their own track candidates based on regional hit patterns. The Global Muon Trigger then combines the information from the three sub-detectors and outputs 4 leading $\mu$ in the full coverage of the muon system, achieving an improved momentum resolution and efficiency comparing with the stand-alone systems.

% GT
The GT decides to accept or reject an event at L1 based on trigger objects delivered by the GCT and GMT. The L1T accept decision is communicated to the sub-detectors through the TTC system. Then raw data corresponding to the triggered bunching crossing is read out from all front-end FIFO memories across the whole detector. The raw data, together with the L1T objects from GT, are sent to the HLT.


\subsection{High Level Trigger}

The event selection at the HLT is performed similarly to that used in the offline processing. For each event, objects such as electrons, muons, jets are reconstructed, and a menu of identification criteria is applied to select the events of physics interest.

% builder-filter
The HLT hardware consists of a CPU processor farm composed of commodity computers, the \acrfull{evf}, running Scientific Linux operating system. The event filter farm consists of thousands of builder-filter units. In the builder units, individual event fragments from the detector are assembled to form complete events. Upon request from a filter unit, the builder unit ships an assembled event to the filter unit. The filter unit then unpacks the raw data into detector-specific data structures and performs the event reconstruction and selection. Associated builder unit and filter unit are located in a single multi-core machine and communicate via a shared memory. In total, the EVF was executed on approximately 13,000 CPU cores at the end of 2012. On average, the HLT processing time per event is about 90 ms. EVF with 13,000 CPU cores allows the L1T output rate up to 100 kHz. With a fixed L1T rate and increasing CPU cores, the allowed time budget per event for HLT can be expended. The output rate of the HLT is about 1 kHz. The output rate is an optimal choice based on the event size, as well as the computing and storage capacity of the offline system.

% filtering and storage
The HLT filtering process uses the full precision of the data from the detector. The selection is based on offline-quality reconstruction algorithms. It works by computing a menu of the HLT paths, in each of which a predefined process of object reconstruction and event selection is executed. If at least one of the HLT paths get past, the event will be accepted and sent to storage and offline processing. Upon the HLT accept decisions are made, the events are sent to the storage manager for archival storage. The event data are stored locally on disk and eventually transferred to the CMS Tier-0 computing center for offline processing and permanent storage. Events are grouped into a set of non-exclusive streams according to the HLT decisions.


\section{Object Reconstruction}
\label{sec:cmsexperiment:reconstruction}

The layer structure of CMS detector, tracking-ECAL-HCAL-Muon, is idea to particle flow reconstruction, which uses information from all subdetector systems and reconstruct each final state particles, including muon, egamma and hadrons. Based on the particle flow candidates, jets met are computed and BTag and hadronic tau are identified in the jet collections. 

Fig~\ref{fig:exp:pfa} illustrates the function of the sub dector and how different final state particles behave in them. Starting from the beam interaction region, particles first enter a tracker, in which charged-particle trajectories (tracks) and origins (vertices) are reconstructed from signals (hits) in the sensitive layers. The tracker is immersed in a magnetic field that bends the trajectories and allows the electric charges and momenta of charged particles to be measured. Electrons and photons are then absorbed in an electromagnetic calorimeter (ECAL). The corresponding electromagnetic showers are detected as clusters of energy recorded in neighbouring cells, from which the energy and direction of the particles can be determined. Charged and neutral hadrons may initiate a hadronic shower in the ECAL as well, which is subsequently fully absorbed in the hadron calorimeter (HCAL). The corresponding clusters are used to estimate their energies and directions. Muons and neutrinos traverse the calorimeters with little or no interactions. While neutrinos escape undetected, muons produce hits in additional tracking layers called muon detectors, located outside the calorimeters. As a result,  muons leaves tracks in the tracker and muon chamber with MIP in ECAL and HCAL. Electrons and phontons deposit energy in the ECAL with and without track correspondence respectively. Charged and neutral hadrons deposit energy in both ECAL and HCAL with and without track correspondence respectively. 

\begin{figure}[ht]
    \centering
    \includegraphics[width=0.8\textwidth]{chapters/CMSExperiment/sectionReconstruction/figures/pfa.png}
    \caption{Caption}
    \label{fig:cmsexperiment:reconstruction:pfa}
\end{figure}

The FPA begins with computing PF element in each subdetector: tracks in the tracker and muon chamber, clusters in ECAL and HCAL. Then PF elements in different subdetectors are linked to form PF Blocks via a linking process. PF Block summary the activity of potential particle candidate in each subdetector. The details of reconstruction of PF elements and their linking can be found in []. In the end, each PF blocks are identified into PF candidates, final state particles including muon, egamma and hadrons based on which JetMET, BTag and Tau ID are further computed.





\subsection{Muon}

The reconstruction of muon involves standalone reconstruction in the muon system, followed by globle muon reconstruction add trajectories in the tracker. The standalone reconstruction starts with the track segments in the individual muon chambers. The state vectors (track position, momentum, and direction) associated with the segments found in the innermost chambers are used to seed the muon trajectories, working from inside out, using the \acrfull{kf} technique \cite{tech:kf:Fruhwirth:1987fm}. The track parameters and the corresponding errors are updated at each step. The procedure is iterated until the outermost measurement surface of the muon system is reached. A backward Kalman-filter is then applied, working from outside in, and the track parameters are defined at the innermost muon station. Finally, the track is extrapolated to the nominal interaction point and a vertex-constrained fit to the track parameters is performed.

The global muon reconstruction consists in extending the muon trajectories to include hits in the silicon tracker. tarting from a standalone reconstructed muon, the muon trajectory is extrapolated from the innermost muon station to the outer tracker surface, taking into account the muon energy loss in the material and the effect of multiple scattering. This extrapolation and the associated uncertainty defines a region of interest in the tracker, where track candidates are seeded by hit doubles and reconstructed using Kalman-filter. A set of trajectory fits to the global hits are carried out to exact the muon momentum and impact parameters. This retain both prompt muons and muons from hadrons decays with the highest possible efficiency.

\begin{figure}[ht]
    \centering
    \includegraphics[width=0.95\textwidth]{chapters/CMSExperiment/sectionReconstruction/figures/resMu.png}
    \caption{Momentum resolution as a function of }
    \label{fig:cmsexperiment:reconstruction:resMu}
\end{figure}


Figure~\ref{fig:exp:resMu} shows the transverse momentum resolution of muons with different energies for standalone reconstruction algorithm (a) and the global reconstruction algorithm (b). A significant improvement is achieved when going from standalone to global muon reconstruction.





\subsection{EGamma}

\begin{figure}[ht]
    \centering
    \includegraphics[width=0.5\textwidth]{chapters/CMSExperiment/sectionReconstruction/figures/resEle.png}
    \includegraphics[width=0.42\textwidth]{chapters/CMSExperiment/sectionReconstruction/figures/resGamma.png}
    \caption{Caption}
    \label{fig:cmsexperiment:reconstruction:resEle}
\end{figure}

The electron reconstruction in CMS is hampered by the amount of tracker material which is discretely distributed in front of the ECAL. The material thickness varies strongly with $\eta$, raising from 0.3 $\chi_0$ at the central barrel to 1.5 $\chi_0$ at the barrel edge, then falling to 0.7 $\chi_0$ in the endcap inner edge. When electrons traversing the silicon layers of the pixel and inner tracker detectors, they radiate collections of bremsstrahlung photons and the energy reaches the ECAL spread in $\phi$. \cite{cms:tdr1:Bayatian:2006nff} provides an good illustration -- ''For electrons at $p^T=$10 GeV, about half of the electrons radiate away more than half of their energy before reaching the surface of the ECAL. In about 10$\%$ of the cases, more than 95$\%$ of the initial electron energy is radiated!'' Further more, the radiated photon can convert into electron-positron pairs, which are usually soft and trapped in the field loosing all energy in the end undetected.

The reconstruction of electron start with making superclusters of ECAL energy deposits. The superclustering algorithm is optimized for the scenarios of energy spread in $\phi$. The supercluster drives the finding of track seeds, which is hit doublets in pixel. If a seed compatible with the supercluster is found, tracks building begins inside-out with a nonlinear filter approach called \acrfull{gsf} \cite{tech:gsf:Adam:2005bya}. For supercluster linked with GFS tracks by particle-flow algorithm, an electron candidate is made and a fit to the GSF track and ECAL superclusters is used to extract the four-momentum measurement under the electron assumption. This not only provides combines the advantages of trackers in low energy region and the ECAL in the high energy region, but also connect the low energy and high energy region smoothly. The energy resolution of electrons using tracker, ECAL and combined is shown in Figure~\ref{fig:exp:resEle} left. For supercluster not linked to GFS tracks, an photon candidate is made and the energy is obtained from the sum of energy deposited in a supercluster of crystals. To quantify the lateral spread of photon shower, a variable R9 is defined as the ratio of sum energy in 3x3 crystals around the highest crystal and the total energy sum in the supercluster. Energy resolution of photons with R9 >0.943 is shown in Figure~\ref{fig:exp:resEle} right.



\subsection{hadrons}
Once muons, electrons, and isolated photons are identified and removed from the PF blocks, the remaining particles to be identified are hadrons from jet fragmentation and hadronization. These particles may be detected as charged hadrons and neutral hadrons, among which $\pi^0$ decays to two non-isolated photons. The ECAL and HCAL clusters not linked to any track give rise to non-isolated photons and neutral hadrons. Within the tracker acceptance ($|\eta|< 2.5$), all these ECAL clusters are turned into non-isolated photons and all these HCAL clusters are turned into neutral hadrons. 

Charge hadrons are made from the remaining calorimeter clusters and tracks. Each of the remaining HCAL clusters of the PF block is linked to one or several tracks and these tracks may in turn be linked to some of the remaining ECAL clusters. It is possible that calorimeter clusters also include unresolved FSR photons or close-by non-isolated neutral hadrons around charged hadrons. To identify unresolved neutral components around charged hadron, a match of calibrated calorimeter energy and tracker momentum is carried out. If the calibrated calorimetric energy is compatible with the sum of the track momenta, no neutral particle is identified. The charged-hadron momenta are redefined by further calibration taking into account information from both tracker and calorimeter. If the calibrated calorimetric energy is in excess of the sum of the track momenta by an amount larger than the expected calorimetric energy resolution for hadrons, the excess may be interpreted as the presence of photons and neutral hadrons. The excess energy is first treated as a non-isolated photon and subtracted from the ECAL energy. If ECAL energy alone is not enough to account for the excess, the remaining excess is treated as a neutral hadron. If the calibrated calorimeter energy is smaller than the tracking momentum, a search for non-isolated muon in the track projection is carried out with relaxed muon reconstruction standard and momentum of the reconstructed muons is subtracted before a re-compare.



\subsection{Jet and Met}

\begin{figure}[ht]
    \centering
    \includegraphics[width=0.6\textwidth]{chapters/CMSExperiment/sectionReconstruction/figures/resJet.png}
    \caption{Caption}
    \label{fig:cmsexperiment:reconstruction:resJet}
\end{figure}

Jet are reconstructed by clustering PF candidates using anti-kt algorithm \cite{tech:antikt:Cacciari:2008gp}. The energy resolution of jet is shown in Figure~\ref{fig:exp:resJet}. Met is computed by balancing the total visible transverse momentum. 



\subsection{bTagging}
Jets originated from heavy flavor quarks are tag via multi-variational method taking into account the displacement of secondary vertices.

\subsection{Tau ID}
 Jets originated from hadonic taus are tagged via iso-plus-striped algorithms.



\section{Simulation}
\label{sec:cmsexperiment:simulation}

\begin{figure}[ht]
    \centering
    \includegraphics[width=0.6\textwidth]{chapters/CMSExperiment/sectionMCSimulation/figures/ps.png}
    \caption{Caption}
    \label{fig:cmsexperiment:simulation:ps}
\end{figure}

    \chapter{$Br(W)$ measurement and LU Test}
\label{sec:analysis}


The chapter presents the physics analysis of $Br(W)$ measurement and LU test based on the CMS Run2016 dataset. 
The works included in the chapter are done together with Nathaniel Odell, postdoc fellow at Northwestern University.
Nate is in charge of the shape analysis and has provided me many advice in the counting analysis. 

% The Section~\ref{sec:analysis:dataset} gives a description of the CMS dataset and the Monte Carlo simulation in the CMS Run 2016 used in the analysis. Section~\ref{sec:analysis:selection} summarizes the criteria of the object and event selection. Section~\ref{sec:analysis:background} explains the data driven method for the estimation of QCD background. Section~\ref{sec:analysis:method} discusses about the approach to extract the W decay branching fraction from the yields in the counting analysis. Section~\ref{sec:analysis:systematics} lists the systematical uncertainties considered in the $Br(W)$ measurement and the approach to combine them. Section~\ref{sec:analysis:result} presents the final result of the $Br(W)$ measurement and test of LU 




This document is structured as follows:

\begin{enumerate}
    \item A description of the datasets used in the analysis
    \item The selection of physics objects used in the analysis and event categories
    \item Methods for estimating backgrounds
    \item The methods used for extracting the branching fractions
    \item Estimation of systematic uncertainties
    \item Results of carrying out the branching fraction measurement and a test of lepton universality
\end{enumerate}
    
    
\section{Dataset and Simulated Events}
\label{sec:analysis:dataset}


\subsection{Data}
\label{sec:analysis:dataset:data}

In this analysis, data is selected based on the presence of at least one muon or one electron. The single muon dataset requires that events contain at least one muon with transverse momentum $\pt > 25\GeV$ and passing the loose track isolation criterion, $\rm Iso_{track} < 0.1$. The single electron dataset requires that there be at least one electron satisfying the requirement that $\pt > 30\GeV$ and that it passes the tight identification requirements as defined by the CMS Physics Object Group for electron and photon (EGamma POG).  The specific dataset names and the associated integrated luminosities are listed in Table~\ref{tab:analysis:dataset:data2016}.

\begin{table}[ht]
    \centering
    \setlength{\tabcolsep}{1em}
    \renewcommand{\arraystretch}{1.1}
    \caption{Data samples produced by CMS in 2016.} \label{tab:analysis:dataset:data2016}
    \begin{table}
    \caption{Data samples produced by CMS in 2016.
        \label{tab:data_2016}}
    \begin{tabular}{l c c}
    \hline
    Sample                                              & Run ranges    & $L_{int} (fb^{-1})$ \\
    \hline
    \texttt{SingleMuon/Run2016B-03Feb2017\_ver2-v2}     & 272007-275376 & 5.33                \\
    \texttt{SingleMuon/Run2016C-03Feb2017-v2}           & 275657-276283 & 2.4                 \\
    \texttt{SingleMuon/Run2016D-03Feb2017-v2}           & 276315-276811 & 4.26                \\
    \texttt{SingleMuon/Run2016E-03Feb2017-v2}           & 276831-277420 & 4.1                 \\
    \texttt{SingleMuon/Run2016F-03Feb2017-v2}           & 277772-278808 & 3.2                 \\
    \texttt{SingleMuon/Run2016G-03Feb2017-v2}           & 278820-280385 & 7.8                 \\
    \texttt{SingleMuon/Run2016H-03Feb2017\_ver*-v1}     & 281613-284044 & 9.2                 \\
    \hline
    \texttt{SingleElectron/Run2016B-03Feb2017\_ver2-v2} & 272007-275376 & 5.33                  \\
    \texttt{SingleElectron/Run2016C-03Feb2017-v2}       & 275657-276283 & 2.4                   \\
    \texttt{SingleElectron/Run2016D-03Feb2017-v2}       & 276315-276811 & 4.26                  \\
    \texttt{SingleElectron/Run2016E-03Feb2017-v2}       & 276831-277420 & 4.1                   \\
    \texttt{SingleElectron/Run2016F-03Feb2017-v2}       & 277772-278808 & 3.2                   \\
    \texttt{SingleElectron/Run2016G-03Feb2017-v2}       & 278820-280385 & 7.8                   \\
    \texttt{SingleElectron/Run2016H-03Feb2017\_ver*-v1} & 281613-284044 & 9.2                   \\
    \hline
    \end{tabular}
\end{table}

\end{table}



\noindent Run ranges where data quality is determined to be insufficient are removed from the datset by applying a luminosity mask. The following file is provided in JSON format from the CMS Physics Performance and Dataset (PPD) group:

\texttt{Cert\_271036-284044\_13TeV\_23Sep2016ReReco\_Collisions16\_JSON.txt}

\noindent The full dataset consists of 35.9\fbinv of integrated luminosity~\cite{cms:lumi2016:CMS-PAS-LUM-17-001}




\subsection{Simulated Dataset}
\label{sec:analysis:dataset:simulation}

Simulated datasets are used for modelling the major SM processes, including SM diboson, $\PW/\PZ/\PGg$ associated with jets, single-top and \ttbar. The background from multijet QCD is estimated by a data-driven approach discussed in Section~\ref{sec:analysis:background} The simulated samples used in modelling the background and signal are shown in Table~\ref{tab:analysis:dataset:mc2016}.  The production of the samples was carried during the Summer 2016 campaign and the production of the mini Analysis Oriented Data format (miniAOD) was done using CMSSW release \texttt{8\_0\_26\_patch2}. The same release was used for processing both the data and the simulated samples. Lepton universality is assumed for the simulated datasets, namely $ \BWl = 10.8\%$. To account for the deviation from the data, some corrections and reweightings of the simulated dataset are applied, which are discussed in Section~\ref{sec:analysis:calibration}.

\begin{table}[ht]
    \centering
    \setlength{\tabcolsep}{2em}
    \renewcommand{\arraystretch}{1.3}
    \caption{Simulated datasets.} \label{tab:analysis:dataset:mc2016}

    \begin{table}
    \centering
    \setlength{\tabcolsep}{1.5em}
    \renewcommand{\arraystretch}{1.25}
    \small

    \begin{tabular}{l l c}
        \hline
        Process                                           & Generator         & $\sigma \times \text{BR} (pb)$ \\
        \hline \hline
        $t\bar{t}$                                        & POWHEG+PYTHIA     & 831.76                         \\
        $t\bar{t}$  (leptonic)                            & POWHEG+PYTHIA     & 87.32                          \\
        $t\bar{t}$  (semi-leptonic)                       & POWHEG+PYTHIA     & 364.35                         \\
        $tW/\bar{t}W$                                     & POWHEG+PYTHIA     & 35.6                           \\
        \hline
        Z+jets                                            &                  &                                \\
        \hspace*{1em} $10 < m_{\ell\ell} < 50$ GeV        & AMC@NLO+PYTHIA   & 18610                          \\
        \hspace*{1em} $m_{\ell\ell} > 50 $GeV             & AMC@NLO+PYTHIA   & 5765                           \\
        \hspace*{1em} $m_{\ell\ell} > 50, N_{j} = 0 $GeV  & AMC@NLO+PYTHIA   & 4757                           \\
        \hspace*{1em} $m_{\ell\ell} > 50, N_{j} = 1 $GeV  & AMC@NLO+PYTHIA   & 884.4                          \\
        \hspace*{1em} $m_{\ell\ell} > 50, N_{j} = 2 $GeV  & AMC@NLO+PYTHIA   & 338.9                          \\
        \hline
        W + 1 jet                                         & MADGRAPH+PYTHIA  & 11486.5                        \\
        W + 2 jet                                         & MADGRAPH+PYTHIA  & 3775.2                         \\
        W + 3 jet                                         & MADGRAPH+PYTHIA  & 1139.8                         \\
        W + 4 jet                                         & MADGRAPH+PYTHIA  & 655.82                         \\
        \hline
        $qq\rightarrow WW \rightarrow 2\ell 2\nu$         & POWHEG           & 12.13                          \\
        $gg\rightarrow WW \rightarrow 2\ell 2\nu$         & POWHEG           & 0.588                          \\
        WZ $\rightarrow 3\ell \nu$                        & POWHEG+PYTHIA    & 5.29                           \\
        WZ $\rightarrow 2\ell 2q$                         & AMC@NLO+PYTHIA   & 5.595                          \\
        ZZ $\rightarrow 2\ell 2\nu$                       & POWHEG+PYTHIA    & 0.564                          \\
        ZZ $\rightarrow 2\ell 2q$                         & AMC@NLO+PYTHIA   & 3.22                           \\
        ZZ $\rightarrow 4\ell$                            & AMC@NLO+PYTHIA   & 1.21                           \\
    \end{tabular}

    \caption{Simulated MC samples.} \label{tab:dat:mc2016}
\end{table}

\end{table}







% \subsection{Standard Reweightings}



% \FloatBarrier



\section{Selection}
\label{sec:analysis:selection}



\subsection{Object Selection}
The event topologies of interest will require reconstructing electrons, muons, hadronically decaying tau leptons, hadronic jets, and missing tansverse energy (MET).  In this section, the reconstruction and selection of these physics objects is described.



\subsubsection{Primary vertex}
Primary vertices (PV) are reconstructed based on information from the tracking subsystem, mainly through the inner pixel detector. Quality cuts are applied to reconstructed PVs to guarantee they come from a proton-proton hard scattering event. These cuts are as follows,
\begin{equation*}
    N_{\rm d.o.f.}> 4; \quad  \left|z\right| < 24~\mathrm{cm}; \quad \sqrt{x^{2} + y^{2}} < 2~\mathrm{cm}.
\end{equation*}


The PVs are ordered based on the sum \pt of tracks used in their reconstruction. Selected physics objects are associated to the PV with the greatest sum \pt. 




\subsubsection{Muon}
Muon candidates are reconstructed using both the muon and tracker systems. The coverage of these two detector systems allows reconstruction of muons within $\left|\eta\right| < 2.4$ and $p_{T}$ as low as 5 GeV~\cite{Chatrchyan:2012xi}. Muons are required to be reconstructed using both the \emph{global muon} and \emph{tracker muon} reconstruction algorithms. These algorithms are distinct in that one begins with tracker information and extrapolates to find consistency with hits in the muons system (\emph{tracker muons}), while the other (\emph{global muon}) inverts the reconstruction steps starting from the muon system and finding tracks that are consistent. The combination of these two algorithms makes for a muon reconstruction that is accurate in predicting muon momentum and efficient in detecting muons within the detector acceptance.

In the interest of detecting muons decaying from vector bosons, a set of identification and isolation requirements are applied~\cite{Sirunyan:2018fpa}. The muon identification requirements are designed to have high selection efficiency and a low probability of misidentifying non-prompt muons originating from non-bosonic decays. The muon POG provided selection criteria are listed in table~\ref{tab:muon_id}.

\begin{table}[ht]
    \centering
    \setlength{\tabcolsep}{2em}
    \renewcommand{\arraystretch}{1.25}
    % \small
    \caption{Tight muon identification criteria as provided by muon POG}
    \label{tab:muon_id}
    \begin{tabular}{l|c}
    variable                            & cut value \\
    \hline
    isGlobal                            & True      \\
    isPF                                & True      \\
    $\chi^{2}$                          & $< 10$    \\
    number of matched stations          & $> 1$     \\
    number of pixel hits                & $> 0$     \\
    number of track layers              & $> 5$     \\
    number of valid hits                & $> 0$     \\
    $|d_{xy}|$                          & $< 0.2$   \\
    $|d_{z}|$                           & $< 0.5$    \\
    \hline
    $ISO_{PF}/p_{T}$ ($\rho$ corrected) & $< 0.15$
    \end{tabular}
\end{table}


\noindent To increase the likelihood of selecting muons produced by the prompt decay of vector bosons, an isolation requirement is placed on all muons. The isolation of the muon is calculated by summing the \pt of all charged hadronic, neutral hadronic, and photon particle flow candidates in a cone of radius $\Delta R = 0.4$ about the muon candidate. This quantity is corrected to remove the contamination of the neutral component due to pileup by subtracting off the average energy deposited by pileup. It is defined as,
\begin{equation}
    I_{\rm PF} = I_{\rm ch. had} + \max \left(0, I_{\rm neu. had} +
    I_{\gamma} - 0.5 I_{\rm PU}\right)
\end{equation}

Simulated events are reweighted to account for differences in the muon reconstruction, identification, and isolation efficiencies with respect to data.







\subsubsection{Electron}

Electrons are reconstructed by combining information from the electromagnetic calorimeter and the tracking system using a gaussian-sum
filter (GSF) method \cite{Baffioni:2006cd}.  Corrections are applied to account for mismeasurement of electron momentum scale and resolution. All electrons are required to have $\pt \geq 20~\GeV$ and $|\eta| < 2.5$.  Electrons are identified using a tight cut-based scheme. The requirements for this selection are listed in table~\ref{tab:slt:electron_id}.

\begin{table}[ht]
    \centering
    \setlength{\tabcolsep}{2em}
    \renewcommand{\arraystretch}{1.25}
    % \small
    \caption{Tight electron identification criteria as provided by the egamma POG.}
    \label{tab:slt:electron_id}

    \begin{tabular}{l|c|c}
        \hline
        variable                          & $|\eta| < 1.4446$ & $|\eta| \geq 1.566$ \\
        \hline
        $\sigma_{i\eta}\sigma_{i\eta}$    & $<0.00998$        & $0.0394$            \\
        $|d\eta|$                         & $<0.00308$        & $0.0292$            \\
        $|d\phi|$                         & $<0.0816$         & $0.00605$           \\
        $H/E$                             & $<0.0414$         & $0.0641$            \\
        $|\frac{1}{E} - \frac{1}{p}|$     & $<0.0129$         & $0.0129$            \\
        missing hits                      & $\leq 1$          & $\leq 1$            \\
        $|d_{0}|$                         & $<1.$             & $<1.$               \\
        \hline
        conversion rejection              & true              & true                \\
        I$_{PF}$/$p_{T}$ (EA corrected) & $< 0.0588$        & $<0.0571$           \\
        \hline
    \end{tabular}
\end{table}

\noindent The electrons are also required to pass a tight isolation criteria. The isolation variable is constructed by summing the energy of charged and neutral particle flow objects within a cone of radius $\Delta R = 0.4$ about the electron candidate and subtracting off the contribution from pileup.  The combined particle flow isolation with the pileup correction is,
\begin{equation}
    \nonumber
    I_{\rm comb} = I_{\rm ch. had.} + \max\left(  0, I_{\rm neu. had.} + I_{\gamma} - \rho EA(|\eta_{e}|) \right).
\end{equation}

The pileup correction is dependent on the parameter $\rho$ which correlates with the average energy due to pileup, and the effective area which changes depending on the $|\eta|$ value of the electron.





\subsubsection{Hadronic Tau}

Hadronically decaying $\tau$ leptons are reconstructed using the hadron-plus-strips algorithm~\cite{ref:cms-tau}. This algorithm constructs candidates seeded by a PF jet that are consistent with either a single or triple charged pion decay of the $\tau$ lepton.  In the single charged pion decay mode, the presence of neutral pions is detected by reconstructing their photonic decays. If the hadronic tau candidate is found to overlap ($\Delta R < 0.3$) with either an electron or muon passing the analysis selections listed above, the tau candidate is rejected. Jets originating from non-$\tau$ decays are rejected with a MVA discriminator that takes into account the pileup contribution to the neutral component of the $\tau$ decay~\cite{CMS-TAU-16-003-001}.  

Reconstructed hadronic taus are required to have $p_{T} > 20$ GeV and $|\eta| < 2.3$ unless noted otherwise.  It is observed that the counting analysis is more sensitive to misidentification of hadronic jets as hadronic tau candidates, so while the tight working point is used in the shape analysis, the very tight working point is used for the counting analysis.

The scale factor accounting for different tau reconstruction and identification efficiencies were measured in several control regions~\cite{CMS-TAU-16-003-001}.  The measurement is carried out in both in two different control regions: one enriched in $Z\to\tau\tau$ production and one enriched in \ttbar.  Because of the large overlap with our signal region in the case of the latter, the former measurement is used so that the datasets that are used are uncorrelated.  For the selection algorithm and tight working point a scale factor of $0.95 \pm 0.05$ is used; for the very tight working point it is $0.92 \pm 0.05$.






\subsubsection{Jet}
Jets are reconstructed from PF candidates \cite{ref:pf}. PF candidates combine information from all of the detector subsystems to facilitate the reconstruction and identification of individual particles.  These PF candidates are clustered using the anti-$k_{t}$ algorigthm \cite{Cacciari:2008gp} with a cone size of $\Delta R = 0.4$. Once reconstructed, a number of corrections are applied to the jets to correct for pileup contamination, differing absolute response in jet \pt, and relative response in $\eta$ \cite{ref:jetscale}.  To reduce contamination from photons and prompt leptons, several ID requirements are placed on the jets and are listed in table~\ref{tab:slt:jet_id_2016}.

\begin{table}[ht]
    \centering
    \setlength{\tabcolsep}{0.8em}
    \renewcommand{\arraystretch}{1.25}
    % \small
    \caption{Jet ID requirements for 2016.}
    \label{tab:slt:jet_id_2016}
    \begin{tabular}{l|ccc}
        \hline
                                    & $|\eta| < 2.4$ & $2.4 < |\eta| \leq 3.0$ & $3.0 < |\eta| \leq 4.7$ \\
        \hline                                                                   
        number of constituents         & $> 1$          & $> 1$                  & -- \\
        neutral hadronic fraction      & $< 0.99$       & $< 0.99$               & -- \\
        neutral EM fraction            & $< 0.99$       & $< 0.99$               & $<0.9$ \\
        charged hadronic fraction      & $> 0$          & --                     & -- \\
        charged EM fraction            & $< 0.99$       & --                     & -- \\
        number of charged constituents & $> 0$          & --                     & -- \\
        number of neutrals             & --             & --                     & $>10$                   \\
        \hline
    \end{tabular}
\end{table}

\noindent In addition to the above requirements, it is required that all jets have $\pt > 30$ GeV and $\left|\eta\right| < 4.7$.  Jets are vetoed if they overlap with a muon, electron, or tau passing the identification requirements described above within a cone size of $\Delta R = 0.3$. 

The identification of jets originating from the decay of b quarks is done using the CSV b-tagging algorithm~\cite{Sirunyan:2298594} is used to optimize the efficiency for identifying b-jets while reducing the misidentification from jets originating from light quark (usdg).  In this analysis, the recommended medium working point ($\text{CSV} > 0.8484$) supplied by the b tag POG is used.  

To account for the difference in b tag efficiency in data and simulation, the b tag status of jets is modified based on a set of scale factors derived by the b tag POG.  The method used for applying the b tag scale factors modifies the status of individual jets to either promote or demote their b tagging status~\cite{twiki:btag_method}.  The method relies on the user measuring the b tag and mistag efficiencies in the simulated samples.  This is described further in appendix~\label{app:btag}.


\FloatBarrier

% ===========================
% Event Selection
% ===========================
\subsection{Event Selection}

The event selection begins by requiring an event pass the lowest \pt theshold single electron or muon trigger that is not prescaled. From these datasets it is possible to select on a number of $\PW\PW$-like final states originating from \ttbar and tW production.  These final states are constructed based on the number of reconstructed leptons, jet multiplicity, and b tag multiplicity.  The categorization of these events differs between the counting and shape analysis.  The common definition of the categories are listed below.


\begin{itemize}
    \singlespacing
    \item $ee$:
    \begin{itemize}
        \item exactly two electrons
        \item $p_{T} > 30, 20$ GeV
        \item reject events with hadronic taus or muons
        \item Z boson veto ($N_{b} \geq 1$ only): $M_{ee} < 75$ or $M_{ee} > 105\,\GeV$
    \end{itemize}
    \item $\mu\mu$:
    \begin{itemize}
        \item exactly two muons
        \item $p_{T} > 25, 10$ GeV
        \item reject events with hadronic taus or electrons
        \item Z boson veto ($N_{b} \geq 1$ only): $M_{\mu\mu} < 75$ or $M_{\mu\mu} > 105\,\GeV$
    \end{itemize}
    \item $e\mu$:
    \begin{itemize}
        \item exactly one electron and one muon
        \item lead muon (electron) $p_{T} > 25 (30)\,\GeV$
        \item trailing muon (electron) $p_{T} > 10\,(20)\,\GeV$
        \item reject events with hadronic taus 
        \item events in electron datastream that fire muon trigger are vetoed
    \end{itemize}
    \item $e\tau$:
    \begin{itemize}
        \item exactly one electron and one hadronic tau
        \item $p_{T} > 30, 20$ GeV
        \item reject events with muons
    \end{itemize}
    \item $\mu\tau$:
    \begin{itemize}
        \item exactly one muon and one hadronic tau
        \item $p_{T} > 25, 20$ GeV
        \item reject events with electrons
    \end{itemize}
    \item e + jets:
    \begin{itemize}
        \item exactly one electron 
        \item $p_{T} > 30$ GeV
        \item reject events with muons or hadronic taus
        \item at least four jets
    \end{itemize}
    \item $\mu + jets$:
    \begin{itemize}
        \item exactly one muon 
        \item $p_{T} > 25$ GeV
        \item reject events with electrons or hadronic taus
        \item at least four jets
    \end{itemize}
\end{itemize}




These selections are designed to primarily target $\ttbar$ production and specific \PW decay modes. The final states will tend to only contain events from a single datastream except for the $e\mu$ selection which has non-negligible overlap between the electron and muon datastreams. Any overlap in events between the two datastreams are removed by only taking the event from single muon datastream. Because the tau can decay to an electron, muon, or hadronically, each of these channels has some mixing between terms arising from $W\rightarrow\ell$ decays and $W\rightarrow\tau\rightarrow\ell$ decays.  The mixing between the selected final states and the underlying W boson decays are shown in table~\ref{tab:signal_breakdown}. These numbers are estimated from simulated \ttbar events and are consequently dependent on the values of branching fractions used in the simulation.  

For the counting analysis, there is always a requirement that there be at least two jets and at least one b tagged jet. The categories are partitioned based on whether there is exactly one b jet or there are two or more b jets.  Additionally, the $e\mu$ selection is split into muon triggered and electron triggered categories: if the muon has the highest \pt and the muon trigger has fired it categorized as a $\mu e$ event; if the electron has highest \pt and the electron trigger fired it is categorized as an $e \mu$ event.

The shape analysis takes advantage of subdividing the data into several additional b tag and jet multiplicity categories to constrain systematics uncertainties, and to take advantage of higher tau reconsruction purity in lower jet multiplicity bins.  The categories are shown in table~\ref{tab:jet_categories}.  As described in the list above, the $ee$ and $\mu\mu$ categories have a Z veto applied in the case that there are one or more b tags; this requirement is not applied in the zero b tag case.  There is also a set of requirements to enhance the proportion of Drell-Yan in the $e\tau$ and $\mu\tau$ categories in the case that the number of jets is 0 or 1 and there are no b tags.  The requirements are constructed to mainly reduce the W boson contribution and are :
\begin{equation*}
    40 \GeV \leq M_{\ell\tau_{h}} \leq 100 \GeV, \quad
    \Delta\phi(\ell, \tau_{h}) > 2.5, \quad
    M_{T}^{\ell} < 60 \GeV,
\end{equation*}

\noindent where $M_{T}^{\ell}$ is the transverse mass of the electron or muon,
\begin{equation}
\label{eq:trans_mass}
    M_{T,\ell} = \sqrt{2 p_{T}^{\ell}\MET (1-\cos\Delta\phi(p_{T}^{\ell}, \MET))}.
\end{equation}


\begin{table}[]
    \centering
    \setlength{\tabcolsep}{1.5em}
    \renewcommand{\arraystretch}{1.1}
    \caption{Categories used in the shape analysis based on jet and b
    tag multiplicities.}
    
    \begin{tabular}{l|c|c|c|c}
                                    & $N_{j} = 0$        & $N_{j} = 1$        & $N_{j} = 2$        & $N_{j} \geq 3$     \\
	\hline
    \multirow{2}{*}{$N_{b} = 0$}    & $e\tau$, $\mu\tau$ & $e\tau$, $\mu\tau$ & \multicolumn{2}{c}{$e\tau$, $\mu\tau$} \\
                                    & $e\mu$             & $e\mu$             & \multicolumn{2}{c}{$ee, \mu\mu, e\mu$} \\
	\hline
    \multirow{3}{*}{$N_{b} = 1$}    &                    & $e\tau$, $\mu\tau$ & $e\tau$, $\mu\tau$ & $e\tau$, $\mu\tau$ \\
	\cline{4-5}
                                    &                    & $e\mu$             & \multicolumn{2}{c}{$ee, \mu\mu, e\mu$}  \\
                                    &                    &                    & \multicolumn{2}{c}{$ej$, $\mu j$}  \\
	\hline
    \multirow{3}{*}{$N_{b} \geq 2$} & \multicolumn{2}{c|}{}                   & $e\tau$, $\mu\tau$ & $e\tau$, $\mu\tau$ \\
	\cline{4-5}
                                    & \multicolumn{2}{c|}{}                   & \multicolumn{2}{c}{$ee, \mu\mu, e\mu$}  \\
                                    & \multicolumn{2}{c|}{}                   & \multicolumn{2}{c}{$ej$, $\mu j$}  \\
	\hline
    \end{tabular}
    
    \label{tab:jet_categories}
\end{table}


% \begin{table}[ht]
	\centering
	\setlength{\tabcolsep}{0.4em}
    \renewcommand{\arraystretch}{1.5}
    \small
    
    \begin{tabular}{|cc|cc|cc|cc|}
    
    %%%%%%%%%%%%%%%%%%
	% mu-trigger
	%%%%%%%%%%%%%%%%%%
    \multicolumn{8}{c}{single muon trigger} \\
    \hline
    \multicolumn{2}{|c|}{$\mu e$} 					& \multicolumn{2}{c|}{$\mu\mu$} 				  & \multicolumn{2}{c|}{$\mu \tau$} 					& \multicolumn{2}{c|}{$\mu + jets$} 			  	\\
    \hline
    1b & 2b                   						& 1b & 2b        	 						      & 1b & 2b        										& 1b & 2b         			        				\\
    \hline 
    \multicolumn{2}{|c|}{$n_{e,\mu,\tau_h} = 1,1,0$}& \multicolumn{2}{c|}{$n_{e,\mu,\tau_h} = 0,2,0$} & \multicolumn{2}{c|}{$n_{e,\mu,\tau_h} = 0,1,1$}      & \multicolumn{2}{c|}{$n_{e,\mu,\tau_h} = 0,1,0$} 	\\
    \multicolumn{2}{|c|}{$p^T_\mu,p^T_e>25,15$ GeV} & \multicolumn{2}{c|}{$p^T_\mu,p^T_\mu>25,10$ GeV}& \multicolumn{2}{c|}{$p^T_\mu,p^T_{\tau_h}>30,20$ GeV}& \multicolumn{2}{c|}{$p^T_\mu>30$ GeV}           	\\
    \multicolumn{2}{|c|}{$p^T_\mu>p^T_e$} 			& \multicolumn{2}{c|}{$|m_{\mu\mu}-m_Z|>15$ GeV } & \multicolumn{2}{c|}{ --- }						     & \multicolumn{2}{c|}{ --- } 						\\
    \multicolumn{2}{|c|}{$n_{jet}\geq2$}			& \multicolumn{2}{c|}{$n_{jet}\geq2$}             &  \multicolumn{2}{c|}{$n_{jet}\geq2$} 				 & \multicolumn{2}{c|}{$n_{jet}\geq4$}             	\\
    \hline

    
    %\multicolumn{8}{c}{} \\
    \multicolumn{8}{c}{single electron trigger} \\
    
    %%%%%%%%%%%%%%%%%%
	% e-trigger
	%%%%%%%%%%%%%%%%%%
    \hline
    \multicolumn{2}{|c|}{$e e$} 					& \multicolumn{2}{c|}{$e \mu$} 				      & \multicolumn{2}{c|}{$e \tau$} 			     		& \multicolumn{2}{c|}{$e + jets$} 			     	\\
    \hline
    1b & 2b                   						& 1b & 2b        	 						      & 1b & 2b        										& 1b & 2b         			        				\\
    \hline 
    \multicolumn{2}{|c|}{$n_{e,\mu,\tau_h} = 2,0,0$}& \multicolumn{2}{c|}{$n_{e,\mu,\tau_h} = 1,1,0$} & \multicolumn{2}{c|}{$n_{e,\mu,\tau_h} = 1,0,1$}      & \multicolumn{2}{c|}{$n_{e,\mu,\tau_h} = 1,0,0$} 	\\
    \multicolumn{2}{|c|}{$p^T_e,p^T_e>30,15$ GeV}   & \multicolumn{2}{c|}{$p^T_e,p^T_\mu>30,10$ GeV}  & \multicolumn{2}{c|}{$p^T_e,p^T_{\tau_h}>30,20$ GeV}  & \multicolumn{2}{c|}{$p^T_e>30$ GeV}           	\\
    \multicolumn{2}{|c|}{$|m_{ee}-m_Z|>15$ GeV }    & \multicolumn{2}{c|}{$p^T_e>p^T_\mu$}   		  & \multicolumn{2}{c|}{ --- }						     & \multicolumn{2}{c|}{ --- } 						\\
    \multicolumn{2}{|c|}{$n_{jet}\geq2$}			& \multicolumn{2}{c|}{$n_{jet}\geq2$}             & \multicolumn{2}{c|}{$n_{jet}\geq2$} 				 & \multicolumn{2}{c|}{$n_{jet}\geq4$}             	\\
    \hline
    
    \end{tabular}
    
    \caption{Analysis selections of 8 channels based on single muon and single electron triggers.}
    \label{tab:slt:eventSelection}
\end{table}


% \begin{figure}[ht]
%     \centering
%     \includegraphics[width=0.99\textwidth]{chapters/Analysis/sectionSelection/figures/counting.png}
%     \includegraphics[width=0.99\textwidth]{chapters/Analysis/sectionSelection/figures/shaping.png}
%     \caption{caption}
%     \label{fig:analysis:selection:yields}
% \end{figure}


% \begin{figure}[ht]
%     \centering
%     \includegraphics[width=0.49\textwidth]{chapters/Analysis/sectionSelection/figures/trgLep_mumu.png}
%     \includegraphics[width=0.49\textwidth]{chapters/Analysis/sectionSelection/figures/trgLep_emu.png}
%     \includegraphics[width=0.49\textwidth]{chapters/Analysis/sectionSelection/figures/trgLep_ee.png}
%     \includegraphics[width=0.49\textwidth]{chapters/Analysis/sectionSelection/figures/trgLep_emu2.png}
%     \caption{caption}
%     \label{fig:analysis:selection:trigger}
% \end{figure}




\begin{sidewaystable}[ht]
    \centering
    \setlength{\tabcolsep}{0.4em}
    \renewcommand{\arraystretch}{1.5}

    \caption{Composition of accepted $t\bar{t}$+$tW$ events, breakdown by 21 WW decay.  Values are in percent.}
    \resizebox{\textwidth}{!}{
    \begin{tabular}{|l|cc|cc|cc|cc|cc|cc|cc|cc|}
    
    
    \hline
    channel & \multicolumn{2}{|c|}{$\mu e$} & \multicolumn{2}{c|}{$\mu\mu$} & \multicolumn{2}{|c|}{$\mu \tau$} & \multicolumn{2}{|c|}{$\mu$+jets} & \multicolumn{2}{|c|}{$ee$} & \multicolumn{2}{|c|}{$e\mu$} & \multicolumn{2}{|c|}{$e \tau$} & \multicolumn{2}{|c|}{$e+jets$} \\
    \hline
    $\rm n_{b tag}$ & $n_b=1$ & $n_b\geq2$ & $n_b=1$ & $n_b\geq2$ & $n_b=1$ & $n_b\geq2$ & $n_b=1$ & $n_b\geq2$ & $n_b=1$ & $n_b\geq2$ & $n_b=1$ & $n_b\geq2$ & $n_b=1$ & $n_b\geq2$ & $n_b=1$ & $n_b\geq2$ \\ 
    \hline
    
    $tt/tW \to ee$                     &   -- &   -- &   -- &   -- &   -- &   -- &   -- &   -- & 87.4 & 87.8 &   -- &   -- &  0.7 &   -- &  3.1 &  3.1 \\ 
    $tt/tW \to \mu\mu$                 &   -- &   -- & 81.6 & 83.0 &   -- &   -- &  1.3 &  1.2 &   -- &   -- &   -- &   -- &   -- &   -- &   -- &   -- \\ 
    $tt/tW \to e\mu$                   & 86.5 & 87.0 &   -- &   -- &  0.8 &  0.5 &  3.3 &  3.3 &   -- &   -- & 82.7 & 84.1 &   -- &   -- &  1.4 &  1.4 \\ 
    $tt/tW \to \tau_{e}\tau_{e}$       &   -- &   -- &   -- &   -- &   -- &   -- &   -- &   -- &   -- &   -- &   -- &   -- &   -- &   -- &   -- &   -- \\ 
    $tt/tW \to \tau_{\mu}\tau_{\mu}$   &   -- &   -- &  0.7 &  0.6 &   -- &   -- &   -- &   -- &   -- &   -- &   -- &   -- &   -- &   -- &   -- &   -- \\ 
    $tt/tW \to \tau_{e}\tau_{\mu}$     &   -- &   -- &   -- &   -- &   -- &   -- &   -- &   -- &   -- &   -- &  0.6 &  0.6 &   -- &   -- &   -- &   -- \\ 
    $tt/tW \to \tau_{e}\tau_{h}$       &   -- &   -- &   -- &   -- &   -- &   -- &   -- &   -- &   -- &   -- &   -- &   -- &  3.1 &  3.2 &   -- &   -- \\ 
    $tt/tW \to \tau_{\mu}\tau_{h}$     &   -- &   -- &   -- &   -- &  3.2 &  3.6 &   -- &   -- &   -- &   -- &   -- &   -- &   -- &   -- &   -- &   -- \\ 
    $tt/tW \to \tau_{h}\tau_{h}$       &   -- &   -- &   -- &   -- &   -- &   -- &   -- &   -- &   -- &   -- &   -- &   -- &   -- &   -- &   -- &   -- \\ 
    $tt/tW \to e\tau_{e}$              &   -- &   -- &   -- &   -- &   -- &   -- &   -- &   -- & 11.7 & 11.5 &   -- &   -- &   -- &   -- &  0.8 &  0.8 \\ 
    $tt/tW \to e\tau_{\mu}$            &  4.1 &  4.0 &   -- &   -- &   -- &   -- &   -- &   -- &   -- &   -- & 11.2 & 11.0 &   -- &   -- &   -- &   -- \\ 
    $tt/tW \to e\tau_{h}$              &   -- &   -- &   -- &   -- &   -- &   -- &   -- &   -- &   -- &   -- &   -- &   -- & 57.5 & 63.6 &  3.4 &  3.6 \\ 
    $tt/tW \to \mu\tau_{e}$            &  8.3 &  8.2 &   -- &   -- &   -- &   -- &  0.6 &  0.7 &   -- &   -- &  3.6 &  3.6 &   -- &   -- &   -- &   -- \\ 
    $tt/tW \to \mu\tau_{\mu}$          &   -- &   -- & 15.7 & 15.8 &   -- &   -- &   -- &   -- &   -- &   -- &   -- &   -- &   -- &   -- &   -- &   -- \\ 
    $tt/tW \to \mu\tau_{h}$            &   -- &   -- &   -- &   -- & 57.4 & 63.6 &  3.4 &  3.6 &   -- &   -- &   -- &   -- &   -- &   -- &   -- &   -- \\ 
    $tt/tW \to eh$                     &   -- &   -- &   -- &   -- &   -- &   -- &   -- &   -- &   -- &   -- &  1.6 &   -- & 35.9 & 30.5 & 85.6 & 85.4 \\ 
    $tt/tW \to \mu h$                  &   -- &   -- &  1.7 &  0.5 & 35.7 & 30.1 & 85.3 & 85.3 &   -- &   -- &   -- &   -- &   -- &   -- &   -- &   -- \\ 
    $tt/tW \to \tau_{e}h$              &   -- &   -- &   -- &   -- &   -- &   -- &   -- &   -- &   -- &   -- &   -- &   -- &  1.9 &  1.6 &  4.8 &  4.7 \\ 
    $tt/tW \to \tau_{\mu}h$            &   -- &   -- &   -- &   -- &  2.1 &  1.6 &  5.0 &  4.9 &   -- &   -- &   -- &   -- &   -- &   -- &   -- &   -- \\ 
    $tt/tW \to \tau_{h}h$              &   -- &   -- &   -- &   -- &   -- &   -- &   -- &   -- &   -- &   -- &   -- &   -- &   -- &   -- &   -- &   -- \\ 
    $tt/tW \to hh$                     &   -- &   -- &   -- &   -- &   -- &   -- &   -- &   -- &   -- &   -- &   -- &   -- &   -- &   -- &   -- &   -- \\ 

    \hline
    \end{tabular}}
    \label{tab:analysis:selection:signal_breakdown}
    
\end{sidewaystable}

\begin{sidewaystable}[ht]
    \centering
    % \rule{1.5\textwidth}{0.8\textwidth}
    % \toprule
    \setlength{\tabcolsep}{0.0em}
    \renewcommand{\arraystretch}{2}
    \footnotesize
    \begin{tabular}{l|cccccccc|cc}
    \hline
        & QCD & VV  & $\gamma$ & Z & W & t & tW & tt & total & data      \\
    \hline
    
    $\mu e$, $n_b=1$                   &       --$\pm$     -- &     90.3$\pm$    4.2 &      0.9$\pm$    0.9 &    202.7$\pm$   37.6 &     13.4$\pm$    5.1 &      9.5$\pm$    2.6 &   2107.6$\pm$   53.1 &  38871.4$\pm$   87.5 &  41295.8$\pm$  109.2 &  41047.0$\pm$  202.6 \\ 
    $\mu e$, $n_b\geq2$                &       --$\pm$     -- &      5.9$\pm$    1.0 &       --$\pm$     -- &       --$\pm$     -- &      3.1$\pm$    2.2 &      2.3$\pm$    1.6 &    625.7$\pm$   28.9 &  22647.7$\pm$   66.8 &  23270.9$\pm$   74.1 &  23918.0$\pm$  154.7 \\ 
    \hline
    $\mu\mu$, $n_b=1$                  &       --$\pm$     -- &    370.4$\pm$    5.8 &      4.1$\pm$    1.8 &  18046.9$\pm$  455.4 &     52.4$\pm$   11.7 &     55.8$\pm$    6.7 &   3406.2$\pm$   68.8 &  62266.6$\pm$  112.4 &  84202.3$\pm$  474.3 &  84284.0$\pm$  290.3 \\ 
    $\mu\mu$, $n_b\geq2$               &       --$\pm$     -- &     45.8$\pm$    1.5 &      0.0$\pm$    0.0 &   1945.7$\pm$  142.0 &      3.6$\pm$    2.6 &      3.9$\pm$    1.8 &    959.3$\pm$   36.2 &  35685.2$\pm$   85.1 &  38643.4$\pm$  169.6 &  39253.0$\pm$  198.1 \\ 
    \hline
    $\mu\tau$, $n_b=1$                 &   1130.7$\pm$  108.8 &     52.3$\pm$    2.6 &     11.8$\pm$    3.2 &    866.7$\pm$   78.7 &    730.8$\pm$   42.9 &    182.6$\pm$   12.4 &   1291.0$\pm$   41.9 &  18430.0$\pm$   60.6 &  22695.9$\pm$  159.6 &  21621.0$\pm$  147.0 \\ 
    $\mu\tau$, $n_b\geq2$              &    346.6$\pm$   51.5 &      5.5$\pm$    0.7 &      0.9$\pm$    0.8 &    103.6$\pm$   29.6 &     56.9$\pm$   14.4 &     36.8$\pm$    5.6 &    322.6$\pm$   21.0 &   9647.6$\pm$   43.7 &  10520.4$\pm$   78.3 &   9934.0$\pm$   99.7 \\ 
    \hline
    $\mu$+jets, $n_b=1$                &  24300.4$\pm$ 3404.9 &    371.0$\pm$    5.2 &   1501.2$\pm$   67.5 &   7533.2$\pm$  265.9 &  49248.1$\pm$  327.3 &   8484.6$\pm$   85.3 &  24447.8$\pm$  187.0 & 514064.6$\pm$  327.2 & 629950.9$\pm$ 3453.3 & 630704.0$\pm$  794.2 \\ 
    $\mu$+jets, $n_b\geq2$             &   4650.7$\pm$ 1399.5 &     61.4$\pm$    2.0 &    248.3$\pm$   31.8 &   1331.9$\pm$  114.0 &   6524.2$\pm$  118.8 &   5172.2$\pm$   66.7 &  10335.6$\pm$  121.4 & 356185.1$\pm$  272.2 & 384509.5$\pm$ 1442.2 & 385397.0$\pm$  620.8 \\ 
    \hline
    $e e$, $n_b=1$                     &       --$\pm$     -- &    138.2$\pm$    3.6 &      2.8$\pm$    1.2 &   4726.5$\pm$  215.7 &      5.4$\pm$    2.8 &      1.1$\pm$    0.8 &   1382.0$\pm$   42.7 &  23447.3$\pm$   66.9 &  29703.3$\pm$  229.9 &  29491.0$\pm$  171.7 \\ 
    $e e$, $n_b\geq2$                  &       --$\pm$     -- &     16.2$\pm$    0.9 &      0.1$\pm$    0.1 &    500.5$\pm$   67.8 &      3.7$\pm$    2.6 &      2.1$\pm$    1.2 &    371.4$\pm$   22.1 &  13412.7$\pm$   50.7 &  14306.6$\pm$   87.5 &  14334.0$\pm$  119.7 \\ 
    \hline
    $e\mu$, $n_b=1$                    &       --$\pm$     -- &    127.2$\pm$    4.9 &     25.5$\pm$   13.2 &    411.9$\pm$   52.7 &     32.8$\pm$    7.2 &     37.6$\pm$    5.4 &   2917.6$\pm$   62.7 &  49878.6$\pm$   99.2 &  53431.1$\pm$  129.8 &  52362.0$\pm$  228.8 \\ 
    $e\mu$, $n_b\geq2$                 &       --$\pm$     -- &      9.0$\pm$    1.3 &      1.9$\pm$    1.1 &     59.0$\pm$   19.5 &      6.5$\pm$    3.2 &      6.1$\pm$    2.2 &    837.9$\pm$   33.8 &  28374.1$\pm$   74.9 &  29294.5$\pm$   84.6 &  29860.0$\pm$  172.8 \\ 
    \hline
    $e\tau$, $n_b=1$                   &    874.2$\pm$   90.3 &     38.0$\pm$    2.1 &    194.5$\pm$   38.8 &    677.8$\pm$   69.3 &    456.3$\pm$   32.9 &    125.3$\pm$   10.0 &    908.2$\pm$   34.6 &  12884.7$\pm$   49.7 &  16159.1$\pm$  139.0 &  15309.0$\pm$  123.7 \\ 
    $e\tau$, $n_b\geq2$                &     94.2$\pm$   46.3 &      3.0$\pm$    0.4 &     10.0$\pm$    2.9 &     53.4$\pm$   21.3 &     28.7$\pm$    8.5 &     43.4$\pm$    6.0 &    196.1$\pm$   15.9 &   6682.4$\pm$   35.8 &   7111.3$\pm$   65.1 &   7006.0$\pm$   83.7 \\ 
    \hline
    $e$+jets, $n_b=1$                  &  25625.1$\pm$ 2941.3 &    494.9$\pm$    5.1 &  12035.7$\pm$  173.0 &  13119.8$\pm$  323.2 &  34481.3$\pm$  266.1 &   5786.3$\pm$   68.8 &  17454.7$\pm$  154.8 & 360917.6$\pm$  268.5 & 469915.4$\pm$ 2992.9 & 464543.0$\pm$  681.6 \\ 
    $e$+jets, $n_b\geq2$               &   3327.4$\pm$ 1476.4 &     84.5$\pm$    2.0 &   2095.3$\pm$   78.4 &   2520.8$\pm$  138.5 &   4696.3$\pm$   98.0 &   3524.2$\pm$   53.7 &   7616.3$\pm$  102.3 & 249557.0$\pm$  223.4 & 273421.8$\pm$ 1509.3 & 274162.0$\pm$  523.6 \\ 
    \hline

    \end{tabular}
    \caption{Estimates of the yields. The estimate of the expected yield is compared to
    the yield observed from data.  Uncertainties are statistical only.
    \label{tab:yields}}
\end{sidewaystable}

\input{chapters/Analysis/sectionSelection/tables/yields_ltau.tex}

\FloatBarrier



\section{Background Estimation}
\label{sec:analysis:background}





This analysis has three main sources of backgrounds:

\begin{itemize}
    \item vector boson plus jets
    \item multijet QCD
    \item diboson production
\end{itemize}

Vector boson (W or Z) production is the most prominent source of backgrounds overall.  These processes are well modelled and the simulation is found to be sufficient for the precision of modelling we require.  Diboson production is similarily well modelled, but is far less significant.  

It is worth pointing out that normalization of the W+jet, WW, and WZ production will all be sensitive to variations of the \PW decay branching fractions.  In general the yield from these processes are very small compared to \ttbar and tW production.  It is also the case that we assume fairly conservative uncertainties on the normalization of these processes ($\geq 5\%$) such that there should not be any sensitivity to the effect of the variation of the branching fractions.  For the case of the shape analysis, the \PW + jets sample is treated as a signal sample and is decomposed based on the decay modes of the \PW boson.  

The multijet QCD background affects the $e\tau$, $e$ + jets, $\mu\tau$, and $\mu$ + jets channels.  The number of generated events for this process is found to be insufficient for accurately modelling the event yields and shapes for this process so data-driven methods are employed.

The breakdown of predicted backgrounds are shown in tables~\ref{tab:yields} and \ref{tab:yields_ltau}.   


In several of the selected channels there is a non-negligible contamination from non-prompt production of leptons, in particular, the channels targetting semi-leptonic \ttbar decays and decays with hadronic taus in the final state.  One production process that gives rise to these events for which there is insufficient MC events is multiplepton QCD.  These events tend to affect selections where there is only one electron or muon.  This background is estimated using two different methods: for the semileptonic \ttbar selections, an estimate based on the fake rate method is used; for the hadronic $\tau$ final states, a sideband selected by inverting the dilepton charge requirement is used for the estimate.




\subsection{Fake rate method}

A commonly used method for estimating backgrounds from misidentified prompt lepton production can be summarized as follows:

\begin{enumerate}
    \item construct a control region that is enhanced in the production of leptons from non-prompt sources,
    
    \item measure the ratio, the ``fake rate", of the number of leptons passing a loose selection criteria to the number passing a tighter selection, i.e., the number of muons passing the analysis isolation requirement to those that pass with no isolation requirement,
    
        \begin{equation}
            f = \frac{N_{\rm pass\ iso}}{N_{\rm no iso}}
        \end{equation}

    \item apply a weight based on the fake rate ($w = f/(1-f)$) to events in the signal region where the leptons are required to pass the loose requirement but fail the tight requirement.
\end{enumerate}

The control region that is used for the fake rate measurement is selected to be enhanced in \PZ plus jet production.  Specifically, it is required that:

\begin{itemize}
    \item there are at least two muons or electrons passing the full analysis requirements,
    \item the two leptons must have opposite signs,
    \item $|M_{\ell\ell} - M_{Z}| < 15~\GeV$,
    \item the dilepton pair that has mass closest to the \PZ boson is selected
    \item one additional lepton (muon or electron) passing all
    identification requirements except the isolation requirement
\end{itemize}

The additional lepton is assumed to originate from an hadronic jet that is produced in association with the \PZ boson, but can frequently arise due to a prompt lepton produced from a diboson process such as WZ or ZZ production.  This is accounted for by subtracting off the estimate of these processes from simulation from the data in the fake rate control region.  Figures~\ref{fig:lepton_fr} show the measured \pt distributions of the electron and muon candidates and the resulting fake rates and the values for each of the \pt bins are shown in table~\ref{tab:lepton_fr}.



The fake rate that is applied to the data in the isolation sideband of the signal region is the one derived from data.  The systematic uncertainty on this background is conservatively treated as being 30\% for both electron and muon fakes.  





The shape of QCD estimation is obtained from inverting the lepton's tight isolation. Then to normalize the QCD component, two approaches are considered. First, an antiiso-to-iso scale factor can be derived from orthogonal $n_j,n_b$ regions. Second, the ht-binned QCD MC can be used for normalization which are less sensitive to MC statistics. The first approach is purely data driven, while the second is a hybrid of data-driven shape and MC-driven normalization. Here gives a description of both the approach. In the end, it turns out that the first approach have some issue about giving a reasonable data/MC match at high electron \pt in $e$jets channel. Meanwhile, the QCD MC has event statistics and gives a more reliable estimation and is chosen. In shape analysis, the normalization from the QCD MC is treated as a free parameters taking account of the LO cross section uncertainties. In counting analysis, the normalization from the QCD MC is assigned with a 30\% uncertainty.

In the first approach, $1\leq n_j<4,n_b\geq1$ orthogonal region is used to measure the antiiso-to-iso scale factor, which is \pt and $\eta$ dependent and defined as
\begin{equation}
SF (\pt, \eta) =  \frac{N^{\rm{iso}}_{\rm{data}} (\pt, \eta) - \sum N^{\rm{iso}}_{\rm{MC}}(\pt, \eta) } 
{N^{\rm{antiiso}}_{\rm{data}} (\pt, \eta)- \sum N^{\rm{antiiso}}_{\rm{MC}}(\pt, \eta) }
\end{equation}

\noindent where ``antiiso'' refers to failing the Tight but passing the Loose working point. The trigger requirement are the same for ``iso'' and ``antiiso''. Figure~\ref{fig:appendix:123j1b} shows the distribution of iso and antiiso region in the $\mu$jet (left two columns) and $e$jet (right two columns) with $1\leq n_j<4,n_b\geq1$. The obtained $SF (\pt, \eta)$ is shown in Figure~\ref{fig:appendix:123j1b_sf}.

\begin{figure}
    \centering
    \includegraphics[width=0.99\textwidth]{chapters/Analysis/sectionBackground/figures/ljets_kinematics/123j1b.png}
    \caption{The iso and anti-iso region of $\mu$+jet (left two columns) and $e$+jet (right two columns) channel 
    with $1\leq n_j <4, n_b\geq1$, which is orthogonal to the $n_j\geq4,n_b\geq1$ signal region.}
    \label{fig:appendix:123j1b}
\end{figure}


\begin{figure}
    \centering
    \includegraphics[width=0.49\textwidth]{chapters/Analysis/sectionBackground/figures/ljets_kinematics/123j1b/SF_mu_1d.png}
    \includegraphics[width=0.49\textwidth]{chapters/Analysis/sectionBackground/figures/ljets_kinematics/123j1b/SF_e_1d.png}
    \includegraphics[width=0.49\textwidth]{chapters/Analysis/sectionBackground/figures/ljets_kinematics/123j1b/SF_mu_2d.png}
    \includegraphics[width=0.49\textwidth]{chapters/Analysis/sectionBackground/figures/ljets_kinematics/123j1b/SF_e_2d.png}
    \caption{iso-to-antiiso SF in the $\mu$+jet (left) and $e$+jet (right) channel 
    with $1\leq n_j <4, n_b\geq1$ side-band region.}
    \label{fig:appendix:123j1b_sf}
\end{figure}



\begin{figure}
    \centering
    \includegraphics[width=0.99\textwidth]{chapters/Analysis/sectionBackground/figures/ljets_kinematics/sf_mu4j.png}
    \caption{iso-to-antiiso SF in the $\mu$+jet all regions.}
    \label{fig:appendix:allsf}
\end{figure}

\begin{figure}
    \centering
    \includegraphics[width=0.99\textwidth]{chapters/Analysis/sectionBackground/figures/ljets_kinematics/sf_e4j.png}
    \caption{iso-to-antiiso SF in the $e$+jet all regions.}
    \label{fig:appendix:allsf}
\end{figure}




In the second approach, the HT-binned QCD MC is used. Figure~\ref{fig:appendix:4j1b} shows iso and antiiso region of $\mu$jet (left two columns) and $e$jet (right two columns) channel
with $n_j\geq4,n_b\geq1$ and the QCD MC in the signal region is shown as red line.

\begin{figure}
    \centering
    \includegraphics[width=0.99\textwidth]{chapters/Analysis/sectionBackground/figures/ljets_kinematics/4j1b.png}
    \caption{The iso and anti-iso region of $\mu$+jet (left two columns) and $e$+jet (right two columns) channel 
    with $n_j\geq4,n_b\geq1$, the signal region.}
    \label{fig:appendix:4j1b}
\end{figure}




\begin{figure}
    \centering
    \includegraphics[width=0.99\textwidth]{chapters/Analysis/sectionBackground/figures/ljets_application/ddNorm_ddShape_mu4j.png}
    \includegraphics[width=0.99\textwidth]{chapters/Analysis/sectionBackground/figures/ljets_application/ddNorm_ddShape_e4j.png}
    \caption{Fully data-driven QCD estimation.}
    \label{fig:app:QCD:application_SFNorm_ddShape}
\end{figure}

\begin{figure}
    \centering
    \includegraphics[width=0.99\textwidth]{chapters/Analysis/sectionBackground/figures/ljets_application/mcNorm_mcShape.png}
    \caption{Fully MC-based QCD estimation}
    \label{fig:app:QCD:application_mc}
\end{figure}

\begin{figure}
    \centering
    \includegraphics[width=0.99\textwidth]{chapters/Analysis/sectionBackground/figures/ljets_application/mcNorm_ddShape.png}
    \caption{Data-driven shape normalized by MC-based normalization.}
    \label{fig:app:QCD:application_mcNorm_ddShape}
\end{figure}
\FloatBarrier


\FloatBarrier



\subsection{Same-sign estimate}

This estimation relies on the dearth of standard model processes that can give rise to same-sign lepton pairs.  Because of this it is expected that most events with same-sign lepton pairs are the result of at least on of the leptons not being the result of a prompt bosonic decay, but are produced by a hadronic jet.  It is further assumed that this process will give rise to misidentifying hadronic jets as leptons in near equal measure between the same sign and opposite sign selections.  This is verified by deriving a scale factor in an isolation inverted region.

The process of deriving the estimate is simple enough: all the same selection requirements that are applied in the nominal analysis selection are applied to the side-band region with the exception of the opposite sign requirement, which is inverted.  The same is done for all of the relevant MC samples in order to determine what component of the same-sign data sample will already be estimated by the MC.  Finally, a correction factor is applied to account for any difference in the probability of the QCD giving rise to opposite sign and same sign final states.

To verify the method, a control region enriched in $\PZ\to\tau\tau$ is examined for the $\mu\tau$ and $e\tau$ final states.  A comparison of data and simulation in both the same sign and opposite sign regions are shown in figure~\ref{fig:ltau_fakes}.

\begin{figure}
    \centering
    \includegraphics[width=0.7\textwidth]{chapters/Analysis/sectionBackground/figures/ltau_kinematics/etau_cr.pdf}
    \includegraphics[width=0.7\textwidth]{chapters/Analysis/sectionBackground/figures/ltau_kinematics/mutau_cr.pdf}
    \caption{Opposite sign (left) and same sign (right) control regions for the \PZ enriched $e\tau$ (top) and $\mu\tau$ (bottom) selection.}
    \label{fig:ltau_fakes}
\end{figure}


In counting analysis, the QCD background in the $e\tau$ and $\mu\tau$ channel with $n_j\geq 2,n_b=1$ and
$n_j\geq 2,n_b\geq2$ is estimated with the QCD in the same-sign region scaled by a ss-to-os scale factor.
The scale factor can be defined as
\begin{equation}
    SF = \frac{N^{\rm{os}}_{\rm{data}} - \sum N^{\rm{os}}_{\rm{MC}} } {N^{\rm{ss}}_{\rm{data}} - \sum N^{\rm{ss}}_{\rm{MC}} }/
\end{equation}
\noindent The scale factor is derived from orthogonal $n_j,n_b$ regions.
Figure~\ref{fig:appendix:qcdsf:ltau, fig:appendix:qcdsf:ltau2} show the ss and os region of the $\mu\tau$ (left two columns) and 
$e\tau$ (right two columns) channel with different orthogonal $n_j,n_b$. Being the closest to the signal
$n_j,n_b$ category, the ss-to-os scale factor derived with $n_j\geq2,n_b=0$ is used to estimate the QCD 
background.


\begin{sidewaysfigure}[htb!]
    \centering
    \includegraphics[width=0.9\textwidth]{chapters/Analysis/sectionBackground/figures/ltau_kinematics/ltau1.png}
    \caption{The $m_{l\tau}$ in the SS and OS region of $\mu\tau$ (left two columns) and $e\tau$ (right two columns) 
    channel. Different rows correspond to different $n_j,n_b$ configuration, which includes
    $n_j=0,n_b=0$, $n_j=1,n_b=0$, $n_j=1,n_b=1$. 
    }
    \label{fig:appendix:qcdsf:ltau}
\end{sidewaysfigure}



\begin{sidewaysfigure}[htb!]
    \centering
    \includegraphics[width=0.9\textwidth]{chapters/Analysis/sectionBackground/figures/ltau_kinematics/ltau2.png}

    \caption{The $m_{l\tau}$ in the SS and OS region of $\mu\tau$ (left two columns) and $e\tau$ (right two columns) 
    channel. Different rows correspond to different $n_j,n_b$ configuration, which includes
    $n_j\geq 2,n_b=0$, $n_j\geq 2,n_b=1$, $n_j\geq 2,n_b\geq 2$. The last two rows dominated by \ttbar are the channels used by the counting analysis.
    }
    \label{fig:appendix:qcdsf:ltau2}
\end{sidewaysfigure}



\FloatBarrier



\section{Statistical Analysis}
\label{sec:analysis:method}



Having carried out the event selection as described in section~\ref{sec:analysis:selection}, the estimation of the W branching fraction is carried out using two different approaches.  Before describing the two approaches, it will be useful to describe the formalism that is common to both.


%%%%%%%%%%%%%%%%%%%%%%%%%%%%%%%%%%%%%%%%
% 1. Determination of Signal Acceptance
%%%%%%%%%%%%%%%%%%%%%%%%%%%%%%%%%%%%%%%%
\subsection{Determination of Signal Efficiency}

For single W decay, also taking into account the $\tau$ decay, we write the branching 
ratio in the form of a vector

\begin{equation}
    \vec{B} = 
    \begin{bmatrix}
        B_e & B_\mu & B_{\tau_e} & B_{\tau_\mu} &  B_{\tau_h} & B_h
	\end{bmatrix} 
\end{equation}

For WW decay, there are 21 possible final states. We can denote the branching
ratio as a 6-by-6 symmetric matrix with 21 independent terms. Mathematically, 
this branching ratio matrix \textit{B} equals to the outer product of vector
$\vec{B}$:

\begin{equation}
	B = \vec{B}^T \vec{B} =
    \begin{bmatrix}
        B_e B_e             & B_e B_\mu         & B_e B_{\tau_e}            & B_e B_{\tau_\mu}          & B_e B_{\tau_h}            & B_e B_h           \\
        B_\mu B_e           & B_\mu B_\mu       & B_\mu B_{\tau_e}          & B_\mu B_{\tau_\mu}        & B_\mu B_{\tau_h}          & B_\mu B_h         \\
        B_{\tau_e} B_e      & B_{\tau_e} B_\mu  & B_{\tau_e} B_{\tau_e}     & B_{\tau_e} B_{\tau_\mu}   & B_{\tau_e} B_{\tau_h}     & B_{\tau_e} B_h    \\
        B_{\tau_\mu} B_e    & B_{\tau_\mu}B_\mu & B_{\tau_\mu} B_{\tau_e}   & B_{\tau_\mu} B_{\tau_\mu} & B_{\tau_\mu} B_{\tau_h}   & B_{\tau_\mu} B_h  \\
        B_{\tau_h} B_e      & B_{\tau_h} B_\mu  & B_{\tau_h} B_{\tau_e}     & B_{\tau_h}  B_{\tau_\mu}  & B_{\tau_h} B_{\tau_h}     & B_{\tau_h} B_h    \\
        B_h B_e             & B_h B_\mu         & B_h B_{\tau_e}            & B_h B_{\tau_\mu}          & B_h  B_{\tau_h}           & B_h  B_h 
	\end{bmatrix}
	\label{eq:br_matrix}
\end{equation}


Similarly, signal efficiency has 21 independent terms, corresponding to 21
WW decay channels. In each channel, we defining $E$ matrix to manage these terms.

\begin{equation}
	E = 
	\begin{bmatrix}
    E_{ee}          & E_{e\mu}          & E_{e\tau_e}       & E_{e\tau_\mu}        & E_{e\tau_h}        & E_{eh}        \\
    E_{\mu e}       & E_{\mu \mu}       & E_{\mu \tau_e}    & E_{\mu \tau_\mu}     & E_{\mu \tau_h}     & E_{\mu h}     \\
    E_{\tau_e e}    & E_{\tau_e \mu}    & E_{\tau_e \tau_e} & E_{\tau_e \tau_\mu}  & E_{\tau_e \tau_h}  & E_{\tau_e h}  \\
    E_{\tau_\mu e}  & E_{\tau_\mu\mu}   & E_{\tau_\mu\tau_e}& E_{\tau_\mu\tau_\mu} & E_{\tau_\mu\tau_h} & E_{\tau_\mu h}\\
    E_{\tau_h e}    & E_{\tau_h \mu}    & E_{\tau_h \tau_e} & E_{\tau_h \tau_\mu}  & E_{\tau_h \tau_h } & E_{\tau_h h}  \\
    E_{he}          & E_{h\mu}          & E_{h\tau_e}       & E_{h\tau_\mu}        & E_{h\tau_h}        & E_{hh} 
	\end{bmatrix}
    \label{eq:eff_matrix}
\end{equation}


In counting analysis, the signal efficiency is therefore slightly differ 
from shape analysis because of slightly higher pT threshold in $\mu\tau$, $\mu4j$ channel
and tighter tau working point. In 8 channels in 1b and 2b cases, the signal efficiency 
is determined from $t\bar{t}$ and {tW} MC events, with result being shown in Figure 
\ref{efficencyMatrix} and Table \ref{efficencyTableMuon}, \ref{efficencyTableElectron}.
In addition, the percentage portion of each of 21 WW decay final states is 
shown in Table \ref{selectionSignals}.

With the E and B matrix, the prediction of yield in a channel can be expressed as

\begin{equation}
    N = \sigma_{sg}L\times \textbf{E} \cdot \textbf{B} + \sum N_{bg}
    \label{prediction}
\end{equation}


\begin{figure}[ht]
    \centering
    channels with $\mu$-trigger-1b \\
    \includegraphics[width=15cm]{chapters/Analysis/sectionStatisticalAnalysis/figures/acc_mu1b.png}
    
    channels with $\mu$-trigger-2b \\
    \includegraphics[width=15cm]{chapters/Analysis/sectionStatisticalAnalysis/figures/acc_mu2b.png}
    
    channels with $e$-trigger-1b \\
    \includegraphics[width=15cm]{chapters/Analysis/sectionStatisticalAnalysis/figures/acc_e1b.png}
    
    channels with $e$-trigger-2b \\
    \includegraphics[width=15cm]{chapters/Analysis/sectionStatisticalAnalysis/figures/acc_e2b.png}
    
    %--------------------------
    \caption{ Efficiency matrices \textbf{E} in trigger and b-tag categories. }
    \label{efficencyMatrix}
\end{figure}

\begin{sidewaystable}[p]
    \centering
    \setlength{\tabcolsep}{0.4em}
    \renewcommand{\arraystretch}{1.5}
    \caption{Efficiency of $t\bar{t}$+$tW$ events, breakdown by 21 WW decay.  Values are in percent.}
    
    \resizebox{\textwidth}{!}{
    \begin{tabular}{|l|cc|cc|cc|cc|cc|cc|cc|cc|}
    
    
    \hline
    channel & \multicolumn{2}{c|}{$\mu e$} & \multicolumn{2}{c|}{$\mu\mu$} & \multicolumn{2}{c|}{$\mu \tau$} & \multicolumn{2}{c|}{$\mu$+jets} & \multicolumn{2}{c|}{$ee$} & \multicolumn{2}{c|}{$e\mu$} & \multicolumn{2}{c|}{$e \tau$} & \multicolumn{2}{c|}{$e+jets$} \\    \hline
    $\rm n_{b tag}$ & $n_b=1$ & $n_b\geq2$ & $n_b=1$ & $n_b\geq2$ & $n_b=1$ & $n_b\geq2$ & $n_b=1$ & $n_b\geq2$ & $n_b=1$ & $n_b\geq2$ & $n_b=1$ & $n_b\geq2$ & $n_b=1$ & $n_b\geq2$ & $n_b=1$ & $n_b\geq2$ \\ 
    \hline
    
    $tt/tW \to ee$                     &    --    &    --    &    --    &    --    &    --    &    --    &    --    &    --    &  5.71(1) &  3.19(1) &    --    &    --    &    --    &    --    &  3.17(1) &  2.14(1) \\ 
    $tt/tW \to \mu\mu$                 &    --    &    --    & 14.14(2) &  8.02(1) &    --    &    --    &  1.80(1) &  1.21(0) &    --    &    --    &    --    &    --    &    --    &    --    &    --    &    --    \\ 
    $tt/tW \to e\mu$                   &  4.69(1) &  2.66(0) &    --    &    --    &    --    &    --    &  2.35(0) &  1.60(0) &    --    &    --    &  5.76(1) &  3.24(1) &    --    &    --    &  0.71(0) &  0.48(0) \\ 
    $tt/tW \to \tau_{e}\tau_{e}$       &    --    &    --    &    --    &    --    &    --    &    --    &    --    &    --    &  0.74(2) &  0.44(2) &    --    &    --    &    --    &    --    &  1.18(4) &  0.81(2) \\ 
    $tt/tW \to \tau_{\mu}\tau_{\mu}$   &    --    &    --    &  3.88(5) &  2.11(4) &    --    &    --    &  1.13(3) &  0.77(2) &    --    &    --    &    --    &    --    &    --    &    --    &    --    &    --    \\ 
    $tt/tW \to \tau_{e}\tau_{\mu}$     &  0.78(2) &  0.45(1) &    --    &    --    &    --    &    --    &  0.92(2) &  0.62(1) &    --    &    --    &  1.24(2) &  0.70(2) &    --    &    --    &  0.40(1) &  0.27(1) \\ 
    $tt/tW \to \tau_{e}\tau_{h}$       &    --    &    --    &    --    &    --    &    --    &    --    &    --    &    --    &    --    &    --    &    --    &    --    &  0.48(1) &  0.26(0) &  0.84(1) &  0.61(1) \\ 
    $tt/tW \to \tau_{\mu}\tau_{h}$     &    --    &    --    &    --    &    --    &  0.74(1) &  0.40(1) &  1.28(1) &  0.92(1) &    --    &    --    &    --    &    --    &    --    &    --    &    --    &    --    \\ 
    $tt/tW \to \tau_{h}\tau_{h}$       &    --    &    --    &    --    &    --    &    --    &    --    &    --    &    --    &    --    &    --    &    --    &    --    &    --    &    --    &    --    &    --    \\ 
    $tt/tW \to e\tau_{e}$              &    --    &    --    &    --    &    --    &    --    &    --    &    --    &    --    &  2.14(1) &  1.18(1) &    --    &    --    &    --    &    --    &  2.29(1) &  1.62(1) \\ 
    $tt/tW \to e\tau_{\mu}$            &  1.28(1) &  0.69(1) &    --    &    --    &    --    &    --    &  0.80(1) &  0.54(0) &    --    &    --    &  4.49(2) &  2.47(1) &    --    &    --    &  1.18(1) &  0.86(1) \\ 
    $tt/tW \to e\tau_{h}$              &    --    &    --    &    --    &    --    &    --    &    --    &    --    &    --    &    --    &    --    &    --    &    --    &  1.59(1) &  0.88(0) &  2.61(1) &  1.90(0) \\ 
    $tt/tW \to \mu\tau_{e}$            &  2.54(1) &  1.43(1) &    --    &    --    &    --    &    --    &  2.61(1) &  1.86(1) &    --    &    --    &  1.42(1) &  0.78(1) &    --    &    --    &  0.23(0) &  0.16(0) \\ 
    $tt/tW \to \mu\tau_{\mu}$          &    --    &    --    &  7.76(2) &  4.37(2) &    --    &    --    &  1.88(1) &  1.35(1) &    --    &    --    &    --    &    --    &    --    &    --    &    --    &    --    \\ 
    $tt/tW \to \mu\tau_{h}$            &    --    &    --    &    --    &    --    &  2.27(1) &  1.27(0) &  3.70(1) &  2.70(1) &    --    &    --    &    --    &    --    &    --    &    --    &    --    &    --    \\ 
    $tt/tW \to eh$                     &    --    &    --    &    --    &    --    &    --    &    --    &    --    &    --    &    --    &    --    &    --    &    --    &  0.10(0) &    --    &  6.83(0) &  4.64(0) \\ 
    $tt/tW \to \mu h$                  &    --    &    --    &    --    &    --    &  0.15(0) &    --    &  9.71(1) &  6.61(0) &    --    &    --    &    --    &    --    &    --    &    --    &    --    &    --    \\ 
    $tt/tW \to \tau_{e}h$              &    --    &    --    &    --    &    --    &    --    &    --    &    --    &    --    &    --    &    --    &    --    &    --    &    --    &    --    &  2.17(1) &  1.44(0) \\ 
    $tt/tW \to \tau_{\mu}h$            &    --    &    --    &    --    &    --    &    --    &    --    &  3.30(1) &  2.19(1) &    --    &    --    &    --    &    --    &    --    &    --    &    --    &    --    \\ 
    $tt/tW \to \tau_{h}h$              &    --    &    --    &    --    &    --    &    --    &    --    &    --    &    --    &    --    &    --    &    --    &    --    &    --    &    --    &    --    &    --    \\ 
    $tt/tW \to hh$                     &    --    &    --    &    --    &    --    &    --    &    --    &    --    &    --    &    --    &    --    &    --    &    --    &    --    &    --    &    --    &    --    \\ 

    \hline
    \end{tabular}}
    
    \label{tab:sigacc}
    
\end{sidewaystable}

\FloatBarrier



%%%%%%%%%%%%%%%%%%%%%%%%%%%%%%%%%%%%%%%%
% 2. Extraction of Parameters
%%%%%%%%%%%%%%%%%%%%%%%%%%%%%%%%%%%%%%%%
\subsection{Extraction of Parameters}

In counting analysis, branching fractions are extracted by solving a set of
three quadratic equations, obtained by setting the expected normalized
yields equal to the measured ones. The four measurements are performed independently 
in four mutually exclusive regions based on the number of b tags (1 or more than 2) 
and trigger type (single electron or muon). Then these four measurements are combined based 
on a $\chi^{2}$ minimization to obtain the final result.

The four groups of channels and their yields
are shown in figure~\ref{fig:signalRegion}.
Channels using single-$\mu$ or single-$e$ trigger are

\begin{itemize}
    \item single-$\mu$ trigger : $\mu e$, $\mu\mu$, $\mu\tau$, $\mu h$.
    \item single-$e$ trigger : $ee$, $e\mu$, $e\tau$, $eh$.
\end{itemize}

where $e\mu$ and $\mu e$ are mutually exclusive -- $e\mu$ channel
requires fired e-trigger and \(p^T_e > p^T_\mu\), while $\mu e$ channel
requires fired $\mu$-trigger and \(p^T_e < p^T_\mu\). 

Besides being formally different from the shape analysis, the thresholds
of leptons $p^T$ and working point for the hadronic tau isolation are
slightly different, as optimizations in counting. This results in slightly different signal
acceptances. For the 8 channels under consideration, the signal efficiency determined from
simulated $t\bar{t}$ and tW events are shown in
tables \ref{efficencyTableMuon} and \ref{efficencyTableElectron}. 
The efficiency matrices \textbf{E} of 
included channels in the four categories are shown in Fig \ref{efficencyMatrix}.



The normalized yields, which are inspired by the definition of branching
fraction, is the ratio of one yield over the sum of all yields in the
trigger category:


\begin{itemize}
    \item single-\(\mu\) trigger : 
    \(X_{e} = \frac{n^{\mu e}}{n^{\mu e} + n^{\mu \mu} + n^{\mu \tau} + n^{\mu h}}\), 
    \(X_{\mu} = \frac{n^{\mu \mu}}{n^{\mu e} + n^{\mu \mu} + n^{\mu \tau} + n^{\mu h}}\), 
    \(X_{\tau} = \frac{n^{\mu \tau}}{n^{\mu e} + n^{\mu \mu} + n^{\mu \tau} + n^{\mu h}}\),

    \item single-\(e\) trigger : 
    \(X_{e} = \frac{n^{e e}}{n^{e e} + n^{e \mu} + n^{e \tau} + n^{e h}}\), 
    \(X_{\mu} = \frac{n^{e \mu}}{n^{e e} + n^{e \mu} + n^{e \tau} + n^{e h}}\), 
    \(X_{\tau} = \frac{n^{e \tau}}{n^{e e} + n^{e \mu} + n^{e \tau} + n^{e h}}\),
\end{itemize}

where \(n^f \equiv N^f - \sum_{k\in bg} N^f_k \) is the yield of channel
\(f\) with background subtracted. Based on Eqn \ref{eq:data_model}, the measured normalized yields
\(\{X_{e},X_{\mu},X_{\tau}\}\) should equal to the calculation with
efficiency \textbf{E} and branching fraction \textbf{B}:

\begin{equation} \label{quadEqA}
    \begin{split}
    X_e &= \frac{ E_{ij}^{te}B^{ij} }{E_{ij}^{te}B^{ij} + E_{ij}^{t\mu}B^{ij} + E_{ij}^{t\tau}B^{ij} + E_{ij}^{th}B^{ij}} \\
    X_\mu &= \frac{ E_{ij}^{t\mu}B^{ij} }{E_{ij}^{te}B^{ij} + E_{ij}^{t\mu}B^{ij} + E_{ij}^{t\tau}B^{ij} + E_{ij}^{th}B^{ij}} \\
    X_\tau &= \frac{ E_{ij}^{t\tau}B^{ij} }{E_{ij}^{te}B^{ij} + E_{ij}^{t\mu}B^{ij} + E_{ij}^{t\tau}B^{ij} + E_{ij}^{th}B^{ij}}
    \end{split}
\end{equation}



% where $n^f \equiv N^f - \sum_{k\in bg} n^f_k $ is the yield of channel $f$ 
% with background subtracted and three normalized yields, 
% $\{r_{e},r_{\mu},r_{\tau}\}$, are measured from data with background subtracted. 

% Based on Eqn \ref{prediction}, the measured normalized yields $\{r_{e},r_{\mu},r_{\tau}\}$ 
% should equal to the calculation with efficiency \textbf{E} and branching fraction \textbf{B}:




where \(t\in \{\mu,e\}\) depends on the trigger category. Plugging in
explicit form of \textbf{E} and \textbf{B} matrices in Eqn \ref{eq:br_matrix} and Eqn \ref{eq:eff_matrix}
and unity condition of branching fraction \(\beta_h = 1- \beta_e -
\beta_\mu - \beta_\tau\), Eq \ref{quadEqA} can be written as a set of
three quadratic equations with
\(\{\beta_{e},\beta_{\mu},\beta_{\tau}\}\) as three unknowns.


\begin{equation} \label{quadEqB}
    \footnotesize
	\begin{split}
        Q_e(\beta_e,\beta_\mu,\beta_\tau) &=
        c_{e1} \beta_e^2 + c_{e2} \beta_\mu^2 + c_{e3} \beta_\tau^2 + 
        c_{e4} \beta_e\beta_\mu + c_{e5} \beta_e\beta_\tau + c_{e6} \beta_\mu\beta_\tau +
        c_{e7} \beta_e + c_{e8} \beta_\mu + c_{e9} \beta_\tau + c_{e0} = 0 \\
        %
        Q_\mu(\beta_e,\beta_\mu,\beta_\tau) &= 
        c_{\mu 1} \beta_e^2 + c_{\mu 2} \beta_\mu^2 + c_{\mu 3} \beta_\tau^2 + 
        c_{\mu 4} \beta_e\beta_\mu + c_{\mu 5} \beta_e\beta_\tau + c_{\mu 6} \beta_\mu\beta_\tau +
        c_{\mu 7} \beta_e + c_{\mu 8} \beta_\mu + c_{\mu 9} \beta_\tau + c_{\mu 0} = 0 \\
        %
        Q_\tau(\beta_e,\beta_\mu,\beta_\tau) &= 
        c_{_\tau1} \beta_e^2 + c_{\tau2} \beta_\mu^2 + c_{\tau3} \beta_\tau^2 + 
        c_{\tau4} \beta_e\beta_\mu + c_{\tau5} \beta_e\beta_\tau + c_{\tau6} \beta_\mu\beta_\tau +
        c_{\tau7} \beta_e + c_{\tau8} \beta_\mu + c_{\tau9} \beta_\tau + c_{\tau0} = 0 
    \end{split}
\end{equation}

where coefficients $c_{ei},c_{\mu i},c_{\tau i}$ are fully determined 
by efficiency \textbf{E} and normalized yields $\{X_{e},X_{\mu},X_{\tau}\}$,
as are listed in table~\ref{quadcoeff}.

\begin{table}[ht]
    \centering
   	\setlength{\tabcolsep}{0.5 em}
    \renewcommand{\arraystretch}{1.6}
    \caption{ Coefficients of quadratic equations $c_{lk}$ in terms of E and X, where the index
    $l\in \{ e,\mu,\tau \} $ corresponds to the three equations $F_e=0,F_\mu=0,F_\tau=0$ and the 
    index $k\in\{ 0,1,2,\dots 9\}$. In the table, 
    $ \Delta_{ij} \equiv E^{tl}_{ij} - X_l \times ( E^{te}_{ij} + E^{t\mu}_{ij} + E^{t\tau}_{ij} + E^{th}_{ij} )$ is 
    a $6\times 6$ matrix, where the lower index $i,j$ are for the $6\times 6$ elements and the upper index denotes the channel. }
    
    \begin{tabular}{c|l}

    \hline
    $c_{l0}$ & $\Delta_{hh}$ \\
    \hline
    $c_{l1}$ & $\Delta_{ee}     - 2\Delta_{eh}   + \Delta_{hh}$ \\
    \hline
    $c_{l2}$ & $\Delta_{\mu\mu} - 2\Delta_{\mu h} + \Delta_{hh}$ \\
    \hline
    
    $c_{l3}$ & $   b^\tau_e   b^\tau_e   \Delta_{\tau_e   \tau_e}  
    			 + b^\tau_\mu b^\tau_\mu \Delta_{\tau_\mu \tau_\mu}
                 + b^\tau_h   b^\tau_h   \Delta_{\tau_h   \tau_h}
                 
                 + 2 b^\tau_e   b^\tau_\mu \Delta_{\tau_e   \tau_\mu} 
    		     + 2 b^\tau_e   b^\tau_h   \Delta_{\tau_e   \tau_h}   
    		     + 2 b^\tau_\mu b^\tau_h   \Delta_{\tau_\mu \tau_h} - $ \\
                 
             & $   2 b^\tau_e   \Delta_{e   \tau_h}
                 - 2 b^\tau_\mu \Delta_{\mu \tau_h}
                 - 2 b^\tau_h   \Delta_{h.  \tau_h} 
                 + \Delta_{hh} $ \\

    \hline
    $c_{l4}$ & $2\Delta_{e\mu} - 2\Delta_{eh} -2\Delta_{\mu h} +2\Delta_{hh}$  \\
    \hline
    $c_{l5}$ & $  2b^\tau_e   \Delta_{e \tau_e} 
    			+ 2b^\tau_\mu \Delta_{e \tau_\mu}
                + 2b^\tau_h   \Delta_{e \tau_h}
                - 2b^\tau_e   \Delta_{\tau_e   h} 
    			- 2b^\tau_\mu \Delta_{\tau_\mu h}
                - 2b^\tau_h   \Delta_{\tau_h   h} 
                - 2\Delta_{eh}   + 2 \Delta_{hh} $ \\
        
    \hline            
    $c_{l6}$ & $  2b^\tau_e   \Delta_{\mu \tau_e} 
    			+ 2b^\tau_\mu \Delta_{\mu \tau_\mu}
                + 2b^\tau_h   \Delta_{\mu \tau_h}
                - 2b^\tau_e   \Delta_{\tau_e   h} 
    			- 2b^\tau_\mu \Delta_{\tau_\mu h}
                - 2b^\tau_h   \Delta_{\tau_h   h} 
                - 2\Delta_{\mu h}   + 2 \Delta_{hh} $ \\
    \hline            
    $c_{l7}$ & $ 2\Delta_{eh}      - 2 \Delta_{hh} $ \\
    \hline
    $c_{l8}$ & $ 2\Delta_{\mu h}   - 2 \Delta_{hh}$ \\
    \hline
    $c_{l9}$ & $  2b^\tau_e   \Delta_{\tau_e   h} 
                 + 2b^\tau_\mu \Delta_{\tau_\mu h} 
                 + 2b^\tau_h   \Delta_{\tau_h   h} 
                 - 2 \Delta_{hh}$ \\
    \hline

    
	\end{tabular}
    

\label{tab:quadcoeff}
    
\end{table}


In the $\{\beta_{e},\beta_{\mu},\beta_{\tau}\}$ parameter space, 
equation~\ref{quadEqB} represents three hyperbolic planes, intersection 
of which is the solution of desired branching fractions, as is shown
in figure~\ref{visualize}.


\begin{figure}[ht]
    \centering
    \includegraphics[width=7cm]{chapters/Analysis/sectionStatisticalAnalysis/figures/visual.png}
    
    %--------------------------
    \caption{Visualization of Eq \ref{quadEqB} in the
    \(\{\beta_{e},\beta_{\mu},\beta_{\tau}\}\) parameter space. Each
    equation in Eq \ref{quadEqB} is a hyperbolic plane, while their
    intersection is the solution Eq \ref{quadEqB}. Mathematically, there
    are 8 possible solutions. However, only one solution is physical,
    located with \(\beta \in (0,1) \). }
    \label{visualize}
\end{figure}

% This approach analytically obtains the coefficient $c_{ij}$ in 
% Eq \ref{quadEqB} from efficiency matrix \textbf{E} and measured 
% normalized yields $\{X_{e},X_{\mu},X_{\tau}\}$. Then it numerically
% solves the branching fractions $\{\beta_{e},\beta_{\mu},\beta_{\tau}\}$,
% using a modification of the Powell hybrid method as implemented in MINPACK-Scipy. 

\begin{equation} 
    \left [
    \begin{tabular}{c}
	    $\beta_{e}$ \\
	    $\beta_{\mu}$ \\
	    $\beta_{\tau}$
    \end{tabular}
    \right ]
    = Solution
    \left [
    \begin{tabular}{c}
	    $Q_e    (\beta_e,\beta_\mu,\beta_\tau) = 0$ \\
	    $Q_\mu  (\beta_e,\beta_\mu,\beta_\tau) = 0$ \\
	    $Q_\tau (\beta_e,\beta_\mu,\beta_\tau) = 0$
    \end{tabular}
\right ]
\end{equation}

\FloatBarrier

%%%%%%%%%%%%%%%%%%%%%%%%%%%%%%%%%%%%%%%%
% 3. Test of Parameters Extraction
%%%%%%%%%%%%%%%%%%%%%%%%%%%%%%%%%%%%%%%%
\subsection{Test of Parameters Extraction}


A test of parameter extraction is performed using
signal MC samples, in which $\beta=10.80\%$ is assumed. The test
takes $N_{mc,sg}$ as input instead of data yields with background 
subtracted $n=N_{data} - N_{mc,bg}$, 
so as to check whether the extracted branching fractions agree with 
the assumption in signal simulation.

we generate 2000 toys, each of which variates the yield $n=N_{sg}$ by $\delta_n=\sqrt{N_{sg}}$.
The distribution of $\{\beta_{e},\beta_{\mu},\beta_{\tau}\}$ extracted from toys are shown in
figure~\ref{test_toy}. The centers of distributions are consistent with
the assumed branching fraction in the MC generator, while widths of distributions are 
consistent with uncertainty calculated by error propagation.


\begin{figure}[ht]
    \centering
    \includegraphics[width=7cm]{chapters/Analysis/sectionStatisticalAnalysis/figures/test_mu1b.png}
    \includegraphics[width=7cm]{chapters/Analysis/sectionStatisticalAnalysis/figures/test_mu2b.png}
    \includegraphics[width=7cm]{chapters/Analysis/sectionStatisticalAnalysis/figures/test_e1b.png}
    \includegraphics[width=7cm]{chapters/Analysis/sectionStatisticalAnalysis/figures/test_e2b.png}
    
    %--------------------------
    \caption{ Distribution of 2000 toys. }
    \label{test_toy}
\end{figure}

\begin{quote}
    
    After establishing parameter extraction, we perform a closure test using 
    signal MC samples. As is described above, the input of parameter extraction 
    is data yields with background subtracted $n=N_{data} - N_{mc,bg}$. 
    But here for testing purpose, we replace $n=N_{data} - N_{mc,bg}$ with $N_{mc,sg}$ as the input,
    as is given in Eqn \ref{eqn:testinput}.
    The pass of the test is that parameter extraction 
    gives back branching fraction assumed in the MC generator, which is $10.80\%$.
    
    
    \begin{equation}
    	n=N_{mc,sg}\pm \sqrt{N_{mc,sg}}
    	\label{eqn:testinput}
    \end{equation}
    
    where $N_{mc,sg}$ comes from tt and tW MC sample normalized to luminosity. 
    It is uncertainty is assumed as a Gaussian error with width $\sqrt{N_{sg}}$, 
    so as to estimate the expected statistical uncertainty of data.
    The extracted branching fraction is listed in Table \ref{test_ana}.

\end{quote}



\begin{table}[ht]
    \centering
	\begin{tabular}{l|ccc}
    \hline
          	 & $B_e$             &   $B_\mu$       	 & 	  $B_\tau$   	 \\
    MC Assumption  & 10.80 		 &  10.80 		 	 & 	  10.80     	 \\
    \hline
  	$\mu-1b$ &   10.802$\pm$0.058  &   10.807$\pm$0.053  &    10.808$\pm$0.127 \\
  	$\mu-2b$ &   10.802$\pm$0.103  &   10.807$\pm$0.092  &    10.808$\pm$0.216 \\
  	$e-1b$   &   10.804$\pm$0.073  &   10.797$\pm$0.059  &    10.794$\pm$0.152 \\
  	$e-2b$   &   10.805$\pm$0.130  &   10.797$\pm$0.104  &    10.794$\pm$0.263 \\
    \hline
	\end{tabular}
	
	%--------------------------
    \caption{Branching fraction extracted from signal MC. The 
    uncertainty is calculated by error propagation. The small 
    deviation from MC assumption is resulted by the difference 
    value of $Br(\tau \to e)$ and $Br(\tau \to \mu)$ in MC and in 
    extractor. The extractor uses 0.1773,0.1731 respectively, while 
    the MC assumption of tau decay needs to be found in generator 
    cards. 
    }
    \label{test_ana}
\end{table}

\FloatBarrier

% In addition, to test the error propagation, we generate 2000 toy experiments, 
% each of which variates the yield $n=N_{sg}$ by $\sqrt{N_{sg}}$.
% The width of distribution of toys is consistent with uncertainty 
% from error propagation list in the Table \ref{test_ana}.
% Also as expected, the center of distribution of toys is consistent with
% the assumed branching fraction in the MC generator.






% Finally, the branching fractions obtained in all four categories are combined:

% \begin{equation}
% 	\beta_i = \frac{ \sum_{cat} \beta_i^{cat} / \sigma^2_{\beta_i^{cat}}}{\sum_{cat} 1 / \sigma^2_{\beta_i^{cat}}} ,
%     \qquad
%     \sigma^2_{\beta_i} = \frac{1}{\sum_{cat} 1 / \sigma^2_{\beta_i^{cat}} }
% \end{equation}

% where $i = e,\mu,\tau$ and categories are single electron or muon trigger with 1 or 2 b-jets, $cat \in \{\mu \text{-} 1b,\mu \text{-} 2b,e \text{-} 1b,e \text{-} 2b \}$


\section{Systematic Uncertainties}
\label{sec:analysis:systematics}


The various sources of uncertainty that have been considered are described in the following sections. As described in Section~\ref{sec:analysis:method},  the treatment of systematics differs between the two analysis approaches: the shape analysis makes use of nuisance parameters; the counting analysis is carried out varying each systematic individually and assessing the variation on the estimates of the branching fractions.


\subsection{Sources of Systematic Uncertainties}
\label{sec:analysis:systematics:source}


\subsubsection{Luminosity} 
The uncertainty on the CMS luminosity measurement is estimated to be 2.5\% for the 2016 run~\cite{CMS-PAS-LUM-17-001}.  This uncertainty effects the overall scale of the all predicted yields in a fully correlated manner. 
% In shape analysis, it is handled as a normalization nuisance parameter. In counting analysis, its effect on signal simulations is cancelled and effect on background simulation gets propagated.


\subsubsection{Data-driven background estimates}

The uncertainty is split into four normalization parameters: one for each of the final states \cet, \cmt, \ceh, and \cmh.  

\subsubsection{Cross section for simulated processes}

\begin{itemize}
    \item \ttbar: 5\%, (and PDF, $\alpha_{s}$, $\mu_{R}/\mu_{F}$ in shape analysis)
    \item \tW:    5\%
    \item \zjets: 10\%, (and PDF, $\alpha_{s}$, $\mu_{R}/\mu_{F}$ in shape analysis)
    \item \wjets: 10\%
    \item \gjets: 10\%
    \item diboson: 10\%
\end{itemize}
% For diboson, shape analysis treats \PWW as a signal and its cross-section is not constrained.
Since shape analysis also incorporate \zjets control regions and is sensitive to the \ttbar cross-section, the cross-section of \zjets and \ttbar is further decomposed into extra nuisance parameters, including PDF, $\alpha_{s}$, $\mu_{R}/\mu_{F}$. The counting analysis does not have \zjets control regions and is insensitive to \ttbar cross-section due to the construction of ratio of channels. So overall uncertainties of \ttbar and \zjets are used.

\subsubsection{\WW \pt reweighting}
Shape analysis treats \WW as a signal. The reweighting of the \WW \pt is accounted for by including two nuisance parameter for the resummation and factorization variations as described in section~\ref{sec:analysis:calibration:genlevel}. In the \ttbar signal region employed by the counting analysis, there is little contamination from \WW process. Thus the \WW \pt reweighting and associated systematic effects are neglected.

\subsubsection{top \pt reweighting}
Top \pt reweighting is applied in nominal case because it is observed as unnecessary, but is used to estimate the associated uncertainties. In shape analysis, the uncertainty on the top \pt scale is included as a one-sided morphing templates generated based on previous top studies in shape analysis and it turns out to be highly constrained. In counting analysis, the construction of ratios allow cancellations of \ttbar related systematics and 1\% of the top \pt scale is used as the size of uncertainty, which gets propagated through the parameter extraction.


\subsubsection{Pileup}

Each event is weighted with a scale factor to account for differences in the pileup spectrum between data and simulation.  The uncertainty on the event weights is mainly due to the uncertainty on the minimum bias cross section.  The nominal minimum bias cross section is $69.2 \pm 3.18~\text{mb}$. The effect of the uncertainty is propagated through the analysis by calculating the distribution of pileup in data while varying the cross section up and down by one standard deviation.  

\subsubsection{Trigger efficiency}

% The uncertainty due to the trigger efficiency scale factors is accounted for by saving the uncertainty of each weight and including the uncertainty in the bin uncertainty of the template histograms.  

\begin{itemize}
    \item \textit{single muon}: 0.5\% normalization uncertainty on all categories where the triggering lepton is a muon.
    \item \textit{single electron}: a $\pt$ and $\eta$ dependent correction is applied to events triggerred with single electron trigger. The correction and uncertainty is measured with tag-and-prob approach described in \ref{sec:analysis:calibration:trigger}. The uncertainty accounts for the statistical uncertainty, the variation due the triggering of the tag lepton, and variation due to the probe electron.
\end{itemize}

\subsubsection{Muon reconstruction}
    \begin{itemize}
    \item \textit{identification/isolation}: the uncertainties are accounted for each muon and are based on values provided by the POG. 
    \item \textit{energy scale}: to account for the muon energy scale, the muon \pt is varied by $\pm 1 \sigma$ (0.2\%).
    \end{itemize}

\subsubsection{Electron reconstruction}
    \begin{itemize}
        \item \textit{identification/isolation}: the uncertainties provided per electron are taken from values provided by the POG.
        \item \textit{reconstructions}: treated the same as the identification uncertainty.  Scale factors and their uncertainties are only $\eta$ dependent. 
        \item \textit{energy scale}: The electron energy scale is assumed to be know at the 0.5\% level, and is assigned a  nuisance parameter that modifies the change to the shape of the relevant kinematic quantity.
        %\item \textit{resolution}
    \end{itemize}

\subsubsection{Tau reconstruction}
    \begin{itemize}
        \item \textit{identification}: the $\PGth$ POG recommends a 5\% uncertainty on the scale factor applied to simulation. In shape analysis, because a control region is included to provide an \emph{in situ} evaluation of the $\PGth$ efficiency scale factor, the scale factors are included as \pt-dependent nuisance parameter in seven \pt bins.
        
        \item \textit{$jet\rightarrow\PGth$}: scale factors and uncertainties for jets faking taus were derived based on a dilepton plus reconstructed tau control region.  A nuisance parameter is assigned to each \pt bin used to measure the scale factor and an overall normalization nuisance parameter is assigned to account for any difference in rate between light and heavy jets.
        
        \item \textit{$\Pe\rightarrow\PGth$}: a single normalization nuisance parameter is included to templates where an electron is misreconstructed as a hadronically decaying tau. The counting analysis neglects this uncertainty because the contribution of electron faking \PGth is only sizable in $\PZ \to \PGte \PGth$ control region, not considered by the counting analysis.  
        
        \item \textit{energy scale}: the tau energy scale is corrected in correspondence with POG recommendations and an uncertainty of 1.2\% per decy mode is assigned.  These are included as three shape uncertainties depending on the reconstructed decay mode of the hadronically decaying tau. 
    \end{itemize}

    In addition, the uncertainties of different \PGth decay widths are investigated. Currently the world average of the major \PGth decay widths have about 0.5-1.0\% uncertainties. Five highest \PGth branching fractions include 
    \begin{equation*}
    \begin{split}
    &   \mathcal{B}(\PGth\to \PGp^\pm)        = 0.1082(5), \quad 
        \mathcal{B}(\PGth\to \PGp^\pm \PGp^0)  = 0.2549(9), \quad 
        \mathcal{B}(\PGth\to \PGp^\pm2\PGp^0)  = 0.0926(10),\\
    &   \mathcal{B}(\PGth\to3\PGp^\pm)        = 0.0931(5), \quad
        \mathcal{B}(\PGth\to3\PGp^\pm \PGp^0)  = 0.0462(5).           
    \end{split}
    \end{equation*}
    \noindent Since different decay modes are reconstructed separately by the CMS \PGth identification and thus have different efficiencies. The uncertainties of different \PGth widths could propagate to the \PGth identification efficiencies and lead to some impacts on the final results. To decide the impact, we tag the generator-level decay mode of reconstructed \PGth and variate the event weight accounting to the relative uncertainty of the tau decay branching fraction. Figure~\ref{fig:analysis:systematics:tauhDecayMode} shows the distribution of gen-level \PGth decay modes of $\ttbar \to cmt$ simulated events selected in the \cmt channel. Considering the above five major \PGth decay mode, the corresponding uncertainties on the \BWl are found to be less then 0.1\% relatively, small enough to neglect.
    
    \begin{figure}
    \centering
    \includegraphics[width=0.99\textwidth]{chapters/Analysis/sectionSystematics/figures/tauBr/tauhDecay_mutau.png}
    \caption{Distribution of gen-level \PGth decay modes of $\ttbar \to \cmt$ simulated events selected in the \cmt channel.}
    \label{fig:analysis:systematics:tauhDecayMode}
\end{figure}




\subsubsection{Jet reconstruction}

    Jet systematics impact the analysis by modifying the acceptance of
    events in the various jet multiplicity categories.  With that in
    mind, the uncertainty is taken into account by varying the various
    sources of jet uncertainties and assessing the resulting effect on
    the jet and \PQb tag multiplicities.

    \begin{itemize}
        \item \textit{energy scale}: the jet energy scale is varied by
            the various uncertainty sources on the jet energy
            corrections provided by the JetMET POG.  These are included
            as 18 shape nuisance parameter's (N.B. the lepton kinematic quantities
            which are fit are generally not affected by variation of the
            jet energy scale, but migration between different \PQb tag
            categories does happen)
        \item \textit{resolution}: the jet energy is corrected in
            simulation to account for the difference in resolution
            between data and simulation.  The correction is applied per
            jet and is dependent on the jet \pt.  Consequently, there is
            an associated uncertainty.  The overall effect of this is
            estimated by varying the scale factor up and down one
            standard deviation and propagating the effect to the
            morphing templates.
    \end{itemize}

\subsubsection{b-tagging}
    
The \PQb tag modelling in simulation is corrected to better describe the
data based on scale factors.  The uncertainty on the correction is
assessed based on up and down variations of \PQb tagging and mistagging
scale factors supplied by the \PQb tag POG.  The \PQb tag uncertainties are
factorized based on the various sources of uncertainty considered in the
calculation of the scale factors.  The variation is propagated through
the analysis through the inclusion of shape nuisance parameters for both
b tagging and mistagging variation.




\subsubsection{Theory/simulation modelling}

In addition to the normalization uncertainties coming from PDF, QCD
scale, and uncertainty on $\alpha_{s}$, several other theory
uncertainties are accounted for.  These are only included for $\ttbar$
processes and are applied as recommended by TOP PAG.

\begin{itemize}
    \item \textit{ISR/FSR}: variations to $\alpha_{S}$ affecting both
        ISR and FSR are evaluated based on dedicated \ttbar MC samples.
        This is done for ISR and FSR independently.  These variations are
        propagated through the analysis through morphing templates.
    \item \textit{ME-PS matching scale}: matrix element to parton shower
        matching is regulated at the generator level by the \textit{hdamp}
        parameter.  This parameter is varied from the nominal value of
        $1.58^{+0.66}_{-0.59}$ in dedicated MC samples and propagated 
        through morphing templates.
    \item \textit{Underlying event}: modelling of the underlying event
        is dependent on the Pythia tune that is used (in the case of this
        analysis, CUETP8M2T4)~\cite{CMS-PAS-TOP-16-021}.  Dedicated samples
        are generated varying the appropriate parameters and the variation
        in efficiency is propagated through the analysis with morphing
        templates.
        %(\emph{Not included in this iteration})
\end{itemize}

There are two issues with these uncertainty sources that are worth
considering.  The first is that the variations due to these sources of
uncertainty are estimated from dedicated MC samples.  This leads to a
fairly sizeable statistical uncertainty, and can lead to exagerated
uncertainties and strange behavior in the morphing templates (e.g., both
the up and down variation will predict yields below/above the nominal
sample).  Also, the size of the uncertainty resulting from the FSR
variation is very large (up to 20\%) in the \ceh categories.  This
level of variation would be corrected for in the scale factors
accounting for the difference in ID/misID efficiency for identified
$\PGth$ candidates, and that uncertainty in general would be much
smaller, $<5\%$. 




% Four dedicated MC samples are used for the \ttbar theoretical uncertainties,
% including FSR, ISR, matrix element parton shower (MEPS), and underline events (UE).
% Based on the four dedicated and the nominal \ttbar samples,
% the shape analysis obtains four types of the template morphine,
% while the counting analysis propagates the four deviations of signal efficiency.

% However, it is noticed that variations of theoretical parameters in these
% four dedicated MC samples could have a large influence on the $\PGth$ identification efficiency
% and the $j\to \PGth$ misidentification rate. 
% Since the uncertainty of the $\PGth$ identification efficiency and the $j\to \PGth$ misidentification rate
% are accounted for separately, the changes related to the tau identification 
% in the dedicated \ttbar samples should be removed to avoid double counting.
% Thus a correction is needed for these dedicated samples.
% Such that, the tau identification and misidentification in the dedicated samples 
% are kept the same as the nominal \ttbar sample.

To derive the correction for these dedicated MC samples, we calculate the
probabilities of reconstructing taus in the nominal and dedicated \ttbar events.
The origins of the reconstructed taus are tagged based on the matching to the
gen-level particles. Both Tight and VTight WP of tau identification are
considered. The changes in the tau id and misid due to the FSR up and down 
variation are shown in Figure~\ref{fig:analysis:systematics:sf_fsr}. The changes
due to the ISR, MEPS, UE up and down variations are shown in Figure~\ref{fig:analysis:systematics:sf_isr_MEPS_UE}.
It is clear that the FSR has a considerable impact on the \PGth id and misid that 
needs removal, while the effect from the ISR, MEPS and UE are neglectable.


\begin{figure}
    \centering
    \includegraphics[width=0.49\textwidth]{chapters/Analysis/sectionSystematics/figures/ttTheoretical/2020_MCRatio_fsr_tauGenFlavor_tauTight.png}
    \includegraphics[width=0.49\textwidth]{chapters/Analysis/sectionSystematics/figures/ttTheoretical/2020_MCRatio_fsr_tauGenFlavor_tauVTight.png}
    \caption{Effect of final state radiation on the $\PGth$ identification and $j \to \PGth$ misidentification obtained from the dedicated and the nominal \ttbar samples. 
    The Tight and VTight WP are shown on the left and right, respectively.
    }
    \label{fig:analysis:systematics:sf_fsr}
\end{figure}

% \begin{figure}
%     \centering
%     \includegraphics[width=0.4\textwidth]{chapters/Analysis/sectionSystematics/figures/ttTheoretical/2020_MCRatio_isr_tauGenFlavor_tauTight.png}
%     \includegraphics[width=0.4\textwidth]{chapters/Analysis/sectionSystematics/figures/ttTheoretical/2020_MCRatio_isr_tauGenFlavor_tauVTight.png}
%     \includegraphics[width=0.4\textwidth]{chapters/Analysis/sectionSystematics/figures/ttTheoretical/2020_MCRatio_meps_tauGenFlavor_tauTight.png}
%     \includegraphics[width=0.4\textwidth]{chapters/Analysis/sectionSystematics/figures/ttTheoretical/2020_MCRatio_meps_tauGenFlavor_tauVTight.png}
%     \includegraphics[width=0.4\textwidth]{chapters/Analysis/sectionSystematics/figures/ttTheoretical/2020_MCRatio_ue_tauGenFlavor_tauTight.png}
%     \includegraphics[width=0.4\textwidth]{chapters/Analysis/sectionSystematics/figures/ttTheoretical/2020_MCRatio_ue_tauGenFlavor_tauVTight.png}
%     \caption{ISR, MEPS, UE effect on the $\PGth$ identification and $j \to \PGth$ misidentification obtained from the dedicated and the nominal \ttbar samples.
%     The Tight and VTight WP are shown on the left and right, respectively.
%     }
%     \label{fig:analysis:systematics:sf_isr_MEPS_UE}
% \end{figure}




% \begin{table}[h]
%     \centering
%     \caption{Comparison between nominal uncertainty (in a snapshot of
%         the analysis) and the uncertainty after applying the corrections
%         to the FSR variation.}
        
%     \begin{tabular}{l|cccc}
%                                   & $W\rightarrow e$ & $W\rightarrow \mu$ & $W\rightarrow \PGth$ & $W\rightarrow h$ \\
%         \hline
%         nominal                   & 1.02             & 0.71               & 2.04                & 0.40             \\
%         w/ $\PGth$ FSR corrections & 1.01             & 0.69               & 1.69                & 0.36             \\
%     \end{tabular}
%     \label{fig:fsr_correction}
% \end{table}


% Table~\label{fig:fsr_correction} shows total uncertainties of $Br(W)$ due 
% to FSR before and after the tau id and misid correction of the dedicated FSR sample.
% Before the correction, the dedicated FSR sample leads to an artificially large
% uncertainties which double counts the tau id and misid systematics.




The FSR dedicated \ttbar samples are corrected using the SF in Figure~\ref{fig:analysis:systematics:sf_fsr}.
The up and down variations given by the dedicated MC samples lead to envelops on the \ttbar event efficiencies.
As discussed in section~\ref{sec:analysis:method}, there are 21 \ttbar event efficiencies corresponding to 21 different
$\WW$ decay scenarios. For VTight WP, the 21 envelops on efficiencies due to FSR, ISR, MEPS, and UE variations are shown in 
Figure~\ref{fig:analysis:systematics:effAfterCorrFSR}-\ref{fig:analysis:systematics:effAfterCorrUE}. 
Due to the finite statistics of the dedicated MC samples, the envelops edges are smear by the MC statistics, 
which are also shown in the Figure~\ref{fig:analysis:systematics:effAfterCorrFSR}-\ref{fig:analysis:systematics:effAfterCorrUE}.


\begin{figure}
    \centering
    \includegraphics[width=0.99\textwidth]{chapters/Analysis/sectionSystematics/figures/ttTheoretical/fsr.png}    
    \caption{envelops on 21 efficiencies due to final state radiation (FSR) uncertainty. VTight WP is shown.}
    \label{fig:analysis:systematics:effAfterCorrFSR}
\end{figure}



\begin{figure}
    \centering
    \includegraphics[width=0.99\textwidth]{chapters/Analysis/sectionSystematics/figures/ttTheoretical/isr.png}
    \caption{Envelops on 21 efficiencies due to initial state radiation (ISR) uncertainty. VTight WP is shown.}
    \label{fig:analysis:systematics:effAfterCorrISR}
\end{figure}


\begin{figure}
    \centering
    \includegraphics[width=0.99\textwidth]{chapters/Analysis/sectionSystematics/figures/ttTheoretical/meps.png}
    \caption{Envelops on 21 efficiencies due to parton shower matching (ME-PS) uncertainty. VTight WP is shown.}
    \label{fig:analysis:systematics:effAfterCorrMEPS}
\end{figure}


\begin{figure}
    \centering
    \includegraphics[width=0.99\textwidth]{chapters/Analysis/sectionSystematics/figures/ttTheoretical/ue.png}
    \caption{Envelops on 21 efficiencies due to underline event (UE) uncertainty. VTight WP is shown.}
    \label{fig:analysis:systematics:effAfterCorrUE}
\end{figure}




% \subsubsection{Tau Hadronic Decay Reweighting}

% The tau decay in the simulation is handled by \PYTHIA using the \PYTHIA default branching fractions, which are about 0.5\% different from the experimental values in the PDG~\ref{pdg2020}. This deviation is corrected by reweighting the simulated events with hadronic taus to match the PDG tau decay branching fractions. The 


% The MC events with $\PGth$ in the \cet and $\mu \PGth$ channel is essential to the sensitivity of the $Br(W\to\PGth)$ measurement. However, the tau's hadronic decay branching fraction $B(\PGth \to  \rm{hadrons})$ in the MC simulation are different from the experimental world average in the PDG. The $\PGth$ selection efficiency could be impacted by such difference because various tau's hadronic  decay mode have different efficiencies in the CMS $\PGth$ reconstruction with SPH algorithm.

% Thus it is necessary to reweight the MC events to correct the deviation of tau's decay in the simulation with respect to the PDG values. For the values in the \PYTHIA simulation assumption and the PDG world average, tau's hadronic decay branching fractions are listed in table~\ref{tab:tauhReweighting}. The difference between values in \PYTHIA8 and PDG is about $0.5\%$. The ratios of PDG and \PYTHIA values are also included, which are the event weights applied for the $\PGth \to h$ reweighting.

    
    
% \begin{table}[ht]
%   \centering
%   \setlength{\tabcolsep}{1 em}
%   \renewcommand{\arraystretch}{1.5}
%   \caption{ The values of $B(\PGth \to  \rm{hadrons})$ in PYTHIA8 and in PDG.}
%   \begin{tabular}{l|c|c|c}
%   \hline
%                               & PDG        & \PYTHIA   & PDG / \PYTHIA \\
%   \hline
%   $B(\PGth\to \PGp^\pm)$       & 0.1082(5)  & 0.1076825 & 1.00481       \\
%   $B(\PGth\to \PGp^\pm+ \PGp^0)$& 0.2549(9)  & 0.2537447 & 1.00455       \\
%   $B(\PGth\to \PGp^\pm+2\PGp^0)$& 0.0926(10) & 0.0924697 & 1.00141       \\
%   $B(\PGth\to3\PGp^\pm)$       & 0.0931(5)  & 0.0925691 & 1.00574       \\
%   $B(\PGth\to3\PGp^\pm+ \PGp^0)$& 0.0462(5)  & 0.0459365 & 1.00574       \\
%   \hline
%   \end{tabular}
%   \label{tab:analysis:calibration:tauhReweighting}
% \end{table}


% \begin{figure}
%     \centering
%     \includegraphics[width=0.99\textwidth]{chapters/Analysis/sectionCalibration/figures/tauBr/tauhDecay_mutau.png}
%     \includegraphics[width=0.99\textwidth]{chapters/Analysis/sectionCalibration/figures/tauBr/tauhDecay_mutau2.png}
%     \includegraphics[width=0.99\textwidth]{chapters/Analysis/sectionCalibration/figures/tauBr/tauhDecay_etau.png}
%     \includegraphics[width=0.99\textwidth]{chapters/Analysis/sectionCalibration/figures/tauBr/tauhDecay_etau2.png}
%     \caption{The gen-level daughter mesons from hadronicly decaying taus in the $tt\to \mu \PGth, e \PGth$ events passing $\mu \PGth$ and $e \PGth$ selection.}
%     \label{fig:appendix:reweightTauhBr:tauhBr}
% \end{figure}


% In MC events, the gen-level daughter mesons from hadronically decaying taus are saved.  The $\PGth$'s daughter mesons in the $tt\to \mu \PGth, e \PGth$ events  passing $\mu \PGth$ and $e \PGth$ selection are shown in Figure~\ref{fig:appendix:reweightTauhBr:tauhBr} The leading contributions to the reconstructed $\PGth$ are $\PGth\to \PGp^\pm+\PGp^0 $, $\PGth\to 3\PGp^\pm$, $\PGth\to \PGp^\pm+2\PGp^0$, $\PGth\to \PGp^\pm$, $\PGth\to 3\PGp^\pm + \PGp^0$.  MC events with taus in those five decay modes are reweighted by 

% \begin{equation}
%   w = \frac{^{\rm PDG} B(\PGth \to  \rm{hadrons}) }{^{\rm \PYTHIA} B( \PGth \to \rm{hadrons} )}. 
% \end{equation} 


% \noindent The uncertainties of the weights are from the the PDG uncertainties.  The systematic uncertainty due to the uncertainties of $B(\PGth \to  \rm{hadrons})$ reweighting can be estimated. The effect of the $B(\PGth \to  \rm{hadrons})$  reweighting on the $B(W)$ result is small. The relative systematics from $B(\PGth \to  \rm{hadrons})$ reweighting are about $0.003 - 0.146 \%$,  shown in table~ \ref{tab:syst_tauhReweighting}.


\FloatBarrier



\subsection{Shape Analysis}

As described previously, each source of uncertainty is accounted for in
the shape analysis by including one or more associated nuisance
parameters.  After minimizing the likelihood, post-fit values for the
nuisance parameters and their associated uncertainties are obtained.
This is illustrated in Figure~\ref{fig:analysis:systematics:pulls_all}.  In general, the
pulls on the nuisance parameters do not exceed two sigma of their
initial uncertainty, but many of the nuisance parameters do become
constrained.  Additionally, the correlations for each parameter can be
obtained and are displayed in Figure~\ref{fig:analysis:systematics:corr_matrix}.  In order to
isolate the effect of each nuisance parameter on the uncertainty of the 
branching fractions, the minimization is repeated while individually fixing each
nuisance parameter either to its post-fit value plus or minus the
post-fit uncertainty.  The result of this process is shown in
Figure~\ref{fig:analysis:systematics:pulls_all}.

\begin{sidewaysfigure}[ht]
    \centering
    \includegraphics[width=\textwidth]{chapters/Analysis/sectionSystematics/figures/pulls_impacts_final.pdf}
    \caption{Pulls and constrain of all non-MC statistic nuisance
        parameters after minimizing the likelihood.}
    \label{fig:analysis:systematics:pulls_all}
\end{sidewaysfigure}

\begin{figure}[ht]
    \centering
    \includegraphics[width=0.99\textwidth]{chapters/Analysis/sectionSystematics/figures/correlation_matrix_full.pdf}
    \caption{Correlation matrix for branching fractions and nuisance
        parameters.  This does not include the nuisance parameters
        associated with bin-by-bin MC statistical uncertainty.}
    \label{fig:analysis:systematics:corr_matrix}
\end{figure}

% \begin{figure}[ht]
%     \centering
%     \includegraphics[width=1.2\textwidth, angle=-90]{chapters/Analysis/sectionSystematics/figures/unblinded_impacts.pdf}
%     \caption{impacts}
%     \label{fig:impacts_all}
% \end{figure}

\FloatBarrier



\subsection{Counting Analysis}

For the counting analysis, the systematics are assessed individually by varying up and down the sources of systematic uncertainties. The same branching fraction extraction is repeated with the variated systematic parameter. The change in the branching fractions with respect to the nominal value is treated as the systematic uncertainty resulting from a given source of systematics.

Recall that channels are divided into four groups based on the trigger types and \PQb tag multiplicities, (\gmb, \gmbb, \geb, \gebb), each of which produces one \BWl measurement. Table~\ref{tab:syst_alt} shows the uncertainties of \BWl in these four groups due to each individual source of systematics. The combine of the four groups assumes
\begin{enumerate}
    \item one single source of systematics is fully correlated among the four groups.
    \item different sources of systematics are mutually independent.
\end{enumerate}

\noindent Therefore, the chi-squared in the combine can be written as
\begin{equation}
    \chi^2 (\beta) = (\beta_0 - \textbf{A} \beta )^T \textbf{V}^{-1} (\beta_0 - \textbf{A} \beta )
\end{equation}

\noindent where $\beta = [\bwe, \bwm, \bwt]^T $ is the combined branching fraction, and
$\beta_0$ is the nominal value of the four measurements in the \gmb, \gmbb, \geb, \gebb group, defined as
% 
\begin{equation}
    \beta_0 = \bigg [
    \bwe^{\gmb},  \bwm^{\gmb},  \bwt^{\gmb}, \quad 
    \bwe^{\gmbb}, \bwm^{\gmbb}, \bwt^{\gmbb}, \quad 
    \bwe^{\geb},  \bwm^{\geb},  \bwt^{\geb}, \quad
    \bwe^{\gebb}, \bwm^{\gebb}, \bwt^{\gebb}
    \bigg ]^T,
\end{equation}

\noindent and $\textbf{A}=[I_{3\times3}, I_{3\times3}, I_{3\times3}, I_{3\times3}]^T$ is a $12 \times 3$  matrix consist of four $3\times 3$ identity matrices. $\textbf{V}$ is the variance matrix for the 12 elements in $\beta_0$, which combines the various sources of statistical and the systematic uncertainties.

\begin{equation}
    \textbf{V} =
    \sum_{n \in \text{data,MC}} \big( \Delta_{n}\beta_0 \big) \otimes   \big( \Delta_{n}\beta_0 \big) +
    \sum_{\theta \in \text{syst}} \big( \Delta_{\theta}\beta_0 \big) \otimes  \big( \Delta_{\theta}\beta_0 \big).
\end{equation}

\noindent where $\Delta_{\theta}\beta_0$ is the variation of \BWl with respective to one sigma of systematic parameter $\theta$, and the absolute value of $\Delta_{\theta}\beta_0$ are the shown as rows in Table~\ref{tab:syst_alt}. The statistical and systematic part of the $\textbf{V}$ matrix is shown in Figure~\ref{fig:corBetaBar}. The combined \bwl can be analytically calculated:
\begin{equation}
    \beta =   (\textbf{A}^T \textbf{V}^{-1} \textbf{A})^{-1}(\textbf{A}^T \textbf{V}^{-1}) \beta_0 , \quad
    \text{with } \textbf{Var}\big[\beta\big]  =   (\textbf{A}^T \textbf{V}^{-1} \textbf{A})^{-1}.
\end{equation}

\begin{figure}[ht]
    \centering
    \includegraphics[width=0.99\textwidth]{chapters/Analysis/sectionSystematics/figures/covarMatrix_total.png}
    \caption{ The statistical and systematic part of the $\textbf{V}$ matrix. }
    \label{fig:corBetaBar}
\end{figure}

\begin{table}[]
  \renewcommand{\arraystretch}{1.1}
  \setlength{\tabcolsep}{0.4em}
  \centering
  \caption{ Statistical and systematic uncertainties in counting analysis. }
  \resizebox{\textwidth}{!}{\begin{sidewaystable}[p]
  \small
  \renewcommand{\arraystretch}{1.2}
  \centering

  \begin{tabular}{|l|ccc|ccc|ccc|ccc|ccc|}
  \hline
  Error Source & \multicolumn{3}{c|}{$\mu$-1b} & \multicolumn{3}{c|}{$\mu$-2b} & \multicolumn{3}{c|}{$e$-1b} & \multicolumn{3}{c|}{$e$-2b} \\
  \hline
                & $B_e$ & $B_\mu$ & $B_\tau$ & $B_e$ & $B_\mu$ & $B_\tau$ & $B_e$ & $B_\mu$ & $B_\tau$ & $B_e$ & $B_\mu$ & $B_\tau$ \\
  \hline
  StatErr of Data                            & 0.543 & 0.533 & 1.243 & 0.714 & 0.637 & 1.492 & 0.743 & 0.557 & 1.520 & 0.904 & 0.707 & 1.807 \\ 
  StatErr of bg MC                           & 0.178 & 0.745 & 0.767 & 0.110 & 0.411 & 0.501 & 0.897 & 0.257 & 1.065 & 0.494 & 0.137 & 0.521 \\ 
  StatErr of sg MC                           & 0.168 & 0.151 & 0.415 & 0.189 & 0.165 & 0.428 & 0.217 & 0.176 & 0.503 & 0.233 & 0.192 & 0.520 \\ 
  \hline
  PDG err of $Br^\tau_e$                     & 0.002 & 0.019 & 0.029 & 0.002 & 0.019 & 0.029 & 0.003 & 0.019 & 0.029 & 0.003 & 0.020 & 0.030 \\ 
  PDG err of $Br^\tau_\mu$                   & 0.047 & 0.017 & 0.098 & 0.047 & 0.017 & 0.099 & 0.041 & 0.013 & 0.101 & 0.043 & 0.013 & 0.106 \\ 
  2.5$\%$ err of luminosity                  & 0.330 & 0.461 & 0.120 & 0.093 & 0.064 & 0.049 & 0.135 & 0.390 & 0.204 & 0.002 & 0.101 & 0.092 \\ 
  5$\%$ err of tt XS                         & 0.002 & 0.000 & 0.151 & 0.009 & 0.015 & 0.032 & 0.021 & 0.011 & 0.148 & 0.011 & 0.002 & 0.003 \\ 
  5$\%$ err of tW XS                         & 0.002 & 0.001 & 0.157 & 0.010 & 0.015 & 0.033 & 0.022 & 0.012 & 0.155 & 0.011 & 0.002 & 0.004 \\ 
  5$\%$ err of t XS                          & 0.062 & 0.062 & 0.033 & 0.053 & 0.052 & 0.058 & 0.063 & 0.060 & 0.032 & 0.052 & 0.054 & 0.040 \\ 
  5$\%$ err of W+Jets XS                     & 0.343 & 0.354 & 0.325 & 0.068 & 0.068 & 0.066 & 0.349 & 0.347 & 0.366 & 0.065 & 0.066 & 0.084 \\ 
  10$\%$ err of Z+Jets XS                    & 0.495 & 2.655 & 0.237 & 0.122 & 0.491 & 0.055 & 1.576 & 0.501 & 0.173 & 0.275 & 0.104 & 0.041 \\ 
  10$\%$ err of $\gamma$+Jets XS             & 0.020 & 0.019 & 0.029 & 0.005 & 0.005 & 0.007 & 0.249 & 0.247 & 0.213 & 0.058 & 0.058 & 0.081 \\ 
  10$\%$ err of VV XS                        & 0.004 & 0.044 & 0.027 & 0.001 & 0.010 & 0.005 & 0.038 & 0.003 & 0.021 & 0.008 & 0.001 & 0.001 \\ 
  25$\%$ err of QCD in $e 4j$                & 0.000 & 0.000 & 0.000 & 0.000 & 0.000 & 0.000 & 1.164 & 1.118 & 2.410 & 0.219 & 0.218 & 0.406 \\ 
  25$\%$ err of QCD in $\mu 4j$              & 0.742 & 0.737 & 1.562 & 0.223 & 0.214 & 0.384 & 0.000 & 0.000 & 0.000 & 0.000 & 0.000 & 0.000 \\ 
  25$\%$ err of QCD in $e\tau$               & 0.000 & 0.000 & 0.000 & 0.000 & 0.000 & 0.000 & 0.372 & 0.498 & 2.651 & 0.069 & 0.092 & 0.503 \\ 
  25$\%$ err of QCD in $\mu\tau$             & 0.345 & 0.465 & 2.360 & 0.185 & 0.250 & 1.285 & 0.000 & 0.000 & 0.000 & 0.000 & 0.000 & 0.000 \\ 
  top pT reweighting                         & 0.000 & 0.000 & 0.032 & 0.002 & 0.003 & 0.007 & 0.004 & 0.002 & 0.031 & 0.002 & 0.000 & 0.001 \\ 
  0.6$\%$ err of $\epsilon_e$ reco           & 0.575 & 0.054 & 0.042 & 0.583 & 0.055 & 0.042 & 0.709 & 0.160 & 0.103 & 0.574 & 0.084 & 0.069 \\ 
  1.4$\%$ err of $\epsilon_e$ id             & 1.386 & 0.129 & 0.101 & 1.410 & 0.133 & 0.101 & 1.766 & 0.335 & 0.275 & 1.456 & 0.163 & 0.197 \\ 
  0.1$\%$ err of $\epsilon_\mu$ reco         & 0.015 & 0.125 & 0.016 & 0.008 & 0.095 & 0.011 & 0.009 & 0.078 & 0.008 & 0.008 & 0.077 & 0.008 \\ 
  0.2$\%$ err of $\epsilon_\mu$ id           & 0.052 & 0.496 & 0.066 & 0.021 & 0.370 & 0.045 & 0.033 & 0.299 & 0.029 & 0.032 & 0.299 & 0.031 \\ 
  5$\%$ err of $\epsilon_\tau$               & 0.745 & 1.004 & 5.091 & 0.694 & 0.937 & 4.823 & 0.723 & 0.967 & 5.146 & 0.700 & 0.937 & 5.111 \\ 
  4.7$\%$ err of $\epsilon_{j\to\tau}$       & 0.460 & 0.620 & 3.145 & 0.307 & 0.414 & 2.129 & 0.458 & 0.613 & 3.260 & 0.290 & 0.388 & 2.115 \\ 
  0.5$\%$ err of $ES_{e}$                    & 0.249 & 0.023 & 0.018 & 0.228 & 0.022 & 0.016 & 0.008 & 0.171 & 0.061 & 0.010 & 0.247 & 0.017 \\ 
  0.2$\%$ err of $ES_{\mu}$                  & 0.095 & 0.092 & 0.033 & 0.093 & 0.092 & 0.035 & 0.013 & 0.116 & 0.011 & 0.012 & 0.114 & 0.012 \\ 
  1.2$\%$ err of $ES_{\tau\to\pi^\pm}$       & 0.034 & 0.046 & 0.232 & 0.035 & 0.047 & 0.244 & 0.034 & 0.046 & 0.245 & 0.030 & 0.040 & 0.216 \\ 
  1.2$\%$ err of $ES_{\tau\to\pi^\pm\pi^0}$  & 0.086 & 0.116 & 0.587 & 0.069 & 0.093 & 0.477 & 0.066 & 0.088 & 0.469 & 0.075 & 0.100 & 0.548 \\ 
  1.2$\%$ err of $ES_{\tau\to3\pi^\pm}$      & 0.026 & 0.035 & 0.175 & 0.026 & 0.034 & 0.177 & 0.024 & 0.032 & 0.172 & 0.024 & 0.032 & 0.176 \\ 
  Single-e Trigger (probe syst)              & 0.218 & 0.020 & 0.016 & 0.222 & 0.021 & 0.016 & 0.029 & 0.032 & 0.004 & 0.036 & 0.004 & 0.009 \\ 
  Single-e Trigger (tag syst)                & 0.495 & 0.046 & 0.036 & 0.503 & 0.047 & 0.036 & 0.063 & 0.088 & 0.080 & 0.037 & 0.013 & 0.038 \\ 
  0.5$\%$ err of $Br_{\tau\to\pi^\pm}$       & 0.008 & 0.011 & 0.047 & 0.009 & 0.012 & 0.050 & 0.008 & 0.011 & 0.047 & 0.009 & 0.012 & 0.055 \\ 
  0.4$\%$ err of $Br_{\tau\to\pi^\pm\pi^0}$  & 0.018 & 0.024 & 0.102 & 0.019 & 0.025 & 0.108 & 0.019 & 0.025 & 0.110 & 0.020 & 0.025 & 0.117 \\ 
  1.1$\%$ err of $Br_{\tau\to\pi^\pm2\pi^0}$ & 0.022 & 0.029 & 0.124 & 0.022 & 0.029 & 0.120 & 0.022 & 0.028 & 0.123 & 0.024 & 0.031 & 0.143 \\ 
  0.5$\%$ err of $Br_{\tau\to3\pi^\pm}$      & 0.015 & 0.021 & 0.094 & 0.017 & 0.022 & 0.102 & 0.016 & 0.021 & 0.100 & 0.017 & 0.022 & 0.106 \\ 
  1.1$\%$ err of $Br_{\tau\to3\pi^\pm\pi^0}$ & 0.009 & 0.011 & 0.043 & 0.010 & 0.012 & 0.046 & 0.009 & 0.011 & 0.043 & 0.010 & 0.012 & 0.046 \\ 
  Pileup                                     & 0.041 & 0.183 & 0.777 & 0.231 & 0.026 & 0.891 & 0.428 & 0.474 & 0.592 & 0.248 & 0.137 & 0.835 \\ 
  JES                                        & 2.300 & 0.750 & 4.421 & 1.823 & 1.543 & 2.968 & 1.681 & 2.370 & 4.577 & 1.681 & 1.773 & 2.993 \\ 
  JER                                        & 0.238 & 0.180 & 0.265 & 0.143 & 0.146 & 0.356 & 0.259 & 0.249 & 0.406 & 0.148 & 0.138 & 0.538 \\ 
  Btag                                       & 0.098 & 0.772 & 0.643 & 0.111 & 0.023 & 0.114 & 0.181 & 0.091 & 0.762 & 0.024 & 0.109 & 0.088 \\ 
  Mistag                                     & 0.100 & 0.141 & 0.035 & 0.100 & 0.056 & 0.090 & 0.077 & 0.142 & 0.124 & 0.030 & 0.096 & 0.135 \\ 
  tt fsr                                     & 0.760 & 0.583 & 0.743 & 0.236 & 0.253 & 0.643 & 0.289 & 0.473 & 0.756 & 1.029 & 0.065 & 1.337 \\ 
  tt isr                                     & 0.724 & 0.747 & 1.105 & 0.720 & 0.723 & 0.876 & 0.317 & 1.060 & 1.414 & 0.043 & 0.830 & 0.062 \\ 
  tt UE                                      & 0.021 & 0.037 & 1.665 & 0.306 & 1.017 & 0.266 & 0.122 & 0.177 & 1.060 & 0.172 & 0.133 & 0.053 \\ 
  tt MEPS                                    & 0.198 & 0.653 & 1.699 & 1.117 & 0.645 & 0.129 & 0.033 & 0.743 & 1.812 & 0.163 & 1.279 & 1.196 \\ 
  \hline
  Total                                      & 3.378 & 3.646 & 8.655 & 3.047 & 2.579 & 6.609 & 3.643 & 3.459 & 9.135 & 2.884 & 2.728 & 6.967 \\ 
  \hline
  \end{tabular}
  \caption{ Statistical and systematic error of four categories. }
  \label{tab:syst_alt}
\end{sidewaystable}
}
 
  \label{tab:syst_alt}
\end{table}



\FloatBarrier



% \subsubsection{Method and Result of Analytic Combining}

% The counting analysis extracts leptonic branching fractions
% $\{\bwe, \bwm, \bwt\}$ simultaneously from yields of mutually exclusive channels, 
% grouped in four trigger-bjet categories, $\mu-1b$,  $\mu-2b$,  $e-1b$ and $e-2b$.
% Its parameter extraction outputs totally 4 sets of $\{\bwe, \bwm, \bwt\}$,
% one set in each category. We denote the output 4 sets of $\{\bwe, \bwm, \bwt\}$
% as a vector $\beta$.

% \begin{equation}
%     \beta = \bigg [
%     \bwe^{\gmb}, \bwm^{\gmb}, \bwt^{\gmb}, \quad 
%     \bwe^{\gmbb}, \bwm^{\gmbb}, \bwt^{\gmbb}, \quad 
%     \bwe^{\geb}, \bwm^{\geb}, \bwt^{\geb}, \quad
%     \bwe^{\gebb}, \bwm^{\gebb}, \bwt^{\gebb}
%     \bigg ]
% \end{equation}

% $\beta$ variates with respect to the statistical fluctuation 
% of event yields, as well as the variation of systematic parameters,
% leading to its statistical and systematic uncertainties.


% The statistical variance of $\beta$ is calculated by propagating 
% the statistical uncertainty of yield in each channel, 
% then summing them in quadrature. As is given in Eqn \ref{eqn:statVar}.
% The summation in quadrature is based on the fact that the statistical 
% fluctuation in each channel is fully independent.

% \begin{equation}
%     \textbf{V}_{stat} = \sum_{i \in ch} 
%     \bigg( \frac{\partial \beta}{\partial n_i} \delta_{n_i} \bigg) \otimes 
%     \bigg( \frac{\partial \beta}{\partial n_i} \delta_{n_i} \bigg)
%     \label{eqn:statVar}
% \end{equation}

% The systematic uncertainty due to a given systematic parameter is
% evaluated by variating this systematic parameter by its 
% own uncertainty and then taking the changes of outcoming $\beta$. 
% The total systematic variance is obtained by summing all systematic 
% uncertainties in quadrature. As is given in Eqn \ref{eqn:systVar}.
% This summation of quadrature is based on the assumption made in counting analysis 
% that all systematic parameters in consideration are independent from each other.


% \begin{equation}
%     \textbf{V}_{syst} = \sum_{\theta \in syst}
%     \big( \Delta_{\theta}\beta \big) \otimes 
%     \big( \Delta_{\theta}\beta \big)
%     \label{eqn:systVar}
% \end{equation}

% The outer product of variations is based on the fact that all elements of
% $\beta$ is fully correlated when tuning up and down a systematic parameters.
% Table \ref{tbl:errors} shows the squared root of diagonal elements of 
% $\textbf{V}_{stat}$ and $\textbf{V}_{syst}$. The full matrix of $\textbf{V}_{stat}$ 
% and $\textbf{V}_{syst}$ are shown in Fig \ref{fig:varStatSyst}.
% The total variance of $\beta$ is summation of statistics and systematics variance.
% The variance matrix not only represents the sensitivity of the measurement 
% to each systematics, but also characterizes the correction among 
% $\{\bwe, \bwm, \bwt\}$ within each category and across all four categories. 

% \begin{equation}
%     \textbf{V} = \textbf{V}_{stat} + \textbf{V}_{syst}
% \end{equation}

% To combine the branching fraction in four categories, we construct $\chi^2$ parametrized
% by average branching fractions, $\bar{\beta}_e,\bar{\beta}_\mu, \bar{\beta}_\PGth$ 
% in Eqn \ref{eqn:DefineChisquared}. 

% \begin{equation}
%     \chi^2 (\bar{\beta}_e, \bar{\beta}_\mu, \bar{\beta}_\PGth) = 
%     (\beta - \bar{\beta} )^T \textbf{V}^{-1} (\beta - \bar{\beta} )
%     \label{eqn:DefineChisquared}
% \end{equation}

% where $\bar{\beta}$ is linearly parametrized by $\bar{\beta}_e, \bar{\beta}_\mu, \bar{\beta}_\PGth$ 
% via a $12 \times 3$ matrix A which consists of four $3\times 3$ identity matrices
% distributing parameters linearly to 4 categories.

% \begin{equation}
%     \bar{\beta} = A \bigg [ \bar{\beta}_e, \bar{\beta}_\mu, \bar{\beta}_\PGth \bigg ]
%     =
%     \bigg [
%     \bar{\beta}_e, \bar{\beta}_\mu, \bar{\beta}_\PGth, \quad 
%     \bar{\beta}_e, \bar{\beta}_\mu, \bar{\beta}_\PGth, \quad 
%     \bar{\beta}_e, \bar{\beta}_\mu, \bar{\beta}_\PGth, \quad 
%     \bar{\beta}_e, \bar{\beta}_\mu, \bar{\beta}_\PGth
%     \bigg ]
% \end{equation}

% Thanks to this linearly parametrizing of $\bar{\beta}$, the first and second 
% derivatives of $\chi^2$ can be calculated analytically.

% \begin{align}
%     &\nabla \chi^2   = -2 (A^T V^{-1} \beta - A^T V^{-1} \bar{\beta} )
%     \\
%     &\nabla^2 \chi^2 = 2 A^T V^{-1} A 
% \end{align}

% The central value of average branching fraction comes from minimizing this $\chi ^2$, or
% $\nabla \chi ^2 = 0$
% The variance of average branching fraction comes from Fisher Information, 
% $I = \frac{1}{2} \nabla^2 \chi^2 $ evaluated at 
% point with the least chi-squared $\bar{\beta}^{LS}$.
% In other words, the covariance of average branching fractions,  
% $U [\bar{\beta}^{LS}]$,
% equals to the inverse of
% Fisher Information at $\bar{\beta} = \bar{\beta}^{LS}$.

% \begin{equation}
%     \bar{\beta}^{LS} =   (A^T V^{-1} A)^{-1}(A^T V^{-1})  \cdot \beta
%     \label{eqn:combineMean}
% \end{equation}

% \begin{equation}
%     U \big[\bar{\beta}^{LS} \big]  =   (A^T V^{-1} A)^{-1}
%     \label{eqn:combineCovar}
% \end{equation}

% These analytic formula for LS estimator of linear parameters in 
% Eqn \ref{eqn:combineMean} and \ref{eqn:combineCovar} are derived as Eqn 7.10 and 7.11
% in Glen Cowan's Statistical Data Analysis.
% With this combining method, the values of average branching 
% fractions are shown in Eqn \ref{eqn:averagebf}. The correlation of 
% $\bar{\beta}_e,\bar{\beta}_\mu, \bar{\beta}_\PGth$ is shown in Fig \ref{fig:corBetaBar}. 

% \begin{align}
%     \bar{\beta}_e^{LS}    &= 0.1080 \times \big[1 \pm 0.37\% \text{ (stat)} \pm 2.06\% \text{ (syst)} \big] \\
%     \bar{\beta}_\mu^{LS}  &= 0.1080 \times \big[1 \pm 0.33\% \text{ (stat)} \pm 2.15\% \text{ (syst)} \big] \\
%     \bar{\beta}_\PGth^{LS} &= 0.1080 \times \big[1 \pm 0.81\% \text{ (stat)} \pm 6.35\% \text{ (syst)} \big]
%     \label{eqn:averagebf}
% \end{align}

% \begin{figure}[ht]
%     \centering
%     % \includegraphics[width=7cm]{section5/figures/covarMatrix_beta.png}
%     \caption{ The correlation of $\bar{\beta}_e,\bar{\beta}_\mu, \bar{\beta}_\PGth$ }
%     \label{fig:corBetaBar}
% \end{figure}





\section{Results}
\label{sec:analysis:result}


This section presents results of the measurements of the branching
fractions using the two methods described earlier. 
The two approaches yield consistent result.
Then, the ratios of branching
fractions and the derived SM quantities are presented based on the branching
fractions measured by the shape analysis, the more precise approach.


\subsection{$\mathrm{W}$ branching fractions}
\label{sec:analysis:result:BWl}
The values of the
branching fractions measured using the two approaches are shown in
Table~\ref{tab:results} and Figure~\ref{fig:analysis:result:wbr_result_1D}.  These plots
also show the current, best measured values of the \PW branching fractions
based on a combination of the measurements done by each of the LEP
experiments~\cite{Schael:2013ita}.  The measured values are strongly
correlated because of the construction of the model and because of the
constraint that the sum of branching fractions for leptonic and hadronic
decay modes be equal to one.  To demonstrate this, two dimensional
contours have been drawn and are shown in Figure~\ref{fig:analysis:result:contours_2D}.

% One important distinction between the two measurements is that the MLE
% method measures the branching fraction simultaneously in all final state
% categories while the semi-analytic approach measures the branching fractions
% separately in different trigger and \PQb tag categories and then combines
% them using a $\chi^{2}$ fit.  

\begin{table}[htb!]
    \centering
    \setlength{\tabcolsep}{1.5em}
    \renewcommand{\arraystretch}{1.5}
    \caption{Values of branching fractions determined in both
        analysis approaches and the PDG values.  The errors include
        statistical and systematic uncertainties.
    \label{tab:results}}
    \begin{tabular}{l|ccc}
                           & counting              & shape                 & LEP \\
    \hline                                                                 
    w/o LU &&& \\
    \hline
    $\BWe$      & $(11.15 \pm 0.27) \%$ & $(10.77 \pm 0.1) \%$  & $(10.71 \pm 0.16)$ \% \\
    $\BWm$      & $(11.13 \pm 0.22) \%$ & $(10.91 \pm 0.08) \%$ & $(10.63 \pm 0.15)$ \% \\
    $\BWt$      & $(10.63 \pm 0.65) \%$ & $(10.89 \pm 0.21) \%$ & $(11.38 \pm 0.21)$ \% \\
    $\BWh$      & $(67.08 \pm 0.72) \%$ & $(67.42 \pm 0.28) \%$ & $(-- \pm --)$ \% \\
    \hline
    w/ LU &&& \\
    \hline
    $\BWl$      & $(-- \pm --)\%$       & $(10.87 \pm 0.07)\%$  & $(10.86 \pm 0.09)\%$  \\
    $\BWh$      & $(-- \pm --)\%$       & $(67.38 \pm 0.22)\%$  & $(67.41 \pm 0.27)\%$  \\
    \end{tabular}
\end{table}


\begin{table}[htb!]
    \centering
    \renewcommand{\arraystretch}{1.5}
    \caption{Correlation matrix of leptonic branching fractions.}
    \label{tab:results_corr}
    \resizebox{0.9\textwidth}{!}{
    \begin{tabular}{ccc}
        counting              & shape                 & LEP \\
      $\begin{bmatrix} 1 &+0.576 &-0.060 \\  +0.576 &1 &-0.265 \\ -0.265 &+0.714 &1 \end{bmatrix}$  
    & $\begin{bmatrix} 1 &+0.439 &+0.138 \\  +0.439 &1 &+0.190 \\ +0.138 &+0.190 &1 \end{bmatrix}$ 
    & $\begin{bmatrix} 1 &+0.136 &-0.201 \\  +0.136 &1 &-0.122 \\ -0.201 &-0.122 &1 \end{bmatrix}$ \\
    \end{tabular}}
\end{table}




            
\begin{figure}[htb!]
    \begin{center}
    \includegraphics[width=0.7\textwidth]{chapters/Analysis/sectionResult/figures/unblinded_summary_plot.pdf}
    \caption{Summary of measured values of leptonic branching fractions.}
    \label{fig:analysis:result:wbr_result_1D}
    \end{center}
\end{figure}

% \begin{figure}[htb!]
%     \begin{center}
%     \includegraphics[width=0.5\textwidth]{chapters/Analysis/sectionResult/figures/correlation_matrix_POI_unblinded.pdf}
%     \caption{Correlation matrix between each branching fraction component.}
%     \label{fig:analysis:result:correlation_matrix_POI}
%     \end{center}
% \end{figure}

\begin{figure}[htb!]
    \begin{center}
    \includegraphics[width=0.99\textwidth]{chapters/Analysis/sectionResult/figures/result_contours_2d_br_dash.pdf}
    \caption{Two dimensional comparisons of leptonic branching
    fractions.  For each pair shown in the panels, the branching
    fraction that is not shown has been marginalized over.  The dashed
    lines correspond to 68\% and 95\% contour levels for the resulting two
    dimensional Gaussian distribution.}
    \label{fig:analysis:result:contours_2D}
    \end{center}
\end{figure}


\FloatBarrier











\subsection{Ratios of Branching Fractions and Derived Quantities}
\label{sec:analysis:result:derived}

Having measured the branching fractions, it is of interest to calculate the
ratios between branching fractions and their pdfs to compare to values obtained
at other experiments where only ratios are measured.  To transform the
likelihood of the branching fractions to the likelihood for ratios, the following
integral transformation is evaluated\cite{10.2307/2334671},

% \begin{equation}
%     f(r) = \int_{-\infty}^{\infty} \left|b_{\ell}\right| g(r b_{\ell}, b_{\ell}) \,db_{\ell},
% \end{equation}


\begin{equation}
    f(r_{\PGt/\Pe}, r_{\PGt/\PGm}) = \int_{-\infty}^{\infty}
    \left| \beta_{\PGt}\right|g(r_{\PGt/\Pe}\beta_{\PGt}, r_{\PGt/\PGm}\beta_{\PGt}, \beta_{\PGt})
    d\beta_{\PGt}
\end{equation}

where $r_{\PGt/\Pe} = \bwt/\bwe$ and $r_{\PGt/\PGm} = \bwt/\bwm$ and $g(\bwe,\bwm, bwt)$  is the PDF of the branching fractions which is normal distribution with parameters
determined from the \BWl measurement.  The resulting ratios are shown in in Figure~\ref{fig:analysis:result:ratios_2D}.
Table~\ref{tab:analysis:result:ratios} shows comparisons between the ratios constructed from
the measurements described above, and those measured by LEP and ATLAS.

% \input{chapters/Analysis/sectionResult/ratiosFomula.tex}


\begin{figure}[htb!]
    \begin{center}
    \includegraphics[width=0.7\textwidth]{chapters/Analysis/sectionResult/figures/result_contours_2d_ratio.pdf}
    \caption{Two dimensional distributions of the ratios $\BWt/\BWe$ vs $\BWt/\BWm$ 
    with comparisons of the CMS result to LEP and ATLAS measurements.}
    \label{fig:analysis:result:ratios_2D}
    \end{center}
\end{figure}

\begin{table}[htb!]
    \centering
    % \setlength{\tabcolsep}{0.5em}
    \renewcommand{\arraystretch}{2}
    \caption{Ratios of branching fractions.}
    \label{tab:analysis:result:ratios}
    \begin{tabular}{c|ccc}
                            & CMS               & LEP               & ATLAS              \\
    \hline                                                                 
    $\BWm / \BWe$           & $1.013 \pm 0.009$ & $0.993 \pm 0.019$ & --                 \\
    $\BWt / \BWe$           & $1.011 \pm 0.020$ & $1.063 \pm 0.027$ & --                 \\
    $\BWt / \BWm$           & $0.998 \pm 0.019$ & $1.070 \pm 0.026$ & $0.992 \pm 0.013$  \\
    $2 \BWt /(\BWe + \BWm)$ & $1.002 \pm 0.019$ & $1.066 \pm 0.025$ & --                 \\
    \end{tabular}
\end{table}

The values of the leptonic branching fractions can also be used as a
check of the unitarity of the CKM matrix elements and calculating the
least well measured of the matrix elements, \absVcs.
To do both of these calculations, the following relation between the
leptonic branching fractions and CKM matrix elements is useful, 

\begin{equation}
    R^\PW_{\mathrm{h}/\ell} = \frac{\BWh }{1- \BWh} = \bigg( 1 + \frac{\alpS(m_\PW)}{\pi}\bigg) \sumCKM,
\end{equation}

\noindent where $\alpS(m_\PW)$ is the strong coupling constant
at the $\PW$ pole. The CMS value for
the ratio of total hadronic and total leptonic branching fraction is 
\begin{equation}
    R^\PW_{\mathrm{h}/\ell} = 2.060\pm 0.021,
\end{equation}
\noindent which leads to the following derived value for $\alpS(m_\PW)$, \sumCKM and \absVcs.

% Solving for
% $\sum_{ij}\left|V_{ij}\right|^{2}$ and using (these are the numbers used
% for the LEP combination) a value of $\alphS(M^{2}_{\PW}) =
% 0.1120 \pm 0.0010$ and sum over CKM matix elements squared excluding
% $V_{cs}$ and $V_{tb}$ equal to $1.0544 \pm 0.0051$, the above expression
% above can be used to calculate a value of $1.991 \pm 0.019$ is obtained.
% Further solving for $\left|V_{cs}\right|$ yields a value of $0.968 \pm
% 0.010$.  The uncertainty on this quantity is almost entirely determined
% by the experimental uncertainty of the leptonic branching fraction
% measurment.



\begin{table}[!h]
    \setlength{\tabcolsep}{0.2em}
    \renewcommand{\arraystretch}{2.5}
    \centering
    \caption{SM quantities can be derived from the CMS measured $R^W_{\rm h/l}$. }
    \resizebox{0.99\textwidth}{!}{
    \begin{tabular}{ccc|ccc}
        Assumption &  &  Quantity & CMS & LEP & CMS+LEP\\
        \hline
                                                            &                   & $R^\PW_{\mathrm{h}/\ell}$ & $2.060\pm0.021$ & $2.068\pm0.025$ & $2.063\pm0.016$ \\ \hline
        CKM Unitarity: $\sumCKM = 2$                        & $\Longrightarrow$ & $\alpS(m_\PW)$            & $0.094\pm0.033$ & $0.108\pm0.040$ & $0.099\pm0.026$ \\ \hline
        PDG~\cite{pdg2020} $\alpS(m_\PW) = 1.1200\pm0.010$  & $\Longrightarrow$ & \sumCKM                   & $1.985\pm0.021$ & $1.997\pm0.025$ & $1.992\pm0.016$ \\ \hline
        $\begin{matrix} 
            \text{PDG~\cite{pdg2020}: } \alpS(m_\PW) = 1.1200\pm0.010 \\ 
            \text{PDG~\cite{pdg2020}: } \sum_{\substack{ud,us,ub\\cd,cb}} |V_{ij}|^2 =1.0490(18) 
        \end{matrix} $ 
                                                            & $\Longrightarrow$ & \absVcs                   & $0.969\pm0.011$ & $0.974\pm0.013$ & $0.971\pm0.008$ \\ 
    \end{tabular}
    }
    \label{tab:analysis:result:derivedQuantity}
\end{table}


  

 \begin{figure}[!h]
    \centering
    \includegraphics[width=0.7\textwidth]{chapters/Introduction/sectionRelatedWorks/figures/vcs.pdf}
    \caption{The \absVcs derived from the \BWl measurement by CMS, LEP and CMS+LEP, in comparison with the direct measurements~\cite{pdg2020}.}
    \label{fig:analysis:result:vcs}
\end{figure}









% \subsection{Branching Ratio}

% Test of lepton flavour universality (LFU) between electron and muons in 
% weak section has been performed to unprecedented precision
% in the past two decades. The tests have been carried out on both
% colliders and fix target experiments. Their results are shown
% in Table \ref{tbl:testlfuemu}. In general, the measurements
% branching ratios between electron and muon agree very well with 
% SM prediction.

% % \input{section6/tables/emutest.tex}

% In contract with agreement on LFU for $e$ and $\PGm$ in weak section, LPU 
% regarding $\PGt$ versus $e$ and $\PGm$, as is discussed in Chapter 1, 
% is significantly challenged by 
% measurements from ALEPH, DELPHI, OPAL and L3 with LEP e+e- collision, 
% as well as Belle, Belle and LHCb with B meson decay.


% Therefore, we are interesed in the ratio of $Br (W\to \PGt \nu)$ with respect to electron
% and muon channels,

% \begin{equation}
%     r = \frac{Br (W\to \PGt \nu)}{Br (W\to l \nu)} , \text{ where } l=e,\PGm
% \end{equation}

% based on the assumption that $Br (W\to \PGm \nu) = Br( W\to e \nu )$, which
% is well justified by the previous precision test of LFU between $e$ and $\PGm$ in weak section.
% This assumption is the same in Belle and BaBar measurements.

% The key to the success of Belle and BaBar measurements is that $tau$ are reconstructed
% by the same method as electron or muon, such that systematics regarding object
% reconstruction and selection are cancelled.
% Following this principle, we are measuring r in purely dilepton channels with muonic and electronic taus.
% Comparing with hadronic taus, this avoids the systematic uncertainty related to hadronic tau efficiency
% and misidentification.
% By using leptonic taus, systematics regarding lepton reconstruction 
% is canceled out to the first order, thus the precision of r is not limited systematically.

% The evolved dilepton channels are $\PGm\PGm$, $ee$ and $e\PGm$ with $n_j \geq 2$ and $n_b = 1,2$,
% where $\PGm\PGm$, $ee$ also include $n_b = 0$ bin for \PZ background normalization purpose.
% r is obtained by simultaneous fit to the pT spectrum of the trailing lepton in $\PGm\PGm$,
% $ee$ and $e\PGm$ channels. The methodology of this template fit is described in Section 5.3.
% The result is in Eqn \ref{eqn:fitr}.

% \begin{equation}
%     \boxed{
%     r = \frac{Br (W\to \PGt \nu)}{Br (W\to l \nu)}
%     = 1.000 \times \big[1 \pm 2.72\% \text{ (stat)} \pm 1.44\% \text{ (syst)} \big]
%     }
%     \label{eqn:fitr}
% \end{equation}

% The correlation matrix of the fit is shown in Fig \ref{fig:analysis:result:covr}.

% The measurement of r using leptonic tau has small systematic uncertainty, thanks to the 
% cancellation of reconstruction efficiency. The precision of r is statistically limited, 
% which is expected to be improved when including 2017 data.


% The improvement of r precision when including more channels is shown in
% Fig \ref{fig:analysis:result:gain}. The gain of adding $e\PGm$ and $\PGm e$ channel is
% significant, while adding $l \PGt$ and $l4j$ channel is small.


% \begin{figure}[p]
%     \centering
%     \includegraphics[width=14cm]{chapters/Analysis/sectionResult/figures/r2}
%     \caption{Fitting the pT spectrum of trailing lepton in $ee$, $\PGm\PGm$ and $e\PGm$ channel.
%     The correlation matrix among r and systematic parameters.
%     }
%     \label{fig:analysis:result:covr}
% \end{figure}
    \chapter{Clustering for HGCAL Reconstruction}
\label{sec:relatedworks}

This chapter discusses CLUE clustering algorithm to reconstruct 2D layer clusters, followed by an introduction of TICL that 
links the layer clusters to form trackersters for particle candidates. This chapter is based on published results in 
\cite{cluepaper}, \cite{Chen:2020mih}, \cite{DiPilato:2020mqs}.




\input{chapters/HGCal/sectionOverview}



\section{CLUE Algorithm}
\label{sec:clue}

\subsection{Overview}
Calorimeters with high lateral and longitudinal readout granularity, capable of providing a fine grained image of electromagnetic and hadronic showers, have been suggested for future high energy physics experiments \cite{calice2012calorimetry}. The  silicon sensor readout cells of the CMS endcap calorimeter (HGCAL) \cite{Collaboration:2293646} for HL-LHC \cite{Apollinari:2284929} have an area of about $1 \mathrm{cm}^2$.
When a particle showers, the deposited energy is collected by the sensors on the layers which the shower traverses. 
The purpose of the clustering algorithm when applied to shower reconstruction is to group together individual energy deposits (hits) originating from a particle shower. Due to the high lateral granularity, the number of hits per layer is large, and it is computationally advantageous to collect together hits in 2D clusters layer-by-layer \cite{Chen:2017btc} and then associate these 2D clusters, representing energy blobs, in different layers \cite{Collaboration:2293646}.



However, a computational challenge emerges as a consequence of the large data scale and limited time budget. %For example, clustering millions of hits in each event is tightly constrained by a millisecond-level execution time.
Coping with this challenge requires the clustering algorithm to be highly efficient while maintaining a low computational complexity. Furthermore, a linear scalability is strongly desired in order to avoid bottlenecking the performance of the entire event reconstruction.  Finally, it is highly preferable to have a fully-parallelizable clustering algorithm to take advantage of the trend of heterogeneous computing with hardware accelerators, such as graphics processing units (GPUs), achieving a higher event throughput and a better energy efficiency.



% input/output, characteristics
The input to the clustering algorithm is a set of $n$ hits, whose number varies from a few thousands to a few millions, depending on the longitudinal and transverse granularity of the calorimeter as well as on the number of particles entering the detector. The output is a set of $k$ clusters whose number is usually one or two order of magnitudes smaller than $n$ and in principle depends on both the number of incoming particles and the number of layers. Assuming that the lateral granularity of sensors is constant and finite, the average number of hits in clusters ($m=n/k$) is also constant and finite. For example, in the CMS HGCAL, $m$ is in the order of 10. This leads to the relation among the number of hits $n$, the number of clusters $k$, and the average number of hits in clusters $m$ as $n > k \gg m$.



Most well-known algorithms do not simultaneously satisfy the requirements on linear scalability and easy parallelization for applications like clustering hits in high granularity calorimeters, which is characterized by low dimension and $n > k \gg m$. It is therefore important to investigate new, fast and parallelizable clustering algorithms, as well as their optimized accompanying spatial index that can be conveniently constructed and queried in parallel.

% In this paper, we describe a novel density-based clustering algorithm (CLUE: CLUsters of Energy) with linear scalability and easy parallelization. Its development was inspired by the work described in ref.~\cite{rodriguez2014clustering}. In Section~\ref{sec:algorithm}, we describe the CLUE algorithm and its accompanying spatial index. Then in Section~\ref{sec:implementation}, some details of GPU implementations are discussed. Finally, in Section~\ref{sec:performance} we present CLUE's ability on non-spherical cluster shapes and noise rejection, followed by its computational performance when executed on CPU and GPU with synthetic data, mimicking hits in high granularity calorimeters.



% ------------------------------------
% ------------------------------------

\subsection{Algorithm}
\label{sec:algorithm}
% review of current popular method
Clustering data is one of the most challenging tasks in several scientific domains. The definition of cluster is itself not trivial, as it strongly depends on the context. Many clustering methods have been developed based on a variety of induction principles \cite{maimon2005data}. Currently popular clustering algorithms include (but are not limited to) partitioning, hierarchical and density-based approaches \cite{maimon2005data,han2011data}. Partitioning approaches, such as k-mean \cite{lloyd1982least}, compose clusters by optimizing a dissimilarity function based on distance. However, in the application to high granularity calorimeters, partitioning approaches are prohibitive because the number of clusters $k$ is not known a priori. Hierarchical methods make clusters by constructing a dendrogram with a recursion of splitting or merging. However, hierarchical methods do not scale well because each decision to merge or split needs to scan over many objects or clusters \cite{han2011data}. Therefore, they are not suitable for our application. Density-based methods, such as DBSCAN \cite{Ester:1996:DAD:3001460.3001507}, OPTICS \cite{Ankerst:1999:OOP:304182.304187} and Clustering by Fast Search and Find Density Peak (CFSFDP) \cite{rodriguez2014clustering}, group points by detecting continuous high-density regions. They are capable of discovering clusters of arbitrary shapes and are efficient for large spatial database. If a spatial index is used, their computational complexity is $O(n\log n)$ \cite{han2011data}. However, one of the potential weaknesses of the currently well-known density-based algorithms is that they intrinsically include serial processes which are hard to parallelize: DBSCAN has to iteratively visit all points within an enclosure of density-connectedness before working on the next cluster \cite{Ester:1996:DAD:3001460.3001507}; OPTICS needs to sequentially add points in an ordered list to obtain a dendrogram of reachability distance \cite{Ankerst:1999:OOP:304182.304187}; CFSFDP needs to sequentially assign points to clusters in order of decreasing density \cite{rodriguez2014clustering}. In the application to high granularity calorimeters, as discussed in Section~\ref{sec:introduction}, linear scalability and fully parallelization are essential to handle a huge dataset efficiently by means of heterogeneous computing.


% our method
In order to satisfy these requirements, we propose a fast and fully-parallelizable density-based algorithm (CLUE) inspired by CFSFDP. For the purpose of the algorithm, each sensor cell on a layer with its energy deposit is taken as a 2D point with an associated weight equaling to its energy value. As in CFSFDP, two key variables are calculated for each point: the local density $\rho$ and the separation $\delta$ defined in Equation~\ref{eqn:algorithm:defineRho} and \ref{eqn:algorithm:defineDelta}, where $\delta$ is the distance to the nearest point with higher density (``nearest-higher'') which is slightly adapted from that in CFSFDP in order to take advantage of the spatial index. Then cluster seeds and outliers are identified based on thresholds on $\rho$ and $\delta$. Differing from cluster assignment in CFSFDP, which sorts density and adds points to clusters in order of decreasing density, CLUE first builds a list of followers for each point by registering each point as a follower to its nearest-higher. Then it expands clusters by passing cluster indices from the seeds to their followers iteratively. Since such expansion of clusters is fully independent from each others, it not only avoids the costly density sorting in CFSFDP, but also enables a $k$-way parallelization. Unlike the noise identification in CFSFDP, CLUE rejects noise by identifying outliers and their iteratively descendant followers, as discussed in Section~\ref{sec:performance:clusteringResults}.

%%%%%%%%%%%%%%%%%%%%%
% Spatial Index with Grid
%%%%%%%%%%%%%%%%%%%%%

\subsubsection{Spatial index with fixed-grid}
Query of neighborhood, which retrieves nearby points within a distance, is one of the most frequent operations in density-based clustering algorithms. CLUE uses a spatial index to access and query spatial data points efficiently. Given that the physical layout of sensor cells is a multi-layer tessellation, it is intuitive to index its data with a fixed-grid, which divides the space into fixed rectangular bins \cite{bentley1979data,levinthal1966molecular}. Comparing with the data-driven structures such as KD-Tree \cite{Bentley:1975:MBS:361002.361007} and R-Tree \cite{Guttman:1984:RDI:971697.602266}, space partition in fixed-grid is independent of any particular distribution of data points \cite{rigaux2001spatial}, thus can be explicitly predefined before loading data points. In addition, both construction and query with a fixed-grid are computationally simple and can be easily parallelized. Therefore, CLUE uses a fixed-grid as spatial index for efficient neighborhood queries.


% search box
\begin{figure}[ht]
    \centering
    \includegraphics[width=0.4\textwidth]{chapters/HGCal/figures/clue/Figure1.png}
    \caption{2D points are indexed with a grid for fast neighborhood query in CLUE. Construction of this spatial index only involves registering the indices of points into the bins of the grid according to points' 2D spatial positions. To query d-neighborhood $N_d(i)$ defined in Equation~\ref{eqn:algorithm:defineNeighborhood}, taking the red (blue) point for example, we first locate its $\Omega_d(i)$ defined in Equation~\ref{eqn:algorithm:defineSearchBox}, a set of all points in the bins touched by a square window $[x_i\pm d,y_i\pm d]$. The $[x_i\pm d,y_i\pm d]$ window is shown as the orange (green) square while $\Omega_d(i)$ is shown as orange (green) points. Then we examine points in $\Omega_d(i)$ to identify those within a distance $d$ from point $i$, shown as the ones contained in the red (blue) circle.}
    \label{fig:algorithm:searchBox}
\end{figure}


For each layer of the calorimeter, a fixed-grid spatial index is constructed by registering the indices of 2D points into the square bins in the grid according to the 2D coordinates of the points. When querying $N_d(i)$, the d-neighborhood of point $i$, CLUE only needs to loop over points in the bins touched by the square window $(x_i\pm d,y_i\pm d)$ as shown in Fig.~\ref{fig:algorithm:searchBox}. We denote those points as $\Omega_d(i)$, defined as:
\begin{equation} \label{eqn:algorithm:defineSearchBox}
    \Omega_d(i) = \{j : j \in \textrm{tiles touched by the square window } [x_i\pm d,y_i\pm d] \}.
\end{equation}
\noindent where $\Omega_d(i)$ is guaranteed to include all neighbors within a distance $d$ from the point $i$. Namely, 
\begin{equation} \label{eqn:algorithm:defineNeighborhood}
    N_d(i)=\{j: d_{ij}<d, j \in \Omega_d(i) \} \subseteq \Omega_d(i).
\end{equation}
\noindent Without any spatial index, the query of $N_d(i)$ requires a sequential scan over all points. In contrast, with the grid spatial index, CLUE only needs to loop over the points in $\Omega_d(i)$ to acquire $N_d(i)$. Given that $d$ is small and the maximum granularity of points is constant, the complexity of querying $N_d(i)$ with a fixed-grid is $O(1)$.


%%%%%%%%%%%%%%%%%%%%%
% Parameters and Procedure
%%%%%%%%%%%%%%%%%%%%%

\subsubsection{Clustering procedure of CLUE}

CLUE requires the following four parameters: $d_c$ is the cutoff distance in the calculation of local density; $\rho_c$ is the minimum density to promote a point as a seed or the maximum density to demote a point as an outlier; $\delta_c$ and $\delta_o$ are the minimum separation requirements for seeds and outliers, respectively. The choice of these four parameters can be based on physics: for example, $d_c$ can be chosen based on the shower size and the lateral granularity of detectors; $\rho_c$ can be chosen to exclude noise; $\delta_c$ and $\delta_o$ can be chosen based on the shower sizes and separations. These four parameters allow more degrees of freedom to tune CLUE for the desired goals of physics. %, such as better energy resolution and better pile-up rejection.

\begin{figure}[ht]
    \centering
    \includegraphics[trim=5cm 0cm 4cm 0cm, clip,width=0.99\textwidth]{chapters/HGCal/figures/clue/Figure2.pdf}
    \caption{Demonstration of CLUE algorithm. Points are distributed inside a $6\times6$ 2D area and CLUE parameters are set to $d_c=0.5,\rho_c=3.9,\delta_c=\delta_o=1$. Before the clustering procedure starts, a fixed-grid spatial index is constructed. In the first step, shown as Fig.~\ref{fig:algorithm:procedure} (a), CLUE calculates the local density $\rho$ for each point, which is defined in Equation~\ref{eqn:algorithm:defineRho}. The color and size of points represent their local densities. In the second step, shown as Fig.~\ref{fig:algorithm:procedure} (b), CLUE calculates the nearest-higher $nh$ and the separation $\delta$ for each point, which are defined in Equation~\ref{eqn:algorithm:defineDelta}. The black arrows represent the relation from the nearest-higher of a point to the point itself. If the nearest-higher of a point is -1, there is no arrow pointing to it. In the third step, shown as Fig.~\ref{fig:algorithm:procedure} (c), CLUE promotes a point as a seed if $\rho,\delta$ are both large, or demote it to an outlier if $\rho$ is small and $\delta$ is large. Promoted seeds and demoted outliers are shown as stars and grey squares, respectively. In the fourth step, shown as Fig.~\ref{fig:algorithm:procedure} (d), CLUE propagates the cluster indices from seeds through their chains of followers defined in Equation~\ref{eqn:algorithm:defineFollowers}. Noise points, which are outliers and their descendant followers, are guaranteed not to receive any cluster ids from any seeds. The color of points represents the cluster ids. A grey square means its cluster id is undefined and the point should be considered as noise.
    }
    \label{fig:algorithm:procedure}
\end{figure}


% rho 
Figure~\ref{fig:algorithm:procedure} illustrates the main steps of CLUE algorithm. The local density $\rho$ in CLUE is defined as:
\begin{equation} \label{eqn:algorithm:defineRho}
    \rho_i = \sum_{j: j \in N_{d_c}(i)} \chi(d_{ij}) w_j,
\end{equation}
\noindent where $w_j$ is the weight of point $j$, $\chi(d_{ij})$ is a convolution kernel, which can be optimized according to specific applications. Obvious possible kernel options include flat, Gaussian and exponential functions. 

% delta
The nearest-higher and the distance to it $\delta$ (separation) in CLUE are defined as:
\begin{equation} \label{eqn:algorithm:defineDelta}
    nh_i = 
    \begin{cases}
        \arg\min_{j \in N'_{d_m}(i) } d_{ij},   & \text{if } |N'_{d_m}(i)| \neq 0  \\
        -1,                                     & \text{otherwise}
    \end{cases}, 
    \quad
    \delta_i = 
    \begin{cases}
        d_{i,nh_i}, & \text{if }  |N'_{d_m}(i)| \neq 0 \\
        +\infty,    & \text{otherwise}
    \end{cases},
\end{equation}
\noindent where $d_m= \max (\delta_o, \delta_c)$ and $N'_{d_m}(i) = \{ j : \rho_j > \rho_i, j \in N_{d_m}(i) \}$ is a subset of $N_{d_m}(i)$, where points have higher local densities than $\rho_i$. 

% expand clusters
After $\rho$ and $\delta$ are calculated, points with density $\rho>\rho_c$ and large separation $\delta>\delta_c$ are promoted as cluster seeds, while points with density $\rho<\rho_c$ and large separation $\delta>\delta_o$ are demoted to outliers. For each point, there is a list of followers defined as:
\begin{equation} \label{eqn:algorithm:defineFollowers}
    F_i = \{j : nh_j=i \}.
\end{equation}
\noindent The lists of followers are built by registering the points which are neither seeds nor outliers to the follower lists of their nearest-highers. The cluster indices, associating a follower with a particular seed, are passed down from seeds through their chains of followers iteratively. Outliers and their descendant followers are guaranteed not to receive any cluster indices from seeds, which grants a noise rejection as shown in Fig.~\ref{fig:performance:outlierCuts}. The calculation of $\rho, \delta$ and the decision of seeds and outliers both support $n$-way parallelization, while the expansion of clusters can be done with $k$-way parallelization.
% theoretical complexity
Pseudocode of CLUE is included in Appendix~\ref{sec:clue:pseudocode}.
% For each of the $n$ points, CLUE computes $\rho$, $\delta$, list of followers and cluster index with a constant complexity granted by grid spatial index, resulting in $O(n)$ computational complexity. Besides, the space complexity is also $O(n)$ because CLUE only keeps a few algorithmic variables for each of $n$ points and does not rely on any $n\times n$ matrix. 




% ------------------------------------
% ------------------------------------
\subsection{GPU Implementation}
\label{sec:implementation}

 To parallelize CLUE on GPU, one GPU thread is assigned to each point, for a total of $n$ threads, to construct spatial index, calculate $\rho$ and $\delta$, promote (demote) seeds (outliers) and register points to the corresponding lists of followers of their nearest-highers. Next, one thread is assigned to each seed, for a total of $k$ threads, to expand clusters iteratively along chains of followers. The block size of all kernels, which in practice does not have a remarkable impact on the speed performance, is set to 1024. In the test in Table~\ref{tbl:performance:breakdown}, changing the block size from 1024 to 256 on GPU leads to only about $0.14$~ms decrease in the sum of kernel execution times. The details of parallelism for each kernel are listed in Table~\ref{tbl:implementation:parallelism}. Since the results of a CLUE step are required in the following steps, it is necessary to guarantee that all the threads are synchronized before moving to the next stage. Therefore, each CLUE step can be implemented as a separate kernel. To optimize the performance of accessing the GPU global memory with coalescing, the points on all layers are stored as a single structure-of-array (SoA), including information of their layer numbers and 2D coordinates and weights. Thus points on all layers are input into kernels in one shot.


\begin{table}[t]
    \renewcommand{\arraystretch}{1.25}
    % \small
    \caption{Kernels and Parallelism}
    \centering
    \begin{tabular}{l|l|c|c}
        \hline
        Kernels                                  & parallelism    & total threads & block size \\
        \hline
        build fixed-grid spatial index           & 1 point/thread & n             & 1024 \\
        calculate local density                  & 1 point/thread & n             & 1024 \\
        calculate nearest-higher and separation  & 1 point/thread & n             & 1024 \\
        decide seeds/outliers, register followers& 1 point/thread & n             & 1024 \\
        expand clusters                          & 1 seed/thread  & k             & 1024 \\
        \hline
    \end{tabular} 
    
    \label{tbl:implementation:parallelism}
\end{table}


When parallelizing CLUE on GPU, thread conflicts to access and modify the same memory address in global memory could happen in the following three cases:

\begin{itemize}
    \item ~multiple points need to register to the same bin simultaneously;
    \item ~multiple points need to register to the list of seeds simultaneously;
    \item ~multiple points need to register as followers to the same point simultaneously.
\end{itemize}


\noindent Therefore, atomic operations are necessary to avoid the race conditions among threads in the global memory. During an atomic operation, a thread is granted with an exclusive access to read from and write to a memory location which is inaccessible to other concurrent threads until the atomic operation finishes. 
%Atomic operation allows each thread in the race of memory manipulation to exclusively conduct operations on a piece of memory until operation finishes.
This inevitably leads to some microscopic serialization among threads in race. The serialization in cases (i) and (iii) is negligible because bins are usually small as well as the number of followers of a given point. In contrast, serialization in case (ii) can be costly because the number of seeds $k$ is large. This can cause delays in the execution of kernel responsible for seed promotion. Since the atomic pushing back to the list of seeds is relatively fast in GPU memory comparing to the data transportation between host and device, the total execution time of CLUE still does not suffer significantly from the serialization in case (ii). The speed performance is further discussed in Section~\ref{sec:performance}.






% ------------------------------------
% ------------------------------------

\subsection{Performance Evaluation}
\label{sec:performance}

% Clustering Results
\subsubsection{Clustering results}
\label{sec:performance:clusteringResults}

% clustering on some non spherical topology
\begin{figure}[ht]
    \centering
    \includegraphics[clip, width=0.7\textwidth]{chapters/HGCal/figures/clue/Figure3_cut_boh.pdf}
    \caption{ 
    Examples of CLUE clustering on synthetic datasets. Each sample includes 1000 2D points with the same weight generated from certain distributions, including uniform noise points. The color of points represent their cluster ids. Black points represent outliers detached from any clusters. The links between pairs of points illustrate the relationship between nearest-higher and follower. The red stars highlight the cluster seeds.
    }
    \label{fig:performance:example}
\end{figure}

We demonstrate the clustering results of CLUE with a set of synthetic datasets, shown in Fig.~\ref{fig:performance:example}. Each example has 1000 2D points and includes spatially uniform noise points. The datasets in Fig.~\ref{fig:performance:example} (a) and (c)  are from the scikit-learn package~\cite{scikit-learn}. The dataset in Fig.~\ref{fig:performance:example} (b) is taken from~\cite{rodriguez2014clustering}. Fig~\ref{fig:performance:example} (a) and (b) include elliptical clusters and Fig~\ref{fig:performance:example} (c) contains two parabolic arcs. CLUE successfully detects
density peaks in Figs.~\ref{fig:performance:example} (a), (b), and (c).

\begin{figure}[ht]
    \centering
    \includegraphics[trim=3.5cm 0cm 3.5cm 0cm, clip,width=0.99\textwidth]{chapters/HGCal/figures/clue/Figure4.pdf}
    \caption{ Noise rejection using different values of $\delta_o$. Noise is either an outlier or a descendant follower of an outlier. In this dataset~\cite{rodriguez2014clustering}, 4000 Points are distributed in $500\times500$ 2D square area. Figure~\ref{fig:performance:outlierCuts} (a) represents the decision plot on the $\rho-\delta$ plane, where fixed $\rho_c=80$ and $\delta_c=40$ values are shown as vertical and horizontal blue lines, respectively. Three different values of $\delta_o$ (10,20,60) are shown as orange dash lines. Figures~\ref{fig:performance:outlierCuts} (b), (c) and (d) show the results with $\delta_o=10,20,60$, respectively, illustrating how increasing $\delta_o$ loosens the continuity requirement and helps to demote outliers. The level of denoise should be chosen according to the user's needs.}
    \label{fig:performance:outlierCuts}
\end{figure}

In the induction principle of density-based clustering, the confidence of assigning a low density point to a cluster is established by maintaining the continuity of the cluster. Low density points with large separation should be deprived of association to any clusters. CFSFDP uses a rather costly technique, which calculates a boarder region of each cluster and defines core-halo points in each cluster, to detach unreliable assignments from clusters~\cite{rodriguez2014clustering}. In contrast, CLUE achieves this using cuts on $\delta_o$ and $\rho_c$ while expanding a cluster, as described in Section~\ref{sec:algorithm}. The example in Fig.~\ref{fig:performance:outlierCuts} shows how cutting at different separation values helps to demote outliers. Figure~\ref{fig:performance:outlierCuts} (a) represents the decision plot on the $\rho-\delta$ plane. Points with density below $\rho_c=80$, shown on the left side of the vertical blue line, could be demoted as outliers if their $\delta$ are larger than a threshold. Once an outlier is demoted, all its descendant followers are disallowed from attaching to any clusters. While keeping $\rho_c=80$ fixed, the effect of using three different values of $\delta_o$ (10, 20, 60), shown as orange dash lines in Fig.~\ref{fig:performance:outlierCuts} (a), has been investigated. The corresponding results are shown in Fig.~\ref{fig:performance:outlierCuts} (b),  (c) and (d), respectively.

% Scalability
\subsubsection{Execution time and scalability}
\label{sec:performance:executionTime}

\begin{figure}[ht!]
    \centering
    \includegraphics[width=0.9\textwidth]{chapters/HGCal/figures/clue/Figure5_patatrack02_1.pdf}
    \caption{ (\emph{Upper}) Execution time of CLUE on the single-threaded CPU, multi-threaded CPU with TBB and GPU scale linearly with number of input points, ranging from $10^5$ to $10^6$ in total. Execution time on single-threaded CPU is shown as blue circle dots and on 10 multi-threaded CPU with TBB is shown as blue square dots, while the time on GPU is shown as green circle dots. The stacked bars represent the decomposition of execution time. The green and blue narrower bars are latency for data traffic between host memory and device memory; wider bars represent time of essential CLUE steps; light grey narrower bars labelled as ``other'' are the difference between the total execution time and the sum of major CLUE steps (and major CUDA API calls if GPU). (\emph{Lower}) Comparing with the single-threaded CPU, the speedup factors of the GPU range from 48 to 112,  while the speedup factors of the multi-threaded CPU with TBB range from 1.2 to 2.0, which is less than the number of concurrent threads on CPU because of atomic pushing to the data containers discussed in Section~\ref{sec:implementation}. Table~\ref{tbl:performance:breakdown} shows the details of the decomposition of the execution time in the case of $10^4$ points per layer. }
    \label{fig:performance:executationTime}
\end{figure}

\begin{sidewaystable}[]
    \renewcommand{\arraystretch}{1.5}
    \caption{Decomposition of CLUE execution time in the case of $10^4$ points per layer with 100 layers. The time of sub-processes on GPU is measured with NVIDIA profiler, while on CPU is measured with \texttt{std::chrono} timers in the C++ code. The uncertainties are the standard deviations of 200 trial runs of the same event (10000 trial runs if GPU). The uncertainties of sub-processes on GPU are neglectable given that the maximum and minimum kernel execution time measured by NVIDIA Profiler are very close. With respect to the single-threaded CPU, the speedup factors of the multi-threaded CPU with TBB and the GPU are given in the bracket. ``other" represents the difference between total execution time and the sum of the execution time of CLUE steps (and major CUDA API calls if GPU).}
        
    % \tiny
    \centering
    % 10000
    \resizebox{0.9\textwidth}{!}{
    \begin{tabular}{l|r@{}l|r@{}l|r@{}l}
    \hline
    CLUE Step                                 & \multicolumn{2}{c}{CPU [1T] (baseline)}         & \multicolumn{2}{c}{CPU TBB [10T]}                    & \multicolumn{2}{c}{GPU}  \\ \hline
    build fixed-grid spatial index            &  59.3 $\pm$&  ~1.6 ms       & 117.7 $\pm$&  ~6.4 ms ( 0.50x)        &   0.28 ms& ~(208.63x)       \\
    calculate local density                   & 218.4 $\pm$&  ~2.5 ms       &  33.7 $\pm$&  ~2.6 ms ( 6.48x)        &   0.51 ms& ~(430.57x)       \\
    calculate nearest-higher and separation   & 326.9 $\pm$&  ~2.9 ms       &  45.5 $\pm$&  ~2.5 ms ( 7.19x)        &   0.89 ms& ~(368.54x)       \\
    decide seeds/outliers, register followers &  54.4 $\pm$&  ~2.5 ms       & 109.4 $\pm$&  ~7.7 ms ( 0.50x)        &   0.34 ms& ~(162.38x)       \\
    expand clusters                           &  17.4 $\pm$&  ~1.5 ms       &   6.1 $\pm$&  ~1.3 ms ( 2.86x)        &   0.35 ms& ~( 49.74x)       \\ \hline
    cuda memcpy                               & --&                         & --&                                  &   2.87 ms&                   \\ 
    cuda memset                               & --&                         & --&                                  &   0.10 ms&                   \\ 
    other                                     &  29.1 $\pm$&  ~1.7 ms       &  44.9 $\pm$& ~15.7 ms                 &   1.30 ms&                  \\ \hline
    \textbf{TOTAL} (10000 points per layer)   & \textbf{705.49 $\pm$}&  ~\textbf{7.93 ms} & \textbf{357.24 $\pm$}& ~\textbf{19.68 ms ( 1.97x)} & \textbf{  6.63 $\pm$ 0.63 ms}& ~\textbf{(106.42x)}  \\
    \hline
    \end{tabular}}

    \label{tbl:performance:breakdown}
\end{sidewaystable}


% testing dataset

We tested the computational performance of CLUE using a synthetic dataset that resembles high occupancy events in high granularity calorimeters operated at HL-LHC. The dataset represents a calorimeter with 100 sensor layers. A fixed number of points on each layer are assigned a unit weight in such a way that the density represents circular clusters of energy whose magnitude decreases radially from the centre of the cluster according to a Gaussian distribution with the standard deviation, $\sigma$, set to $3$~cm.
$5$\% of the points represent noise distributed
uniformly over the layers. When clustering with CLUE, the bin size is set to $5$~cm comparable with the width of the clusters and the
algorithm parameters are set to $d_c=3 \text{ cm},\delta_o=\delta_c=5 \text{ cm},\rho_c=8$. To test CLUE's linear scalability, the number of points on each layer is incremented from 1000 to 10000 in 10 equaling steps. A total of 100 layers are input to CLUE simultaneously which simulates the proposed CMS HGCAL design~\cite{Collaboration:2293646}. Therefore the total number of points in the test ranges from $10^5$ to $10^6$. 
%%The clustering result on toy dataset is included in the Appendix \ref{app:toyDetector}.
The linear scalability of execution time are validated in Fig.~\ref{fig:performance:executationTime}.



% testing software/hardware platform
The single-threaded version of the CLUE algorithm on CPU has been implemented in C++, while the one on GPU has been implemented in C with CUDA~\cite{nvidia2011nvidia}. The multi-threaded version of CLUE on CPU uses the Thread Building Block (TBB) library~\cite{reinders2007intel} and has been implemented using the Abstraction Library for Parallel Kernel Acceleration (Alpaka)~\cite{zenker2016alpaka}. The test of the execution time is performed on an Intel Xeon Silver 4114 CPU and NVIDIA Tesla V100 GPU connected by PCIe Gen-3 link. The time of each GPU kernel and CUDA API call is measured using the NVIDIA profiler. The total execution time is averaged over 200 identical events (10000 identical events if GPU). Since CLUE is performed event by event and it is not necessary to repeat memory allocation and release for each event when running on GPU, we perform a one-time allocation of enough GPU memory before processing events and a one-time GPU memory deallocation after finishing all events. Therefore, the one-time \emph{cudaMalloc} and \emph{cudaFree} are not included in the average execution time. Such exclusion is legit because the number of events is extremely massive in high energy physics experiments and the execution time of the one-time \emph{cudaMalloc} and \emph{cudaFree} reused by each individual event is negligible.




% performance result
In Fig.~\ref{fig:performance:executationTime} (\emph{upper}), the scalability of CLUE is linear, consistent with the expectation. The execution time on the single-threaded CPU, multi-threaded CPU with TBB and GPU increases linearly with the total number of points. The stacked bars represent the decomposition of execution time. In the decomposition, unique to the GPU implementation is the latency of data transfer between host and device, which is shown as blue and green narrower bars, while common to all the three implementations are the five CLUE steps. Comparing with the single-threaded CPU, when building spatial index and deciding seeds, shown as red and pink bars, the multi-threaded CPU using TBB does not give a notable speedup due to the implementation of atomic operations in Alpaka~\cite{zenker2016alpaka} as discussed in Section~\ref{sec:implementation}, while the GPU has a prominent outperformance thanks to its larger parallelization scale. For GPU, the kernel of seed promotion in which serialization exists due to atomic appending of points in the list of seeds, does not affect the total execution time significantly if compared with other sub-processes. In the two most computing-intense steps, calculating density and separation, there are no thread conflicts or inevitable atomic operations. Therefore, both the multi-threaded CPU using TBB and the GPU provide a significant speedup. The details of the decomposition of execution time in the case of $10^4$ points per layer are listed in Table~\ref{tbl:performance:breakdown}. 

Fig.~\ref{fig:performance:executationTime} (\emph{lower}) shows the speedup factors. Compared to the single-threaded CPU, the CUDA implementation on GPU is 48-112 times faster, while the multi-threaded version using TBB via Alpaka with 10 threads on CPU is about 1.2-2.0 times faster. The speedup factors are constrained to be smaller than the number of concurrent threads because of the atomic operations. In Table~\ref{tbl:performance:breakdown}, the speedup factors of multi-threaded CPU using TBB reduce to less than $1$ in the sub-process steps of building spatial index and promoting seeds and registering followers, where atomic operations happen and bottleneck the overall speedup factor.

\section{Heterogeneous Computing of CLUE in the CMS Software Framework}
\label{sec:cmsswClue}


% The future High Luminosity LHC (HL-LHC) is expected to deliver about 5 times higher instantaneous luminosity than the present LHC, resulting in pile-up up to 200 interactions per bunch crossing (PU200). As part of the phase-II upgrade program, the CMS collaboration is developing a new end-cap calorimeter system, the High Granularity Calorimeter (HGCAL), featuring highly-segmented hexagonal silicon sensors and scintillators with more than 6 million channels. For each event, the HGCAL clustering algorithm needs to group more than $10^5$ hits into clusters. As consequence of both high pile-up and the high granularity, the HGCAL clustering algorithm is confronted with an unprecedented computing load. CLUE (CLUsters of Energy) is a fast fully-parallelizable density-based clustering algorithm, optimized for high pile-up scenarios in high granularity calorimeters. 

In this section, we present both CPU and GPU implementations of CLUE in the application of HGCAL clustering in the CMS Software framework (CMSSW). Comparing with the previous HGCAL clustering algorithm, CLUE on CPU (GPU) in CMSSW is 30x (180x) faster in processing PU200 events while outputting almost the same clustering results.


\subsection{Implementation in the CMSSW}



% image algo
The previous clustering algorithm \cite{Chen:2017btc} used in the CMS HGCAL reconstruction was based on Clustering by Fast Search and Find Density Peak (CFSFDP) \cite{rodriguez2014clustering} and exploited a KD-Tree spatial index \cite{Bentley:1975:MBS:361002.361007}. In the step of calculating local density $\rho$, KD-Tree provides a significant speedup comparing with not using any spatial index \cite{Chen:2017btc}. However, it has three crucial computing weaknesses: first, KD-Tree does not provide the optimal spatial index for HGCAL, because its window-query is of $O(n\log n)$ complexity and it is hard to construct or query on the GPUs; second, the calculation of separation $\delta$ does not take advantage of spatial index but still relies on a costly $O(n^2)$ loop; third, the expansion of clusters happens in sequential order of decreasing density, which is not only costly because of sorting but also hard to parallelize.


% clue
CLUsters of Energy (CLUE) \cite{cluepaper} is a recently-proposed parallelizable high-speed clustering algorithm. It overcomes the above three computing weaknesses and achieves an average $O(n)$ computational complexity in the applications like HGCAL where $n> k \gg m$. CLUE uses a spatial index \cite{bentley1979data} for fast querying of neighbours. Figure \ref{fig:algorithm:procedure} is a demonstration of CLUE procedure provided in \cite{cluepaper}. Both the CPU and the GPU version of CLUE, referred as CLUE-CPU and CLUE-GPU in this paper, have been implemented in CMSSW for HGCAL reconstruction. CLUE-CPU is implemented in C++, while CLUE-GPU is implemented using CUDA. Figure~\ref{fig:cmssw} shows the workflow of CLUE-GPU within CMSSW: hits are offloaded from CPU to GPU after energy calibration; then CLUE steps are carried out on GPU; in the end, the clustering results are transported back to CPU for post processing and other downstream HGCAL reconstruction related to 3D linkage of CLUE clusters. 

% \begin{figure}[ht]
%     \centering
%     \includegraphics[trim=5cm 0cm 4cm 0cm, clip,width=0.99\textwidth]{chapters/HGCal/figures/chep/Figure2.pdf}
%     \caption{Demonstration of CLUE procedure \cite{cluepaper}. The definitions of four internal variables $\{ \rho, \delta, nh, followers\}$ are also given in \cite{cluepaper}. Before the clustering procedure starts, a fixed-grid spatial index is constructed. In the first step, shown as (a), CLUE calculates the local density $\rho$ for each point. The color and size of points represent their local densities. In the second step, shown as (b), for each point CLUE calculates its nearest-higher $nh$ (defined as the nearest hit with higher density) and its separation $\delta$ (defined as the distance to $nh$). The black arrows represent the relation from the nearest-higher of a point to the point itself. If the nearest-higher of a point is -1, there is no arrow pointing to it. In the third step, shown as (c), CLUE promotes a point as a seed if $\rho,\delta$ are both large, or demote it to an outlier if $\rho$ is small and $\delta$ is large. Promoted seeds and demoted outliers are shown as stars and grey squares, respectively. In the fourth step, shown as (d), CLUE propagates the cluster indices from seeds through their chains of followers. Noise points, which are outliers and their descendant followers, are guaranteed not to receive cluster ids from any seeds. The color of points represents the cluster ids. A grey square means its cluster id is undefined and the point should be considered as noise.
%     }
%     \label{fig:algorithm:procedure}
% \end{figure}


\begin{figure}[ht]
    \centering
    \includegraphics[trim=0.5cm 0cm 0.5cm 0cm, clip,width=0.99\textwidth]{chapters/HGCal/figures/chep/CMSSWFollow.png}
    \caption{ Workflow of CLUE-GPU in CMSSW. Hits are offloaded from CPU to GPU after energy calibration. Then CLUE process are carried out on GPU. In the end, the cluster indices of all hits are transported back to CPU for post processing.}
    \label{fig:cmssw}
\end{figure}


% validate CLUE result
\begin{figure}[ht]
    \centering
    \includegraphics[trim=0cm 0cm 0cm 0cm, clip, width=0.90\textwidth]{chapters/HGCal/figures/chep/results.png}
    \caption{ Example of clustering result from previous algorithm in CMSSW\_10\_6\_X (\emph{left}) and CLUE-CPU (\emph{middle}) and CLUE-GPU (\emph{right}). The example shows a small region on the 12$^{th}$ layer of a simulated of $t\bar{t}$ event.
    }
    \label{fig:results}
\end{figure}

To validate the implementation of CLUE in CMSSW, results of CLUE-CPU and CLUE-GPU are compared with the previous clustering algorithm in CMSSW version 10.6, referred as CMSSW\_10\_6\_X. Based on the simulated $t\bar{t}$ events, CLUE-CPU and CLUE-GPU completely agree with each other, while both of them show some rare disagreements with the previous clustering algorithm implemented in CMSSW\_10\_6\_X. Such disagreements are caused by the different ordering of hits with exactly equal $\rho$ or equal $\delta$ when using different data structures, namely grid in CLUE and KD-Tree in CMSSW\_10\_6\_X. An example of clustering result is shown in Figure~\ref{fig:results}, where from left to right are results from CMSSW\_10\_6\_X, CLUE-CPU and CLUE-GPU. In this example, CLUE-CPU and CLUE-GPU provide almost the same result as the clusters in CMSSW\_10\_6\_X. However, a small notable difference is the blue cluster, which includes 4 hits in CMSSW\_10\_6\_X but 2 in CLUE. This is because the hit at about (x=60, y=106) cm is equally close to the two neighbouring hits in orange cluster and blue cluster, and its two different assignments, caused by different ordering of these two neighbors in spatial index, are equally correct. The topology of blue cluster in both cases are acceptable. Therefore, it is reasonable to conclude that CLUE in CMSSW gives almost the same clustering result as CMSSW\_10\_6\_X with neglectable differences.





\subsection{Performance in the CMSSW}


\begin{figure}[ht]
    \centering
    \includegraphics[trim=0cm 0cm 0cm 0cm, clip,width=0.99\textwidth]{chapters/HGCal/figures/chep/performance.png}
    \caption{ 
    Average execution time of HGCAL clustering for PU200 events. The testing platform is based on Intel i7-4770K CPU and NVIDIA GTX 1080 GPU. Blue, orange and green bars represent execution time of CMSSW\_10\_6\_X, CLUE-CPU and CLUE-GPU respectively. Both CMSSW\_10\_6\_X and CLUE-CPU use a single CPU thread. Three green bars are three evolving versions of CLUE-GPU and the most updated one is version 3 with 32 ms execution time, shown as the bottom-most green bar.
    }
    \label{fig:performance}
\end{figure}

The execution time of HGCAL clustering are tested using PU200 events. The testing platform is based on Intel i7-4770K CPU and NVIDIA GTX 1080 GPU. The average execution time is shown in Figure~\ref{fig:performance}, where measured time includes all clustering steps and all necessary data transfer between CPU and GPU. 

The previous clustering algorithm in CMSSW\_10\_6\_X using a single thread CPU takes 6110 ms on average. In comparison, CLUE-CPU takes only 203 ms using the same single thread CPU, producing almost the same result but 30x faster. The GPU implementation in CMSSW includes three versions. The first version is a plain CUDA implementation of CLUE-CPU and average execution time is 159 ms. The second version combines the data of all hits in the entire HGCAL as a single Structure of Array (SoA) to improve access to global memory and to allow parallelization of hits on different layers. The average execution time of the second version is reduced to 50 ms. The third version uses one-time GPU memory allocation and memory release before and after processing all events respectively. It further reduces execution time to 32 ms, which is decomposed into 6 ms for kernel execution, 20 ms for host-device data transportation and 6 ms for SoA conversion. The 6 ms total kernel execution time is comparable with that in \cite{cluepaper}. The speedup factor of CLUE-GPU over CLUE-CPU is about 6x. 

In the future, the latency due to data traffic and SoA conversion can be shared with other reconstruction processes if more processes are also offloaded to GPU. Such latency can also be partially hidden if multiple CUDA streams work on different events simultaneously to keep the GPU occupied.

% \section{TICL Trackerster reconstruction after CLUE}
\label{sec:hgcal:reco}


% The CMS endcap calorimeter upgrade for the High Luminosity LHC in 2027 uses silicon sensors to achieve radiation tolerance, with the further benefit of a very high readout granularity. Small scintillator tiles with individual SiPM readout are used in regions permitted by the radiation levels. A reconstruction framework is being developed to fully exploit the granularity and other significant features of the detector like precision timing, especially in the high pileup environment of HL-LHC. 

% An iterative clustering framework (TICL) has been put in place, and is being actively developed. The framework takes as input the clusters of energy deposited in individual calorimeter layers delivered by the CLUE algorithm, which has recently been revised and tuned. Mindful of the projected extreme pressure on computing capacity in the HL-LHC era, the algorithms are being designed with modern parallel architectures in mind. Important speedup has recently been obtained for the clustering algorithm by running it on GPUs. Machine learning techniques are being developed and integrated into the reconstruction framework. This paper will describe the approaches being considered and show first results.



% The reconstruction software in HGCAL is being developed with speed and portability in mind. The expected CPU trend in the next years will improve software performance by a factor~$ \sim3$~\cite{4}, while offline workflows and computing in CMS will require a factor~$ \sim30$, mainly driven by the reconstruction of simulated events~\cite{5}. In order to gain the missing factor~$ \sim10$ in performance, the HGCAL software reconstruction cannot rely on any other existing sub-detector software: this new detector represents a unique opportunity to exploit modern architectures and technologies. Therefore, two main solutions are being adopted to provide the needed improvements in CMS performance for Phase-2 runs: heterogeneous computing and machine learning. On the hardware side, GPUs have shown great results in terms of speedup in recent years and the use of hybrid architectures is spreading in many fields of science. In addition, NVIDIA GPUs can be programmed with CUDA (Compute Unified Device Architecture), a parallel computing platform and programming model designed to work with programming languages such as C, C++, Fortran and Python, allowing to easily accelerate compute intensive portions of the applications~\cite{6}. On the software side, machine learning models are largely exploited to accomplish an enormous variety of tasks and can provide better results than traditional methods in many cases. Furthermore, some machine learning algorithms (e.g. Convolutional Neural Networks) can be executed on GPU, thus reducing both training and inference times, thanks to frameworks like Tensorflow~\cite{7} and Keras~\cite{8}.

% \subsection{TICL: The Iterative CLustering}
\noindent The development of the HGCAL reconstruction software is driven by three main concepts:

\begin{itemize}
    \item Particles deposit energy and create \emph{RecHits};
    \item \emph{RecHits} on each layer are clustered together to form \emph{LayerClusters} (2D objects);
    \item \emph{LayerClusters} are linked together to form \emph{Tracksters}, collections of \emph{LayerClusters}.
\end{itemize}

% In order to build the reconstruction chain exploiting the full potential of HGCAL, a modular framework has been developed and is constantly evolving. TICL (The Iterative CLustering)~\cite{9} modules and interfaces are defined such that new developers don't need a deep knowledge of the official CMS software core framework (CMSSW) and can easily contribute. In addition, its flexibility and modularity allow users to test their own algorithms and compare performances, as they can be plugged on top of the framework without applying any strong modification to the existing workflow. 

\noindent For \emph{LayerClusters}, the CLUE algorithm and its CMSSW application have been discussed in Section~\ref{sec:clue} and \ref{sec:cmsswClue}. Based on the \emph{LayerClusters} from CLUE, TICL framework is designed to link the \emph{LayerClusters} together to construct \emph{Tracksters} that representing particle flow candidates.
The design of TICL framework is shown in Figure~\ref{fig:ticl}. This section focuses on the two main aspects of TICL reconstruction. First, a typical TICL iteration that links \emph{LayerClusters} together to produce \emph{Tracksters} will be described in Section~\ref{sec:iter}.  Second, the preliminary results of particle identification and energy regression performed on a \emph{Trackster} with a Convolutional Neural Network are shown in Section~\ref{sec:pid}.

\begin{figure}[tbp]
    \centering % \begin{center}/\end{center} takes some additional vertical space
    \includegraphics[width=.9\textwidth]{chapters/HGCal/figures/chef/ticl.pdf} 
    \caption{\label{fig:ticl} Design of TICL framework. Arrows show connections between different parts of the reconstruction, pointing from the output of a step to the input of a next step. Double pointing arrows indicate a strong connection between two parts of the reconstruction. \emph{Tracks} and \emph{Timing} (orange cells) are information coming from other subdetectors (respectively Tracker and MTD)~\cite{ticlwebsite}.}
\end{figure}


\subsection{Iterations}
\label{sec:iter}

\emph{Tracksters} produced by TICL iterations are suitable for representing real final state particles. In order to guarantee such a correspondence, it's necessary to build a reconstruction workflow that takes into account the specific physics process that different particles undergo into the detector. Four different types of iteration exist now in TICL: \emph{track-seeded} (collects information from the Tracker to get the entry point and momentum direction of charged particles at the front face of the HGCAL detector), \emph{MIP} (\emph{Minimum Ionizing Particle}), \emph{electromagnetic} and \emph{hadronic}. The structure of a TICL iteration is shown in Figure~\ref{fig:iter}. A seeding region is first defined as a window in the [$\eta$, $\phi$] space on a certain layer and a pattern recognition algorithm is applied to all the available \emph{LayerClusters} within the seeding region. Then, linking, cleaning and classification tasks follow, exploiting timing information when possible. At the end of each iteration, the \emph{Tracksters} are required to pass quality criteria and particle identification: all the \emph{LayerClusters} belonging to the selected \emph{Tracksters} are masked out and are not available for the next iteration.

\begin{figure}[tbp]
    \centering % \begin{center}/\end{center} takes some additional vertical space
    \includegraphics[width=.4\textwidth]{chapters/HGCal/figures/chef/iter}
    \qquad
    \includegraphics[width=.4\textwidth]{chapters/HGCal/figures/chef/pralgo}
    % "\includegraphics" from the "graphicx" permits to crop (trim+clip)
    % and rotate (angle) and image (and much more)
    \caption{\label{fig:iter} (\emph{Left}) Structure of a typical iteration in TICL. (\emph{Right}) Scheme of \emph{Cellular Automaton} pattern recognition implemented in TICL~\cite{ticlwebsite}.}
\end{figure}

Thanks to the modular design of the framework, several pattern recognition algorithms can be tested to produce the best results. Currently, a \emph{Cellular Automaton} algorithm is being used based on the experience with CMS Track reconstruction~\cite{Funke:2014dga}. It consists of six main steps:

\begin{enumerate}
\itemsep0em
    \item Start from a Layer\textsubscript{N} and consider a specific \emph{LayerCluster}
    \item Open a window in the [$\eta$, $\phi$] space around it and project it onto the next Layer\textsubscript{N+1}
    \item Consider all the \emph{LayerClusters} inside this region and try to establish a \emph{Doublet} connection between the \emph{LayerClusters} on the two adjacent layers
    \item Apply some compatibility criteria to decide if the \emph{LayerClusters} should be linked or not (i.e. geometry constraints, energy, timing compatibility, etc.)
    \item Repeat this same procedure for all the \emph{LayerClusters} on Layer\textsubscript{N}
    \item Repeat this same procedure for all pairs of contiguous layers [Layer\textsubscript{K}, Layer\textsubscript{K+1}]
\end{enumerate}

At the end of the process, pairs of \emph{LayerClusters} will be connected into \emph{Doublets}. Consecutive \emph{Doublets} (i.e. doublets that share the ``middle'' \emph{LayerCluster}) will be linked together if configurable alignment requirements are satisfied. The set of all connected \emph{Doublets} will form a direct acyclic graph that serves as building block for a \emph{Trackster}. Note that this procedure can be properly configured to allow missing consecutive \emph{LayerClusters} and establish links between non-adjacent layers (e.g. in \emph{MIP} iteration). 





\subsection{Particle identification and energy regression}
\label{sec:pid}
The final purpose of the TICL is to reconstruct particle-flow objects with energies and probabilities of particle identification. To accomplish this task, some preliminary studies were conducted with a single particle produced in front of HGCAL in events where no pile-up is simulated. The momentum of the particle pointed from the vertex (0,0,0), center of the CMS detector. Events were simulated in such a way that particle showers (in case of electrons, photons and charged hadrons) could be fully contained inside the detector.

Since \emph{Tracksters} are suitable for representing real physics objects, a Convolutional Neural Network was designed to perform particle identification and energy regression on \emph{Tracksters} built by TICL electromagnetic iterations. In order to have fixed-sized inputs to the network, each \emph{Trackster} is represented as an image $50 \times 10 \times 3$, where the dimensions represent respectively the number of HGCAL layers per endcap, the maximum number of \emph{LayerClusters} on each layer and the number of features (energy, $\eta$, $\phi$). In this representation, each pixel of the image corresponds to a \emph{LayerCluster} that belongs to the \emph{Trackster}. Furthermore, \emph{LayerClusters} on each layer have been sorted by decreasing energy, applying a zero-padding whenever a layer featured less than 10 clusters, while removing some low energy \emph{LayerClusters} in layers with more than 10.

\begin{figure}[t]
    \centering
    \includegraphics[width=.4\textwidth]{chapters/HGCal/figures/chef/pid}
    \caption{\label{fig:pid} Confusion matrix showing the performance of particle identification.}
\end{figure}

\begin{figure}[h]
    \centering
    \includegraphics[width=.99\textwidth]{chapters/HGCal/figures/chef/er}
    \caption{\label{fig:er} Energy regression preliminary results for photons, electrons and charged hadrons.}
\end{figure}

A preliminary performance study was conducted on a 4-classes model (electron, photon, muon, charged hadron). The dataset consisted of 40 thousand events (10 thousand per particle type): $80\%$ has been used for training, $10\%$ for validation and the remaining $10\%$ for testing. The CNN was trained for 15 epochs (passes of the algorithm through the entire dataset), using the sum of cross-entropy and mean squared error as loss function to account for particle ID and energy regression, respectively. In order to have the value of the two functions of the same order of magnitude during training, the energies of the \emph{Tracksters} were normalized with respect to the data sample. The CNN was trained with Tensorflow~\cite{tensorflow2015-whitepaper}. Results are shown in Figure~\ref{fig:pid} and Figure~\ref{fig:er}. The confusion between electrons and photons is expected to be solved in future by exploiting information coming from the Tracker. The energy regression must be fine tuned and improvements are needed especially in the case of charged hadrons, since part of the shower is reconstructed by hadronic iteration. More classes will be added and the evolution of TICL is expected to improve the performance of this task.






    \chapter{Conclusion}
\label{sec:conclusion}


In this thesis, a precision measurement of the \PW leptonic and inclusive hadronic branching fraction has been performed using the LHC $\sqrt{s}=13\TeV$ p-p collision data collected by the CMS detector during run 2016. 
Dataset are triggered with single electron and single muon trigger. The final states corresponding the topology of leptonic and semileptonic \ttbar are selected. 
The selected sample is split into seven mutually-exclusive channels based on the multiplicity of electron, muon and hadronic tau, requiring at least one muon or one electron that enables trigger.
For each channel, further partition relying on the jet and b-tag multiplicities is designed to separate regions with different signal purity.

Two approaches are used for the measurement. The shape analysis template fit the \pt distribution of the sensitive leptons in different channels and $n_j n_b$ categories. It exploits the $WW$ and $w+jets$ region for more \PW statistics, and $Z+jets$ region for controlling the systematics related to the identification of hadronic tau. The counting analysis constructs ratios of yields for channels with the same trigger and analytically solve three leptonic branching fractions from a set of quadratic equations. It eliminates the shape information and uses only the \ttbar concentrated regions. It is designed to cross-check the shape analysis. 


From shape analysis, the $W\to e\nu$, $W\to \mu\nu$, $W\to \tau\nu$ branching fraction and \PW total hadronic branching fraction are $10.83(10)\%$, $10.94(08)\%$, $10.77(21)\%$ and $67.46(28)\%$, respectively.
From counting analysis, $11.15(27)\%$, $11.13(22)\%$, $10.63(65)\%$ and $67.08(72)\%$, respectively. The value from the two approaches consistent with each other in one sigma. The shape analysis is about 3 times more precise than the counting analysis, owing to the sensitivity from the \pt spectrum, controlling the $\tau_h$ systematics and wider selection regions embracing $WW$ and $W$+Jets events. 

Based on the more precise shape analysis, the ratios between pair-wised leptonic channels are calculated. Assuming universality between electron and muon, the ratio between tauonic branching fraction and the average of electronic and muonic branching fraction is determined as
\begin{equation*}
    R_{\tau/(e,\mu)} = 2 B^W_\tau /(B^W_e +  B^W_\mu) = 1.002\pm0.019,
\end{equation*}

\noindent consistent with the SM lepton universality. This solves the LEP's tension with the SM which is undetermined for more than a decade. Assuming lepton universality among three generations, the leptonic and total hadronic branching fraction is estimated as $10.89(08)\%$ and $67.32(23)\%$ respectively, leading to the ratio of \PW total leptonic and total hadronic $R^W_{h/l}=2.060 \pm 0.021$. With 
$
    R^W_{\rm h/l} = \frac{B_h}{1-B_h} = 
    \left[1 + \frac{\alpha_{S}(M^{2}_{\mathrm{W}})}{\pi}\right]
    \sum_{\substack{i = (\mathrm{u,c}), \\ j=(\mathrm{d, s, b})}}
    \left|\rm V_{ij}\right|^{2}
$, 
three standard model quantities are subsequently derived: the sum square of elements in the first two rows of the Cabibbo--Kobayashi--Maskawa (CKM) matrix  $\sum{\left|V_{ij}\right|^{2}} = 1.991 \pm 0.019$, the CKM element $V_{cs} = 0.970 \pm 0.008$, and the strong coupling constant at the W mass pole $\alpha_{S}(m_\mathrm{W}) = 0.099 \pm 0.026$.


This CMS measurement of \PW branching fraction successfully improves the experimental precision from LEP. In addition, by a good agreement with SM LU, it clearly responds to the LEP's tension with the SM LU undetermined for more than a decade.
    
    
    %%%%%%%%%%%%%%%%%%%%%%%%%%%
    % some ending stuff       %
    %%%%%%%%%%%%%%%%%%%%%%%%%%%
    \begin{appendix}
        

\chapter{An Outline of the Quantum Field Theory of the Standard Model}
\label{sec:physics:qft}


Quantum Field Theory combines special relativity and quantum mechanics to describe the Lorentz invariant rules for fields representing particles and forces. The foundation of QFT is the principle of least action.
\begin{equation}
    \delta s = \delta \int \mathcal{L}(\phi, \partial_\mu \phi) dx = 0,
    \label{eqn:physics:qft:leastAction}
\end{equation}

\noindent which leads to the Euler-Lagrange equation of motion for the fields:
\begin{equation}
    \frac{\partial}{\partial x_\mu} \bigg( \frac{\partial \mathcal{L}}{\partial(\partial \phi / \partial x_\mu)}\bigg) - \frac{\partial \mathcal{L}}{ \partial \phi} = 0,
    \label{eqn:physics:qft:lagrangeEoM}
\end{equation}

\noindent where $\mathcal{L}(\phi, \partial_\mu \phi)$ is the Lagrangian of the quantum fields. In such a framework, the behaviors of the quantum fields are then fully dictated by their Lagrangian via the Euler-Lagrangian Equation of Motion in Equation~\ref{eqn:physics:qft:lagrangeEoM}. Therefore, this allows us to encode our understanding of the dynamics and interactions of field into the Lagrangian. For example, Klein-Gordon Equations and Dirac Equations, which are two versions of the generalization of Schrodinger Equation in the special relativity domain, can be derived from the Lagrangian of the massive scalar field $\phi$ and massive spinor filed $\psi$, as shown in Equation~\ref{eqn:physics:qft:kleinGorden} and \ref{eqn:physics:qft:dirac}. Another example is the electromagnetic field. Based on the Gauss's Law $\nabla \cdot \vec{E} = 0$ and $\nabla \cdot \vec{B} = 0$, Faraday's law of induction $\nabla \times \vec{E} = - \partial_t \vec{B}$ and Ampere's circuital law $\nabla \times \vec{B} = (1/c) \partial_t \vec{E}$, the Maxwell equations for electromagnetic field in the free space can be achieved by defining a Lagrangian of a massless vector field $A_\mu$, as shown in Equation~\ref{eqn:physics:qft:maxwell}.
\begin{align}
    \text{scalar:} \,
    \mathcal{L}_\phi &= \frac{1}{2} \partial_\mu\phi \partial^\mu \phi - \frac{1}{2} m^2 \phi^2 
        &\Longrightarrow& \;  \text{ Klein-Gordon Equation:} \, \partial_\mu \partial^\mu \phi + m^2 \phi = 0 \label{eqn:physics:qft:kleinGorden}\\
    \text{spinor:} \,
    \mathcal{L}_\psi &= i \bar{\psi}\gamma^\mu \partial_\mu \psi - m \bar{\psi} \psi 
        &\Longrightarrow& \; \text{ Dirac Equation:}  \, (i\gamma^\mu\partial_\mu - m) \psi = 0 \label{eqn:physics:qft:dirac} \\
    \text{vector:} \,
    \mathcal{L}_A &= -\frac{1}{4}F_{\mu\nu} F^{\mu \nu} 
        &\Longrightarrow& \; \text{ Maxwell Equations:} \, \partial_\mu \partial^\mu A_\mu = 0 \label{eqn:physics:qft:maxwell}
\end{align}


The first well-established quantum field theory is Quantum Electrodynamics (QED). Its formulation in the early 20th century was a joint effort from many great physicists such as Paul Dirac, Wolfgang Pauli, Werner Heisenberg, and Enrico Fermi. Since its establishment, QED was able to successfully explain many atomic phenomena that involve photon and charged particles, such as spontaneous photon emission in the atoms. However, in the late 1930s, physicists realized that QED calculation could diverge during the next-to-leading order. This causes skepticism to QED to make meaningful predictions when involving loops. This problem was solved by proving that the divergence in the fermion vacuum polarization and interaction vertices can exactly cancel out each other. Therefore QED is renormalizable. 1965 Nobel Prize in Physics was awarded to Shinichiro Tomonaga, Julian Schwinger, and Richard Feynman for their contributions to the QED renormalization. SM is an extension based on QED, which extends the gauge symmetry in QED from $U(1)$ to $U(1)_Y \times SU(2)_L \times SU(3)_c$, including electroweak and strong force. Several milestones along the establishment of SM involves

\begin{itemize}
    \item \textbf{Yang-Mills Theory.} In 1954, Chen-Ning Yang and Robert Mills described the gauge theory for the non-abelian group \cite{PhysRev.96.191}. It serves as one of the most impotent theoretical frameworks for SM. Based on Yang-Mills Theory, SM QCD is developed and Electroweak force is unified. 
    
    \item \textbf{Higgs mechanism.} In 1964, the Higgs mechanism to generate the mass of gauge field via spontaneous symmetry breaking was proposed by three independent groups: Robert Brout and François Englert \cite{PhysRevLett.13.321}; by Peter Higgs \cite{PhysRevLett.13.508}; and by Gerald Guralnik, C. R. Hagen, and Tom Kibble \cite{PhysRevLett.13.585}. 
    
    \item \textbf{GWS Model.} In 1961 Gheldon Glashow combines the electromagnetic and weak force based on Yang-Mills gauge field \cite{Glashow:1961tr}; then in 1967, Steven Weinberg and Abdus Salam incorporated the Higgs Mechanism into Glashow's electroweak theory \cite{PhysRevLett.19.1264}. In 1972, the massive Yang-mills gauge fields with gauge boson mass generated by the Higgs mechanism is proven renormalizable by Gerard 't Hooft and Martinus Veltman \cite{tHooft:1972tcz}. In 1973, CKM matrix was added to GWS theory to allow quark mixing and CP violation.
    
    \item \textbf{Quark Model and QCD.} In 1964, the quark model was proposed by Murray Gell-Mann in order to classify a increasing number of newly discovered mesons and baryons. Quark model soon obtained supports from experiments, such as deep inelastic scattering experiments at SLAC starting in 1969, the discoveries of $J/\psi (c\bar{c})$ at Brookhaven National Laboratory and SLAC in 1974, Ubsilon $\Upsilon (b\bar{b})$ at Fermilab in 1977, and top quark at Fermilab in 1995.  In 1973, asymptotic freedom was proposed by David Gross and Frank Wilczek, and independently by David Politzer, to explain the quark confinement: strong interaction allows perturbation calculation at high energy while confined at low energy.
\end{itemize}

This section gives a brief summery of SM skeleton in aspect of the Yang-Mills theory, higgs mechanism, GWS Theory, Quark asymptotic freedom in QCD. Before starting, we may remind ourselves the final form of the SM Lagrangian:

\begin{equation}
\begin{split}
    \mathcal{L}_{U(1)\times SU(2) \times SU(3)} =&   - \frac{1}{4}B_{\mu\nu}B^{\mu\nu} - \frac{1}{4}W^a_{\mu\nu}W^{\mu\nu}_a - \frac{1}{4}G^a_{\mu\nu}G^{\mu\nu}_a\\
    & + \bar{\chi}_L \gamma^\mu \big( i \partial_\mu -g \frac{\tau_a}{2} W^a_\mu -g'\frac{Y}{2} B_\mu \big) \chi_L 
    + \bar{\psi}_R \gamma^\mu \big( i \partial_\mu -g'\frac{Y}{2} B_\mu \big) \psi_R 
    - g_s (\bar{q}\gamma^\mu  T_{a} q) G_\mu^a \\
    & + \left\lvert  \big( i \partial_\mu -g \frac{\tau_a}{2} W^a_\mu -g'\frac{Y}{2} B_\mu \big)\phi \right\rvert ^2 - V(\phi) \\
    & -(y_1 \bar{\chi}_L \phi \psi_R + y_2 \bar{\chi}_L \phi_c \psi_R + \text{hermitian conjugate}),
\end{split}
\label{eqn:physics:qft:smLagrangian} 
\end{equation}

\noindent where the first to the forth row represents the gauge sector, fermion sector, higgs sector and fermion mass sector respectively. This total SM Lagrangian functions as an indexing map of the discussion in this section. One of the beauty of SM is its minimality. We can count the number of free parameters in the SM Lagrangian. Thanks to the symmetries imposed in the SM, only 18 basic free parameters to begin with are needed in the model.
\begin{itemize}
    \item 3 gauge coupling strength for hypercharge, isospin and color. $g$, $g'$, $g_s$.
    \item 2 parameters $\mu$ and $\lambda$ in the higgs potential field $V(\phi)=\frac{1}{2} \mu^2 \bar{\phi}\phi + \frac{1}{4} \lambda(\bar{\phi}\phi )^2 $
    \item 9 Yukawa couplings between higgs and 9 charged fermions: $y_e$,$y_\mu$,$y_\tau$,$y_u$,$y_d$,$y_c$,$y_s$,$y_t$,$y_b$
    \item 4 parameters in the CKM matrix, 3 Euler angles $\theta_{12}$, $\theta_{23}$, $\theta_{23}$ and CP violating phase $\delta$.
\end{itemize}

The values of these free SM parameters are determined from the experiments. In addition to the above 18 basic parameters, neutrino oscillation indicates neutrinos are not massless and their flavor eigenstates are a mixing of the mass eigenstates. Accordingly, three additional Yukawa couplings are needed for neutrino mass. Analogical to the CKM matrix for the quark mixing, the neutrino mixing is described by PMNS matrix which has 4 free parameters corresponding to three rotation angles and a CP violation phase. Moreover, the CP violation in QCD could also be allowed by adding an extra parameter $\delta_{CP}$. However, in the experiment, the QCD CP violation is not observed in contrast with the considerable CP violation observed in the weak interaction. This CP conservation in QCD is often referred to as "strong CP problem". So $\delta_{CP}$ could be treated as a free SM parameter with a very small value yet to be measured.



%  The Standard Model of particle physics is is a quantum field theory which is gauge sysmmetric. In 1920s, Dirac 


% With a set of amazing theoretical achievements and experimental discoveries since 1900s, we come to the modern understanding of particles physics, the Standard Model. The milestone achievements alone the path includes but not limited to the Yang-Mills theory, the spontaneous symmetry breaking and higgs mechanism, the GWS electroweak theory, the quark model, the renormalization of EW and QCD.  the P violation in Weak interaction, the CP violation in K meson decay, the discovery on WZ boson at CERN, the ,  Perhaps, the SM is the most ultimate answer the humanity have devised so far to the question of "what is the world made of". 







\section{Yang-Mills Gauge Theory}
\label{sec:physics:qft:gaugeSymmetry}

Gauge theory is a type of quantum field theory, the Lagrangian of which is invariant under local phase transformations or gauge transformations. The term ``gauge" should be understood as the regularization of the redundant degrees of freedom in the Lagrangian. The transformations between different gauges form a Lie group, which characterizes this gauge theory. Yang-Mills gauge theory is the gauge theory for the non-abelian Lie groups. (``abelian" or ``non-abelian" tells whether two gauge transformations in the group are commutable or not.) The standard model is built based on the Yang-Mills gauge theory. But why does SM have to respect the gauge symmetry? The primordial reason is that by enhancing the global phase symmetry to local phase symmetry, we can introduce massless gauge bosons and consequently obtain the couplings between the fermions and the gauge boson. Another benefit anchors in the renormalization. A QFT is useful only if it is renormalizable to make finite meaningful predictions. And gauge theory is proven renormalizable. Besides obtaining force and renormalization, a relatively modern understanding of gauge symmetry is that it is not a symmetry in nature but an artificial consequence of the redundant degree of freedom in the theory. According to Noether's theorem, a nature symmetry corresponds to a conservation law. For example, the global gauge symmetry of QED gives rise to the conservation of electric charge. But the local symmetry in the QED, which is related to the redundant degree of freedom in the mathematical description of the photo polarization, does not lead to any corresponding conserved current. The thinking about the essence of the gauge symmetry is nicely presented in Schwartz's book, \textit{Quantum Field Theory and the Standard Model} \cite{schwartz2014quantum}. Here provide a description of the gauge symmetry of the abelian U(1) group and the non-abelian SU(2), SU(3) groups, which is crucial for SM electroweak and QCD, as well as beyond SM models for lepton non-universality in Section~\ref{sec:physics:bsm}.


\subsection{U(1) Gauge Symmetry}
The Lagrangian of a U(1) gauge theory with spinnor field $\psi$ and the associated gauge vector field $B_\mu$ is
% 
\begin{equation}
    \mathcal{L}_{U(1)} = i\bar{\psi}\gamma^\mu D_\mu \psi  - m\bar{\psi} \psi  - \frac{1}{4}B_{\mu\nu}B^{\mu\nu},
    \label{eqn:physics:qft:u1Lagrangian}
\end{equation}

\noindent where the covariant derivative $D_\mu$ and covariant field strength tensor $B_{\mu\nu}$ defined as
% 
\begin{equation}
    D_\mu \equiv \partial_\mu +i g' \frac{Y}{2} B_\mu , \;\;\; 
    B_{\mu\nu} \equiv  \partial_\mu B_\nu - \partial_\nu B_\mu.
\end{equation}

\noindent $\mathcal{L}_{U(1)}$ is invariant under U(1) local transformation
% 
\begin{equation}
	\psi \longmapsto e^{i\alpha(x_\mu)} \psi ,\;\;\; 
	B_\mu  \longmapsto  B_\mu - \frac{1}{g'\frac{Y}{2}}\partial_\mu \alpha(x_\mu).
\end{equation}

\noindent The interaction between the U(1) charge current $j^\mu \equiv g' \bar{\psi}\gamma^\mu \psi$ and the gauge field $B_\mu$ is $-j^\mu A_\mu$ and is embedded in the covariant derivative $D_\mu$. The QED is a U(1) Gauge theory. So the QED Lagrangian takes the form of equation~\ref{eqn:physics:qft:u1Lagrangian}. If we choose $g'\frac{Y}{2} = -e$ and use the conventional notation $A_\mu$ for the QED gauge field, equation~\ref{eqn:physics:qft:u1Lagrangian} becomes the common form of QED Lagrangian.


\subsection{SU(2) Gauge Symmetry}

The three generators of SU(2) group $T_a$ with $a \in {1,2,3 }$ are usually represented by the half of Pauli Matrices $T_a = \frac{1}{2} \tau_a = \frac{1}{2} \sigma_a$, where the Pauli Matrices are 
%
\begin{equation}
    \sigma_1 = \begin{bmatrix} 0 & 1 \\ 1 & 0\end{bmatrix}, \;\;\; 
    \sigma_2 = \begin{bmatrix} 0 & -i \\ i & 0\end{bmatrix}, \;\;\; 
    \sigma_3 = \begin{bmatrix} 1 & 0 \\ 0 & -1\end{bmatrix}.
\end{equation}

\noindent The commutation relation of the group generators can be represented as $[T_a, T_b] = i f_{abc} T_c$, where $f_{abc}$ is the structure constant of the group. For SU(2) group, the structure constant is the Levi-Civita symbol $f_{abc}=\epsilon_{abc}$. In SM, the left-handed neutrino and charged lepton in the same generation form a doublets described by $SU(2)$ group. The same scenario is for the up and down type left-handed quark in the same generation. The higgs doublet in the SM also transforms as a global $SU(2)$ group. Other than the applications in the SM, SU(2) group is also useful in the description of nucleons with proton-neutron doublet $[n, p]$. 

\noindent Now we consider a SU(2) gauge theory. Suppose there are two spinor fields $\psi_1$ and $\psi_2$, which compose a spinor doublet $\chi = [ \psi_1, \psi_2 ]^T$. the Lagrangian of the spinnor doublet $\chi$ and the gauge vector triplet $W^a$ is
%
\begin{equation}
\begin{split}
    \mathcal{L}_{SU(2)}  &= (i\bar{\psi}_1\gamma^\mu D_\mu \psi_1  - m_1\bar{\psi_1} \psi_1) + (i\bar{\psi}_2\gamma^\mu D_\mu \psi_2  - m_2\bar{\psi_2} \psi_2)  - \frac{1}{4}W^a_{\mu\nu}W^{\mu\nu}_a \\
    &= i\bar{\chi}\gamma^\mu D_\mu \chi  - m\bar{\chi} \chi  - \frac{1}{4}W^a_{\mu\nu}W^{\mu\nu}_a,
\end{split}
\label{eqn:physics:qft:su2Lagrangian}
\end{equation}

\noindent where the covariant derivative $D_\mu$ and covariant field strength tensor $W^a_{\mu\nu}$ defined as
%
\begin{equation}
    D_\mu \equiv \partial_\mu +i g \frac{\tau_a}{2} W^a_\mu , \;\;\; 
    W^a_{\mu\nu} \equiv  \partial_\mu W^a_\nu - \partial_\nu W^a_\mu - g f_{abc} W^b_\mu W^c_\nu
\end{equation}

\noindent $\mathcal{L}_{SU(2)}$ is invariant under SU(2) local transformation 
%
\begin{equation}
	\chi \longmapsto  e^{i\alpha^a (x_\mu) \frac{\tau_a}{2}} \chi , \;\;\; 
    W^a_\mu \longmapsto  W^a_\mu - \frac{1}{g}\partial_\mu \alpha^a(x_\mu) - f_{abc}\alpha^b(x_\mu) W^c_\mu 
\end{equation}

\noindent Because SU(2) group is non-abelian, the nontrivial group structure $f_{abc}$ has its contribution to the covariant field strength tensor $W^a_{\mu\nu}$. This leads to the fact that the term $\frac{1}{4}W^a_{\mu\nu}W^{\mu\nu}_a$ in the equation~\ref{eqn:physics:qft:su2Lagrangian} not only involves the kinetic energy of the gauge field but also includes $WWW$ and $WWWW$ terms representing three points and four-point self-interaction of the gauge field. The gauge boson can self interact when gauge symmetry is imposed on a non-abelian group, a unique feature of the non-abelian gauge theories. As we will see in the GWS theory in Section~\ref{sec:physics:qft:gws}, this leads to the Triple-Gauge-Coupling (TGC) and Quatic-gauge-coupling (QGC) in the SM.




\subsection{SU(3) Gauge Symmetry}
The eight generators of SU(3) group $T_a$ with $a \in {1,2,\dots 8}$ are usually represented by the half of Gell-mann Matrices $T_a = \frac{1}{2} \lambda_a$, where the Gell-mann Matrices are 
%
\begin{equation}
\begin{split}
    \lambda_1 &= \begin{bmatrix} 0 & 1 & 0 \\ 1 & 0 & 0 \\ 0 & 0 & 0\end{bmatrix}, \;\;\; 
    \lambda_2 = \begin{bmatrix} 0 &-i & 0 \\ i & 0 & 0 \\ 0 & 0 & 0\end{bmatrix}, \;\;\; 
    \lambda_3 = \begin{bmatrix} 1 & 0 & 0 \\ 0 &-1 & 0 \\ 0 & 0 & 0\end{bmatrix}, \;\;\; 
    \lambda_8 = \begin{bmatrix} 1 & 0 & 0 \\ 0 & 1 & 0 \\ 0 & 0 &-2\end{bmatrix}, \\
    %
    \lambda_4 &= \begin{bmatrix} 0 & 0 & 1 \\ 0 & 0 & 0 \\ 1 & 0 & 0\end{bmatrix}, \;\;\; 
    \lambda_5 = \begin{bmatrix} 0 & 0 &-i \\ 0 & 0 & 0 \\ i & 0 & 0\end{bmatrix}, \;\;\; 
    \lambda_6 = \begin{bmatrix} 0 & 0 & 0 \\ 0 & 0 & 1 \\ 0 & 1 & 0\end{bmatrix}, \;\;\; 
    \lambda_7 = \begin{bmatrix} 0 & 0 & 0 \\ 0 & 0 &-i \\ 0 & i & 0\end{bmatrix}.
\end{split}
\end{equation}

\noindent The SU(3) group has non-trivial structure constants which are $f_{123}=1, f_{147}= f_{246}=f_{257}= f_{345}= f_{516}= f_{637}=\frac{1}{2}, f_{458} = f_{678}=\frac{\sqrt{3}}{2}$. Besides the structure constant $f_{abc}$, there are also three useful group constants $T_R, C_F, C_A$ defined as below, with their values for SU(3) group on the right side
%
\begin{align}
	Tr(T^aT^b)=T_R \delta^{ab}  &\longrightarrow  T_R^{SU(3)}  = \frac{1}{2} \\
    T_a^{i,k}T^a_{k,j} = C_F \delta_{ij}  &\longrightarrow  C_F^{SU(3)} = \frac{N_c^2-1}{2N_c} = \frac{4}{3} \\
    f_{acd}f^{bcd} = C_A \delta_{ab} &\longrightarrow C_A^{SU(3)}  =N_c = 3
\end{align}

\noindent where $N_c$ is the number of charges or colors. These constants often appear in the calculation of the renormalization of the group. In the SM, SU(3) group is used to describe the triplet of three colors $r,g,b$ in the QCD. In addition to the application in SM, SU(3) group is also useful to describe light mesons and baryons which consist of $[u,d,s]$ quarks. For light mesons, two light quarks form $SU(3) \times SU(3)$ group, while for light baryon, three light quarks form $SU(3) \times SU(3) \times SU(3)$ group.


\noindent Now we consider SU(3) gauge theory. Suppose there are three spinor fields $\psi_r$, $\psi_g$, $\psi_b$, which compose a spinor triplet $q = [ \psi_r, \psi_g, \psi_b ]^T$. Lagrangian of the spinnor triplet $q$ and the gauge field octolet $G^a$ is
\begin{equation}
\begin{split}
    \mathcal{L}_{SU(3)}  &= \sum_{k \in \{r,g,b\}} \big( i\bar{\psi_k}\gamma^\mu D_\mu \psi_k  - m\bar{\psi_k} \psi_k \big) - \frac{1}{4}G^a_{\mu\nu}G^{\mu\nu}_a \\
    &= i\bar{q}\gamma^\mu D_\mu q  - m\bar{q} q - \frac{1}{4}G^a_{\mu\nu}G^{\mu\nu}_a
\end{split}
\label{eqn:physics:qft:su3Lagrangian}
\end{equation}

\noindent where the covariant derivative $D_\mu$ and covariant field strength tensor $G^a_{\mu\nu}$ defined as
\begin{equation}
    D_\mu \equiv \partial_\mu +i g T_a G^a_\mu , \;\;\; 
    G^a_{\mu\nu} \equiv \partial_\mu G^a_\nu - \partial_\nu G^a_\mu - g_s f_{abc} G^b_\mu G^c_\nu
    \label{eqn:physics:qft:su3Covariant}
\end{equation}

\noindent $\mathcal{L}_{SU(3)}$ is invariant under SU(3) local transformation 
\begin{equation}
	q \longmapsto  e^{i\alpha_a (x_\mu) T^a} q , \;\;\; 
    G^a_\mu \longmapsto  G^a_\mu - \frac{1}{g''}\partial_\mu \alpha^a(x_\mu) - f_{abc}\alpha^b(x_\mu) G^c_\mu 
\end{equation}



\noindent The same as the $SU(2)$  scenario, the covariant field strength tensor $G^a_{\mu\nu}$ in Equation~\ref{eqn:physics:qft:su3Covariant}  has contributions from the non-trivial group structure $f_{abc}$ of $SU(3)$ group, so the kinematic term of gauge field $\frac{1}{4}G^a_{\mu\nu}G^{\mu\nu}_a$ in the Lagrangian in Equation~\ref{eqn:physics:qft:su3Lagrangian} not only involves the kinematic energy of the gauge field but also include $GGG$ and $GGGG$ terms representing the three-point and four-point self-interaction of the gauge field. For QCD in SM, gluon's self-interaction leads to many unique phenomenologies in the strong interactions, such as quark confinement, evolution of the parton distribution functions and final state ration, which will be discussed in Section~\ref{sec:physics:qft:qcd}.





\section{The Higgs Mechanism}
\label{sec:physics:qft:higgsMechanism}
One might notice that the gauge fields discussed above in Section~\ref{sec:physics:qft:gaugeSymmetry} are all massless: the Lagrangians do not have any terms for the gauge fields' mass because directly adding such mass terms breaks the gauge symmetry. The massless gauge field does not cause problems in the QED and QCD, where photon and gluon are indeed massless. But for weak interaction, it is known that weak force is short-range, and thus the weak bosons must be massive. But if we naively add a mass term for the weak boson by hand, e.g. $m W_\mu W^\mu$, and give up the gauge symmetry, we will come across divergence in the loop integrals related to the propagator and end up with an un-renormalizable theory failing to make any meaningful predictions at high energy scale. The way to get around is the ``higgs mechanism" which generates mass for gauge bosons via spontaneous symmetry breaking while maintaining the gauge symmetry. It first introduces a scalar field $\phi$ with spontaneously-broken global symmetry. $\phi$ has gauge charge, and thus couples via the covariant derivative with the gauge filed that desires mass. Eventually, it is the gauge covariant derivatives of a spontaneously-broken $\phi$ that provides the mass for the gauge field. For this reason, the mass of the gauge particle is often intuitively interpreted as the ``resistance" when the gauge boson moves in the $\phi$ field and interacts with it. In this subsection, the higgs mechanism with $U(1)$ and $SU(2)$ spontaneously broken symmetry are illustrated.

\begin{figure}[ht]
    \centering
    \includegraphics[width=0.5\textwidth]{chapters/Physics/sectionQFT/figures/Higgs.png}
    \caption{Spontaneous symmetry breaking of the scalar field $\phi$  with a $U(1)$  global symmetric potential. The shape of potential on the complex plane looks like a Mexican hat. The scalar field's minimum potential shifts from the origin by the amount of Higgs vacuum expectation value, forming a ring of positions with minimal potential. The scalar field has to choose one of the positions on the ring to settle down. Such a choice is so-called spontaneous symmetry breaking. The radial and lateral perturbation mode around this minimal position gives rise to the Higgs field and the Goldstone field. }
    \label{fig:physics:qft:higgsPotential}
\end{figure}



\subsection{U(1) Spontaneous Symmetry Breaking}

Consider a $U(1)$ gauge theory with a scalar field $\phi$, which has a $U(1)$ global symmetric potential
\begin{equation}
    V(\phi) = \frac{1}{2} \mu^2 \phi^*\phi + \frac{1}{4}\lambda(\phi^*\phi )^2.
\end{equation}


\noindent When engaging local symmetry, we introduce the associated gauge field $B_{\mu}$. The total Lagrangian for the scalar field and the gauge field is
\begin{equation}
\begin{split}
	\mathcal{L} & = D^\mu\phi^* D_\mu\phi-V(\phi)   - \frac{1}{4}B_{\mu\nu}B^{\mu\nu} \\
	& = D^\mu\phi^* D_\mu\phi- \big(\frac{1}{2} \mu^2 \phi^*\phi + \frac{1}{4} \lambda(\phi^*\phi )^2 \big)   - \frac{1}{4}B_{\mu\nu}B^{\mu\nu},
\end{split}
\label{eqn:physics:qft:u1Higgs}
\end{equation}

\noindent where the covariant derivative $D_\mu$ and covariant field strength tensor $B_{\mu\nu}$ defined as
\begin{equation}
    D_\mu \equiv \partial_\mu +i g' \frac{Y}{2} B_\mu ,\;\;\; 
    B_{\mu\nu} \equiv  \partial_\mu B_\nu - \partial_\nu B_\mu,
\end{equation}

\noindent and the gauge transformation of the scalar field and gauge field is
\begin{equation}
	\phi \longmapsto e^{i\alpha(x_\mu)} \phi ,\;\;\; 
	B_\mu  \longmapsto  B_\mu - \frac{1}{g'\frac{Y}{2}}\partial_\mu \alpha(x_\mu).
    \label{eqn:physics:qft:u1HiggsGaugeTransform}
\end{equation}


\noindent The Lagrangian in Equation~\label{eqn:physics:qft:u1Higgs} is invariant under $U(1)$ gauge transformation in Equation~\label{eqn:physics:qft:u1HiggsGaugeTransform}. At this moment, $B_\mu$ is massless. The spontaneous symmetry breaking is concerning the scalar's potential $V(\phi)$. When $\mu^2<0$ , the scalar field's potential $V(\phi)$ on the complex plane looks like a Mexican hat, shown in Figure~\ref{fig:physics:qft:higgsPotential}. It has an infinite number of positions with minimal potential on a ring with $|\phi|^2 = -\frac{\mu^2}{\lambda} = \nu^2$, where $\nu$ is the vacuum expectation value or VEV of the scalar $\phi$. Because of nontrivial vev, nature has to choose one of these ground states for $\phi$ instead of the complete vacuum $\phi=0$. This choice is the so-called spontaneous symmetry break. The term "symmetry break" implies that choosing one specific VEV breaks the $U(1)$ global symmetry of the scalar potential; the term "spontaneous" suggests that the symmetry breaking is induced completely by the scalar itself when $\mu^2<0$. During the SSB, it turns out that it does not matter which one of the VEV's is chosen because the complex phase of $\phi$ will eventually be absorbed by the gauge field $B_\mu$. For convenience, we could choose a VEV $\phi_0 = \nu e^{i 0/\nu}$. The scalar field $\phi$ can be treated as the vibration around $\phi_0$:
\begin{equation}
    \phi = \frac{\nu + h}{\sqrt{2}}e^{i\theta/\nu}, 
\end{equation}

\noindent where the $h$ is the real scalar field for the perturbation in the radial direction, while $\theta$ is the real scalar field for the perturbation in the lateral direction. The radial and lateral vibration h and $\theta$ is the Higgs and Goldstone field respectively, which transform under the gauge transformation as following
\begin{equation}
    h  \longmapsto  h ,\;\;\; 
    \theta  \longmapsto  \theta + \alpha(x_\mu).
\end{equation}

\noindent Rewrite Lagrangian in the Equation~\ref{eqn:physics:qft:u1higgs} in terms of the Higgs field $h$ and Goldstone field $\theta$, one gets
\begin{equation}
\begin{split}
    \mathcal{L} =&  D^\mu\phi^* D_\mu\phi- \big(\frac{1}{2} \mu^2 \phi^*\phi + \frac{1}{4} \lambda(\phi^*\phi )^2 \big)   - \frac{1}{4}B_{\mu\nu}B^{\mu\nu} \\
    =&  (\partial_\mu +i g' \frac{Y}{2} B_\mu) \phi^* (\partial_\mu +i g' \frac{Y}{2} B_\mu) \phi- \big(\frac{1}{2} \mu^2 \phi^*\phi + \frac{1}{4} \lambda(\phi^*\phi )^2 \big)   - \frac{1}{4}B_{\mu\nu}B^{\mu\nu} \\
    =&  - \frac{1}{4}\mathcal{B}_{\mu\nu}\mathcal{B}^{\mu\nu} +  \frac{Y^2}{8} g'^2 \nu^2\mathcal{B}^\mu \mathcal{B}_\mu \;\; \text{ (Gauge boson kinetics and mass) } \\
    & + \big(\frac{1}{2} (\partial_\mu h)^2 -\lambda\nu^2h^2\big)  \;\; \text{ (Higgs kinetics and mass) }\\
    & - \big ( \lambda \nu h^3 + \frac{1}{4}\lambda h^4 \big) \;\; \text{ (Higgs self-coupling) } \\
    & + \big( \frac{Y^2}{8} 2\nu g'^2 \mathcal{B}^\mu \mathcal{B}_\mu h  + \frac{Y^2}{8} g'^2  \mathcal{B}^\mu \mathcal{B}_\mu h^2 \big)   \;\; \text{ (coupling between Higgs and Gauge boson) }
\end{split}
\end{equation}

\noindent where
\begin{equation}
    \mathcal{B}_\mu = B_\mu - \frac{1}{g'\frac{Y}{2}} \partial_\mu \theta/\nu
\end{equation}

\noindent is the gauge field after absorbing the Goldstone field. Intuitively, it means the gauge boson eats the Goldstone boson. Comparing with Equation~\ref{eqn:physics:qft:u1higgs}, this Lagrangian also invariant under $U(1)$ gauge transformation, but the gauge field become massive. It is straight-forward to identify the mass term of the gauge field in the Lagrangian and the mass of the gauge boson and higgs boson turn out to be
\begin{equation}
    m_B = g'\frac{Y}{2} \nu ,\;\;\; 
    m_h = \sqrt{2\lambda\nu^2}
\end{equation}

\noindent Now, the gauge boson acquires its mass! To summary, what is happening is the following: because of the non-trivial vev of the scalar field, SSB happens and produces the Higgs field and the Goldstone field; the Goldstone boson is eaten by the the gauge boson; the gauge boson then become massive and digest the degree of freedom of the Goldstone boson into the transverse polarization which is necessary for massive particles; the higgs field is revealed after the SSB, which predicts a new massive scalar higgs boson.


\subsection{SU(2) Spontaneous Symmetry Breaking}
The Higgs mechanism with SSB for $SU(2)$ symmetry is similar to $U(1)$ breaking but requires two scalar fields forming a scalar doublet. This is the same structure as the scalar field in the SM with $U(1)\times SU(2) \to U(1)$ SSB discussed in Section~\ref{sec:physics:qft:gws}. Therefore,  $SU(2)$ breaking in this section provides an illustration of SSB with the scalar doublet. More complex scalar structures, such as 2 higgs doublets (2HDM) considered in Section~\ref{sec:physics:bsm:chargedHiggs} or higgs triplet, break following the same principle. Here we illustrate the Higgs mechanism with $SU(2)$ SSB by considering a doublet of two complex scalar fields, $\phi = (\phi^+, \phi^0)^T$, which has a $SU(2)$  global symmetric potential as
\begin{equation}
    V(\phi) = \frac{1}{2} \mu^2 \phi^\dagger\phi + \frac{1}{4} \lambda(\phi^\dagger\phi )^2.
\end{equation}

\noindent SSB happens when $\mu<0$. For convenience, we choose a specific VEV $\phi = (0, \nu/\sqrt{2})^T$ for the SSB and expend the scalar fields around this VEV
\begin{equation}
    \phi = \begin{bmatrix} \phi^+ \\ \phi_0 \end{bmatrix} =
    \begin{bmatrix} 0 \\ (\nu + h)/\sqrt{2} \end{bmatrix} e^{i \frac{\tau_a}{2} \theta^a  /\nu}
    \simeq \begin{bmatrix} \theta_2/2 + i\theta_1/2 \\ \nu + h - i\theta_3/2 \end{bmatrix} /\sqrt{2},
    \label{eqn:physics:qft:su2Higgs}
\end{equation}

\noindent where the Higgs field $h$ corresponds to the radial oscillation around the VEV, while three Goldstone field $\theta_1,\theta_2,\theta_3$ corresponds to the three oscillation components in the three rotational direction around $\frac{\tau_a}{2}$ generator axes. The Higgs field and three Goldstone fields transformes under the gauge transformation as 
\begin{equation}
    h  \longmapsto  h ,\;\;\; 
    \theta_a  \longmapsto  \theta_a + \alpha_a(x_\mu).
\end{equation}





% We expend the scalar field around a


% When imposing SU(2) symmetry to the scalar doublet (gauge the scalar doublet), we have to introduce a set of three gauge fields \PW and compose the covariant derivative from it $D_\mu  = \partial_\mu +i g T_a W^a_\mu$. The Lagrangian 

% \begin{equation}
% \begin{split}
%     \mathcal{L}_{SU(2)} =& (D_\mu \phi)^\dagger D_\mu \phi - V(\phi) - \frac{1}{4} W^a_{\mu\nu}W^{\mu\nu}_a \\
%     = & (D_\mu \phi)^\dagger D_\mu \phi - \big(\frac{1}{2} \mu^2 \phi^\dagger\phi + \frac{1}{4} \lambda(\phi^\dagger\phi )^2 \big) - \frac{1}{4} W^a_{\mu\nu}W^{\mu\nu}_a
% \end{split}
% \end{equation}

% when expend the scalar doublet at VEV $\phi_0 = (0, \nu/\sqrt{2})^T$, we have 




\noindent Then we can write the SU(2) Lagrangian for the scalar doublet $\phi$ in terms of Higgs field using the expansion of $\phi$ around VEV in Equation~\label{eqn:physics:qft:u1Higgs}
\begin{equation}
\begin{split}
    \mathcal{L} =& (D^\mu \phi)^\dagger D_\mu \phi - \big(\frac{1}{2} \mu^2 \phi^\dagger\phi + \frac{1}{4} \lambda(\phi^\dagger\phi )^2 \big) - \frac{1}{4} W^a_{\mu\nu}W^{\mu\nu}_a \\
    =& \big( (\partial_\mu +i g \frac{\tau_a}{2} W^a_\mu)  \phi \big)^\dagger (\partial_\mu +i g \frac{\tau_a}{2} W^a_\mu ) \phi - \big(\frac{1}{2} \mu^2 \phi^\dagger\phi + \frac{1}{4} \lambda(\phi^\dagger\phi )^2 \big) - \frac{1}{4} W^a_{\mu\nu}W^{\mu\nu}_a \\
    = & - \frac{1}{4} \mathcal{W}^a_{\mu\nu} \mathcal{W}^{\mu\nu}_a + \frac{1}{8} g^2 \nu^2 \mathcal{W}^{\mu}_a \mathcal{W}_{\mu}^a \;\; \text{ (Gauge boson kinematics and mass) } \\
    & + \big(\frac{1}{2} (\partial_\mu h)^2 -\lambda\nu^2h^2\big)  \;\; \text{ (Higgs kinematics and mass) } \\
    & - \big ( \lambda \nu h^3 + \frac{1}{4}\lambda h^4 \big) \;\; \text{ (Higgs self coupling) } \\
    & + \big( \frac{1}{8} 2\nu g^2 \mathcal{W}^{\mu}_a \mathcal{W}_{\mu}^a h  + \frac{1}{8} g^2  \mathcal{W}^{\mu}_a \mathcal{W}_{\mu}^a h^2 \big)   \;\; \text{ (coupling between Higgs and Gauge boson) }
\end{split}
\end{equation}

\noindent where $\mathcal{W}^a_\mu = W^a_\mu - \frac{1}{g} \partial_\mu \theta^a / \nu - f_{abc}\theta^b W^c_\mu$ are three massive gauge bosons after absorbing three Goldstone bosons.

% \subsection{$U(1)\times SU(2) \to U(1)$ Breaking}

% \begin{equation}
% \begin{split}
%     \mathcal{L} =& (D^\mu \phi)^\dagger D_\mu \phi - \big(\frac{1}{2} \mu^2 \phi^\dagger\phi + \frac{1}{4} \lambda(\phi^\dagger\phi )^2 \big) - \frac{1}{4}B_{\mu\nu}B^{\mu\nu} - \frac{1}{4} W^a_{\mu\nu}W^{\mu\nu}_a
%     % =& \big( (\partial_\mu +i g' \frac{Y}{2} B_\mu +i g \frac{\tau_a}{2} W^a_\mu)  \phi \big)^\dagger (\partial_\mu +i g' \frac{Y}{2} B_\mu + i g \frac{\tau_a}{2} W^a_\mu ) \phi \\
%     % & - \big(\frac{1}{2} \mu^2 \phi^\dagger\phi + \frac{1}{4} \lambda(\phi^\dagger\phi )^2 \big) - \frac{1}{4}B_{\mu\nu}B^{\mu\nu} - \frac{1}{4} W^a_{\mu\nu}W^{\mu\nu}_a
% \end{split}
% \end{equation}
% \noindent where the covariant derivative is $D_\mu = \partial_\mu +i g' \frac{Y}{2} B_\mu +i g \frac{\tau_a}{2} W^a_\mu$ with hypercharge of the complex doublet chosen to be one, $Y=1$. The same process can be used in 

\section{Glashow-Weinberg-Salam Electroweak Model}
\label{sec:physics:qft:gws}
% Glashow-Weinberg-Salam 
The $U(1) \times SU(2)$ gauge symmetric Lagrangian of GWS model for the SM electroweak unification reads 
\begin{equation}
\begin{split}
    \mathcal{L}_{U(1)\times SU(2)} =&  - \frac{1}{4}W^a_{\mu\nu}W^{\mu\nu}_a - \frac{1}{4}B_{\mu\nu}B^{\mu\nu} \\
    & + \bar{\chi}_L \gamma^\mu \big( i \partial_\mu -g \frac{\tau_a}{2} W^a_\mu -g'\frac{Y}{2} B_\mu \big) \chi_L 
    + \bar{\psi}_R \gamma^\mu \big( i \partial_\mu -g'\frac{Y}{2} B_\mu \big) \psi_R \\
    & + \left\lvert  \big( i \partial_\mu -g \frac{\tau_a}{2} W^a_\mu -g'\frac{Y}{2} B_\mu \big)\phi \right\rvert ^2 - V(\phi) \\
    & -(G_1 \bar{\chi}_L \phi \psi_R + G_2 \bar{\chi}_L \phi_c \psi_R + \text{hermitian conjugate}) ,
\end{split}
\label{eqn:physics:qft:gws:lagragian}
\end{equation}

\noindent where the left-handed leptons and quarks in the same family form isospin doublets $\chi_L$  with isospin $\frac{1}{2}$, while all right-handed fermions are isospin singlet $\psi_R$ with isospin 0
\begin{equation}
\begin{split}
    \text{leptons: }
    \chi_L &= 
    \begin{pmatrix} \nu_{e,L} \\ e^-_{L} \end{pmatrix}, 
    \begin{pmatrix} \nu_{\mu,L} \\ \mu^-_{L} \end{pmatrix},
    \begin{pmatrix} \nu_{\tau,L} \\ \tau^-_{L} \end{pmatrix}, 
    \;\; \psi_R  = e^-_R, \mu^-_R, \tau^-_R \\
    % quark
    \text{quarks: }
    \chi_L &= 
    \begin{pmatrix} u_{L} \\ d_{L} \end{pmatrix}, 
    \begin{pmatrix} c_{L} \\ s_{L} \end{pmatrix},
    \begin{pmatrix} t_{L} \\ b_{L} \end{pmatrix}, 
    \;\; \psi_R = u_R, d_R, c_R, s_R, t_R, b_R \\
\end{split}
\end{equation}

\noindent The Lagrangian in Equation~\ref{eqn:physics:qft:gws:lagragian} contains many crucial ideas to realize the unification of electromagnetic and weak interaction. Here list three of the most fundamental ideas. First, the electroweak mixing angle mixes the $U(1)$ and $SU(2)$ gauge bosons. Second, the Higgs mechanism with $U(1)\times SU(2)\to U(1)$ breaking generates mass for gauge bosons under the electroweak mixing angle. Third, the Yukawa couplings generate the fermion mass and result in the fixing between the fermions' flavor eigenstates and mass eigenstates.


\subsection{Electroweak Mixing Angle}
GWS model allows an electroweak mixing angle $\theta_W$  in the $\mathcal{L}_{U(1)\times SU(2)}$ such that 
\begin{equation}
    g sin(\theta_W) = g' cos(\theta_W) = e.
\end{equation}

\noindent As consequences, the mixing angle leads to rotated gauge boson fields and mixing features in the gauge-fermion coupling and gauge self-coupling. With the electroweak mixing angle $\theta_W$ , the electroweak gauge field $B$ and $W^{a}$ can be rewritten as a set of new gauge fields $A,Z,W^\pm$:
\begin{align}
    A_\mu &= (g'W_\mu^3 + gB_\mu)/\sqrt{g^2+g'^2} = cos\theta_WB_\mu  + sin\theta_W W_\mu^3 \\
    Z_\mu &= (g'W_\mu^3 - gB_\mu)/\sqrt{g^2+g'^2} = cos\theta_WB_\mu  - sin\theta_W W_\mu^3\\
    W^\pm_\mu &= (W_\mu^1 \mp iW_\mu^2)/\sqrt{2}.
\end{align} 

\noindent Similarly, the hypercharge current $J^\mu_Y = \bar{\psi}  \gamma^\mu \frac{Y}{2} \psi$  and three isospin current $J^{\mu,a} _\tau = \bar{\chi}_L  \gamma^\mu \frac{\tau_{a}}{2}  \chi_L$ can be rewritten as four new currents of electroweak quantum number: the electromagnetic current $J_{EM}$ , the weak neutral current $J_{NC}$ and the two weak charge current $J_{\pm}$ defined as:
\begin{align}
    J^{\mu}_{EM} &= \bar{\chi}_L \gamma^\mu   \frac{\tau_3}{2} \chi_L + \bar{\psi} \gamma^\mu   \frac{Y}{2} \psi \\
    J^{\mu}_{NC} &= cos^2\theta_W \,  \bar{\chi}_L \gamma^\mu   \frac{\tau_3}{2} \chi_L - sin^2 \theta_W \,  \bar{\psi} \gamma^\mu   \frac{Y}{2} \psi \\
    J^{\mu}_{\pm}&= \bar{\chi}_L  \gamma^\mu \frac{\tau_\pm}{2} \chi_L ,
\end{align}


\noindent where the charge raising and lowering matrix $\tau_{\pm}$ are linear combination of Pauli matrices:
\begin{equation}
    \tau_+ = (\tau_1+i\tau_2)/\sqrt{2} = \sqrt{2} \begin{bmatrix} 0 & 1 \\ 0 & 0\end{bmatrix} , \;\;\; \tau_- = (\tau_1-i\tau_2)/\sqrt{2} = \sqrt{2} \begin{bmatrix} 0 & 0 \\ 1 & 0\end{bmatrix} .
\end{equation}

\noindent By adding the EW mixing angle and work with the EW mixed gauge fields and currents, the Lagrangian of interaction between fermion and gauge boson can be rewritten as the sum of electromagnetic coupling, weak neutral and weak charge current gauge coupling.
\begin{equation}
\begin{split}
    \mathcal{L}_{U(1)\times SU(2)}^{\text{$\psi$-gauge int}} 
    =& -ig \bar{\chi}_L  \gamma^\mu \frac{\tau_a}{2} W^a_\mu \chi_L - ig' \bar{\psi}  \gamma^\mu \frac{Y}{2} B_\mu \psi  \\
    =& -ig \bar{\chi}_L  \gamma^\mu \frac{\tau_+}{2} \chi_L W^+_\mu -ig \bar{\chi}_L  \gamma^\mu \frac{\tau_-}{2} \chi_L W^-_\mu  \;\;\; \text{(Change Current Gauge coupling)} \\
    & -i \big( g\,cos\theta_W \,  \bar{\chi}_L \gamma^\mu   \frac{\tau_3}{2} \chi_L - g'\,sin\theta_W \,  \bar{\psi} \gamma^\mu   \frac{Y}{2} \psi \big)  \, Z_\mu  \;\;\; \text{(Nuetral Current Gauge coupling)} \\
    & -i \big( g\,sin\theta_W \,  \bar{\chi}_L \gamma^\mu   \frac{\tau_3}{2} \chi_L + g'\,cos\theta_W \,  \bar{\psi} \gamma^\mu   \frac{Y}{2} \psi \big)  \, A_\mu  \;\;\; \text{(EM Current Gauge coupling)}\\
    = &-ig J^\mu_\pm W_\mu^\pm - \frac{ig}{cos\theta_W} J^\mu_{NC} Z_\mu - ie J^{\mu}_{EM} A_\mu
\end{split}
\label{eqn:physics:qft:gws:gaugeIntLagragian}
\end{equation}

\noindent The QED electric charge can be composed by the hypercharge and the third isospin value as $Q = T_3  +\frac{Y}{2}$. In Equation~\ref{eqn:physics:qft:gws:gaugeIntLagragian}, on one hand, the electromagnetic current $J_{EM}$ couples with photon field $A$ with a coupling strength constant $e$, which is exactly the component corresponding to the QED gauge interaction. On the other hand, the weak neutral current $J_{NC}$ couples with $Z$ boson with a coupling constant $g/cos\theta_W$ and the weak charged current couples with $W^\pm$ with coupling constant $g$, which corresponds to the weak gauge interaction. Therefore, in this way, the electroweak interaction is comprised of the QED and weak components.

Besides a mixing pattern in the fermion-gauge couplings in Equation~\ref{eqn:physics:qft:gws:gaugeIntLagragian}, the electroweak mixing angle also leads to a mixing in the gauge self-couplings.  As discussed in the Section~\ref{sec:physics:qft:gaugeSymmetry}, the non-abelian group has non-trivial structure constant $f_{abc}$ built into the gauge fields, which results in the gauge self-couplings among $W^{a=1,2,3}$ in the gauge kinematic energy term. Due to electroweak mixing $\theta_W$, the self-gauge coupling in the $W^{a=1,2,3}$ basis can be transformed into $W^\pm Z \gamma$ basis and takes a slightly more complex form, which includes 2 vertices of triple-gauge-coupling shown in Figure~\ref{fig:physics:qft:ewGaugeCoupling} and 4 vertices of four-quatic-gauge couplings shown in Figure~\ref{fig:physics:qft:ewQuadGaugeCoupling}

% \begin{figure}
%     \centering
%     \includegraphics[width=0.4\textwidth]{tgc.png}
%     \includegraphics[width=0.9\textwidth]{qgc.png}
%     \caption{Electroweak Gauge Coupling is a result of gauging the non-abelian U(1)xSU(2) group and electroweak mixing.}
%     \label{fig:physics:qft:ewGaugeCoupling}
% \end{figure}

\begin{figure}[ht]
    \centering
    % \includegraphics[width=0.6\textwidth]{gluon.png}
    % Using the layered layout
    \feynmandiagram [small, horizontal=a to b] {
      a [particle=\(\gamma\)] -- [photon] b,
      f1 [particle=\(W^{-}\)] -- [photon]  b -- [photon] f2 [particle=\(W^{+}\)], 
    }; \qquad
    \feynmandiagram [small, horizontal=a to b] {
      a [particle=\(Z\)] -- [photon] b,
      f1 [particle=\(W^{-}\)] -- [photon] b  -- [photon] f2 [particle=\(W^{+}\)], 
    }; \qquad 
    \caption{Vertices of SM electroweak Triple-Gauge-Couplings. $WW$ pairs couple to $Z/\gamma$ because of the electroweak mixing $\theta_W$. The Triple-Gauge-Couplings is one of the major process of  WW pair production in the electron-positron collider $e^- e^+ \to Z/\gamma \to W^+ W^-$  such as LEP2 which is discussed in Section~\ref{sec:physics:lu:W}.  }
    \label{fig:physics:qft:ewTripleGaugeCouplingripleGaugeCoupling}
\end{figure}


\begin{figure}[ht]
    \centering
    % \includegraphics[width=0.6\textwidth]{gluon.png}
    % Using the layered layout
    \feynmandiagram [small, horizontal=i1 to f1] {
      {i1 [particle=\(\gamma\)],i2[particle=\(\gamma\)]} -- [photon] a [dot] -- [photon] {f1[particle=\(W^{-}\)],f2[particle=\(W^{+}\)]},
    };
    \feynmandiagram [small, horizontal=i1 to f1] {
      {i1 [particle=\(Z\)],i2[particle=\(\gamma\)]} -- [photon] a [dot] -- [photon] {f1[particle=\(W^{-}\)],f2[particle=\(W^{+}\)]},
    };
    \feynmandiagram [small, horizontal=i1 to f1] {
      {i1 [particle=\(Z\)],i2[particle=\(Z\)]} -- [photon] a [dot] -- [photon] {f1[particle=\(W^{-}\)],f2[particle=\(W^{+}\)]},
    };
    \feynmandiagram [small, horizontal=i1 to f1] {
      {i1 [particle=\(W^{-}\)],i2[particle=\(W^{+}\)]} -- [photon] a [dot] -- [photon] {f1[particle=\(W^{-}\)],f2[particle=\(W^{+}\)]},
    };
    \caption{Vertices of SM electroweak Quatic-Gauge-Couplings.}
    \label{fig:physics:qft:ewQuadGaugeCoupling}
\end{figure}




\subsection{Gauge Boson Mass}
The weak gauge boson must be massive since the weak force is indicated short-ranged by the experiments. Nowadays, the masses of W, Z boson have been measured as $M_W =80.379\pm 0.012 $ Gev and $M_Z=91.1876\pm0.0021$ GeV, which in the GWS model is generated by the Higgs mechanism with $U(1)_Y \times SU(2)_L \to U(1)_{EM}$ spontaneous symmetry breaking. The breaking is implemented by a complex scalar doublet field $\phi = [\phi^+, \phi^0]$, the same structure as the Higgs mechanism with $SU(2)$ breaking. While the isospin of the scalar doublet is determined by the $SU(2)$ generators, the choice of the doublet's hypercharge determines the mass of $Z$ and A field after the breaking. To obtain massless photon, the hypercharge of $\phi$ is therefore chosen to be $Y_{\phi} = 1$. The derivation of the photon mass and weak boson mass with $Y_{\phi} = 1$ is as following 
\begin{equation}
\begin{split}
    \mathcal{L}_{U(1)\times SU(2)}^{\text{gauge mass}} 
    =& \left\lvert  (-ig\frac{\tau_a}{2} W_\mu^a - ig'\frac{Y}{2}B_\mu ) \phi \right\rvert^2 \\ 
    = & \frac{1}{8} \left\lvert 
        \begin{bmatrix} 
             gW_\mu^3 + g'B_\mu & g(W^1_\mu-iW^2_\mu) \\
            g(W^1_\mu+iW^2_\mu) & -gW_\mu^3 + g'B_\mu 
        \end{bmatrix}
        \begin{bmatrix} 0 \\ \nu \end{bmatrix} \right\rvert^2 \\
    = & \frac{1}{8}\nu^2g^2 (W^1_\mu W^\mu_1 +W^2_\mu W^\mu_2) + \frac{1}{8}\nu^2(g'B_\mu-gW^3_\nu)(g'B^\mu-gW^{3\nu}) \\
    = &  (\frac{1}{2}\nu g)^2 W^+_\mu W^{-\mu} +  \frac{1}{2} (\frac{1}{2}\nu \sqrt{g'^2+g^2})^2 Z_\mu Z^\mu + 0 A_\mu A^\mu ,
\end{split}
\label{eqn:physics:qft:gws:gaugeMassLagragian}
\end{equation}

\noindent where shown is the Lagrangian terms about the gauge boson mass, the term with gauge field squares coming from the squared covariant derivative of the scalar doublet $|D_\mu \phi|^2$ in the Higgs sector. The VEV $ (0, \nu/\sqrt{2})^T$ is used for the scalar doublet and the underline gauge fields $W^a, B$  are replaced with the physical gauge field $W^\pm, Z, A$ with final mass of
\begin{equation}
    M_W = \frac{1}{2}\nu g, \; M_Z = \frac{1}{2}\nu \sqrt{g'^2+g^2}= \frac{1}{2}\nu \frac{g}{cos\theta_W}, \;  M_A= 0 . \; 
\end{equation}

\noindent One of the immediate consequences of the choice of $Y_{\phi}=1$ for massless photon is that the ratio between \PW and \PZ boson is related to the electroweak mixing angle $\theta_W$:
\begin{equation}
\frac{M_W}{M_Z} = cos\theta_W
\end{equation}

\noindent While Equation~\ref{eqn:physics:qft:gws:gaugeMassLagragian} shows a part of the Lagrangian in the Higgs sector that only have gauge field, the full Lagrangian in the Higgs sector with $\phi = (0, (\nu+h) /\sqrt{2} )^T $  and  $Y_{\phi} = 1$ reads as
\begin{equation}
\begin{split}
    \mathcal{L}_{U(1)\times SU(2)}^{\text{higgs}} 
    = & \left\lvert  \big( i \partial_\mu -g \frac{\tau_a}{2} W^a_\mu -g'\frac{Y}{2} B_\mu \big)\phi \right\rvert ^2 - \big(\frac{1}{2} \mu^2 \phi^\dagger\phi + \frac{1}{4} \lambda(\phi^\dagger\phi )^2 \big)\\
    = &  (\frac{1}{2}\nu g)^2 W^+_\mu W^{-\mu} +  \frac{1}{2} (\frac{1}{2}\nu \sqrt{g'^2+g^2})^2 Z_\mu Z^\mu + 0 A_\mu A^\mu  \;\;\; \text{(gauge mass)} \\
    & + \big( 2 \frac{h}{\nu} + \frac{h^2}{\nu^2} \big) \, \big( (\frac{1}{2}\nu g)^2 W^+_\mu W^-_\mu + \frac{1}{2} (\frac{1}{2}\nu \sqrt{g'^2+g^2})^2  Z_\mu Z_\mu \big) \;\;\; \text{(Higgs-gauge coupling)}\\
    & + \big(\frac{1}{2} (\partial_\mu h)^2 -\lambda\nu^2h^2\big)  - \big ( \lambda \nu h^3 + \frac{1}{4}\lambda h^4 \big) \;\;\; \text{(Higgs and self-coupling)}
\end{split}
\end{equation}

\noindent where the first row corresponds to the gauge boson mass; the second row is responsible for the couplings between gauge bosons; the third row describes the kinematic energy, mass, and self-coupling of Higgs boson. The mass of Higgs boson can be identified as $M_h=\sqrt{2\lambda \nu^2}$. The coupling strength between the gauge boson and the Higgs boson exactly equals the gauge boson mass. Since the photon is massless, it does not couple to the Higgs. As will be discussed in the following paragraph, the coupling strength between the fermion and the Higgs boson is proportional to the fermion mass. Therefore, this property of the Higgs boson is often described as ``higgs couples to mass", which is essentially an experimental observable of the theory that `` particle mass originates from interacting with higgs''. The SM vertices for the Higgs coupling to gauge boson and fermions are shown in Figure~\ref{fig:physics:qft:ewHiggsGaugeCoupling}.

\begin{figure}[ht]
    \centering
    % \includegraphics[width=0.6\textwidth]{gluon.png}
    % Using the layered layout
    \feynmandiagram [small, horizontal=a to b] {
      a [particle=\(H\)] -- [scalar] b,
      f1 [particle=\(W^{-}\)] -- [photon]  b -- [photon] f2 [particle=\(W^{+}\)], 
    }; \qquad
    \feynmandiagram [small, horizontal=a to b] {
      a [particle=\(H\)] -- [scalar] b,
      f1 [particle=\(Z\)] -- [photon] b  -- [photon] f2 [particle=\(Z\)], 
    }; \qquad 
    \feynmandiagram [small, horizontal=a to b] {
      a [particle=\(H\)] -- [scalar] b,
      f1 [particle=\(f\)] -- [fermion] b  -- [fermion] f2 [particle=\(f\)], 
    }; \qquad 
    \caption{Vertices of SM Higgs-gauge couplings and Higgs-fermion coupling. The coupling strength is proportional to the mass of the coupled gauge boson or fermion. }
    \label{fig:physics:qft:ewHiggsGaugeCoupling}
\end{figure}


\subsection{Fermion Mass and Fermion Mixing}

In GWS model, the fermion mass is allowed by introducing Yukawa coupling between the fermion and the scalar doublet $\phi$ with the coupling constant $y_f$. After the spontaneous symmetry breaking $U(1)_Y \times SU(2)_L \to U(1)_{EM}$, $\phi$ is expended around its VEV as $\phi=(0, (\nu+h)/\sqrt{2})^T$. Accordingly, the Lagrangian of the Yukawa coupling between the fermion and the scalar doublet $\phi$ is transformed into the terms for fermion mass and terms for Higgs-fermion interaction. For example, considering only the first family of leptons and quarks, the Lagrangian in the Yukawa coupling sector reads as
\begin{equation}
\begin{split}
	 \mathcal{L}_{U(1)\times SU(2)}^{\text{yukawa}} =& \big[ -y_e (\bar{\nu}_e,\bar{e})_L \phi e_R  -y_d (\bar{u},\bar{d})_L \phi d_R - y_u(\bar{u},\bar{d})_L \phi_c u_R \big ] + h.c. \\
     = & -\frac{y_e}{\sqrt{2}}\nu(\bar{e}_L e_R + \bar{e}_R e_L)   -\frac{y_u}{\sqrt{2}}\nu(\bar{u}_L u_R + \bar{u}_R u_L)  -\frac{y_d}{\sqrt{2}}\nu(\bar{d}_L d_R + \bar{d}_R d_L)  \\
     & -\frac{y_e}{\sqrt{2}}h(\bar{e}_L e_R + \bar{e}_R e_L)  - \frac{y_u}{\sqrt{2}}h(\bar{u}_L u_R + \bar{u}_R u_L)  - \frac{y_d}{\sqrt{2}}h(\bar{d}_L d_R + \bar{d}_R d_L)  \\
     = & -\frac{y_e\nu}{\sqrt{2}}\bar{e} e   -\frac{y_u\nu}{\sqrt{2}}\bar{u}u -\frac{y_d\nu}{\sqrt{2}} \bar{d} d \;\;\; \text{(fermion mass)} \\
     & -\frac{y_e}{\sqrt{2}}h\bar{e} e   -\frac{y_u}{\sqrt{2}}h\bar{u}u -\frac{y_d}{\sqrt{2}} h \bar{d} d \;\;\; \text{(Higgs-fermion coupling)},
\end{split}
\label{eqn:physics:qft:gws:yukawaLagragian}
\end{equation}


\noindent where the fermion mass is then proportional to the Yukawa coupling strength $y_f$ 
\begin{equation}
    M_f =\frac{y_f v}{\sqrt{2}} ,  %, \;\;\; f \in { e,\mu,\tau,u,d,c,s,t,b}
\end{equation}
\noindent and $\phi_c$ is the charge conjugate of the scalar doublet 
\begin{equation}
    \phi_c = -i\tau_2 \phi^* = -i\tau_2 \begin{bmatrix} \phi^+ \\ \phi^0 \end{bmatrix}^* = -i\tau_2 \begin{bmatrix} \phi^- \\ \bar{\phi}^0 \end{bmatrix} = \begin{bmatrix} - \bar{\phi}^0 \\ \phi^-   \end{bmatrix}.
\end{equation}


\noindent From Equation~\ref{eqn:physics:qft:gws:yukawaLagragian}, the coupling strength between fermion and Higgs boson  is proportional to the fermion mass $\frac{y_f}{\sqrt{2}} = \frac{M_f}{\nu}$. Here $\chi_L$ and $\psi$  denote the mass eigenstates of the fermions. It turns out that for quarks, the flavor eigenstates participating in the weak interaction is a linear mixing of mass eigenstates. Such mixing allows the quark transition between two different generations in the flavor changing charged current (FCCC) weak process, confirmed in many experiments. The quark mixing is mathematically expressed as a 3x3 unitary matrix known as the CKM matrix. For down-type quarks, the CKM matrix composes their weak flavor eigenstates $(d',s',b')$ by linear combining their mass eigenstate $(d,s,b)$. For up-type quarks (up, charm, top), their flavor eigenstates are set equal to their mass eigenstates. By rotating $(d,s,b)$ with CKM matrix, the weak coupling strengths between $d,s,b$ quark and $u,c,t$ quark are scaled by the nine elements in the CKM matrix. The CKM rotation from mass eigenstates $(d,s,b)$ to flavor eigenstates $(d',s',b')$ can be expressed as
\begin{equation}
    \begin{bmatrix}
        d' \\ s' \\ b' 
    \end{bmatrix} = 
     \begin{bmatrix}
        V_{ud} & V_{us} & V_{ub} \\ V_{cd} & V_{cs} & V_{cb} \\ V_{td} & V_{ts} & V_{tb}
    \end{bmatrix}   
    \begin{bmatrix}
        d \\ s \\ b
    \end{bmatrix}
\end{equation}

\noindent Similar to quark mixing, the discovery of the neutrino oscillation implies that three neutrinos should also be mixed by a 3x3 unitary matrix. More specifically, the neutrinos oscillation is a consequence of the two facts: firstly, they are massive; secondly, their flavor eigenstates are a mixture of their mass eigenstates. Thus SM is modified accordingly to cope with the neutrino oscillation: three Yukawa coupling constants for neutrino are added for neutrino mass; the 3x3 PMNS matrix is postulated to implement the neutrino mixing. Analogical to the quark mixing, the neutrino flavor eigenstates are ``PMNS-rotated" mass eigenstates
\begin{equation}
    \begin{bmatrix}
        \nu_e \\ \nu_\mu \\ \nu_\tau 
    \end{bmatrix} = 
     \begin{bmatrix}
        U_{e1} & U_{e2} & U_{e3} \\ U_{\mu 1} & U_{\mu 2} & U_{\mu 3} \\ U_{\tau 1} & U_{\tau 2} & U_{\tau 3}
    \end{bmatrix}   
    \begin{bmatrix}
        \nu_1 \\ \nu_2 \\ \nu_3
    \end{bmatrix}
\end{equation}

\noindent  For both CMK and PMNS matrix, though there are 9 elements in the matrix, only four independent parameters are needed to fully parametrize the matrix. The parameters include $\theta_{12},\theta_{13},\theta_{23}, $ for the rotation angle alone third, second, and first axis, respectively, and $\delta_{13}$ for the CP violation during the weak interaction between the first and the third family. Sometimes, Wolfenstein parameters $[A,\lambda, \rho, \eta ]$ are used instead of  $ [ \theta_{12},\theta_{13},\theta_{23},\delta_{13} ]$  for a polynomial parametrization. The two parametrizations of the CKM and PMNS matrix can be expressed as
\begin{equation}
\begin{split}
    V, U =  & 
    \begin{bmatrix}
        1 &  0      & 0      \\ 
        0 &  c_{23} & s_{23} \\
        0 & -s_{23} & c_{23}  
    \end{bmatrix}
    \begin{bmatrix}
        c_{13} &  0      & s_{13} e^{-i\delta_{13}}     \\ 
        0 &  1 & \\
        -s_{13} e^{i\delta_{13}}    & 1 & c_{13}  
    \end{bmatrix}
    \begin{bmatrix}
        c_{12} &  s_{12}  & 0      \\ 
        -s_{12} &  c_{13} & 0 \\
        0 & 0 & 1  
    \end{bmatrix}\\
    = &\begin{bmatrix}
        c_{12}c_{13} & s_{12}c_{13} & s_{13}e^{-i\delta} \\ 
        -s_{12}c_{23}-c_{12}s_{23}s_{13}e^{i\delta} & c_{12}c_{23}-s_{12}s_{23}s_{13}e^{i\delta} & s_{23}c_{13} \\ 
        s_{12}s_{23}-c_{12}c_{23}s_{13}e^{i\delta} & -c_{12}s_{23}-s_{12}c_{23}s_{13}e^{i\delta} & c_{23}c_{13} 
    \end{bmatrix}\\
    =& \begin{bmatrix}
        1-\lambda^2/2 & \lambda & A\lambda^3(\rho-i\eta) \\ 
        -\lambda & 1-\lambda^2/2 & A\lambda^2 \\ 
        A\lambda^3(1-\rho-i\eta) & -A\lambda^2 & 1
    \end{bmatrix} + O(\lambda^4)  \\
\end{split}
\end{equation}

\noindent where the parameters are determined by the experiments. For parametrization with rotation angle $[ \theta_{12},\theta_{13},\theta_{23},\delta_{13}] $, the world average experimental measurements are
\begin{align}
    \text{CMK: } & \theta_{12}=13.04\pm0.15, \; \theta_{13}=0.201\pm0.011, \; \theta_{23}=1.23\pm0.06, \; \delta = 68.8\pm 4.6  \\
    \text{PMNS: } &\theta_{12}=33.62 ^{+0.78}_{-0.76}, \;  \theta_{13}=47.2  ^{+1.9}_{-3.9}, \; \theta_{23}= 8.54 ^{+0.15}_{-0.15}, \;  \delta = 234 ^{+43}_{-31}.
\end{align}


\section{Quantum Chromodynamics}
\label{sec:physics:qft:qcd}



\begin{figure}[ht]
    \centering
    % \includegraphics[width=0.6\textwidth]{gluon.png}
    % Using the layered layout
    \feynmandiagram [small, horizontal=a to b] {
      i1 [particle=\(q\)] -- [fermion] a  [dot] -- [fermion] i2 [particle=\(q\)], 
      a -- [gluon] b [particle=\(g\)],
    }; \qquad
    \feynmandiagram [small, horizontal=a to b] {
      i1 [particle=\(g\)] -- [gluon] a [dot] -- [gluon] i2 [particle=\(g\)], 
      a -- [gluon] b [particle=\(g\)],
    }; \qquad
    \feynmandiagram [small, horizontal=i1 to f1] {
      {i1 [particle=\(g\)],i2[particle=\(g\)]} -- [gluon] a [dot] -- [gluon] {f1[particle=\(g\)],f2[particle=\(g\)]},
    };    
    \caption{Vertices of the SM QCD interaction. Quarks and gluons both carry colors allowing quark-gluon interaction and gluon self-interaction. }
    \label{fig:my_label}
\end{figure}

QCD assumes three color charges with three anti-colors, analogical to the QED's positive and negative electric charge. The potential between two color charges can be attractive or repulsive in the short-distance range, depending on the state of the two-color system:
\begin{align}
	 V(r)_{singlet} &= -\frac{4\alpha_s}{3r} + \lambda r \;\;\; \text{(attractive in short-distance range)}\\
    V(r)_{octet} &= \frac{\alpha_s}{6r} + \lambda r \;\;\; \text{(repulsive in short-distance range)}, 
\end{align}


\noindent where  the $\frac{1}{r}$ term dominates in the short-distance range, and $\lambda r$ term dominates in the long-distance range. In the short-distance range, if the two interacting colors form a color singlet state, the strong force between them is attractive; but repulsive if the two form a color octet state. For example, the two quarks in a meson form a color singlet state, and thus the strong force between them is attractive. In contrast, in the quark-quark scattering process where gluons are exchanged, the two quarks form a color octet state with a repulsive strong force. In the long-distance range, when two colors split far from each other, the strong force potential increases linearly with the separation, creating cylindrical ``color tubes" in the space in between. When the separation become large, the potential energy in the ``color tube'' will be enough to create new quark-antiquark pairs from the vacuum, guaranteeing that quarks cannot be separated and no quarks exist alone. This is an intuitive approach to understand quark confinement. A formal derivation of the QCD quark confinement is through the running of couplings presented in the next paragraph in Equation~\ref{eqn:physics:qft:qcd:lagragian}. Though color charge and gluon in the QCD can be analogical to electric charge and photons in the QED  in many ways, QCD has many unique phenomena, such as asymptotic freedom, quark confinement, color anti-screening. Theoretically speaking, these unique figures is essentially originated from two facts:
\begin{itemize}
\item The SU(3) group is non-abelian.
\item the number of quarks is $n_f=6$ 
\end{itemize}

\noindent Non-trivial $f_{abc}$ of SU(3) group leads to the gluon self-coupling, which together with $n_f=6$ further determines the unique running of the QCD coupling: the QCD coupling increases when the energy scale goes low; meanwhile, it decreases in the high-energy region.  This can be demonstrated by the calculation of the beta function in the QCD renormalization group equation.  The SU(3) gauge symmetric Lagrangian for QCD is
\begin{equation}
    \mathcal{L}^{QCD}_{SU(3)} = i\bar{q}\gamma^\mu D_\mu q  - m\bar{q} q - \frac{1}{4}G^a_{\mu\nu}G^{\mu\nu}_a, 
    \label{eqn:physics:qft:qcd:lagragian}
\end{equation}

\noindent where the first term expresses the quark kinematic energy and interaction with gluons, the third term $\frac{1}{4}G^a_{\mu\nu}G^{\mu\nu}_a$  gives gluon kinematic energy and self-interaction. The gluon self-coupling plays a core role in the QCD running couplings: the running coupling is driven by the vacuum polarization; the QCD vacuum polarization of the gluon propagator includes not only the quark bubbles but also has a large contribution from the gluon bubbles. The general form of renormalization group equation (RGE) is
\begin{equation}
	\mu \frac{\partial}{\partial \mu} \alpha(\mu) \equiv \beta(\alpha) = -2\alpha\big [  (\frac{\alpha}{4\pi}) \beta_0 +  (\frac{\alpha}{4\pi}) ^2 \beta_1  +  (\frac{\alpha}{4\pi}) ^3 \beta_2  + \dots \big], 
    \label{eqn:physics:qft:qcd:rge}
\end{equation}

\noindent where the running of coupling with respect to the energy scale $\mu \frac{\partial}{\partial \mu} \alpha(\mu) $ is represented as ``beta function" $\beta(\alpha)$ which can be expanded as a power series of the coupling $\alpha$. The coefficiencies of the power series $\beta_0, \beta_1 ,\beta_2  ,\beta_3 \dots $ are calculated from the considering 1-loop, 2-loop , 3-loop $\dots$ vacuum polarization of the gluon. If only considering the leading order beta function by taking into account only the 1-loop contribution, the renormalization group equation reduces to a simple first-order linear differential equation $\mu \frac{\partial \alpha(\mu) }{\partial \mu} = \frac{\beta_0}{2\pi} \alpha^2$ , which is easily solved by


\begin{equation}
	\alpha(\mu) = \frac{\alpha(\mu_0) }{1+\alpha(\mu_0) \frac{\beta_0}{2\pi}  ln(\frac{\mu}{\mu_0})}=\frac{1}{\frac{\beta_0}{4\pi}  ln(\frac{\mu^2}{\Lambda^2})},
    \label{eqn:physics:qft:qcd:zeroOrderRunning}
\end{equation}
\noindent where $\mu$ is the running scale, $\mu_0$ is a reference scale and $\Lambda$  is the energy scale which diverges coupling $\alpha(\Lambda) \gg 1$ and the condition for power expansion of beta function breaks. The energy scale $\Lambda$  is called ``Landau Pole'' and is a constant fixed by the boundary condition of the RGE. The $\beta_0$ is obtained by evaluating and summing 1-loop diagrams contributing to the vacuum polarization. 

For QED, photon's vacuum polarization only includes fermion bubbles, and summing all 1-loop diagrams yields $\beta_0 = -4/3$. This is a negative number, which means the QED coupling strength increases as the energy scale $\mu$ increases. Phenomenologically speaking,  the closer one gets to the bare electric charge, the larger the effective electric charge he will be able to feel. In other words, the photon's vacuum polarization creates a charge ``screening" effect for the bare charge. The screening effect is analogical to charge screening effect created by the electric polarization in the dielectric media surrounding an electric charge. The Landau pole of QED can be estimated from $\alpha(m_e=511keV) = \frac{1}{137}$ which gives $\Lambda_{QED}=10^{286}$ eV, a very large energy scale much beyond the Planck scale. So QED is valid and renormalizable in a vast range of energy scales.



For QCD, the gluon's vacuum polarization includes both the fermion bubble and the gluon bubble. Summing all 1-loop diagrams yields
\begin{equation}
	\beta_0 = \frac{11C_A}{3} - \frac{2 n_f}{3} = 7 , 
\end{equation}

\noindent where $n_f=6$ is the number of quarks. The sign of $\beta_0$  in QCD is positive, opposite to $\beta_0$ in the QED. Therefore the QCD coupling $\alpha_s$  increases when approaching to lower energy scale. When energy is low, $\alpha_S$ is large, and quarks are confined; when energy is high, $\alpha_s$ is small, and quarks get asymptotic freedom. The screening effect in the QCD is opposite to that in the QED. When going away from a bare color charge, one feels a larger effective color charge. In other words, the gluon's vacuum polarization creates a color ``anti-screening" effect on the bare color. In fact, the effect is anti-screening as long as the number of quarks is less than $n_f<17$. The QCD's Landau pole $\Lambda_{QCD}$ thereby sets a lower limit for the energy scale, which is approximated by experiments to be around $200$ GeV. For energy scale below $\Lambda_{QCD}$ , perturbative approach no longer works and Lattice QCD is invented to solve the related problems at the low energy scale.


Beyond the 1-loop diagram, the running of coupling can be determined by solving higher order RGE and evaluating $\beta_1,\beta_2,\beta_3$ with the higher-order loops diagrams. The higher-order up to $O(\alpha^3)$ solution to the RGE in Equation~\ref{eqn:physics:qft:qcd:rge} reads as \cite{schwartz2014quantum}
\begin{equation}
    \alpha_s(\mu) = \alpha_s(\mu_R) - \frac{\alpha^2_s(\mu_R)}{2\pi}\beta_0 \ln \frac{\mu}{\mu_R} + \frac{\alpha^3_s(\mu_R)}{8\pi^2} \bigg[ -\beta_1 \ln \frac{\mu}{\mu_R}  + 2 \beta_0 ^2 \ln^2 \frac{\mu}{\mu_R} \bigg ] + O(\alpha^4_s(\mu_R)), 
    \label{eqn:physics:qft:qcd:higherOrderRunning}
    % Schwartz eq	 26.102
\end{equation}


where the higher order of beta coefficiencies  $\beta_1,\beta_2,\beta_3$ are calculated by summing higher-order up to four-loop diagrams
\begin{equation*}
\begin{split}
\beta_1 &= 102 - 10 n_f - \frac{8}{3} n_f  , \;\;\;\;\;\;
\beta_2 = \frac{325}{54}n_f^2 - \frac{5033}{18}n_f   +\frac{2857}{2} \\
\beta_3  &= \frac{1093}{729}n_f^3  + \big ( \frac{50065}{162}+\frac{6472}{81}\zeta_3 \big ) n_f^2   -  \big ( \frac{1078361}{162}+\frac{6508}{27}\zeta_3 \big ) n_f + 3564\zeta_3+\frac{149753}{6}
\end{split}
\end{equation*}


        
\chapter{Plots for the Analysis of \PW Branching Fractions}

\section{Kinematics Plots in Counting Analysis}


%  emu channel
\begin{figure}[ht]
    \centering
    $ \mu  e - 1b$ \\
    \includegraphics[width=0.49\textwidth]{chapters/Analysis/sectionPlots/figures/kinematics_pickles/emu/1b/emu_1b_lepton1_pt.pdf}
    \includegraphics[width=0.49\textwidth]{chapters/Analysis/sectionPlots/figures/kinematics_pickles/emu/1b/emu_1b_lepton1_eta.pdf}
    \includegraphics[width=0.49\textwidth]{chapters/Analysis/sectionPlots/figures/kinematics_pickles/emu/1b/emu_1b_lepton2_pt.pdf}
    \includegraphics[width=0.49\textwidth]{chapters/Analysis/sectionPlots/figures/kinematics_pickles/emu/1b/emu_1b_lepton2_eta.pdf}
    \includegraphics[width=0.49\textwidth]{chapters/Analysis/sectionPlots/figures/kinematics_pickles/emu/1b/emu_1b_nJets.pdf}
    \includegraphics[width=0.49\textwidth]{chapters/Analysis/sectionPlots/figures/kinematics_pickles/emu/1b/emu_1b_nBJets.pdf}
    
    \caption{$\mu e$ channel with $n_j\geq2, n_b=1$.}
\end{figure}

\begin{figure}[ht]
    \centering
    $ \mu e- 2b$ \\
    \includegraphics[width=0.49\textwidth]{chapters/Analysis/sectionPlots/figures/kinematics_pickles/emu/2b/emu_2b_lepton1_pt.pdf}
    \includegraphics[width=0.49\textwidth]{chapters/Analysis/sectionPlots/figures/kinematics_pickles/emu/2b/emu_2b_lepton1_eta.pdf}
    \includegraphics[width=0.49\textwidth]{chapters/Analysis/sectionPlots/figures/kinematics_pickles/emu/2b/emu_2b_lepton2_pt.pdf}
    \includegraphics[width=0.49\textwidth]{chapters/Analysis/sectionPlots/figures/kinematics_pickles/emu/2b/emu_2b_lepton2_eta.pdf}
    \includegraphics[width=0.49\textwidth]{chapters/Analysis/sectionPlots/figures/kinematics_pickles/emu/2b/emu_2b_nJets.pdf}
    \includegraphics[width=0.49\textwidth]{chapters/Analysis/sectionPlots/figures/kinematics_pickles/emu/2b/emu_2b_nBJets.pdf}
    
    \caption{$\mu e$ channel with $n_j\geq2, n_b\geq2$.}
\end{figure}

%  mumu channel
\begin{figure}[ht]
    \centering
    $\mu\mu - 1b$ \\
    \includegraphics[width=0.49\textwidth]{chapters/Analysis/sectionPlots/figures/kinematics_pickles/mumu/1b/mumu_1b_lepton1_pt.pdf}
    \includegraphics[width=0.49\textwidth]{chapters/Analysis/sectionPlots/figures/kinematics_pickles/mumu/1b/mumu_1b_lepton1_eta.pdf}
    \includegraphics[width=0.49\textwidth]{chapters/Analysis/sectionPlots/figures/kinematics_pickles/mumu/1b/mumu_1b_lepton2_pt.pdf}
    \includegraphics[width=0.49\textwidth]{chapters/Analysis/sectionPlots/figures/kinematics_pickles/mumu/1b/mumu_1b_lepton2_eta.pdf}
    \includegraphics[width=0.49\textwidth]{chapters/Analysis/sectionPlots/figures/kinematics_pickles/mumu/1b/mumu_1b_nJets.pdf}
    \includegraphics[width=0.49\textwidth]{chapters/Analysis/sectionPlots/figures/kinematics_pickles/mumu/1b/mumu_1b_nBJets.pdf}
    
    \caption{$\mu \mu$ channel with $n_j\geq2, n_b=1$.}
\end{figure}

\begin{figure}[ht]
    \centering
    $\mu\mu - 2b$ \\
    \includegraphics[width=0.49\textwidth]{chapters/Analysis/sectionPlots/figures/kinematics_pickles/mumu/2b/mumu_2b_lepton1_pt.pdf}
    \includegraphics[width=0.49\textwidth]{chapters/Analysis/sectionPlots/figures/kinematics_pickles/mumu/2b/mumu_2b_lepton1_eta.pdf}
    \includegraphics[width=0.49\textwidth]{chapters/Analysis/sectionPlots/figures/kinematics_pickles/mumu/2b/mumu_2b_lepton2_pt.pdf}
    \includegraphics[width=0.49\textwidth]{chapters/Analysis/sectionPlots/figures/kinematics_pickles/mumu/2b/mumu_2b_lepton2_eta.pdf}
    \includegraphics[width=0.49\textwidth]{chapters/Analysis/sectionPlots/figures/kinematics_pickles/mumu/2b/mumu_2b_nJets.pdf}
    \includegraphics[width=0.49\textwidth]{chapters/Analysis/sectionPlots/figures/kinematics_pickles/mumu/2b/mumu_2b_nBJets.pdf}
    
    \caption{$\mu\mu$ channel with $n_j\geq2, n_b\geq2$.}
\end{figure}


%  mutau channel
\begin{figure}[ht]
    \centering
    $\mu\tau - 1b$ \\
    \includegraphics[width=0.49\textwidth]{chapters/Analysis/sectionPlots/figures/kinematics_pickles/mutau/1b/mutau_1b_lepton1_pt.pdf}
    \includegraphics[width=0.49\textwidth]{chapters/Analysis/sectionPlots/figures/kinematics_pickles/mutau/1b/mutau_1b_lepton1_eta.pdf}
    \includegraphics[width=0.49\textwidth]{chapters/Analysis/sectionPlots/figures/kinematics_pickles/mutau/1b/mutau_1b_lepton2_pt.pdf}
    \includegraphics[width=0.49\textwidth]{chapters/Analysis/sectionPlots/figures/kinematics_pickles/mutau/1b/mutau_1b_lepton2_eta.pdf}
    \includegraphics[width=0.49\textwidth]{chapters/Analysis/sectionPlots/figures/kinematics_pickles/mutau/1b/mutau_1b_nJets.pdf}
    \includegraphics[width=0.49\textwidth]{chapters/Analysis/sectionPlots/figures/kinematics_pickles/mutau/1b/mutau_1b_nBJets.pdf}
    
    \caption{$\mu\tau$ channel with $n_j\geq2, n_b=1$.}
\end{figure}

\begin{figure}[ht]
    \centering
    $\mu\tau - 2b$ \\
    \includegraphics[width=0.49\textwidth]{chapters/Analysis/sectionPlots/figures/kinematics_pickles/mutau/2b/mutau_2b_lepton1_pt.pdf}
    \includegraphics[width=0.49\textwidth]{chapters/Analysis/sectionPlots/figures/kinematics_pickles/mutau/2b/mutau_2b_lepton1_eta.pdf}
    \includegraphics[width=0.49\textwidth]{chapters/Analysis/sectionPlots/figures/kinematics_pickles/mutau/2b/mutau_2b_lepton2_pt.pdf}
    \includegraphics[width=0.49\textwidth]{chapters/Analysis/sectionPlots/figures/kinematics_pickles/mutau/2b/mutau_2b_lepton2_eta.pdf}
    \includegraphics[width=0.49\textwidth]{chapters/Analysis/sectionPlots/figures/kinematics_pickles/mutau/2b/mutau_2b_nJets.pdf}
    \includegraphics[width=0.49\textwidth]{chapters/Analysis/sectionPlots/figures/kinematics_pickles/mutau/2b/mutau_2b_nBJets.pdf}
    
    \caption{$\mu\tau$ channel with $n_j\geq2, n_b\geq2$.}
\end{figure}


% muj channel
\begin{figure}[ht]
    \centering
    $\mu j- 1b$ \\
    \includegraphics[width=0.49\textwidth]{chapters/Analysis/sectionPlots/figures/kinematics_pickles/mu4j/1b/mu4j_1b_lepton1_pt.pdf}
    \includegraphics[width=0.49\textwidth]{chapters/Analysis/sectionPlots/figures/kinematics_pickles/mu4j/1b/mu4j_1b_lepton1_eta.pdf}
    \includegraphics[width=0.49\textwidth]{chapters/Analysis/sectionPlots/figures/kinematics_pickles/mu4j/1b/mu4j_1b_nJets.pdf}
    \includegraphics[width=0.49\textwidth]{chapters/Analysis/sectionPlots/figures/kinematics_pickles/mu4j/1b/mu4j_1b_nBJets.pdf}
    
    \caption{$\mu$jet channel with $n_j\geq4, n_b=1$.}
\end{figure}

\begin{figure}[ht]
    \centering
    $\mu j - 2b$ \\
    \includegraphics[width=0.49\textwidth]{chapters/Analysis/sectionPlots/figures/kinematics_pickles/mu4j/2b/mu4j_2b_lepton1_pt.pdf}
    \includegraphics[width=0.49\textwidth]{chapters/Analysis/sectionPlots/figures/kinematics_pickles/mu4j/2b/mu4j_2b_lepton1_eta.pdf}
    \includegraphics[width=0.49\textwidth]{chapters/Analysis/sectionPlots/figures/kinematics_pickles/mu4j/2b/mu4j_2b_nJets.pdf}
    \includegraphics[width=0.49\textwidth]{chapters/Analysis/sectionPlots/figures/kinematics_pickles/mu4j/2b/mu4j_2b_nBJets.pdf}
    
    \caption{$\mu$jet channel with $n_j\geq4, n_b\geq2$.}
\end{figure}













%  ee channel
\begin{figure}[ht]
    \centering
    $ee - 1b$ \\
    \includegraphics[width=0.49\textwidth]{chapters/Analysis/sectionPlots/figures/kinematics_pickles/ee/1b/ee_1b_lepton1_pt.pdf}
    \includegraphics[width=0.49\textwidth]{chapters/Analysis/sectionPlots/figures/kinematics_pickles/ee/1b/ee_1b_lepton1_eta.pdf}
    \includegraphics[width=0.49\textwidth]{chapters/Analysis/sectionPlots/figures/kinematics_pickles/ee/1b/ee_1b_lepton2_pt.pdf}
    \includegraphics[width=0.49\textwidth]{chapters/Analysis/sectionPlots/figures/kinematics_pickles/ee/1b/ee_1b_lepton2_eta.pdf}
    \includegraphics[width=0.49\textwidth]{chapters/Analysis/sectionPlots/figures/kinematics_pickles/ee/1b/ee_1b_nJets.pdf}
    \includegraphics[width=0.49\textwidth]{chapters/Analysis/sectionPlots/figures/kinematics_pickles/ee/1b/ee_1b_nBJets.pdf}
    
    \caption{$ee$ channel with $n_j\geq4, n_b=1$.}
\end{figure}

\begin{figure}[ht]
    \centering
    $ee - 2b$ \\
    \includegraphics[width=0.49\textwidth]{chapters/Analysis/sectionPlots/figures/kinematics_pickles/ee/2b/ee_2b_lepton1_pt.pdf}
    \includegraphics[width=0.49\textwidth]{chapters/Analysis/sectionPlots/figures/kinematics_pickles/ee/2b/ee_2b_lepton1_eta.pdf}
    \includegraphics[width=0.49\textwidth]{chapters/Analysis/sectionPlots/figures/kinematics_pickles/ee/2b/ee_2b_lepton2_pt.pdf}
    \includegraphics[width=0.49\textwidth]{chapters/Analysis/sectionPlots/figures/kinematics_pickles/ee/2b/ee_2b_lepton2_eta.pdf}
    \includegraphics[width=0.49\textwidth]{chapters/Analysis/sectionPlots/figures/kinematics_pickles/ee/2b/ee_2b_nJets.pdf}
    \includegraphics[width=0.49\textwidth]{chapters/Analysis/sectionPlots/figures/kinematics_pickles/ee/2b/ee_2b_nBJets.pdf}
    
    \caption{$ee$ channel with $n_j\geq2, n_b\geq2$.}
\end{figure}

%  emu channel
\begin{figure}[ht]
    \centering
    $e \mu - 1b$ \\
    \includegraphics[width=0.49\textwidth]{chapters/Analysis/sectionPlots/figures/kinematics_pickles/emu2/1b/emu2_1b_lepton1_pt.pdf}
    \includegraphics[width=0.49\textwidth]{chapters/Analysis/sectionPlots/figures/kinematics_pickles/emu2/1b/emu2_1b_lepton1_eta.pdf}
    \includegraphics[width=0.49\textwidth]{chapters/Analysis/sectionPlots/figures/kinematics_pickles/emu2/1b/emu2_1b_lepton2_pt.pdf}
    \includegraphics[width=0.49\textwidth]{chapters/Analysis/sectionPlots/figures/kinematics_pickles/emu2/1b/emu2_1b_lepton2_eta.pdf}
    \includegraphics[width=0.49\textwidth]{chapters/Analysis/sectionPlots/figures/kinematics_pickles/emu2/1b/emu2_1b_nJets.pdf}
    \includegraphics[width=0.49\textwidth]{chapters/Analysis/sectionPlots/figures/kinematics_pickles/emu2/1b/emu2_1b_nBJets.pdf}
    
    \caption{$e\mu$ channel with $n_j\geq4, n_b=1$.}
\end{figure}

\begin{figure}[ht]
    \centering
    $e\mu - 2b$ \\
    \includegraphics[width=0.49\textwidth]{chapters/Analysis/sectionPlots/figures/kinematics_pickles/emu2/2b/emu2_2b_lepton1_pt.pdf}
    \includegraphics[width=0.49\textwidth]{chapters/Analysis/sectionPlots/figures/kinematics_pickles/emu2/2b/emu2_2b_lepton1_eta.pdf}
    \includegraphics[width=0.49\textwidth]{chapters/Analysis/sectionPlots/figures/kinematics_pickles/emu2/2b/emu2_2b_lepton2_pt.pdf}
    \includegraphics[width=0.49\textwidth]{chapters/Analysis/sectionPlots/figures/kinematics_pickles/emu2/2b/emu2_2b_lepton2_eta.pdf}
    \includegraphics[width=0.49\textwidth]{chapters/Analysis/sectionPlots/figures/kinematics_pickles/emu2/2b/emu2_2b_nJets.pdf}
    \includegraphics[width=0.49\textwidth]{chapters/Analysis/sectionPlots/figures/kinematics_pickles/emu2/2b/emu2_2b_nBJets.pdf}
    
    \caption{$e\mu$ channel with $n_j\geq2, n_b\geq2$.}
\end{figure}


%  etau channel
\begin{figure}[ht]
    \centering
    $e\tau - 1b$ \\
    \includegraphics[width=0.49\textwidth]{chapters/Analysis/sectionPlots/figures/kinematics_pickles/etau/1b/etau_1b_lepton1_pt.pdf}
    \includegraphics[width=0.49\textwidth]{chapters/Analysis/sectionPlots/figures/kinematics_pickles/etau/1b/etau_1b_lepton1_eta.pdf}
    \includegraphics[width=0.49\textwidth]{chapters/Analysis/sectionPlots/figures/kinematics_pickles/etau/1b/etau_1b_lepton2_pt.pdf}
    \includegraphics[width=0.49\textwidth]{chapters/Analysis/sectionPlots/figures/kinematics_pickles/etau/1b/etau_1b_lepton2_eta.pdf}
    \includegraphics[width=0.49\textwidth]{chapters/Analysis/sectionPlots/figures/kinematics_pickles/etau/1b/etau_1b_nJets.pdf}
    \includegraphics[width=0.49\textwidth]{chapters/Analysis/sectionPlots/figures/kinematics_pickles/etau/1b/etau_1b_nBJets.pdf}
    
    \caption{$e\tau$ channel with $n_j\geq4, n_b=1$.}
\end{figure}

\begin{figure}[ht]
    \centering
    $e\tau - 2b$ \\
    \includegraphics[width=0.49\textwidth]{chapters/Analysis/sectionPlots/figures/kinematics_pickles/etau/2b/etau_2b_lepton1_pt.pdf}
    \includegraphics[width=0.49\textwidth]{chapters/Analysis/sectionPlots/figures/kinematics_pickles/etau/2b/etau_2b_lepton1_eta.pdf}
    \includegraphics[width=0.49\textwidth]{chapters/Analysis/sectionPlots/figures/kinematics_pickles/etau/2b/etau_2b_lepton2_pt.pdf}
    \includegraphics[width=0.49\textwidth]{chapters/Analysis/sectionPlots/figures/kinematics_pickles/etau/2b/etau_2b_lepton2_eta.pdf}
    \includegraphics[width=0.49\textwidth]{chapters/Analysis/sectionPlots/figures/kinematics_pickles/etau/2b/etau_2b_nJets.pdf}
    \includegraphics[width=0.49\textwidth]{chapters/Analysis/sectionPlots/figures/kinematics_pickles/etau/2b/etau_2b_nBJets.pdf}
    
    \caption{$e\tau$ channel with $n_j\geq2, n_b\geq2$.}
\end{figure}


% ej channel
\begin{figure}[ht]
    \centering
    $e j- 1b$ \\
    \includegraphics[width=0.49\textwidth]{chapters/Analysis/sectionPlots/figures/kinematics_pickles/e4j/1b/e4j_1b_lepton1_pt.pdf}
    \includegraphics[width=0.49\textwidth]{chapters/Analysis/sectionPlots/figures/kinematics_pickles/e4j/1b/e4j_1b_lepton1_eta.pdf}
    \includegraphics[width=0.49\textwidth]{chapters/Analysis/sectionPlots/figures/kinematics_pickles/e4j/1b/e4j_1b_nJets.pdf}
    \includegraphics[width=0.49\textwidth]{chapters/Analysis/sectionPlots/figures/kinematics_pickles/e4j/1b/e4j_1b_nBJets.pdf}
    
    \caption{$e$jet channel with $n_j\geq4, n_b=1$.}
\end{figure}

\begin{figure}[ht]
    \centering
    $e j - 2b$ \\
    \includegraphics[width=0.49\textwidth]{chapters/Analysis/sectionPlots/figures/kinematics_pickles/e4j/2b/e4j_2b_lepton1_pt.pdf}
    \includegraphics[width=0.49\textwidth]{chapters/Analysis/sectionPlots/figures/kinematics_pickles/e4j/2b/e4j_2b_lepton1_eta.pdf}
    \includegraphics[width=0.49\textwidth]{chapters/Analysis/sectionPlots/figures/kinematics_pickles/e4j/2b/e4j_2b_nJets.pdf}
    \includegraphics[width=0.49\textwidth]{chapters/Analysis/sectionPlots/figures/kinematics_pickles/e4j/2b/e4j_2b_nBJets.pdf}
    
    \caption{$e$jet channel with $n_j\geq2, n_b\geq2$.}
\end{figure}

\FloatBarrier



\section{Kinematics Plots in Shape Analysis}

\subsubsection{ee}

\begin{figure}[htb!]
    \centering
    \includegraphics[width=0.4\textwidth]{chapters/Analysis/sectionPlots/figures/data_mc_overlays/ee_2016_cat_gt2_eq1_b_signal_linear_lepton_lepton1_pt}
    \includegraphics[width=0.4\textwidth]{chapters/Analysis/sectionPlots/figures/data_mc_overlays/ee_2016_cat_gt2_eq1_b_signal_linear_lepton_lepton1_eta}

    \includegraphics[width=0.4\textwidth]{chapters/Analysis/sectionPlots/figures/data_mc_overlays/ee_2016_cat_gt2_eq1_b_signal_linear_lepton_lepton2_pt}
    \includegraphics[width=0.4\textwidth]{chapters/Analysis/sectionPlots/figures/data_mc_overlays/ee_2016_cat_gt2_eq1_b_signal_linear_lepton_lepton2_eta}
    \caption{\pt and $\eta$ distributions for leading (top) and trailing
    (bottom) electrons in the $ee$ channel with $N_{j} \geq 2$, $N_{b}
    = 1$, and Z boson veto.}
    \label{fig:analysis:plots:ee_1_kinematic}
\end{figure}

\begin{figure}[htb!]
    \centering
    \includegraphics[width=0.3\textwidth]{chapters/Analysis/sectionPlots/figures/data_mc_overlays/ee_2016_cat_gt2_eq1_b_signal_linear_lepton_dilepton1_mass}
    \includegraphics[width=0.3\textwidth]{chapters/Analysis/sectionPlots/figures/data_mc_overlays/ee_2016_cat_gt2_eq1_b_signal_linear_lepton_dilepton1_pt}
    \includegraphics[width=0.3\textwidth]{chapters/Analysis/sectionPlots/figures/data_mc_overlays/ee_2016_cat_gt2_eq1_b_signal_linear_lepton_dilepton1_delta_r}
    \caption{Dielectron mass, \pt, and $\Delta R$ in the $ee$ channel
    with $N_{j} \geq 2$, $N_{b} = 1$, and Z boson veto.}
    \label{fig:analysis:plots:ee_1_dilepton}
\end{figure}

\begin{figure}[htb!]
    \centering
    \includegraphics[width=0.4\textwidth]{chapters/Analysis/sectionPlots/figures/data_mc_overlays/ee_2016_cat_gt2_gt2_b_signal_linear_lepton_lepton1_pt}
    \includegraphics[width=0.4\textwidth]{chapters/Analysis/sectionPlots/figures/data_mc_overlays/ee_2016_cat_gt2_gt2_b_signal_linear_lepton_lepton1_eta}

    \includegraphics[width=0.4\textwidth]{chapters/Analysis/sectionPlots/figures/data_mc_overlays/ee_2016_cat_gt2_gt2_b_signal_linear_lepton_lepton2_pt}
    \includegraphics[width=0.4\textwidth]{chapters/Analysis/sectionPlots/figures/data_mc_overlays/ee_2016_cat_gt2_gt2_b_signal_linear_lepton_lepton2_eta}
    \caption{\pt and $\eta$ distributions for leading (top) and trailing
    (bottom) electrons in the $ee$ channel with $N_{j} \geq 2$, $N_{b}
    \geq 2$, and Z boson veto.}
    \label{fig:analysis:plots:ee_2_kinematic}
\end{figure}

\begin{figure}[htb!]
    \centering
    \includegraphics[width=0.3\textwidth]{chapters/Analysis/sectionPlots/figures/data_mc_overlays/ee_2016_cat_gt2_gt2_b_signal_linear_lepton_dilepton1_mass}
    \includegraphics[width=0.3\textwidth]{chapters/Analysis/sectionPlots/figures/data_mc_overlays/ee_2016_cat_gt2_gt2_b_signal_linear_lepton_dilepton1_pt}
    \includegraphics[width=0.3\textwidth]{chapters/Analysis/sectionPlots/figures/data_mc_overlays/ee_2016_cat_gt2_gt2_b_signal_linear_lepton_dilepton1_delta_r}
    \caption{Dielectron mass, \pt, and $\Delta R$ in the $ee$ channel
    with $N_{j} \geq 2$, $N_{b} \geq 2$, and Z boson veto.}
    \label{fig:analysis:plots:ee_2_dilepton}
\end{figure}

\begin{figure}[htb!]
    \centering
    \includegraphics[width=0.4\textwidth]{chapters/Analysis/sectionPlots/figures/data_mc_overlays/ee_2016_inclusive_linear_jet_n_bjets}
    \includegraphics[width=0.4\textwidth]{chapters/Analysis/sectionPlots/figures/data_mc_overlays/ee_2016_inclusive_linear_jet_n_jets}
    \includegraphics[width=0.3\textwidth]{chapters/Analysis/sectionPlots/figures/data_mc_overlays/ee_2016_inclusive_linear_misc_met_mag}
    \caption{Multiplicity of b tagged jets, non-tagged jets, and MET in
    $ee$ channel with $N_{j} \geq 2$.}
    \label{fig:analysis:plots:ee_jetmet}
\end{figure}


\FloatBarrier
\input{chapters/Analysis/sectionPlots/figures/mumu_plots}
\FloatBarrier
\subsubsection{emu}

\begin{figure}[htb!]
    \centering
    \includegraphics[width=0.4\textwidth]{chapters/Appendix/sectionPlots/figures/data_mc_overlays/emu_2016_cat_eq0_eq0_a_signal_linear_lepton_lepton1_pt}
    \includegraphics[width=0.4\textwidth]{chapters/Appendix/sectionPlots/figures/data_mc_overlays/emu_2016_cat_eq0_eq0_a_signal_linear_lepton_lepton1_eta}

    \includegraphics[width=0.4\textwidth]{chapters/Appendix/sectionPlots/figures/data_mc_overlays/emu_2016_cat_eq0_eq0_a_signal_linear_lepton_lepton2_pt}
    \includegraphics[width=0.4\textwidth]{chapters/Appendix/sectionPlots/figures/data_mc_overlays/emu_2016_cat_eq0_eq0_a_signal_linear_lepton_lepton2_eta}
    \caption{\pt and $\eta$ distributions for leading (top) and trailing
    (bottom) electrons in the $e\mu$ channel with $N_{j} = 0$.}
    \label{fig:emu_1_kinematic}
\end{figure}

\begin{figure}[htb!]
    \centering
    \includegraphics[width=0.3\textwidth]{chapters/Appendix/sectionPlots/figures/data_mc_overlays/emu_2016_cat_eq0_eq0_a_signal_linear_lepton_dilepton1_mass}
    \includegraphics[width=0.3\textwidth]{chapters/Appendix/sectionPlots/figures/data_mc_overlays/emu_2016_cat_eq0_eq0_a_signal_linear_lepton_dilepton1_pt}
    \includegraphics[width=0.3\textwidth]{chapters/Appendix/sectionPlots/figures/data_mc_overlays/emu_2016_cat_eq0_eq0_a_signal_linear_lepton_dilepton1_delta_r}
    \caption{Dielectron mass, \pt, and $\Delta R$ in the $e\mu$ channel
    with $N_{j} = 0$.}
    \label{fig:emu_1_dilepton}
\end{figure}

\begin{figure}[htb!]
    \centering
    \includegraphics[width=0.4\textwidth]{chapters/Appendix/sectionPlots/figures/data_mc_overlays/emu_2016_cat_eq1_eq0_a_signal_linear_lepton_lepton1_pt}
    \includegraphics[width=0.4\textwidth]{chapters/Appendix/sectionPlots/figures/data_mc_overlays/emu_2016_cat_eq1_eq0_a_signal_linear_lepton_lepton1_eta}

    \includegraphics[width=0.4\textwidth]{chapters/Appendix/sectionPlots/figures/data_mc_overlays/emu_2016_cat_eq1_eq0_a_signal_linear_lepton_lepton2_pt}
    \includegraphics[width=0.4\textwidth]{chapters/Appendix/sectionPlots/figures/data_mc_overlays/emu_2016_cat_eq1_eq0_a_signal_linear_lepton_lepton2_eta}
    \caption{\pt and $\eta$ distributions for leading (top) and trailing
        (bottom) electrons in the $e\mu$ channel with $N_{j} = 1$ and
        $N_{b} = 0$.}
    \label{fig:emu_2_kinematic}
\end{figure}

\begin{figure}[htb!]
    \centering
    \includegraphics[width=0.3\textwidth]{chapters/Appendix/sectionPlots/figures/data_mc_overlays/emu_2016_cat_eq1_eq0_a_signal_linear_lepton_dilepton1_mass}
    \includegraphics[width=0.3\textwidth]{chapters/Appendix/sectionPlots/figures/data_mc_overlays/emu_2016_cat_eq1_eq0_a_signal_linear_lepton_dilepton1_pt}
    \includegraphics[width=0.3\textwidth]{chapters/Appendix/sectionPlots/figures/data_mc_overlays/emu_2016_cat_eq1_eq0_a_signal_linear_lepton_dilepton1_delta_r}
    \caption{Dielectron mass, \pt, and $\Delta R$ in the $e\mu$ channel
    with $N_{j} = 0$ and $N_{b} = 0$.}
    \label{fig:emu_2_dilepton}
\end{figure}

\begin{figure}[htb!]
    \centering
    \includegraphics[width=0.4\textwidth]{chapters/Appendix/sectionPlots/figures/data_mc_overlays/emu_2016_cat_eq1_eq1_a_signal_linear_lepton_lepton1_pt}
    \includegraphics[width=0.4\textwidth]{chapters/Appendix/sectionPlots/figures/data_mc_overlays/emu_2016_cat_eq1_eq1_a_signal_linear_lepton_lepton1_eta}

    \includegraphics[width=0.4\textwidth]{chapters/Appendix/sectionPlots/figures/data_mc_overlays/emu_2016_cat_eq1_eq1_a_signal_linear_lepton_lepton2_pt}
    \includegraphics[width=0.4\textwidth]{chapters/Appendix/sectionPlots/figures/data_mc_overlays/emu_2016_cat_eq1_eq1_a_signal_linear_lepton_lepton2_eta}
    \caption{\pt and $\eta$ distributions for leading (top) and trailing
        (bottom) electrons in the $e\mu$ channel with $N_{j} = 1$ and
        $N_{b} = 1$.}
    \label{fig:emu_3_kinematic}
\end{figure}

\begin{figure}[htb!]
    \centering
    \includegraphics[width=0.3\textwidth]{chapters/Appendix/sectionPlots/figures/data_mc_overlays/emu_2016_cat_eq1_eq1_a_signal_linear_lepton_dilepton1_mass}
    \includegraphics[width=0.3\textwidth]{chapters/Appendix/sectionPlots/figures/data_mc_overlays/emu_2016_cat_eq1_eq1_a_signal_linear_lepton_dilepton1_pt}
    \includegraphics[width=0.3\textwidth]{chapters/Appendix/sectionPlots/figures/data_mc_overlays/emu_2016_cat_eq1_eq1_a_signal_linear_lepton_dilepton1_delta_r}
    \caption{Dielectron mass, \pt, and $\Delta R$ in the $e\mu$ channel
    with $N_{j} = 1$ and $N_{b} = 1$.}
    \label{fig:emu_3_dilepton}
\end{figure}

\begin{figure}[htb!]
    \centering
    \includegraphics[width=0.4\textwidth]{chapters/Appendix/sectionPlots/figures/data_mc_overlays/emu_2016_cat_gt2_eq0_signal_linear_lepton_lepton1_pt}
    \includegraphics[width=0.4\textwidth]{chapters/Appendix/sectionPlots/figures/data_mc_overlays/emu_2016_cat_gt2_eq0_signal_linear_lepton_lepton1_eta}

    \includegraphics[width=0.4\textwidth]{chapters/Appendix/sectionPlots/figures/data_mc_overlays/emu_2016_cat_gt2_eq0_signal_linear_lepton_lepton2_pt}
    \includegraphics[width=0.4\textwidth]{chapters/Appendix/sectionPlots/figures/data_mc_overlays/emu_2016_cat_gt2_eq0_signal_linear_lepton_lepton2_eta}
    \caption{\pt and $\eta$ distributions for leading (top) and trailing
        (bottom) electrons in the $e\mu$ channel with $N_{j} \geq 2$ and
        $N_{b} = 0$.}
    \label{fig:emu_4_kinematic}
\end{figure}

\begin{figure}[htb!]
    \centering
    \includegraphics[width=0.3\textwidth]{chapters/Appendix/sectionPlots/figures/data_mc_overlays/emu_2016_cat_gt2_eq0_signal_linear_lepton_dilepton1_mass}
    \includegraphics[width=0.3\textwidth]{chapters/Appendix/sectionPlots/figures/data_mc_overlays/emu_2016_cat_gt2_eq0_signal_linear_lepton_dilepton1_pt}
    \includegraphics[width=0.3\textwidth]{chapters/Appendix/sectionPlots/figures/data_mc_overlays/emu_2016_cat_gt2_eq0_signal_linear_lepton_dilepton1_delta_r}
    \caption{Dielectron mass, \pt, and $\Delta R$ in the $e\mu$ channel
    with $N_{j} \geq 2$ and $N_{b} = 0$.}
    \label{fig:emu_4_dilepton}
\end{figure}

\begin{figure}[htb!]
    \centering
    \includegraphics[width=0.4\textwidth]{chapters/Appendix/sectionPlots/figures/data_mc_overlays/emu_2016_cat_gt2_eq1_a_signal_linear_lepton_lepton1_pt}
    \includegraphics[width=0.4\textwidth]{chapters/Appendix/sectionPlots/figures/data_mc_overlays/emu_2016_cat_gt2_eq1_a_signal_linear_lepton_lepton1_eta}

    \includegraphics[width=0.4\textwidth]{chapters/Appendix/sectionPlots/figures/data_mc_overlays/emu_2016_cat_gt2_eq1_a_signal_linear_lepton_lepton2_pt}
    \includegraphics[width=0.4\textwidth]{chapters/Appendix/sectionPlots/figures/data_mc_overlays/emu_2016_cat_gt2_eq1_a_signal_linear_lepton_lepton2_eta}
    \caption{\pt and $\eta$ distributions for leading (top) and trailing
        (bottom) electrons in the $e\mu$ channel with $N_{j} \geq 2$ and
        $N_{b} = 1$.}
    \label{fig:emu_5_kinematic}
\end{figure}

\begin{figure}[htb!]
    \centering
    \includegraphics[width=0.3\textwidth]{chapters/Appendix/sectionPlots/figures/data_mc_overlays/emu_2016_cat_gt2_eq1_a_signal_linear_lepton_dilepton1_mass}
    \includegraphics[width=0.3\textwidth]{chapters/Appendix/sectionPlots/figures/data_mc_overlays/emu_2016_cat_gt2_eq1_a_signal_linear_lepton_dilepton1_pt}
    \includegraphics[width=0.3\textwidth]{chapters/Appendix/sectionPlots/figures/data_mc_overlays/emu_2016_cat_gt2_eq1_a_signal_linear_lepton_dilepton1_delta_r}
    \caption{Dielectron mass, \pt, and $\Delta R$ in the $e\mu$ channel
    with $N_{j} \geq 2$ and $N_{b} = 1$.}
    \label{fig:emu_5_dilepton}
\end{figure}

\begin{figure}[htb!]
    \centering
    \includegraphics[width=0.4\textwidth]{chapters/Appendix/sectionPlots/figures/data_mc_overlays/emu_2016_cat_gt2_gt2_a_signal_linear_lepton_lepton1_pt}
    \includegraphics[width=0.4\textwidth]{chapters/Appendix/sectionPlots/figures/data_mc_overlays/emu_2016_cat_gt2_gt2_a_signal_linear_lepton_lepton1_eta}

    \includegraphics[width=0.4\textwidth]{chapters/Appendix/sectionPlots/figures/data_mc_overlays/emu_2016_cat_gt2_gt2_a_signal_linear_lepton_lepton2_pt}
    \includegraphics[width=0.4\textwidth]{chapters/Appendix/sectionPlots/figures/data_mc_overlays/emu_2016_cat_gt2_gt2_a_signal_linear_lepton_lepton2_eta}
    \caption{\pt and $\eta$ distributions for leading (top) and trailing
        (bottom) electrons in the $e\mu$ channel with $N_{j} \geq 2$ and
        $N_{b} = 1$.}
    \label{fig:emu_6_kinematic}
\end{figure}

\begin{figure}[htb!]
    \centering
    \includegraphics[width=0.3\textwidth]{chapters/Appendix/sectionPlots/figures/data_mc_overlays/emu_2016_cat_gt2_gt2_a_signal_linear_lepton_dilepton1_mass}
    \includegraphics[width=0.3\textwidth]{chapters/Appendix/sectionPlots/figures/data_mc_overlays/emu_2016_cat_gt2_gt2_a_signal_linear_lepton_dilepton1_pt}
    \includegraphics[width=0.3\textwidth]{chapters/Appendix/sectionPlots/figures/data_mc_overlays/emu_2016_cat_gt2_gt2_a_signal_linear_lepton_dilepton1_delta_r}
    \caption{Dielectron mass, \pt, and $\Delta R$ in the $e\mu$ channel
    with $N_{j} \geq 2$ and $N_{b} = 1$.}
    \label{fig:emu_6_dilepton}
\end{figure}

\begin{figure}[htb!]
    \centering
    \includegraphics[width=0.4\textwidth]{chapters/Appendix/sectionPlots/figures/data_mc_overlays/emu_2016_inclusive_linear_jet_n_bjets}
    \includegraphics[width=0.4\textwidth]{chapters/Appendix/sectionPlots/figures/data_mc_overlays/emu_2016_inclusive_linear_jet_n_jets}
    \includegraphics[width=0.3\textwidth]{chapters/Appendix/sectionPlots/figures/data_mc_overlays/emu_2016_inclusive_linear_misc_met_mag}
    \caption{Multiplicity of b tagged jets, non-tagged jets, and MET in
    $e\mu$ channel.}
    \label{fig:emu_jetmet}
\end{figure}


\FloatBarrier
\subsubsection{etau}

\begin{figure}[htb!]
    \centering
    \includegraphics[width=0.4\textwidth]{chapters/Analysis/sectionPlots/figures/data_mc_overlays/etau_2016_cat_eq0_eq0_signal_linear_lepton_lepton1_pt}
    \includegraphics[width=0.4\textwidth]{chapters/Analysis/sectionPlots/figures/data_mc_overlays/etau_2016_cat_eq0_eq0_signal_linear_lepton_lepton1_eta}

    \includegraphics[width=0.4\textwidth]{chapters/Analysis/sectionPlots/figures/data_mc_overlays/etau_2016_cat_eq0_eq0_signal_linear_lepton_lepton2_pt}
    \includegraphics[width=0.4\textwidth]{chapters/Analysis/sectionPlots/figures/data_mc_overlays/etau_2016_cat_eq0_eq0_signal_linear_lepton_lepton2_eta}
    \caption{\pt and $\eta$ distributions for leading (top) and trailing
    (bottom) electrons in the $e\tau$ channel with $N_{j} = 0$.}
    \label{fig:analysis:plots:etau_1_kinematic}
\end{figure}

\begin{figure}[htb!]
    \centering
    \includegraphics[width=0.3\textwidth]{chapters/Analysis/sectionPlots/figures/data_mc_overlays/etau_2016_cat_eq0_eq0_signal_linear_lepton_dilepton1_mass}
    \includegraphics[width=0.3\textwidth]{chapters/Analysis/sectionPlots/figures/data_mc_overlays/etau_2016_cat_eq0_eq0_signal_linear_lepton_dilepton1_pt}
    \includegraphics[width=0.3\textwidth]{chapters/Analysis/sectionPlots/figures/data_mc_overlays/etau_2016_cat_eq0_eq0_signal_linear_lepton_dilepton1_delta_r}
    \caption{Dielectron mass, \pt, and $\Delta R$ in the $e\tau$ channel
    with $N_{j} = 0$.}
    \label{fig:analysis:plots:etau_1_dilepton}
\end{figure}

\begin{figure}[htb!]
    \centering
    \includegraphics[width=0.4\textwidth]{chapters/Analysis/sectionPlots/figures/data_mc_overlays/etau_2016_cat_eq1_eq0_signal_linear_lepton_lepton1_pt}
    \includegraphics[width=0.4\textwidth]{chapters/Analysis/sectionPlots/figures/data_mc_overlays/etau_2016_cat_eq1_eq0_signal_linear_lepton_lepton1_eta}

    \includegraphics[width=0.4\textwidth]{chapters/Analysis/sectionPlots/figures/data_mc_overlays/etau_2016_cat_eq1_eq0_signal_linear_lepton_lepton2_pt}
    \includegraphics[width=0.4\textwidth]{chapters/Analysis/sectionPlots/figures/data_mc_overlays/etau_2016_cat_eq1_eq0_signal_linear_lepton_lepton2_eta}
    \caption{\pt and $\eta$ distributions for leading (top) and trailing
        (bottom) electrons in the $e\tau$ channel with $N_{j} = 1$ and
        $N_{b} = 0$.}
    \label{fig:analysis:plots:etau_2_kinematic}
\end{figure}

\begin{figure}[htb!]
    \centering
    \includegraphics[width=0.3\textwidth]{chapters/Analysis/sectionPlots/figures/data_mc_overlays/etau_2016_cat_eq1_eq0_signal_linear_lepton_dilepton1_mass}
    \includegraphics[width=0.3\textwidth]{chapters/Analysis/sectionPlots/figures/data_mc_overlays/etau_2016_cat_eq1_eq0_signal_linear_lepton_dilepton1_pt}
    \includegraphics[width=0.3\textwidth]{chapters/Analysis/sectionPlots/figures/data_mc_overlays/etau_2016_cat_eq1_eq0_signal_linear_lepton_dilepton1_delta_r}
    \caption{Dielectron mass, \pt, and $\Delta R$ in the $e\tau$ channel
    with $N_{j} = 0$ and $N_{b} = 0$.}
    \label{fig:analysis:plots:etau_2_dilepton}
\end{figure}

\begin{figure}[htb!]
    \centering
    \includegraphics[width=0.4\textwidth]{chapters/Analysis/sectionPlots/figures/data_mc_overlays/etau_2016_cat_eq1_eq1_signal_linear_lepton_lepton1_pt}
    \includegraphics[width=0.4\textwidth]{chapters/Analysis/sectionPlots/figures/data_mc_overlays/etau_2016_cat_eq1_eq1_signal_linear_lepton_lepton1_eta}

    \includegraphics[width=0.4\textwidth]{chapters/Analysis/sectionPlots/figures/data_mc_overlays/etau_2016_cat_eq1_eq1_signal_linear_lepton_lepton2_pt}
    \includegraphics[width=0.4\textwidth]{chapters/Analysis/sectionPlots/figures/data_mc_overlays/etau_2016_cat_eq1_eq1_signal_linear_lepton_lepton2_eta}
    \caption{\pt and $\eta$ distributions for leading (top) and trailing
        (bottom) electrons in the $e\tau$ channel with $N_{j} = 1$ and
        $N_{b} = 1$.}
    \label{fig:analysis:plots:etau_3_kinematic}
\end{figure}

\begin{figure}[htb!]
    \centering
    \includegraphics[width=0.3\textwidth]{chapters/Analysis/sectionPlots/figures/data_mc_overlays/etau_2016_cat_eq1_eq1_signal_linear_lepton_dilepton1_mass}
    \includegraphics[width=0.3\textwidth]{chapters/Analysis/sectionPlots/figures/data_mc_overlays/etau_2016_cat_eq1_eq1_signal_linear_lepton_dilepton1_pt}
    \includegraphics[width=0.3\textwidth]{chapters/Analysis/sectionPlots/figures/data_mc_overlays/etau_2016_cat_eq1_eq1_signal_linear_lepton_dilepton1_delta_r}
    \caption{Dielectron mass, \pt, and $\Delta R$ in the $e\tau$ channel
    with $N_{j} = 1$ and $N_{b} = 1$.}
    \label{fig:analysis:plots:etau_3_dilepton}
\end{figure}

\begin{figure}[htb!]
    \centering
    \includegraphics[width=0.4\textwidth]{chapters/Analysis/sectionPlots/figures/data_mc_overlays/etau_2016_cat_gt2_eq0_signal_linear_lepton_lepton1_pt}
    \includegraphics[width=0.4\textwidth]{chapters/Analysis/sectionPlots/figures/data_mc_overlays/etau_2016_cat_gt2_eq0_signal_linear_lepton_lepton1_eta}

    \includegraphics[width=0.4\textwidth]{chapters/Analysis/sectionPlots/figures/data_mc_overlays/etau_2016_cat_gt2_eq0_signal_linear_lepton_lepton2_pt}
    \includegraphics[width=0.4\textwidth]{chapters/Analysis/sectionPlots/figures/data_mc_overlays/etau_2016_cat_gt2_eq0_signal_linear_lepton_lepton2_eta}
    \caption{\pt and $\eta$ distributions for leading (top) and trailing
        (bottom) electrons in the $e\tau$ channel with $N_{j} \geq 2$ and
        $N_{b} = 0$.}
    \label{fig:analysis:plots:etau_4_kinematic}
\end{figure}

\begin{figure}[htb!]
    \centering
    \includegraphics[width=0.3\textwidth]{chapters/Analysis/sectionPlots/figures/data_mc_overlays/etau_2016_cat_gt2_eq0_signal_linear_lepton_dilepton1_mass}
    \includegraphics[width=0.3\textwidth]{chapters/Analysis/sectionPlots/figures/data_mc_overlays/etau_2016_cat_gt2_eq0_signal_linear_lepton_dilepton1_pt}
    \includegraphics[width=0.3\textwidth]{chapters/Analysis/sectionPlots/figures/data_mc_overlays/etau_2016_cat_gt2_eq0_signal_linear_lepton_dilepton1_delta_r}
    \caption{Dielectron mass, \pt, and $\Delta R$ in the $e\tau$ channel
    with $N_{j} \geq 2$ and $N_{b} = 0$.}
    \label{fig:analysis:plots:etau_4_dilepton}
\end{figure}

\begin{figure}[htb!]
    \centering
    \includegraphics[width=0.4\textwidth]{chapters/Analysis/sectionPlots/figures/data_mc_overlays/etau_2016_cat_eq2_eq1_signal_linear_lepton_lepton1_pt}
    \includegraphics[width=0.4\textwidth]{chapters/Analysis/sectionPlots/figures/data_mc_overlays/etau_2016_cat_eq2_eq1_signal_linear_lepton_lepton1_eta}

    \includegraphics[width=0.4\textwidth]{chapters/Analysis/sectionPlots/figures/data_mc_overlays/etau_2016_cat_eq2_eq1_signal_linear_lepton_lepton2_pt}
    \includegraphics[width=0.4\textwidth]{chapters/Analysis/sectionPlots/figures/data_mc_overlays/etau_2016_cat_eq2_eq1_signal_linear_lepton_lepton2_eta}
    \caption{\pt and $\eta$ distributions for leading (top) and trailing
        (bottom) electrons in the $e\tau$ channel with $N_{j} = 2$ and
        $N_{b} = 1$.}
    \label{fig:analysis:plots:etau_5_kinematic}
\end{figure}

\begin{figure}[htb!]
    \centering
    \includegraphics[width=0.3\textwidth]{chapters/Analysis/sectionPlots/figures/data_mc_overlays/etau_2016_cat_eq2_eq1_signal_linear_lepton_dilepton1_mass}
    \includegraphics[width=0.3\textwidth]{chapters/Analysis/sectionPlots/figures/data_mc_overlays/etau_2016_cat_eq2_eq1_signal_linear_lepton_dilepton1_pt}
    \includegraphics[width=0.3\textwidth]{chapters/Analysis/sectionPlots/figures/data_mc_overlays/etau_2016_cat_eq2_eq1_signal_linear_lepton_dilepton1_delta_r}
    \caption{Dielectron mass, \pt, and $\Delta R$ in the $e\tau$ channel
    with $N_{j} = 2$ and $N_{b} = 1$.}
    \label{fig:analysis:plots:etau_5_dilepton}
\end{figure}

\begin{figure}[htb!]
    \centering
    \includegraphics[width=0.4\textwidth]{chapters/Analysis/sectionPlots/figures/data_mc_overlays/etau_2016_cat_eq2_eq2_signal_linear_lepton_lepton1_pt}
    \includegraphics[width=0.4\textwidth]{chapters/Analysis/sectionPlots/figures/data_mc_overlays/etau_2016_cat_eq2_eq2_signal_linear_lepton_lepton1_eta}

    \includegraphics[width=0.4\textwidth]{chapters/Analysis/sectionPlots/figures/data_mc_overlays/etau_2016_cat_eq2_eq2_signal_linear_lepton_lepton2_pt}
    \includegraphics[width=0.4\textwidth]{chapters/Analysis/sectionPlots/figures/data_mc_overlays/etau_2016_cat_eq2_eq2_signal_linear_lepton_lepton2_eta}
    \caption{\pt and $\eta$ distributions for leading (top) and trailing
        (bottom) electrons in the $e\tau$ channel with $N_{j} = 2$ and
        $N_{b} = 2$.}
    \label{fig:analysis:plots:etau_6_kinematic}
\end{figure}

\begin{figure}[htb!]
    \centering
    \includegraphics[width=0.3\textwidth]{chapters/Analysis/sectionPlots/figures/data_mc_overlays/etau_2016_cat_eq2_eq2_signal_linear_lepton_dilepton1_mass}
    \includegraphics[width=0.3\textwidth]{chapters/Analysis/sectionPlots/figures/data_mc_overlays/etau_2016_cat_eq2_eq2_signal_linear_lepton_dilepton1_pt}
    \includegraphics[width=0.3\textwidth]{chapters/Analysis/sectionPlots/figures/data_mc_overlays/etau_2016_cat_eq2_eq2_signal_linear_lepton_dilepton1_delta_r}
    \caption{Dielectron mass, \pt, and $\Delta R$ in the $e\tau$ channel
    with $N_{j} = 2$ and $N_{b} = 2$.}
    \label{fig:analysis:plots:etau_6_dilepton}
\end{figure}

\begin{figure}[htb!]
    \centering
    \includegraphics[width=0.4\textwidth]{chapters/Analysis/sectionPlots/figures/data_mc_overlays/etau_2016_cat_gt3_eq1_signal_linear_lepton_lepton1_pt}
    \includegraphics[width=0.4\textwidth]{chapters/Analysis/sectionPlots/figures/data_mc_overlays/etau_2016_cat_gt3_eq1_signal_linear_lepton_lepton1_eta}

    \includegraphics[width=0.4\textwidth]{chapters/Analysis/sectionPlots/figures/data_mc_overlays/etau_2016_cat_gt3_eq1_signal_linear_lepton_lepton2_pt}
    \includegraphics[width=0.4\textwidth]{chapters/Analysis/sectionPlots/figures/data_mc_overlays/etau_2016_cat_gt3_eq1_signal_linear_lepton_lepton2_eta}
    \caption{\pt and $\eta$ distributions for leading (top) and trailing
        (bottom) electrons in the $e\tau$ channel with $N_{j} \geq 3$ and
        $N_{b} = 1$.}
    \label{fig:analysis:plots:etau_7_kinematic}
\end{figure}

\begin{figure}[htb!]
    \centering
    \includegraphics[width=0.3\textwidth]{chapters/Analysis/sectionPlots/figures/data_mc_overlays/etau_2016_cat_gt3_eq1_signal_linear_lepton_dilepton1_mass}
    \includegraphics[width=0.3\textwidth]{chapters/Analysis/sectionPlots/figures/data_mc_overlays/etau_2016_cat_gt3_eq1_signal_linear_lepton_dilepton1_pt}
    \includegraphics[width=0.3\textwidth]{chapters/Analysis/sectionPlots/figures/data_mc_overlays/etau_2016_cat_gt3_eq1_signal_linear_lepton_dilepton1_delta_r}
    \caption{Dielectron mass, \pt, and $\Delta R$ in the $e\tau$ channel
    with $N_{j} \geq 3$ and $N_{b} = 1$.}
    \label{fig:analysis:plots:etau_7_dilepton}
\end{figure}

\begin{figure}[htb!]
    \centering
    \includegraphics[width=0.4\textwidth]{chapters/Analysis/sectionPlots/figures/data_mc_overlays/etau_2016_cat_gt3_gt2_signal_linear_lepton_lepton1_pt}
    \includegraphics[width=0.4\textwidth]{chapters/Analysis/sectionPlots/figures/data_mc_overlays/etau_2016_cat_gt3_gt2_signal_linear_lepton_lepton1_eta}

    \includegraphics[width=0.4\textwidth]{chapters/Analysis/sectionPlots/figures/data_mc_overlays/etau_2016_cat_gt3_gt2_signal_linear_lepton_lepton2_pt}
    \includegraphics[width=0.4\textwidth]{chapters/Analysis/sectionPlots/figures/data_mc_overlays/etau_2016_cat_gt3_gt2_signal_linear_lepton_lepton2_eta}
    \caption{\pt and $\eta$ distributions for leading (top) and trailing
        (bottom) electrons in the $e\tau$ channel with $N_{j} \geq 3$ and
        $N_{b} \geq 2$.}
    \label{fig:analysis:plots:etau_8_kinematic}
\end{figure}

\begin{figure}[htb!]
    \centering
    \includegraphics[width=0.3\textwidth]{chapters/Analysis/sectionPlots/figures/data_mc_overlays/etau_2016_cat_gt3_gt2_signal_linear_lepton_dilepton1_mass}
    \includegraphics[width=0.3\textwidth]{chapters/Analysis/sectionPlots/figures/data_mc_overlays/etau_2016_cat_gt3_gt2_signal_linear_lepton_dilepton1_pt}
    \includegraphics[width=0.3\textwidth]{chapters/Analysis/sectionPlots/figures/data_mc_overlays/etau_2016_cat_gt3_gt2_signal_linear_lepton_dilepton1_delta_r}
    \caption{Dielectron mass, \pt, and $\Delta R$ in the $e\tau$ channel
        with $N_{j} \geq 3$ and $N_{b} \geq 2$.}
    \label{fig:analysis:plots:etau_8_dilepton}
\end{figure}

\begin{figure}[htb!]
    \centering
    \includegraphics[width=0.4\textwidth]{chapters/Analysis/sectionPlots/figures/data_mc_overlays/etau_2016_inclusive_linear_jet_n_bjets}
    \includegraphics[width=0.4\textwidth]{chapters/Analysis/sectionPlots/figures/data_mc_overlays/etau_2016_inclusive_linear_jet_n_jets}
    \includegraphics[width=0.3\textwidth]{chapters/Analysis/sectionPlots/figures/data_mc_overlays/etau_2016_inclusive_linear_misc_met_mag}
    \caption{Multiplicity of b tagged jets, non-tagged jets, and MET in
        $e\tau$ channel.}
    \label{fig:analysis:plots:etau_jetmet}
\end{figure}


\FloatBarrier
\subsubsection{mutau}

\begin{figure}[htb!]
    \centering
    \includegraphics[width=0.4\textwidth]{chapters/Appendix/sectionPlots/figures/data_mc_overlays/mutau_2016_cat_eq0_eq0_signal_linear_lepton_lepton1_pt}
    \includegraphics[width=0.4\textwidth]{chapters/Appendix/sectionPlots/figures/data_mc_overlays/mutau_2016_cat_eq0_eq0_signal_linear_lepton_lepton1_eta}

    \includegraphics[width=0.4\textwidth]{chapters/Appendix/sectionPlots/figures/data_mc_overlays/mutau_2016_cat_eq0_eq0_signal_linear_lepton_lepton2_pt}
    \includegraphics[width=0.4\textwidth]{chapters/Appendix/sectionPlots/figures/data_mc_overlays/mutau_2016_cat_eq0_eq0_signal_linear_lepton_lepton2_eta}
    \caption{\pt and $\eta$ distributions for leading (top) and trailing
    (bottom) electrons in the $\mu\tau$ channel with $N_{j} = 0$.}
    \label{fig:mutau_1_kinematic}
\end{figure}

\begin{figure}[htb!]
    \centering
    \includegraphics[width=0.3\textwidth]{chapters/Appendix/sectionPlots/figures/data_mc_overlays/mutau_2016_cat_eq0_eq0_signal_linear_lepton_dilepton1_mass}
    \includegraphics[width=0.3\textwidth]{chapters/Appendix/sectionPlots/figures/data_mc_overlays/mutau_2016_cat_eq0_eq0_signal_linear_lepton_dilepton1_pt}
    \includegraphics[width=0.3\textwidth]{chapters/Appendix/sectionPlots/figures/data_mc_overlays/mutau_2016_cat_eq0_eq0_signal_linear_lepton_dilepton1_delta_r}
    \caption{Dielectron mass, \pt, and $\Delta R$ in the $\mu\tau$ channel
    with $N_{j} = 0$.}
    \label{fig:mutau_1_dilepton}
\end{figure}

\begin{figure}[htb!]
    \centering
    \includegraphics[width=0.4\textwidth]{chapters/Appendix/sectionPlots/figures/data_mc_overlays/mutau_2016_cat_eq1_eq0_signal_linear_lepton_lepton1_pt}
    \includegraphics[width=0.4\textwidth]{chapters/Appendix/sectionPlots/figures/data_mc_overlays/mutau_2016_cat_eq1_eq0_signal_linear_lepton_lepton1_eta}

    \includegraphics[width=0.4\textwidth]{chapters/Appendix/sectionPlots/figures/data_mc_overlays/mutau_2016_cat_eq1_eq0_signal_linear_lepton_lepton2_pt}
    \includegraphics[width=0.4\textwidth]{chapters/Appendix/sectionPlots/figures/data_mc_overlays/mutau_2016_cat_eq1_eq0_signal_linear_lepton_lepton2_eta}
    \caption{\pt and $\eta$ distributions for leading (top) and trailing
        (bottom) electrons in the $\mu\tau$ channel with $N_{j} = 1$ and
        $N_{b} = 0$.}
    \label{fig:mutau_2_kinematic}
\end{figure}

\begin{figure}[htb!]
    \centering
    \includegraphics[width=0.3\textwidth]{chapters/Appendix/sectionPlots/figures/data_mc_overlays/mutau_2016_cat_eq1_eq0_signal_linear_lepton_dilepton1_mass}
    \includegraphics[width=0.3\textwidth]{chapters/Appendix/sectionPlots/figures/data_mc_overlays/mutau_2016_cat_eq1_eq0_signal_linear_lepton_dilepton1_pt}
    \includegraphics[width=0.3\textwidth]{chapters/Appendix/sectionPlots/figures/data_mc_overlays/mutau_2016_cat_eq1_eq0_signal_linear_lepton_dilepton1_delta_r}
    \caption{Dielectron mass, \pt, and $\Delta R$ in the $\mu\tau$ channel
    with $N_{j} = 0$ and $N_{b} = 0$.}
    \label{fig:mutau_2_dilepton}
\end{figure}

\begin{figure}[htb!]
    \centering
    \includegraphics[width=0.4\textwidth]{chapters/Appendix/sectionPlots/figures/data_mc_overlays/mutau_2016_cat_eq1_eq1_signal_linear_lepton_lepton1_pt}
    \includegraphics[width=0.4\textwidth]{chapters/Appendix/sectionPlots/figures/data_mc_overlays/mutau_2016_cat_eq1_eq1_signal_linear_lepton_lepton1_eta}

    \includegraphics[width=0.4\textwidth]{chapters/Appendix/sectionPlots/figures/data_mc_overlays/mutau_2016_cat_eq1_eq1_signal_linear_lepton_lepton2_pt}
    \includegraphics[width=0.4\textwidth]{chapters/Appendix/sectionPlots/figures/data_mc_overlays/mutau_2016_cat_eq1_eq1_signal_linear_lepton_lepton2_eta}
    \caption{\pt and $\eta$ distributions for leading (top) and trailing
        (bottom) electrons in the $\mu\tau$ channel with $N_{j} = 1$ and
        $N_{b} = 1$.}
    \label{fig:mutau_3_kinematic}
\end{figure}

\begin{figure}[htb!]
    \centering
    \includegraphics[width=0.3\textwidth]{chapters/Appendix/sectionPlots/figures/data_mc_overlays/mutau_2016_cat_eq1_eq1_signal_linear_lepton_dilepton1_mass}
    \includegraphics[width=0.3\textwidth]{chapters/Appendix/sectionPlots/figures/data_mc_overlays/mutau_2016_cat_eq1_eq1_signal_linear_lepton_dilepton1_pt}
    \includegraphics[width=0.3\textwidth]{chapters/Appendix/sectionPlots/figures/data_mc_overlays/mutau_2016_cat_eq1_eq1_signal_linear_lepton_dilepton1_delta_r}
    \caption{Dielectron mass, \pt, and $\Delta R$ in the $\mu\tau$ channel
    with $N_{j} = 1$ and $N_{b} = 1$.}
    \label{fig:mutau_3_dilepton}
\end{figure}

\begin{figure}[htb!]
    \centering
    \includegraphics[width=0.4\textwidth]{chapters/Appendix/sectionPlots/figures/data_mc_overlays/mutau_2016_cat_gt2_eq0_signal_linear_lepton_lepton1_pt}
    \includegraphics[width=0.4\textwidth]{chapters/Appendix/sectionPlots/figures/data_mc_overlays/mutau_2016_cat_gt2_eq0_signal_linear_lepton_lepton1_eta}

    \includegraphics[width=0.4\textwidth]{chapters/Appendix/sectionPlots/figures/data_mc_overlays/mutau_2016_cat_gt2_eq0_signal_linear_lepton_lepton2_pt}
    \includegraphics[width=0.4\textwidth]{chapters/Appendix/sectionPlots/figures/data_mc_overlays/mutau_2016_cat_gt2_eq0_signal_linear_lepton_lepton2_eta}
    \caption{\pt and $\eta$ distributions for leading (top) and trailing
        (bottom) electrons in the $\mu\tau$ channel with $N_{j} \geq 2$ and
        $N_{b} = 0$.}
    \label{fig:mutau_4_kinematic}
\end{figure}

\begin{figure}[htb!]
    \centering
    \includegraphics[width=0.3\textwidth]{chapters/Appendix/sectionPlots/figures/data_mc_overlays/mutau_2016_cat_gt2_eq0_signal_linear_lepton_dilepton1_mass}
    \includegraphics[width=0.3\textwidth]{chapters/Appendix/sectionPlots/figures/data_mc_overlays/mutau_2016_cat_gt2_eq0_signal_linear_lepton_dilepton1_pt}
    \includegraphics[width=0.3\textwidth]{chapters/Appendix/sectionPlots/figures/data_mc_overlays/mutau_2016_cat_gt2_eq0_signal_linear_lepton_dilepton1_delta_r}
    \caption{Dielectron mass, \pt, and $\Delta R$ in the $\mu\tau$ channel
    with $N_{j} \geq 2$ and $N_{b} = 0$.}
    \label{fig:mutau_4_dilepton}
\end{figure}

\begin{figure}[htb!]
    \centering
    \includegraphics[width=0.4\textwidth]{chapters/Appendix/sectionPlots/figures/data_mc_overlays/mutau_2016_cat_eq2_eq1_signal_linear_lepton_lepton1_pt}
    \includegraphics[width=0.4\textwidth]{chapters/Appendix/sectionPlots/figures/data_mc_overlays/mutau_2016_cat_eq2_eq1_signal_linear_lepton_lepton1_eta}

    \includegraphics[width=0.4\textwidth]{chapters/Appendix/sectionPlots/figures/data_mc_overlays/mutau_2016_cat_eq2_eq1_signal_linear_lepton_lepton2_pt}
    \includegraphics[width=0.4\textwidth]{chapters/Appendix/sectionPlots/figures/data_mc_overlays/mutau_2016_cat_eq2_eq1_signal_linear_lepton_lepton2_eta}
    \caption{\pt and $\eta$ distributions for leading (top) and trailing
        (bottom) electrons in the $\mu\tau$ channel with $N_{j} = 2$ and
        $N_{b} = 1$.}
    \label{fig:mutau_5_kinematic}
\end{figure}

\begin{figure}[htb!]
    \centering
    \includegraphics[width=0.3\textwidth]{chapters/Appendix/sectionPlots/figures/data_mc_overlays/mutau_2016_cat_eq2_eq1_signal_linear_lepton_dilepton1_mass}
    \includegraphics[width=0.3\textwidth]{chapters/Appendix/sectionPlots/figures/data_mc_overlays/mutau_2016_cat_eq2_eq1_signal_linear_lepton_dilepton1_pt}
    \includegraphics[width=0.3\textwidth]{chapters/Appendix/sectionPlots/figures/data_mc_overlays/mutau_2016_cat_eq2_eq1_signal_linear_lepton_dilepton1_delta_r}
    \caption{Dielectron mass, \pt, and $\Delta R$ in the $\mu\tau$ channel
    with $N_{j} = 2$ and $N_{b} = 1$.}
    \label{fig:mutau_5_dilepton}
\end{figure}

\begin{figure}[htb!]
    \centering
    \includegraphics[width=0.4\textwidth]{chapters/Appendix/sectionPlots/figures/data_mc_overlays/mutau_2016_cat_eq2_eq2_signal_linear_lepton_lepton1_pt}
    \includegraphics[width=0.4\textwidth]{chapters/Appendix/sectionPlots/figures/data_mc_overlays/mutau_2016_cat_eq2_eq2_signal_linear_lepton_lepton1_eta}

    \includegraphics[width=0.4\textwidth]{chapters/Appendix/sectionPlots/figures/data_mc_overlays/mutau_2016_cat_eq2_eq2_signal_linear_lepton_lepton2_pt}
    \includegraphics[width=0.4\textwidth]{chapters/Appendix/sectionPlots/figures/data_mc_overlays/mutau_2016_cat_eq2_eq2_signal_linear_lepton_lepton2_eta}
    \caption{\pt and $\eta$ distributions for leading (top) and trailing
        (bottom) electrons in the $\mu\tau$ channel with $N_{j} = 2$ and
        $N_{b} = 2$.}
    \label{fig:mutau_6_kinematic}
\end{figure}

\begin{figure}[htb!]
    \centering
    \includegraphics[width=0.3\textwidth]{chapters/Appendix/sectionPlots/figures/data_mc_overlays/mutau_2016_cat_eq2_eq2_signal_linear_lepton_dilepton1_mass}
    \includegraphics[width=0.3\textwidth]{chapters/Appendix/sectionPlots/figures/data_mc_overlays/mutau_2016_cat_eq2_eq2_signal_linear_lepton_dilepton1_pt}
    \includegraphics[width=0.3\textwidth]{chapters/Appendix/sectionPlots/figures/data_mc_overlays/mutau_2016_cat_eq2_eq2_signal_linear_lepton_dilepton1_delta_r}
    \caption{Dielectron mass, \pt, and $\Delta R$ in the $\mu\tau$ channel
    with $N_{j} = 2$ and $N_{b} = 2$.}
    \label{fig:mutau_6_dilepton}
\end{figure}

\begin{figure}[htb!]
    \centering
    \includegraphics[width=0.4\textwidth]{chapters/Appendix/sectionPlots/figures/data_mc_overlays/mutau_2016_cat_gt3_eq1_signal_linear_lepton_lepton1_pt}
    \includegraphics[width=0.4\textwidth]{chapters/Appendix/sectionPlots/figures/data_mc_overlays/mutau_2016_cat_gt3_eq1_signal_linear_lepton_lepton1_eta}

    \includegraphics[width=0.4\textwidth]{chapters/Appendix/sectionPlots/figures/data_mc_overlays/mutau_2016_cat_gt3_eq1_signal_linear_lepton_lepton2_pt}
    \includegraphics[width=0.4\textwidth]{chapters/Appendix/sectionPlots/figures/data_mc_overlays/mutau_2016_cat_gt3_eq1_signal_linear_lepton_lepton2_eta}
    \caption{\pt and $\eta$ distributions for leading (top) and trailing
        (bottom) electrons in the $\mu\tau$ channel with $N_{j} \geq 3$ and
        $N_{b} = 1$.}
    \label{fig:mutau_7_kinematic}
\end{figure}

\begin{figure}[htb!]
    \centering
    \includegraphics[width=0.3\textwidth]{chapters/Appendix/sectionPlots/figures/data_mc_overlays/mutau_2016_cat_gt3_eq1_signal_linear_lepton_dilepton1_mass}
    \includegraphics[width=0.3\textwidth]{chapters/Appendix/sectionPlots/figures/data_mc_overlays/mutau_2016_cat_gt3_eq1_signal_linear_lepton_dilepton1_pt}
    \includegraphics[width=0.3\textwidth]{chapters/Appendix/sectionPlots/figures/data_mc_overlays/mutau_2016_cat_gt3_eq1_signal_linear_lepton_dilepton1_delta_r}
    \caption{Dielectron mass, \pt, and $\Delta R$ in the $\mu\tau$ channel
    with $N_{j} \geq 3$ and $N_{b} = 1$.}
    \label{fig:mutau_7_dilepton}
\end{figure}

\begin{figure}[htb!]
    \centering
    \includegraphics[width=0.4\textwidth]{chapters/Appendix/sectionPlots/figures/data_mc_overlays/mutau_2016_cat_gt3_gt2_signal_linear_lepton_lepton1_pt}
    \includegraphics[width=0.4\textwidth]{chapters/Appendix/sectionPlots/figures/data_mc_overlays/mutau_2016_cat_gt3_gt2_signal_linear_lepton_lepton1_eta}

    \includegraphics[width=0.4\textwidth]{chapters/Appendix/sectionPlots/figures/data_mc_overlays/mutau_2016_cat_gt3_gt2_signal_linear_lepton_lepton2_pt}
    \includegraphics[width=0.4\textwidth]{chapters/Appendix/sectionPlots/figures/data_mc_overlays/mutau_2016_cat_gt3_gt2_signal_linear_lepton_lepton2_eta}
    \caption{\pt and $\eta$ distributions for leading (top) and trailing
        (bottom) electrons in the $\mu\tau$ channel with $N_{j} \geq 3$ and
        $N_{b} \geq 2$.}
    \label{fig:mutau_8_kinematic}
\end{figure}

\begin{figure}[htb!]
    \centering
    \includegraphics[width=0.3\textwidth]{chapters/Appendix/sectionPlots/figures/data_mc_overlays/mutau_2016_cat_gt3_gt2_signal_linear_lepton_dilepton1_mass}
    \includegraphics[width=0.3\textwidth]{chapters/Appendix/sectionPlots/figures/data_mc_overlays/mutau_2016_cat_gt3_gt2_signal_linear_lepton_dilepton1_pt}
    \includegraphics[width=0.3\textwidth]{chapters/Appendix/sectionPlots/figures/data_mc_overlays/mutau_2016_cat_gt3_gt2_signal_linear_lepton_dilepton1_delta_r}
    \caption{Dielectron mass, \pt, and $\Delta R$ in the $\mu\tau$ channel
    with $N_{j} \geq 3$ and $N_{b} \geq 2$.}
    \label{fig:mutau_8_dilepton}
\end{figure}

\begin{figure}[htb!]
    \centering
    \includegraphics[width=0.4\textwidth]{chapters/Appendix/sectionPlots/figures/data_mc_overlays/mutau_2016_inclusive_linear_jet_n_bjets}
    \includegraphics[width=0.4\textwidth]{chapters/Appendix/sectionPlots/figures/data_mc_overlays/mutau_2016_inclusive_linear_jet_n_jets}
    \includegraphics[width=0.3\textwidth]{chapters/Appendix/sectionPlots/figures/data_mc_overlays/mutau_2016_inclusive_linear_misc_met_mag}
    \caption{Multiplicity of b tagged jets, non-tagged jets, and MET in
    $\mu\tau$ channel.}
    \label{fig:mutau_jetmet}
\end{figure}



\FloatBarrier

\subsubsection{ejet}


\begin{figure}[htb!]
    \centering
    \includegraphics[width=0.4\textwidth]{chapters/Analysis/sectionPlots/figures/data_mc_overlays/ejet_2016_cat_gt4_eq1_signal_linear_lepton_lepton1_pt}
    \includegraphics[width=0.4\textwidth]{chapters/Analysis/sectionPlots/figures/data_mc_overlays/ejet_2016_cat_gt4_eq1_signal_linear_lepton_lepton1_eta}
    \caption{\pt and $\eta$ distributions for the muon in the $e$ + jets
    channel with $N_{j} \geq 4$ and $N_{b} = 1$.
    \label{fig:analysis:plots:ejet_1_kinematic}}
\end{figure}

\begin{figure}[htb!]
    \centering
    \includegraphics[width=0.3\textwidth]{chapters/Analysis/sectionPlots/figures/data_mc_overlays/ejet_2016_cat_gt4_eq1_signal_linear_jet_n_bjets}
    \includegraphics[width=0.3\textwidth]{chapters/Analysis/sectionPlots/figures/data_mc_overlays/ejet_2016_cat_gt4_eq1_signal_linear_jet_n_jets}
    \includegraphics[width=0.3\textwidth]{chapters/Analysis/sectionPlots/figures/data_mc_overlays/ejet_2016_cat_gt4_eq1_signal_linear_misc_met_mag}
    \caption{Multiplicity of b tagged jets, non-tagged jets, and MET in
    $e$ + jets channel with $N_{j} \geq 4$ and $N_{b} = 1$.
    \label{fig:analysis:plots:ejet_1_jetmet}}
\end{figure}

\begin{figure}[htb!]
    \centering
    \includegraphics[width=0.4\textwidth]{chapters/Analysis/sectionPlots/figures/data_mc_overlays/ejet_2016_cat_gt4_gt2_signal_linear_lepton_lepton1_pt}
    \includegraphics[width=0.4\textwidth]{chapters/Analysis/sectionPlots/figures/data_mc_overlays/ejet_2016_cat_gt4_gt2_signal_linear_lepton_lepton1_eta}
    \caption{\pt and $\eta$ distributions for the muon in the $e$ + jets
    channel with $N_{j} \geq 2$ and $N_{b} \geq 2$.
    \label{fig:analysis:plots:ejet_2_kinematic}}
\end{figure}

\begin{figure}[htb!]
    \centering
    \includegraphics[width=0.3\textwidth]{chapters/Analysis/sectionPlots/figures/data_mc_overlays/ejet_2016_cat_gt4_gt2_signal_linear_jet_n_bjets}
    \includegraphics[width=0.3\textwidth]{chapters/Analysis/sectionPlots/figures/data_mc_overlays/ejet_2016_cat_gt4_gt2_signal_linear_jet_n_jets}
    \includegraphics[width=0.3\textwidth]{chapters/Analysis/sectionPlots/figures/data_mc_overlays/ejet_2016_cat_gt4_gt2_signal_linear_misc_met_mag}
    \caption{Multiplicity of b tagged jets, non-tagged jets, and MET in
    $e$ + jets channel with $N_{j} \geq 4$ and $N_{b} \geq 2$.
    \label{fig:analysis:plots:ejet_2_jetmet}}
\end{figure}

\FloatBarrier
\begin{figure}[htb!]
    \centering
    \includegraphics[width=0.4\textwidth]{figures/data_mc_overlays/mujet_2016/cat_gt4_eq1_signal/linear/lepton/lepton1_pt}
    \includegraphics[width=0.4\textwidth]{figures/data_mc_overlays/mujet_2016/cat_gt4_eq1_signal/linear/lepton/lepton1_eta}
    \caption{\pt and $\eta$ distributions for the muon in the $\mu$ + jets
    channel with $N_{j} \geq 4$ and $N_{b} = 1$.
    \label{fig:mujet_1_kinematic}}
\end{figure}

\begin{figure}[htb!]
    \centering
    \includegraphics[width=0.3\textwidth]{figures/data_mc_overlays/mujet_2016/cat_gt4_eq1_signal/linear/jet/n_bjets}
    \includegraphics[width=0.3\textwidth]{figures/data_mc_overlays/mujet_2016/cat_gt4_eq1_signal/linear/jet/n_jets}
    \includegraphics[width=0.3\textwidth]{figures/data_mc_overlays/mujet_2016/cat_gt4_eq1_signal/linear/misc/met_mag}
    \caption{Multiplicity of b tagged jets, non-tagged jets, and MET in
    $\mu$ + jets channel with $N_{j} \geq 4$ and $N_{b} = 1$.
    \label{fig:mujet_1_jetmet}}
\end{figure}

\begin{figure}[htb!]
    \centering
    \includegraphics[width=0.4\textwidth]{figures/data_mc_overlays/mujet_2016/cat_gt4_gt2_signal/linear/lepton/lepton1_pt}
    \includegraphics[width=0.4\textwidth]{figures/data_mc_overlays/mujet_2016/cat_gt4_gt2_signal/linear/lepton/lepton1_eta}
    \caption{\pt and $\eta$ distributions for the muon in the $\mu$ + jets
    channel with $N_{j} \geq 2$ and $N_{b} \geq 2$.
    \label{fig:mujet_2_kinematic}}
\end{figure}

\begin{figure}[htb!]
    \centering
    \includegraphics[width=0.3\textwidth]{figures/data_mc_overlays/mujet_2016/cat_gt4_gt2_signal/linear/jet/n_bjets}
    \includegraphics[width=0.3\textwidth]{figures/data_mc_overlays/mujet_2016/cat_gt4_gt2_signal/linear/jet/n_jets}
    \includegraphics[width=0.3\textwidth]{figures/data_mc_overlays/mujet_2016/cat_gt4_gt2_signal/linear/misc/met_mag}
    \caption{Multiplicity of b tagged jets, non-tagged jets, and MET in
    $\mu$ + jets channel with $N_{j} \geq 4$ and $N_{b} \geq 2$.
    \label{fig:mujet_2_jetmet}}
\end{figure}

\FloatBarrier

        --------------------------------------------------
        \chapter{Supplement Materials for CLUE clustering algorithm}
        
\section{Pseudocode}

Pseudocode of CLUE in serialized implementation. Parallelization is discussed in Section~\ref{sec:implementation}.


\begin{algorithm}[h]
% \SetAlgoLined
    \For{$i \in$ points}{
        $\rho_{[i]} = 0$ \\
        \For{$j \in \Omega_{d_c}(i)$ }{
            \If{$ dist(i,j) < d_c$}{
                $\rho_{[i]}$ += $w_{[j]}$
            }
        }
    }
\caption{calculate $\rho$}    
\end{algorithm}




\begin{algorithm}[h]
% \SetAlgoLined
    \For{$i \in$ points}{
        $\delta_{[i]} = +\infty$ \\
        $nh_{[i]} = -1$ \\
        \For{$j \in \Omega_{d_m}(i)$ }{
            \If{ $dist(i,j) < d_m$ \textbf{and} $\rho_{[j]} > \rho_{[i]}$}{
                \If {$dist(i,j) < \delta_{[i]} $}{
                    $nh_{[i]} = j$  \\
                    $\delta_{[i]} = d_{ij}$ \\
                }
            }
        }
    }
\caption{calculate $\delta$}    
\end{algorithm}



\begin{algorithm}[h]
% \SetAlgoLined
    k = 0\;
    stack = [] \;
    \For{i $\in$ points}{
        $isSeed = \rho_{[i]} > \rho_c \textbf{ and } \delta_{[i]} > \delta_c$ \\
        $isOutlier = \rho_{[i]} < \rho_c \textbf{ and } \delta_{[i]} > \delta_o$ \\
        \eIf {$isSeed$}{
            $clusterId_{[i]}$ = k \\
            k++ \\
            stack.pushback(i) \\
        } { \If{not $isOutlier$ }{
            $followers_{[nh_{[i]}]}.pushback(i)$
            }
        }
    }
    
    \While{stack.size $>$ 0}{
        i = stack.back \\
        stack.popback \\
        \For{$j \in followers_{[i]}$}{
            $clusterId_{[j]} = clusterId_{[i]}$ \\
            stack.pushback(j) \\
        }
    }
    
\caption{find seeds and outliers, assign clusters}
\label{algo:algorithm:assignClusters}
\end{algorithm}
        


\section{Extra Performance Tests}




%%%%%%%%%%%%%%%%%%%%%%%%%%%
% ryzen
%%%%%%%%%%%%%%%%%%%%%%%%%%%
\subsection{Ryzen}
\begin{figure}[ht!]
    \centering
    \includegraphics[width=0.82\textwidth]{chapters/HGCal/figures/clue/private/Figure5_1.pdf}
    \caption{Execution time of CLUE on CPU and GPU both scale linearly with number of input points, ranging from $10^5$ to $10^6$ in total on 100 layers. Mean and standard deviation are based on 200 trial runs. \texttt{AMD Ryzen 2700X} and \texttt{NVIDIA GTX 1080Ti}.}
\end{figure}
\begin{figure}[ht!]
    \centering
    \includegraphics[trim=3cm 0cm 3cm 0cm, clip,width=0.9\textwidth]{chapters/HGCal/figures/clue/private/addition_ryzen.pdf}
    \caption{Stability during 200 trial runs}
\end{figure}

\newpage
\begin{landscape}
\begin{table}[ht!]
    \renewcommand{\arraystretch}{1.25}
    \tiny
    \centering

    % 1000
    \scalebox{0.7}{\begin{tabular}{l|c|c|c|c|c|c}
    \hline
    CLUE Step                                 & CPU [1T] (baseline)         & CPU TBB [1T]                          & CPU TBB [4T]                          & CPU TBB [8T]                          & CPU TBB [16T]                         & GPU                       \\ \hline
    build fixed-grid spatial index            &   5.48 $\pm$  0.09 ms       &   6.45 $\pm$  0.05 ms ( 0.85x)        &   3.79 $\pm$  0.10 ms ( 1.45x)        &   3.41 $\pm$  0.12 ms ( 1.61x)        &   3.25 $\pm$  0.46 ms ( 1.69x)        &   0.06 ms ( 89.15x)       \\
    calculate local density                   &  10.91 $\pm$  0.07 ms       &  11.98 $\pm$  0.02 ms ( 0.91x)        &   3.20 $\pm$  0.10 ms ( 3.40x)        &   1.86 $\pm$  0.08 ms ( 5.87x)        &   1.20 $\pm$  0.05 ms ( 9.10x)        &   0.15 ms ( 70.40x)       \\
    calculate nearest-higher and separation   &  14.37 $\pm$  0.07 ms       &  16.26 $\pm$  0.01 ms ( 0.88x)        &   4.18 $\pm$  0.09 ms ( 3.43x)        &   2.35 $\pm$  0.12 ms ( 6.11x)        &   1.47 $\pm$  0.08 ms ( 9.79x)        &   0.20 ms ( 70.81x)       \\
    decide seeds/outliers, register followers &   2.99 $\pm$  0.14 ms       &   3.45 $\pm$  0.02 ms ( 0.87x)        &   3.12 $\pm$  0.03 ms ( 0.96x)        &   3.20 $\pm$  0.19 ms ( 0.94x)        &   3.14 $\pm$  0.41 ms ( 0.95x)        &   0.04 ms ( 79.48x)       \\
    expand clusters                           &   0.30 $\pm$  0.03 ms       &   2.14 $\pm$  0.01 ms ( 0.14x)        &   0.52 $\pm$  0.01 ms ( 0.58x)        &   0.33 $\pm$  0.02 ms ( 0.90x)        &   0.27 $\pm$  0.01 ms ( 1.10x)        &   0.04 ms (  6.99x)       \\ \hline
    cuda memcpy, memset                       & --                          & --                                    & --                                    & --                                    & --                                    &   0.31 ms,   0.06 ms      \\ 
    other                                     &   5.66 $\pm$  0.28 ms       &  16.58 $\pm$  1.67 ms                 &  17.29 $\pm$  1.62 ms                 &  17.38 $\pm$  1.65 ms                 &  17.48 $\pm$  1.62 ms                 &   0.60 ms                 \\ \hline
    \textbf{TOTAL} ( 1000 points per layer)   & \textbf{ 39.71 $\pm$  0.54 ms} & \textbf{ 56.86 $\pm$  1.67 ms ( 0.70x)} & \textbf{ 32.12 $\pm$  1.84 ms ( 1.24x)} & \textbf{ 28.53 $\pm$  1.86 ms ( 1.39x)} & \textbf{ 26.81 $\pm$  1.94 ms ( 1.48x)} & \textbf{  1.47 $\pm$  0.03 ms ( 27.09x)}  \\
    \hline 
    \multicolumn{4}{c}{} 
    \end{tabular}}
    \linebreak


    % 2000
    \scalebox{0.7}{\begin{tabular}{l|c|c|c|c|c|c}
    \hline
    CLUE Step                                 & CPU [1T] (baseline)         & CPU TBB [1T]                          & CPU TBB [4T]                          & CPU TBB [8T]                          & CPU TBB [16T]                         & GPU                       \\ \hline
    build fixed-grid spatial index            &  10.21 $\pm$  0.05 ms       &  12.14 $\pm$  0.05 ms ( 0.84x)        &   7.04 $\pm$  0.22 ms ( 1.45x)        &   6.34 $\pm$  0.13 ms ( 1.61x)        &   5.88 $\pm$  0.17 ms ( 1.73x)        &   0.12 ms ( 84.32x)       \\
    calculate local density                   &  26.51 $\pm$  0.05 ms       &  28.54 $\pm$  0.03 ms ( 0.93x)        &   7.33 $\pm$  0.09 ms ( 3.62x)        &   4.04 $\pm$  0.12 ms ( 6.56x)        &   2.49 $\pm$  0.07 ms (10.66x)        &   0.28 ms ( 94.16x)       \\
    calculate nearest-higher and separation   &  34.56 $\pm$  0.04 ms       &  37.12 $\pm$  0.02 ms ( 0.93x)        &   9.48 $\pm$  0.03 ms ( 3.65x)        &   5.12 $\pm$  0.10 ms ( 6.75x)        &   3.13 $\pm$  0.03 ms (11.04x)        &   0.40 ms ( 87.20x)       \\
    decide seeds/outliers, register followers &   6.24 $\pm$  0.05 ms       &   6.98 $\pm$  0.03 ms ( 0.89x)        &   6.49 $\pm$  0.33 ms ( 0.96x)        &   6.44 $\pm$  0.17 ms ( 0.97x)        &   5.99 $\pm$  0.03 ms ( 1.04x)        &   0.09 ms ( 71.07x)       \\
    expand clusters                           &   1.44 $\pm$  0.01 ms       &   3.69 $\pm$  0.01 ms ( 0.39x)        &   1.02 $\pm$  0.02 ms ( 1.41x)        &   0.68 $\pm$  0.03 ms ( 2.12x)        &   0.58 $\pm$  0.02 ms ( 2.47x)        &   0.08 ms ( 17.05x)       \\ \hline
    cuda memcpy, memset                       & --                          & --                                    & --                                    & --                                    & --                                    &   0.61 ms,   0.08 ms      \\ 
    other                                     &   7.61 $\pm$  0.16 ms       &  17.68 $\pm$  1.79 ms                 &  18.44 $\pm$  1.74 ms                 &  18.45 $\pm$  1.78 ms                 &  18.57 $\pm$  1.74 ms                 &   0.83 ms                 \\ \hline
    \textbf{TOTAL} ( 2000 points per layer)   & \textbf{ 86.57 $\pm$  0.28 ms} & \textbf{106.15 $\pm$  1.79 ms ( 0.82x)} & \textbf{ 49.79 $\pm$  1.91 ms ( 1.74x)} & \textbf{ 41.08 $\pm$  1.90 ms ( 2.11x)} & \textbf{ 36.65 $\pm$  1.96 ms ( 2.36x)} & \textbf{  2.49 $\pm$  0.01 ms ( 34.74x)}  \\
    \hline 
    \multicolumn{4}{c}{} 
    \end{tabular}}
    \linebreak


    % 3000
    \scalebox{0.7}{\begin{tabular}{l|c|c|c|c|c|c}
    \hline
    CLUE Step                                 & CPU [1T] (baseline)         & CPU TBB [1T]                          & CPU TBB [4T]                          & CPU TBB [8T]                          & CPU TBB [16T]                         & GPU                       \\ \hline
    build fixed-grid spatial index            &  14.79 $\pm$  0.08 ms       &  17.44 $\pm$  0.16 ms ( 0.85x)        &  11.11 $\pm$  0.22 ms ( 1.33x)        &   9.43 $\pm$  0.16 ms ( 1.57x)        &   8.78 $\pm$  0.20 ms ( 1.69x)        &   0.18 ms ( 83.24x)       \\
    calculate local density                   &  41.55 $\pm$  0.04 ms       &  44.54 $\pm$  0.05 ms ( 0.93x)        &  11.41 $\pm$  0.10 ms ( 3.64x)        &   6.05 $\pm$  0.14 ms ( 6.87x)        &   3.77 $\pm$  0.02 ms (11.03x)        &   0.47 ms ( 89.11x)       \\
    calculate nearest-higher and separation   &  53.70 $\pm$  0.05 ms       &  57.14 $\pm$  0.06 ms ( 0.94x)        &  14.54 $\pm$  0.11 ms ( 3.69x)        &   7.62 $\pm$  0.19 ms ( 7.05x)        &   4.70 $\pm$  0.03 ms (11.42x)        &   0.65 ms ( 82.77x)       \\
    decide seeds/outliers, register followers &   9.30 $\pm$  0.09 ms       &  10.25 $\pm$  0.08 ms ( 0.91x)        &  10.77 $\pm$  0.25 ms ( 0.86x)        &   9.66 $\pm$  0.29 ms ( 0.96x)        &   8.84 $\pm$  0.03 ms ( 1.05x)        &   0.13 ms ( 69.24x)       \\
    expand clusters                           &   2.55 $\pm$  0.02 ms       &   5.01 $\pm$  0.03 ms ( 0.51x)        &   1.53 $\pm$  0.04 ms ( 1.67x)        &   1.00 $\pm$  0.05 ms ( 2.54x)        &   0.89 $\pm$  0.03 ms ( 2.86x)        &   0.13 ms ( 19.99x)       \\ \hline
    cuda memcpy, memset                       & --                          & --                                    & --                                    & --                                    & --                                    &   0.91 ms,   0.08 ms      \\ 
    other                                     &   9.51 $\pm$  0.08 ms       &  18.78 $\pm$  1.88 ms                 &  19.62 $\pm$  1.86 ms                 &  19.75 $\pm$  1.85 ms                 &  19.67 $\pm$  1.86 ms                 &   1.02 ms                 \\ \hline
    \textbf{TOTAL} ( 3000 points per layer)   & \textbf{131.40 $\pm$  0.24 ms} & \textbf{153.16 $\pm$  1.92 ms ( 0.86x)} & \textbf{ 68.99 $\pm$  1.91 ms ( 1.90x)} & \textbf{ 53.51 $\pm$  1.92 ms ( 2.46x)} & \textbf{ 46.65 $\pm$  2.06 ms ( 2.82x)} & \textbf{  3.57 $\pm$  0.01 ms ( 36.77x)}  \\
    \hline 
    \multicolumn{4}{c}{} 
    \end{tabular}}
    \linebreak


    % 4000
    \scalebox{0.7}{\begin{tabular}{l|c|c|c|c|c|c}
    \hline
    CLUE Step                                 & CPU [1T] (baseline)         & CPU TBB [1T]                          & CPU TBB [4T]                          & CPU TBB [8T]                          & CPU TBB [16T]                         & GPU                       \\ \hline
    build fixed-grid spatial index            &  18.33 $\pm$  0.09 ms       &  22.95 $\pm$  0.17 ms ( 0.80x)        &  15.00 $\pm$  0.13 ms ( 1.22x)        &  12.99 $\pm$  0.20 ms ( 1.41x)        &  12.14 $\pm$  0.21 ms ( 1.51x)        &   0.22 ms ( 81.71x)       \\
    calculate local density                   &  61.94 $\pm$  0.14 ms       &  64.51 $\pm$  0.04 ms ( 0.96x)        &  16.41 $\pm$  0.08 ms ( 3.77x)        &   8.62 $\pm$  0.19 ms ( 7.18x)        &   5.22 $\pm$  0.04 ms (11.86x)        &   0.63 ms ( 99.09x)       \\
    calculate nearest-higher and separation   &  84.05 $\pm$  0.22 ms       &  86.08 $\pm$  0.04 ms ( 0.98x)        &  21.84 $\pm$  0.29 ms ( 3.85x)        &  11.35 $\pm$  0.29 ms ( 7.40x)        &   6.85 $\pm$  0.04 ms (12.26x)        &   0.95 ms ( 88.93x)       \\
    decide seeds/outliers, register followers &  13.11 $\pm$  0.13 ms       &  15.44 $\pm$  0.18 ms ( 0.85x)        &  15.13 $\pm$  0.14 ms ( 0.87x)        &  13.37 $\pm$  0.31 ms ( 0.98x)        &  12.27 $\pm$  0.04 ms ( 1.07x)        &   0.19 ms ( 68.13x)       \\
    expand clusters                           &   4.12 $\pm$  0.02 ms       &   7.18 $\pm$  0.02 ms ( 0.57x)        &   2.25 $\pm$  0.04 ms ( 1.83x)        &   1.56 $\pm$  0.07 ms ( 2.64x)        &   1.37 $\pm$  0.03 ms ( 3.00x)        &   0.23 ms ( 17.69x)       \\ \hline
    cuda memcpy, memset                       & --                          & --                                    & --                                    & --                                    & --                                    &   1.21 ms,   0.08 ms      \\ 
    other                                     &  11.43 $\pm$  0.03 ms       &  19.95 $\pm$  1.99 ms                 &  20.71 $\pm$  1.94 ms                 &  20.84 $\pm$  1.97 ms                 &  20.84 $\pm$  1.97 ms                 &   1.03 ms                 \\ \hline
    \textbf{TOTAL} ( 4000 points per layer)   & \textbf{192.97 $\pm$  0.32 ms} & \textbf{216.12 $\pm$  1.97 ms ( 0.89x)} & \textbf{ 91.34 $\pm$  2.09 ms ( 2.11x)} & \textbf{ 68.73 $\pm$  2.06 ms ( 2.81x)} & \textbf{ 58.69 $\pm$  2.17 ms ( 3.29x)} & \textbf{  4.54 $\pm$  0.01 ms ( 42.50x)}  \\
    \hline 
    \multicolumn{4}{c}{} 
    \end{tabular}}
    \linebreak


    % 5000
    \scalebox{0.7}{\begin{tabular}{l|c|c|c|c|c|c}
    \hline
    CLUE Step                                 & CPU [1T] (baseline)         & CPU TBB [1T]                          & CPU TBB [4T]                          & CPU TBB [8T]                          & CPU TBB [16T]                         & GPU                       \\ \hline
    build fixed-grid spatial index            &  22.48 $\pm$  0.11 ms       &  27.53 $\pm$  0.24 ms ( 0.82x)        &  18.13 $\pm$  0.21 ms ( 1.24x)        &  15.50 $\pm$  0.26 ms ( 1.45x)        &  14.47 $\pm$  0.18 ms ( 1.55x)        &   0.27 ms ( 82.06x)       \\
    calculate local density                   &  77.69 $\pm$  0.07 ms       &  81.28 $\pm$  0.12 ms ( 0.96x)        &  20.64 $\pm$  0.22 ms ( 3.76x)        &  10.74 $\pm$  0.23 ms ( 7.23x)        &   6.52 $\pm$  0.05 ms (11.91x)        &   0.82 ms ( 94.31x)       \\
    calculate nearest-higher and separation   & 107.57 $\pm$  0.05 ms       & 109.03 $\pm$  0.12 ms ( 0.99x)        &  27.62 $\pm$  0.16 ms ( 3.90x)        &  14.31 $\pm$  0.36 ms ( 7.52x)        &   8.67 $\pm$  0.05 ms (12.40x)        &   1.18 ms ( 91.42x)       \\
    decide seeds/outliers, register followers &  16.26 $\pm$  0.16 ms       &  19.08 $\pm$  0.15 ms ( 0.85x)        &  18.82 $\pm$  0.15 ms ( 0.86x)        &  16.88 $\pm$  0.51 ms ( 0.96x)        &  15.40 $\pm$  0.05 ms ( 1.06x)        &   0.24 ms ( 66.83x)       \\
    expand clusters                           &   5.28 $\pm$  0.02 ms       &   8.90 $\pm$  0.04 ms ( 0.59x)        &   2.83 $\pm$  0.04 ms ( 1.87x)        &   1.96 $\pm$  0.08 ms ( 2.69x)        &   1.72 $\pm$  0.02 ms ( 3.08x)        &   0.33 ms ( 15.97x)       \\ \hline
    cuda memcpy, memset                       & --                          & --                                    & --                                    & --                                    & --                                    &   1.51 ms,   0.08 ms      \\ 
    other                                     &  13.22 $\pm$  0.08 ms       &  21.01 $\pm$  2.11 ms                 &  21.83 $\pm$  2.07 ms                 &  21.98 $\pm$  2.07 ms                 &  22.02 $\pm$  2.04 ms                 &   1.04 ms                 \\ \hline
    \textbf{TOTAL} ( 5000 points per layer)   & \textbf{242.50 $\pm$  0.24 ms} & \textbf{266.83 $\pm$  2.18 ms ( 0.91x)} & \textbf{109.85 $\pm$  2.28 ms ( 2.21x)} & \textbf{ 81.38 $\pm$  2.16 ms ( 2.98x)} & \textbf{ 68.80 $\pm$  2.21 ms ( 3.52x)} & \textbf{  5.48 $\pm$  0.01 ms ( 44.26x)}  \\
    \hline 
    \end{tabular}}
    \linebreak



\end{table}
\end{landscape}


\newpage
\begin{landscape}
\begin{table}[ht!]
    \renewcommand{\arraystretch}{1.25}
    \tiny
    \centering

    % 6000
    \scalebox{0.7}{\begin{tabular}{l|c|c|c|c|c|c}
    \hline
    CLUE Step                                 & CPU [1T] (baseline)         & CPU TBB [1T]                          & CPU TBB [4T]                          & CPU TBB [8T]                          & CPU TBB [16T]                         & GPU                       \\ \hline
    build fixed-grid spatial index            &  26.47 $\pm$  0.17 ms       &  33.49 $\pm$  0.26 ms ( 0.79x)        &  21.67 $\pm$  0.19 ms ( 1.22x)        &  18.62 $\pm$  0.17 ms ( 1.42x)        &  17.34 $\pm$  0.40 ms ( 1.53x)        &   0.33 ms ( 81.38x)       \\
    calculate local density                   &  92.83 $\pm$  0.39 ms       &  96.12 $\pm$  0.04 ms ( 0.97x)        &  24.37 $\pm$  0.11 ms ( 3.81x)        &  12.61 $\pm$  0.20 ms ( 7.36x)        &   7.67 $\pm$  0.10 ms (12.11x)        &   0.94 ms ( 98.95x)       \\
    calculate nearest-higher and separation   & 128.26 $\pm$  0.36 ms       & 130.39 $\pm$  0.04 ms ( 0.98x)        &  32.94 $\pm$  0.28 ms ( 3.89x)        &  16.90 $\pm$  0.31 ms ( 7.59x)        &  10.35 $\pm$  0.04 ms (12.39x)        &   1.34 ms ( 95.96x)       \\
    decide seeds/outliers, register followers &  18.88 $\pm$  0.31 ms       &  23.21 $\pm$  0.13 ms ( 0.81x)        &  22.63 $\pm$  0.12 ms ( 0.83x)        &  20.23 $\pm$  0.19 ms ( 0.93x)        &  18.40 $\pm$  0.08 ms ( 1.03x)        &   0.29 ms ( 66.09x)       \\
    expand clusters                           &   6.59 $\pm$  0.03 ms       &  10.49 $\pm$  0.02 ms ( 0.63x)        &   3.36 $\pm$  0.03 ms ( 1.96x)        &   2.34 $\pm$  0.07 ms ( 2.82x)        &   2.10 $\pm$  0.35 ms ( 3.14x)        &   0.43 ms ( 15.20x)       \\ \hline
    cuda memcpy, memset                       & --                          & --                                    & --                                    & --                                    & --                                    &   1.82 ms,   0.08 ms      \\ 
    other                                     &  14.89 $\pm$  0.30 ms       &  22.17 $\pm$  2.20 ms                 &  22.92 $\pm$  2.16 ms                 &  23.10 $\pm$  2.14 ms                 &  23.20 $\pm$  2.14 ms                 &   1.08 ms                 \\ \hline
    \textbf{TOTAL} ( 6000 points per layer)   & \textbf{287.93 $\pm$  0.72 ms} & \textbf{315.87 $\pm$  2.21 ms ( 0.91x)} & \textbf{127.89 $\pm$  2.27 ms ( 2.25x)} & \textbf{ 93.80 $\pm$  2.34 ms ( 3.07x)} & \textbf{ 79.05 $\pm$  2.41 ms ( 3.64x)} & \textbf{  6.29 $\pm$  0.01 ms ( 45.76x)}  \\
    \hline 
    \multicolumn{4}{c}{} 
    \end{tabular}}
    \linebreak


    % 7000
    \scalebox{0.7}{\begin{tabular}{l|c|c|c|c|c|c}
    \hline
    CLUE Step                                 & CPU [1T] (baseline)         & CPU TBB [1T]                          & CPU TBB [4T]                          & CPU TBB [8T]                          & CPU TBB [16T]                         & GPU                       \\ \hline
    build fixed-grid spatial index            &  29.94 $\pm$  0.16 ms       &  37.70 $\pm$  0.16 ms ( 0.79x)        &  24.82 $\pm$  0.20 ms ( 1.21x)        &  21.27 $\pm$  0.43 ms ( 1.41x)        &  19.83 $\pm$  0.19 ms ( 1.51x)        &   0.36 ms ( 83.23x)       \\
    calculate local density                   & 114.19 $\pm$  0.21 ms       & 117.11 $\pm$  0.11 ms ( 0.98x)        &  29.66 $\pm$  0.14 ms ( 3.85x)        &  15.36 $\pm$  0.37 ms ( 7.43x)        &   9.26 $\pm$  0.04 ms (12.33x)        &   1.14 ms (100.50x)       \\
    calculate nearest-higher and separation   & 164.45 $\pm$  0.56 ms       & 166.15 $\pm$  0.96 ms ( 0.99x)        &  41.58 $\pm$  0.42 ms ( 3.96x)        &  21.32 $\pm$  0.58 ms ( 7.71x)        &  12.92 $\pm$  0.06 ms (12.73x)        &   1.65 ms ( 99.46x)       \\
    decide seeds/outliers, register followers &  25.02 $\pm$  0.16 ms       &  27.71 $\pm$  0.18 ms ( 0.90x)        &  27.16 $\pm$  0.14 ms ( 0.92x)        &  24.14 $\pm$  0.87 ms ( 1.04x)        &  22.13 $\pm$  0.07 ms ( 1.13x)        &   0.35 ms ( 70.63x)       \\
    expand clusters                           &   9.16 $\pm$  0.04 ms       &  13.49 $\pm$  0.05 ms ( 0.68x)        &   4.36 $\pm$  0.04 ms ( 2.10x)        &   3.15 $\pm$  0.15 ms ( 2.91x)        &   2.75 $\pm$  0.04 ms ( 3.33x)        &   0.63 ms ( 14.45x)       \\ \hline
    cuda memcpy, memset                       & --                          & --                                    & --                                    & --                                    & --                                    &   2.12 ms,   0.08 ms      \\ 
    other                                     &  17.33 $\pm$  0.23 ms       &  23.38 $\pm$  2.30 ms                 &  24.11 $\pm$  2.30 ms                 &  24.25 $\pm$  2.26 ms                 &  24.28 $\pm$  2.26 ms                 &   1.07 ms                 \\ \hline
    \textbf{TOTAL} ( 7000 points per layer)   & \textbf{360.09 $\pm$  0.74 ms} & \textbf{385.54 $\pm$  2.66 ms ( 0.93x)} & \textbf{151.67 $\pm$  2.57 ms ( 2.37x)} & \textbf{109.50 $\pm$  2.38 ms ( 3.29x)} & \textbf{ 91.17 $\pm$  2.46 ms ( 3.95x)} & \textbf{  7.41 $\pm$  0.01 ms ( 48.61x)}  \\
    \hline 
    \multicolumn{4}{c}{} 
    \end{tabular}}
    \linebreak


    % 8000
    \scalebox{0.7}{\begin{tabular}{l|c|c|c|c|c|c}
    \hline
    CLUE Step                                 & CPU [1T] (baseline)         & CPU TBB [1T]                          & CPU TBB [4T]                          & CPU TBB [8T]                          & CPU TBB [16T]                         & GPU                       \\ \hline
    build fixed-grid spatial index            &  33.54 $\pm$  0.46 ms       &  42.64 $\pm$  0.08 ms ( 0.79x)        &  28.78 $\pm$  0.24 ms ( 1.17x)        &  24.86 $\pm$  0.49 ms ( 1.35x)        &  23.19 $\pm$  0.24 ms ( 1.45x)        &   0.40 ms ( 83.50x)       \\
    calculate local density                   & 134.98 $\pm$  0.49 ms       & 137.13 $\pm$  0.32 ms ( 0.98x)        &  34.59 $\pm$  0.24 ms ( 3.90x)        &  17.89 $\pm$  0.41 ms ( 7.54x)        &  10.80 $\pm$  0.19 ms (12.50x)        &   1.35 ms ( 99.72x)       \\
    calculate nearest-higher and separation   & 194.15 $\pm$  0.21 ms       & 192.84 $\pm$  0.08 ms ( 1.01x)        &  48.59 $\pm$  0.45 ms ( 4.00x)        &  25.00 $\pm$  0.69 ms ( 7.77x)        &  15.16 $\pm$  0.12 ms (12.80x)        &   1.96 ms ( 99.29x)       \\
    decide seeds/outliers, register followers &  27.42 $\pm$  1.47 ms       &  32.62 $\pm$  0.15 ms ( 0.84x)        &  31.34 $\pm$  0.16 ms ( 0.88x)        &  27.74 $\pm$  0.94 ms ( 0.99x)        &  25.34 $\pm$  0.09 ms ( 1.08x)        &   0.42 ms ( 65.18x)       \\
    expand clusters                           &  10.74 $\pm$  0.04 ms       &  15.57 $\pm$  0.09 ms ( 0.69x)        &   5.05 $\pm$  0.05 ms ( 2.13x)        &   3.63 $\pm$  0.16 ms ( 2.96x)        &   3.15 $\pm$  0.06 ms ( 3.41x)        &   0.70 ms ( 15.24x)       \\ \hline
    cuda memcpy, memset                       & --                          & --                                    & --                                    & --                                    & --                                    &   2.43 ms,   0.08 ms      \\ 
    other                                     &  18.81 $\pm$  0.29 ms       &  24.46 $\pm$  2.41 ms                 &  25.22 $\pm$  2.40 ms                 &  25.50 $\pm$  2.39 ms                 &  25.48 $\pm$  2.40 ms                 &   1.11 ms                 \\ \hline
    \textbf{TOTAL} ( 8000 points per layer)   & \textbf{419.64 $\pm$  1.77 ms} & \textbf{445.26 $\pm$  2.42 ms ( 0.94x)} & \textbf{173.56 $\pm$  2.65 ms ( 2.42x)} & \textbf{124.62 $\pm$  2.49 ms ( 3.37x)} & \textbf{103.12 $\pm$  2.61 ms ( 4.07x)} & \textbf{  8.45 $\pm$  0.02 ms ( 49.65x)}  \\
    \hline 
    \multicolumn{4}{c}{} 
    \end{tabular}}
    \linebreak


    % 9000
    \scalebox{0.7}{\begin{tabular}{l|c|c|c|c|c|c}
    \hline
    CLUE Step                                 & CPU [1T] (baseline)         & CPU TBB [1T]                          & CPU TBB [4T]                          & CPU TBB [8T]                          & CPU TBB [16T]                         & GPU                       \\ \hline
    build fixed-grid spatial index            &  37.03 $\pm$  0.22 ms       &  46.53 $\pm$  0.17 ms ( 0.80x)        &  31.69 $\pm$  0.21 ms ( 1.17x)        &  27.36 $\pm$  0.24 ms ( 1.35x)        &  25.48 $\pm$  0.25 ms ( 1.45x)        &   0.43 ms ( 85.19x)       \\
    calculate local density                   & 153.48 $\pm$  0.25 ms       & 154.23 $\pm$  0.19 ms ( 1.00x)        &  39.07 $\pm$  0.21 ms ( 3.93x)        &  20.12 $\pm$  0.33 ms ( 7.63x)        &  12.14 $\pm$  0.10 ms (12.64x)        &   1.50 ms (102.43x)       \\
    calculate nearest-higher and separation   & 226.39 $\pm$  0.38 ms       & 223.16 $\pm$  0.33 ms ( 1.01x)        &  56.40 $\pm$  0.55 ms ( 4.01x)        &  28.85 $\pm$  0.52 ms ( 7.85x)        &  17.55 $\pm$  0.33 ms (12.90x)        &   2.20 ms (102.93x)       \\
    decide seeds/outliers, register followers &  30.01 $\pm$  1.17 ms       &  38.39 $\pm$  0.74 ms ( 0.78x)        &  35.75 $\pm$  0.19 ms ( 0.84x)        &  31.82 $\pm$  0.30 ms ( 0.94x)        &  28.94 $\pm$  0.09 ms ( 1.04x)        &   0.47 ms ( 63.41x)       \\
    expand clusters                           &  12.68 $\pm$  0.06 ms       &  18.30 $\pm$  0.05 ms ( 0.69x)        &   5.88 $\pm$  0.07 ms ( 2.15x)        &   4.28 $\pm$  0.14 ms ( 2.97x)        &   3.75 $\pm$  0.03 ms ( 3.38x)        &   0.89 ms ( 14.25x)       \\ \hline
    cuda memcpy, memset                       & --                          & --                                    & --                                    & --                                    & --                                    &   2.73 ms,   0.08 ms      \\ 
    other                                     &  21.05 $\pm$  0.46 ms       &  25.49 $\pm$  2.52 ms                 &  26.35 $\pm$  2.49 ms                 &  26.52 $\pm$  2.49 ms                 &  26.68 $\pm$  2.48 ms                 &   1.10 ms                 \\ \hline
    \textbf{TOTAL} ( 9000 points per layer)   & \textbf{480.63 $\pm$  1.15 ms} & \textbf{506.10 $\pm$  2.70 ms ( 0.95x)} & \textbf{195.14 $\pm$  2.70 ms ( 2.46x)} & \textbf{138.94 $\pm$  2.96 ms ( 3.46x)} & \textbf{114.53 $\pm$  2.71 ms ( 4.20x)} & \textbf{  9.40 $\pm$  0.02 ms ( 51.12x)}  \\
    \hline 
    \multicolumn{4}{c}{} 
    \end{tabular}}
    \linebreak


    % 10000
    \scalebox{0.7}{\begin{tabular}{l|c|c|c|c|c|c}
    \hline
    CLUE Step                                 & CPU [1T] (baseline)         & CPU TBB [1T]                          & CPU TBB [4T]                          & CPU TBB [8T]                          & CPU TBB [16T]                         & GPU                       \\ \hline
    build fixed-grid spatial index            &  40.65 $\pm$  0.26 ms       &  52.40 $\pm$  0.25 ms ( 0.78x)        &  35.23 $\pm$  0.21 ms ( 1.15x)        &  30.44 $\pm$  0.28 ms ( 1.34x)        &  28.16 $\pm$  0.21 ms ( 1.44x)        &   0.47 ms ( 85.74x)       \\
    calculate local density                   & 170.90 $\pm$  0.38 ms       & 171.69 $\pm$  0.05 ms ( 1.00x)        &  43.51 $\pm$  0.34 ms ( 3.93x)        &  22.40 $\pm$  0.37 ms ( 7.63x)        &  13.42 $\pm$  0.06 ms (12.74x)        &   1.66 ms (102.72x)       \\
    calculate nearest-higher and separation   & 255.18 $\pm$  0.58 ms       & 250.64 $\pm$  0.08 ms ( 1.02x)        &  63.29 $\pm$  0.58 ms ( 4.03x)        &  32.37 $\pm$  0.60 ms ( 7.88x)        &  19.64 $\pm$  0.07 ms (12.99x)        &   2.44 ms (104.68x)       \\
    decide seeds/outliers, register followers &  33.13 $\pm$  0.43 ms       &  42.01 $\pm$  0.14 ms ( 0.79x)        &  40.08 $\pm$  0.20 ms ( 0.83x)        &  35.96 $\pm$  0.36 ms ( 0.92x)        &  32.90 $\pm$  0.09 ms ( 1.01x)        &   0.53 ms ( 62.48x)       \\
    expand clusters                           &  14.77 $\pm$  0.06 ms       &  20.48 $\pm$  0.04 ms ( 0.72x)        &   6.66 $\pm$  0.06 ms ( 2.22x)        &   4.86 $\pm$  0.15 ms ( 3.04x)        &   4.23 $\pm$  0.03 ms ( 3.49x)        &   1.07 ms ( 13.87x)       \\ \hline
    cuda memcpy, memset                       & --                          & --                                    & --                                    & --                                    & --                                    &   3.03 ms,   0.08 ms      \\ 
    other                                     &  23.70 $\pm$  0.47 ms       &  26.67 $\pm$  2.62 ms                 &  27.58 $\pm$  2.59 ms                 &  27.78 $\pm$  2.57 ms                 &  27.76 $\pm$  2.63 ms                 &   1.15 ms                 \\ \hline
    \textbf{TOTAL} (10000 points per layer)   & \textbf{538.33 $\pm$  0.78 ms} & \textbf{563.90 $\pm$  2.64 ms ( 0.95x)} & \textbf{216.35 $\pm$  2.90 ms ( 2.49x)} & \textbf{153.81 $\pm$  2.95 ms ( 3.50x)} & \textbf{126.10 $\pm$  2.82 ms ( 4.27x)} & \textbf{ 10.42 $\pm$  0.02 ms ( 51.67x)}  \\
    \hline 
    \end{tabular}}
    \linebreak


\end{table}
\end{landscape}






%%%%%%%%%%%%%%%%%%%%%%%%%%%
% patatrack02
%%%%%%%%%%%%%%%%%%%%%%%%%%%

\newpage

\subsection{Patatrack02}
\begin{figure}[ht!]
    \centering
    \includegraphics[width=0.82\textwidth]{chapters/HGCal/figures/clue/private/Figure5_patatrack02_1.pdf}
    \caption{ Execution time of CLUE on CPU and GPU both scale linearly with number of input points, ranging from $10^5$ to $10^6$ in total on 100 layers.  Mean and standard deviation are based on 200 trial runs. (10000 trial runs if GPU of Patatrack02) \texttt{Intel Xeon Silver 4114} and \texttt{NVIDIA Tesla V100}}
\end{figure}
\begin{figure}[ht!]
    \centering
    \includegraphics[trim=3cm 0cm 3cm 0cm, clip,width=0.9\textwidth]{chapters/HGCal/figures/clue/private/addition_pttrk02.pdf}
    \caption{Stability during 200 trial runs}
\end{figure}


\newpage

\begin{figure}[ht!]
    \centering
    \includegraphics[trim=3cm 0cm 3cm 0cm, clip,width=\textwidth]{chapters/HGCal/figures/clue/private/addition_pttrk022.pdf}
    \caption{On GPU of Patatrack02, 10000 trials are performed}
\end{figure}




\newpage
\begin{landscape}
\begin{table}[ht!]
    \renewcommand{\arraystretch}{1.25}
    \tiny
    \centering
     % 1000
    \scalebox{0.7}{\begin{tabular}{l|c|c|c|c|c|c}
    \hline
    CLUE Step                                 & CPU [1T] (baseline)         & CPU TBB [1T]                          & CPU TBB [10T]                         & CPU TBB [20T]                         & CPU TBB [40T]                         & GPU                       \\ \hline
    build fixed-grid spatial index            &   5.50 $\pm$  0.82 ms       &   6.87 $\pm$  0.98 ms ( 0.80x)        &   9.38 $\pm$  1.70 ms ( 0.59x)        &  14.44 $\pm$  2.05 ms ( 0.38x)        &  10.40 $\pm$  2.64 ms ( 0.53x)        &   0.03 ms (187.44x)       \\
    calculate local density                   &  15.58 $\pm$  2.53 ms       &  19.42 $\pm$  2.27 ms ( 0.80x)        &   2.61 $\pm$  0.65 ms ( 5.97x)        &   2.27 $\pm$  0.40 ms ( 6.86x)        &   0.93 $\pm$  0.46 ms (16.77x)        &   0.06 ms (278.53x)       \\
    calculate nearest-higher and separation   &  19.92 $\pm$  4.06 ms       &  24.70 $\pm$  2.13 ms ( 0.81x)        &   3.11 $\pm$  0.79 ms ( 6.40x)        &   2.70 $\pm$  0.45 ms ( 7.38x)        &   1.09 $\pm$  0.33 ms (18.33x)        &   0.08 ms (238.02x)       \\
    decide seeds/outliers, register followers &   4.20 $\pm$  0.82 ms       &   4.98 $\pm$  0.47 ms ( 0.84x)        &   9.15 $\pm$  1.33 ms ( 0.46x)        &  13.25 $\pm$  1.35 ms ( 0.32x)        &   9.65 $\pm$  0.69 ms ( 0.43x)        &   0.02 ms (202.73x)       \\
    expand clusters                           &   0.47 $\pm$  0.11 ms       &   2.43 $\pm$  0.25 ms ( 0.19x)        &   0.43 $\pm$  0.10 ms ( 1.10x)        &   0.36 $\pm$  0.03 ms ( 1.30x)        &   0.22 $\pm$  0.02 ms ( 2.11x)        &   0.05 ms (  9.26x)       \\ \hline
    cuda memcpy, memset                       & --                          & --                                    & --                                    & --                                    & --                                    &   0.17 ms,   0.08 ms      \\ 
    other                                     &   6.31 $\pm$  2.57 ms       &  22.32 $\pm$  4.19 ms                 &  20.38 $\pm$  3.76 ms                 &  23.61 $\pm$  2.58 ms                 &  20.84 $\pm$  5.98 ms                 &   0.58 ms                 \\ \hline
    \textbf{TOTAL} ( 1000 points per layer)   & \textbf{ 51.98 $\pm$  8.94 ms} & \textbf{ 80.72 $\pm$  8.43 ms ( 0.64x)} & \textbf{ 45.06 $\pm$  7.25 ms ( 1.15x)} & \textbf{ 56.63 $\pm$  5.69 ms ( 0.92x)} & \textbf{ 43.13 $\pm$  8.20 ms ( 1.21x)} & \textbf{  1.07 $\pm$  0.14 ms ( 48.47x)}  \\
    \hline 
    \multicolumn{4}{c}{} 
    \end{tabular}}
    \linebreak


    % 2000
    \scalebox{0.7}{\begin{tabular}{l|c|c|c|c|c|c}
    \hline
    CLUE Step                                 & CPU [1T] (baseline)         & CPU TBB [1T]                          & CPU TBB [10T]                         & CPU TBB [20T]                         & CPU TBB [40T]                         & GPU                       \\ \hline
    build fixed-grid spatial index            &  11.01 $\pm$  0.67 ms       &  12.89 $\pm$  0.99 ms ( 0.85x)        &  23.52 $\pm$  4.06 ms ( 0.47x)        &  24.77 $\pm$  4.16 ms ( 0.44x)        &  29.30 $\pm$  6.06 ms ( 0.38x)        &   0.07 ms (167.61x)       \\
    calculate local density                   &  35.98 $\pm$  1.77 ms       &  41.51 $\pm$  2.90 ms ( 0.87x)        &   7.51 $\pm$  1.30 ms ( 4.79x)        &   4.15 $\pm$  0.65 ms ( 8.68x)        &   2.78 $\pm$  2.26 ms (12.94x)        &   0.11 ms (318.84x)       \\
    calculate nearest-higher and separation   &  45.00 $\pm$  2.30 ms       &  50.14 $\pm$  4.23 ms ( 0.90x)        &   8.83 $\pm$  1.43 ms ( 5.09x)        &   5.01 $\pm$  0.73 ms ( 8.98x)        &   3.22 $\pm$  0.52 ms (13.97x)        &   0.14 ms (320.98x)       \\
    decide seeds/outliers, register followers &   8.39 $\pm$  0.60 ms       &   9.00 $\pm$  0.69 ms ( 0.93x)        &  21.76 $\pm$  2.50 ms ( 0.39x)        &  22.51 $\pm$  2.67 ms ( 0.37x)        &  23.95 $\pm$  2.29 ms ( 0.35x)        &   0.06 ms (145.20x)       \\
    expand clusters                           &   1.63 $\pm$  0.13 ms       &   3.61 $\pm$  0.15 ms ( 0.45x)        &   0.88 $\pm$  0.17 ms ( 1.86x)        &   0.61 $\pm$  0.04 ms ( 2.68x)        &   0.46 $\pm$  0.10 ms ( 3.53x)        &   0.08 ms ( 21.75x)       \\ \hline
    cuda memcpy, memset                       & --                          & --                                    & --                                    & --                                    & --                                    &   0.33 ms,   0.09 ms      \\ 
    other                                     &   8.45 $\pm$  1.58 ms       &  19.80 $\pm$ 10.11 ms                 &  25.36 $\pm$ 14.49 ms                 &  24.11 $\pm$  4.43 ms                 &  26.29 $\pm$  9.01 ms                 &   0.76 ms                 \\ \hline
    \textbf{TOTAL} ( 2000 points per layer)   & \textbf{110.45 $\pm$  6.22 ms} & \textbf{136.95 $\pm$ 14.78 ms ( 0.81x)} & \textbf{ 87.86 $\pm$ 18.09 ms ( 1.26x)} & \textbf{ 81.15 $\pm$ 10.80 ms ( 1.36x)} & \textbf{ 86.01 $\pm$ 13.72 ms ( 1.28x)} & \textbf{  1.63 $\pm$  0.12 ms ( 67.66x)}  \\
    \hline 
    \multicolumn{4}{c}{} 
    \end{tabular}}
    \linebreak


    % 3000
    \scalebox{0.7}{\begin{tabular}{l|c|c|c|c|c|c}
    \hline
    CLUE Step                                 & CPU [1T] (baseline)         & CPU TBB [1T]                          & CPU TBB [10T]                         & CPU TBB [20T]                         & CPU TBB [40T]                         & GPU                       \\ \hline
    build fixed-grid spatial index            &  17.82 $\pm$  1.88 ms       &  19.68 $\pm$  1.08 ms ( 0.91x)        &  37.75 $\pm$  4.77 ms ( 0.47x)        &  43.22 $\pm$  5.46 ms ( 0.41x)        &  46.26 $\pm$  6.08 ms ( 0.39x)        &   0.10 ms (184.57x)       \\
    calculate local density                   &  56.15 $\pm$  4.39 ms       &  62.31 $\pm$  3.40 ms ( 0.90x)        &  11.11 $\pm$  1.42 ms ( 5.05x)        &   6.67 $\pm$  1.11 ms ( 8.42x)        &   4.06 $\pm$  1.01 ms (13.83x)        &   0.14 ms (415.21x)       \\
    calculate nearest-higher and separation   &  71.16 $\pm$  7.17 ms       &  76.14 $\pm$  3.79 ms ( 0.93x)        &  13.11 $\pm$  1.42 ms ( 5.43x)        &   7.81 $\pm$  0.47 ms ( 9.12x)        &   4.90 $\pm$  0.41 ms (14.53x)        &   0.21 ms (343.11x)       \\
    decide seeds/outliers, register followers &  12.63 $\pm$  1.68 ms       &  13.50 $\pm$  1.25 ms ( 0.94x)        &  32.54 $\pm$  2.13 ms ( 0.39x)        &  34.89 $\pm$  2.24 ms ( 0.36x)        &  35.23 $\pm$  1.51 ms ( 0.36x)        &   0.09 ms (142.97x)       \\
    expand clusters                           &   2.47 $\pm$  0.34 ms       &   5.06 $\pm$  0.09 ms ( 0.49x)        &   1.19 $\pm$  0.05 ms ( 2.07x)        &   0.81 $\pm$  0.15 ms ( 3.07x)        &   0.73 $\pm$  1.07 ms ( 3.37x)        &   0.08 ms ( 30.39x)       \\ \hline
    cuda memcpy, memset                       & --                          & --                                    & --                                    & --                                    & --                                    &   0.51 ms,   0.10 ms      \\ 
    other                                     &  11.27 $\pm$  1.78 ms       &  20.76 $\pm$  7.07 ms                 &  24.48 $\pm$  4.21 ms                 &  29.02 $\pm$ 12.28 ms                 &  28.85 $\pm$ 11.58 ms                 &   0.97 ms                 \\ \hline
    \textbf{TOTAL} ( 3000 points per layer)   & \textbf{171.50 $\pm$ 15.29 ms} & \textbf{197.45 $\pm$ 10.59 ms ( 0.87x)} & \textbf{120.18 $\pm$ 11.08 ms ( 1.43x)} & \textbf{122.41 $\pm$ 15.98 ms ( 1.40x)} & \textbf{120.03 $\pm$ 16.28 ms ( 1.43x)} & \textbf{  2.19 $\pm$  0.17 ms ( 78.21x)}  \\
    \hline 
    \multicolumn{4}{c}{} 
    \end{tabular}}
    \linebreak


    % 4000
    \scalebox{0.7}{\begin{tabular}{l|c|c|c|c|c|c}
    \hline
    CLUE Step                                 & CPU [1T] (baseline)         & CPU TBB [1T]                          & CPU TBB [10T]                         & CPU TBB [20T]                         & CPU TBB [40T]                         & GPU                       \\ \hline
    build fixed-grid spatial index            &  24.92 $\pm$  3.37 ms       &  25.17 $\pm$  0.82 ms ( 0.99x)        &  55.06 $\pm$  2.77 ms ( 0.45x)        &  53.85 $\pm$  3.04 ms ( 0.46x)        &  58.18 $\pm$  2.91 ms ( 0.43x)        &   0.13 ms (195.20x)       \\
    calculate local density                   &  83.62 $\pm$  9.08 ms       &  86.14 $\pm$  2.17 ms ( 0.97x)        &  16.10 $\pm$  0.67 ms ( 5.19x)        &   8.67 $\pm$  1.54 ms ( 9.65x)        &   5.12 $\pm$  0.35 ms (16.34x)        &   0.20 ms (413.61x)       \\
    calculate nearest-higher and separation   & 111.91 $\pm$ 11.02 ms       & 111.97 $\pm$  2.84 ms ( 1.00x)        &  19.50 $\pm$  0.62 ms ( 5.74x)        &  10.75 $\pm$  0.75 ms (10.41x)        &   6.90 $\pm$  0.49 ms (16.23x)        &   0.30 ms (369.89x)       \\
    decide seeds/outliers, register followers &  18.50 $\pm$  2.46 ms       &  19.26 $\pm$  0.46 ms ( 0.96x)        &  44.05 $\pm$  1.74 ms ( 0.42x)        &  44.40 $\pm$  1.91 ms ( 0.42x)        &  45.22 $\pm$  1.54 ms ( 0.41x)        &   0.12 ms (149.11x)       \\
    expand clusters                           &   4.97 $\pm$  0.72 ms       &   7.72 $\pm$  0.19 ms ( 0.64x)        &   1.90 $\pm$  0.10 ms ( 2.61x)        &   1.15 $\pm$  0.13 ms ( 4.31x)        &   1.05 $\pm$  0.04 ms ( 4.71x)        &   0.12 ms ( 40.35x)       \\ \hline
    cuda memcpy, memset                       & --                          & --                                    & --                                    & --                                    & --                                    &   0.67 ms,   0.10 ms      \\ 
    other                                     &  14.49 $\pm$  3.37 ms       &  20.97 $\pm$  4.58 ms                 &  27.48 $\pm$  4.34 ms                 &  30.24 $\pm$ 11.54 ms                 &  30.11 $\pm$  4.34 ms                 &   1.63 ms                 \\ \hline
    \textbf{TOTAL} ( 4000 points per layer)   & \textbf{258.41 $\pm$ 26.14 ms} & \textbf{271.23 $\pm$ 10.62 ms ( 0.95x)} & \textbf{164.08 $\pm$  7.61 ms ( 1.57x)} & \textbf{149.07 $\pm$ 13.18 ms ( 1.73x)} & \textbf{146.58 $\pm$  7.21 ms ( 1.76x)} & \textbf{  3.28 $\pm$  0.14 ms ( 78.88x)}  \\
    \hline 
    \multicolumn{4}{c}{} 
    \end{tabular}}
    \linebreak


    % 5000
    \scalebox{0.7}{\begin{tabular}{l|c|c|c|c|c|c}
    \hline
    CLUE Step                                 & CPU [1T] (baseline)         & CPU TBB [1T]                          & CPU TBB [10T]                         & CPU TBB [20T]                         & CPU TBB [40T]                         & GPU                       \\ \hline
    build fixed-grid spatial index            &  30.44 $\pm$  2.04 ms       &  33.77 $\pm$  3.01 ms ( 0.90x)        &  63.89 $\pm$  5.23 ms ( 0.48x)        &  64.53 $\pm$  5.32 ms ( 0.47x)        &  70.36 $\pm$  3.08 ms ( 0.43x)        &   0.16 ms (191.80x)       \\
    calculate local density                   & 102.24 $\pm$  4.98 ms       & 111.40 $\pm$  7.26 ms ( 0.92x)        &  18.78 $\pm$  0.79 ms ( 5.44x)        &  10.27 $\pm$  1.32 ms ( 9.95x)        &   5.99 $\pm$  0.29 ms (17.07x)        &   0.25 ms (411.02x)       \\
    calculate nearest-higher and separation   & 139.52 $\pm$  9.76 ms       & 145.92 $\pm$  9.00 ms ( 0.96x)        &  22.96 $\pm$  0.71 ms ( 6.08x)        &  12.75 $\pm$  0.71 ms (10.94x)        &   8.35 $\pm$  0.41 ms (16.71x)        &   0.41 ms (342.51x)       \\
    decide seeds/outliers, register followers &  23.19 $\pm$  2.01 ms       &  25.69 $\pm$  2.93 ms ( 0.90x)        &  51.20 $\pm$  3.66 ms ( 0.45x)        &  53.37 $\pm$  2.50 ms ( 0.43x)        &  54.61 $\pm$  1.34 ms ( 0.42x)        &   0.15 ms (150.13x)       \\
    expand clusters                           &   5.89 $\pm$  0.45 ms       &  10.10 $\pm$  1.33 ms ( 0.58x)        &   2.28 $\pm$  0.11 ms ( 2.58x)        &   1.48 $\pm$  0.23 ms ( 3.97x)        &   1.36 $\pm$  0.03 ms ( 4.34x)        &   0.14 ms ( 41.11x)       \\ \hline
    cuda memcpy, memset                       & --                          & --                                    & --                                    & --                                    & --                                    &   1.31 ms,   0.09 ms      \\ 
    other                                     &  16.09 $\pm$  1.77 ms       &  24.58 $\pm$ 11.43 ms                 &  28.63 $\pm$  3.42 ms                 &  33.26 $\pm$ 13.64 ms                 &  31.55 $\pm$  4.34 ms                 &   0.72 ms                 \\ \hline
    \textbf{TOTAL} ( 5000 points per layer)   & \textbf{317.37 $\pm$ 18.46 ms} & \textbf{351.46 $\pm$ 25.59 ms ( 0.90x)} & \textbf{187.74 $\pm$ 10.64 ms ( 1.69x)} & \textbf{175.67 $\pm$ 16.69 ms ( 1.81x)} & \textbf{172.21 $\pm$  6.79 ms ( 1.84x)} & \textbf{  3.24 $\pm$  0.20 ms ( 98.09x)}  \\
    \hline 
    \end{tabular}}

\end{table}
\end{landscape}


\newpage
\begin{landscape}
\begin{table}[ht!]
    \renewcommand{\arraystretch}{1.25}
    \tiny
    \centering
 
    % 6000
    \scalebox{0.7}{\begin{tabular}{l|c|c|c|c|c|c}
    \hline
    CLUE Step                                 & CPU [1T] (baseline)         & CPU TBB [1T]                          & CPU TBB [10T]                         & CPU TBB [20T]                         & CPU TBB [40T]                         & GPU                       \\ \hline
    build fixed-grid spatial index            &  35.76 $\pm$  1.87 ms       &  39.43 $\pm$  0.97 ms ( 0.91x)        &  75.72 $\pm$  3.02 ms ( 0.47x)        &  74.47 $\pm$  3.41 ms ( 0.48x)        &  84.78 $\pm$  4.55 ms ( 0.42x)        &   0.19 ms (192.49x)       \\
    calculate local density                   & 119.51 $\pm$  3.10 ms       & 131.76 $\pm$  3.67 ms ( 0.91x)        &  21.33 $\pm$  0.68 ms ( 5.60x)        &  11.09 $\pm$  0.39 ms (10.78x)        &   7.58 $\pm$  3.10 ms (15.78x)        &   0.28 ms (426.93x)       \\
    calculate nearest-higher and separation   & 164.32 $\pm$  2.80 ms       & 173.73 $\pm$  5.06 ms ( 0.95x)        &  26.29 $\pm$  0.65 ms ( 6.25x)        &  14.47 $\pm$  0.46 ms (11.36x)        &   9.72 $\pm$  1.26 ms (16.91x)        &   0.46 ms (354.68x)       \\
    decide seeds/outliers, register followers &  25.90 $\pm$  1.13 ms       &  30.38 $\pm$  2.83 ms ( 0.85x)        &  61.15 $\pm$  1.32 ms ( 0.42x)        &  61.14 $\pm$  1.65 ms ( 0.42x)        &  65.08 $\pm$  3.81 ms ( 0.40x)        &   0.19 ms (137.26x)       \\
    expand clusters                           &   6.25 $\pm$  0.18 ms       &  12.17 $\pm$  0.65 ms ( 0.51x)        &   2.69 $\pm$  0.10 ms ( 2.32x)        &   1.80 $\pm$  0.06 ms ( 3.47x)        &   1.40 $\pm$  0.36 ms ( 4.48x)        &   0.15 ms ( 42.16x)       \\ \hline
    cuda memcpy, memset                       & --                          & --                                    & --                                    & --                                    & --                                    &   1.73 ms,   0.09 ms      \\ 
    other                                     &  18.50 $\pm$  1.30 ms       &  26.08 $\pm$ 12.39 ms                 &  32.71 $\pm$  4.59 ms                 &  33.13 $\pm$  4.71 ms                 &  40.31 $\pm$ 29.49 ms                 &   0.44 ms                 \\ \hline
    \textbf{TOTAL} ( 6000 points per layer)   & \textbf{370.24 $\pm$  8.84 ms} & \textbf{413.56 $\pm$ 17.25 ms ( 0.90x)} & \textbf{219.89 $\pm$  7.16 ms ( 1.68x)} & \textbf{196.10 $\pm$  7.53 ms ( 1.89x)} & \textbf{208.85 $\pm$ 31.51 ms ( 1.77x)} & \textbf{  3.52 $\pm$  0.22 ms (105.19x)}  \\
    \hline 
    \multicolumn{4}{c}{} 
    \end{tabular}}
    \linebreak


    % 7000
    \scalebox{0.7}{\begin{tabular}{l|c|c|c|c|c|c}
    \hline
    CLUE Step                                 & CPU [1T] (baseline)         & CPU TBB [1T]                          & CPU TBB [10T]                         & CPU TBB [20T]                         & CPU TBB [40T]                         & GPU                       \\ \hline
    build fixed-grid spatial index            &  42.27 $\pm$  1.83 ms       &  45.61 $\pm$  2.51 ms ( 0.93x)        &  87.13 $\pm$  3.34 ms ( 0.49x)        &  85.61 $\pm$  4.38 ms ( 0.49x)        &  94.75 $\pm$  3.33 ms ( 0.45x)        &   0.21 ms (201.17x)       \\
    calculate local density                   & 147.56 $\pm$  2.92 ms       & 159.23 $\pm$  5.17 ms ( 0.93x)        &  24.76 $\pm$  0.75 ms ( 5.96x)        &  14.06 $\pm$  3.09 ms (10.49x)        &   8.07 $\pm$  0.24 ms (18.28x)        &   0.35 ms (423.74x)       \\
    calculate nearest-higher and separation   & 211.28 $\pm$  3.12 ms       & 215.99 $\pm$  4.35 ms ( 0.98x)        &  31.21 $\pm$  0.65 ms ( 6.77x)        &  17.59 $\pm$  1.06 ms (12.01x)        &  11.53 $\pm$  0.48 ms (18.32x)        &   0.58 ms (366.26x)       \\
    decide seeds/outliers, register followers &  35.20 $\pm$  1.69 ms       &  36.91 $\pm$  3.75 ms ( 0.95x)        &  73.20 $\pm$  2.03 ms ( 0.48x)        &  73.44 $\pm$  3.88 ms ( 0.48x)        &  74.77 $\pm$  1.26 ms ( 0.47x)        &   0.22 ms (156.60x)       \\
    expand clusters                           &  10.24 $\pm$  0.29 ms       &  16.13 $\pm$  0.24 ms ( 0.63x)        &   3.69 $\pm$  0.13 ms ( 2.78x)        &   2.48 $\pm$  0.56 ms ( 4.13x)        &   2.02 $\pm$  0.09 ms ( 5.07x)        &   0.20 ms ( 50.39x)       \\ \hline
    cuda memcpy, memset                       & --                          & --                                    & --                                    & --                                    & --                                    &   2.19 ms,   0.10 ms      \\ 
    other                                     &  20.72 $\pm$  1.35 ms       &  27.36 $\pm$ 12.44 ms                 &  34.20 $\pm$  3.31 ms                 &  38.92 $\pm$ 16.94 ms                 &  35.56 $\pm$  4.77 ms                 &   1.55 ms                 \\ \hline
    \textbf{TOTAL} ( 7000 points per layer)   & \textbf{467.27 $\pm$  7.95 ms} & \textbf{501.22 $\pm$ 18.18 ms ( 0.93x)} & \textbf{254.19 $\pm$  6.65 ms ( 1.84x)} & \textbf{232.11 $\pm$ 19.73 ms ( 2.01x)} & \textbf{226.70 $\pm$  6.86 ms ( 2.06x)} & \textbf{  5.41 $\pm$  2.12 ms ( 86.44x)}  \\
    \hline 
    \multicolumn{4}{c}{} 
    \end{tabular}}
    \linebreak


    % 8000
    \scalebox{0.7}{\begin{tabular}{l|c|c|c|c|c|c}
    \hline
    CLUE Step                                 & CPU [1T] (baseline)         & CPU TBB [1T]                          & CPU TBB [10T]                         & CPU TBB [20T]                         & CPU TBB [40T]                         & GPU                       \\ \hline
    build fixed-grid spatial index            &  47.62 $\pm$  1.75 ms       &  59.22 $\pm$  7.75 ms ( 0.80x)        &  97.24 $\pm$  5.76 ms ( 0.49x)        &  97.56 $\pm$  6.56 ms ( 0.49x)        & 106.56 $\pm$  5.34 ms ( 0.45x)        &   0.23 ms (204.46x)       \\
    calculate local density                   & 172.60 $\pm$  2.42 ms       & 194.45 $\pm$ 19.03 ms ( 0.89x)        &  27.69 $\pm$  0.99 ms ( 6.23x)        &  15.27 $\pm$  1.44 ms (11.30x)        &   9.44 $\pm$  1.63 ms (18.29x)        &   0.39 ms (440.43x)       \\
    calculate nearest-higher and separation   & 248.12 $\pm$  2.24 ms       & 264.91 $\pm$ 22.50 ms ( 0.94x)        &  35.13 $\pm$  0.99 ms ( 7.06x)        &  19.97 $\pm$  1.79 ms (12.43x)        &  12.94 $\pm$  0.48 ms (19.17x)        &   0.70 ms (352.76x)       \\
    decide seeds/outliers, register followers &  40.82 $\pm$  0.81 ms       &  50.10 $\pm$  7.07 ms ( 0.81x)        &  82.35 $\pm$  4.08 ms ( 0.50x)        &  84.42 $\pm$  3.94 ms ( 0.48x)        &  85.42 $\pm$  2.79 ms ( 0.48x)        &   0.26 ms (155.20x)       \\
    expand clusters                           &  11.65 $\pm$  0.48 ms       &  21.73 $\pm$  3.58 ms ( 0.54x)        &   4.29 $\pm$  0.19 ms ( 2.72x)        &   2.69 $\pm$  0.39 ms ( 4.33x)        &   1.89 $\pm$  0.50 ms ( 6.16x)        &   0.23 ms ( 51.68x)       \\ \hline
    cuda memcpy, memset                       & --                          & --                                    & --                                    & --                                    & --                                    &   2.85 ms,   0.09 ms      \\ 
    other                                     &  23.45 $\pm$  1.44 ms       &  31.35 $\pm$ 12.78 ms                 &  36.38 $\pm$  5.38 ms                 &  41.84 $\pm$ 18.94 ms                 &  39.75 $\pm$ 16.78 ms                 &   0.11 ms                 \\ \hline
    \textbf{TOTAL} ( 8000 points per layer)   & \textbf{544.27 $\pm$  5.99 ms} & \textbf{621.76 $\pm$ 57.28 ms ( 0.88x)} & \textbf{283.09 $\pm$ 11.45 ms ( 1.92x)} & \textbf{261.75 $\pm$ 21.76 ms ( 2.08x)} & \textbf{256.00 $\pm$ 19.76 ms ( 2.13x)} & \textbf{  4.86 $\pm$  0.43 ms (111.94x)}  \\
    \hline 
    \multicolumn{4}{c}{} 
    \end{tabular}}
    \linebreak


    % 9000
    \scalebox{0.7}{\begin{tabular}{l|c|c|c|c|c|c}
    \hline
    CLUE Step                                 & CPU [1T] (baseline)         & CPU TBB [1T]                          & CPU TBB [10T]                         & CPU TBB [20T]                         & CPU TBB [40T]                         & GPU                       \\ \hline
    build fixed-grid spatial index            &  53.36 $\pm$  2.02 ms       &  58.60 $\pm$  3.45 ms ( 0.91x)        & 104.46 $\pm$  7.57 ms ( 0.51x)        & 103.58 $\pm$  5.05 ms ( 0.52x)        & 118.62 $\pm$  9.16 ms ( 0.45x)        &   0.26 ms (208.86x)       \\
    calculate local density                   & 196.28 $\pm$  2.15 ms       & 209.87 $\pm$  3.99 ms ( 0.94x)        &  30.25 $\pm$  1.03 ms ( 6.49x)        &  16.07 $\pm$  0.54 ms (12.22x)        &  10.38 $\pm$  2.00 ms (18.90x)        &   0.45 ms (435.32x)       \\
    calculate nearest-higher and separation   & 290.55 $\pm$  3.50 ms       & 293.58 $\pm$  4.95 ms ( 0.99x)        &  39.61 $\pm$  0.97 ms ( 7.34x)        &  22.19 $\pm$  0.89 ms (13.09x)        &  14.70 $\pm$  0.92 ms (19.76x)        &   0.80 ms (361.26x)       \\
    decide seeds/outliers, register followers &  47.10 $\pm$  0.92 ms       &  50.71 $\pm$  4.60 ms ( 0.93x)        &  90.56 $\pm$  7.19 ms ( 0.52x)        &  93.06 $\pm$  3.41 ms ( 0.51x)        &  95.67 $\pm$  4.23 ms ( 0.49x)        &   0.30 ms (156.05x)       \\
    expand clusters                           &  14.33 $\pm$  0.13 ms       &  24.46 $\pm$  0.28 ms ( 0.59x)        &   5.03 $\pm$  0.32 ms ( 2.85x)        &   3.15 $\pm$  0.12 ms ( 4.54x)        &   2.27 $\pm$  0.44 ms ( 6.31x)        &   0.30 ms ( 48.26x)       \\ \hline
    cuda memcpy, memset                       & --                          & --                                    & --                                    & --                                    & --                                    &   2.95 ms,   0.10 ms      \\ 
    other                                     &  25.80 $\pm$  0.88 ms       &  30.43 $\pm$ 13.83 ms                 &  35.64 $\pm$  4.18 ms                 &  35.91 $\pm$  5.76 ms                 &  47.22 $\pm$ 29.27 ms                 &   0.51 ms                 \\ \hline
    \textbf{TOTAL} ( 9000 points per layer)   & \textbf{627.43 $\pm$  6.88 ms} & \textbf{667.66 $\pm$ 19.02 ms ( 0.94x)} & \textbf{305.54 $\pm$ 15.13 ms ( 2.05x)} & \textbf{273.97 $\pm$ 10.69 ms ( 2.29x)} & \textbf{288.86 $\pm$ 31.83 ms ( 2.17x)} & \textbf{  5.66 $\pm$  0.35 ms (110.84x)}  \\
    \hline 
    \multicolumn{4}{c}{} 
    \end{tabular}}
    \linebreak


    % 10000
    \scalebox{0.7}{\begin{tabular}{l|c|c|c|c|c|c}
    \hline
    CLUE Step                                 & CPU [1T] (baseline)         & CPU TBB [1T]                          & CPU TBB [10T]                         & CPU TBB [20T]                         & CPU TBB [40T]                         & GPU                       \\ \hline
    build fixed-grid spatial index            &  59.27 $\pm$  1.60 ms       &  70.34 $\pm$ 10.21 ms ( 0.84x)        & 117.71 $\pm$  6.43 ms ( 0.50x)        & 111.76 $\pm$  5.29 ms ( 0.53x)        & 122.71 $\pm$  3.46 ms ( 0.48x)        &   0.28 ms (208.63x)       \\
    calculate local density                   & 218.42 $\pm$  2.47 ms       & 235.04 $\pm$ 12.82 ms ( 0.93x)        &  33.69 $\pm$  2.61 ms ( 6.48x)        &  17.41 $\pm$  0.50 ms (12.55x)        &  10.93 $\pm$  0.38 ms (19.99x)        &   0.51 ms (430.57x)       \\
    calculate nearest-higher and separation   & 326.89 $\pm$  2.86 ms       & 333.39 $\pm$ 14.81 ms ( 0.98x)        &  45.47 $\pm$  2.54 ms ( 7.19x)        &  24.30 $\pm$  0.89 ms (13.45x)        &  15.87 $\pm$  0.45 ms (20.59x)        &   0.89 ms (368.54x)       \\
    decide seeds/outliers, register followers &  54.43 $\pm$  2.54 ms       &  60.57 $\pm$  7.66 ms ( 0.90x)        & 109.43 $\pm$  7.70 ms ( 0.50x)        & 104.43 $\pm$  3.81 ms ( 0.52x)        & 104.05 $\pm$  1.49 ms ( 0.52x)        &   0.34 ms (162.38x)       \\
    expand clusters                           &  17.37 $\pm$  1.46 ms       &  30.66 $\pm$  6.13 ms ( 0.57x)        &   6.09 $\pm$  1.35 ms ( 2.86x)        &   3.59 $\pm$  0.13 ms ( 4.84x)        &   2.97 $\pm$  0.09 ms ( 5.85x)        &   0.35 ms ( 49.74x)       \\ \hline
    cuda memcpy, memset                       & --                          & --                                    & --                                    & --                                    & --                                    &   2.87 ms,   0.10 ms      \\ 
    other                                     &  29.11 $\pm$  1.66 ms       &  32.04 $\pm$  6.03 ms                 &  44.86 $\pm$ 15.73 ms                 &  39.56 $\pm$  4.60 ms                 &  40.56 $\pm$  6.21 ms                 &   1.30 ms                 \\ \hline
    \textbf{TOTAL} (10000 points per layer)   & \textbf{705.49 $\pm$  7.93 ms} & \textbf{762.03 $\pm$ 52.43 ms ( 0.93x)} & \textbf{357.24 $\pm$ 19.68 ms ( 1.97x)} & \textbf{301.04 $\pm$ 10.11 ms ( 2.34x)} & \textbf{297.09 $\pm$  8.48 ms ( 2.37x)} & \textbf{  6.63 $\pm$  0.63 ms (106.42x)}  \\
    \hline 
    \end{tabular}}


\end{table}
\end{landscape}




    \end{appendix}
    
    
    
    
     

    \begin{singlespace}
    \bibliographystyle{unsrt}
    \bibliography{
        chapters/Introduction/chapterReference,
        chapters/Physics/chapterReference,
        chapters/CMSExperiment/chapterReference,
        chapters/Analysis/chapterReference,
        chapters/HGCal/chapterReference
        }
    \end{singlespace}
    

    
\end{document}
